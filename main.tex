\documentclass[headings=small,a4paper,12pt,oneside]{scrreprt}

\setcounter{tocdepth}{3}
\setcounter{secnumdepth}{3}

\usepackage{setspace}
\usepackage[right=4cm,top=3cm,left=2cm,bottom=3cm]{geometry}
\usepackage[utf8]{inputenc}
\usepackage[ngerman]{babel}
\usepackage[T1]{fontenc}
\usepackage{lmodern}
\usepackage{scrlayer-scrpage}
\usepackage{blindtext}
\usepackage{graphicx}
\usepackage{hyperref}

\graphicspath{ {./images/} }

\pagestyle{scrheadings}
\clearpairofpagestyles
\chead{Open Science in den Geschichtswissenschaften?}
\cfoot{Seite \thepage}
%\ofoot{page \thepage~of \totalpages}
\setlength{\headheight}{14.5pt}

\begin{document}

\newgeometry{right=2cm,top=2cm,left=2cm,bottom=2cm}
\singlespacing
\titlehead{Humboldt-Universität zu Berlin\\
Philosophische Fakultät\\
Institut für Geschichtswissenschaften\vspace*{4cm}}
\subject{Masterarbeit}
\title{Open Science in den Geschichtswissenschaften?}
\subtitle{Konzeption eines offenen Forschungsdatenmanagements am Beispiel von Forschungsdaten zu Jüdischen Gewerbebetrieben im Nationalsozialismus}
\author{vorgelegt von:\\Sophie Eckenstaler}
\date{am 07.06.2022\vspace*{3.7cm}}
\publishers{\normalsize
%Layout für markdown-Generierung
Erstbetreuer: Prof. Dr. Rüdiger Hohls, Institut für Geschichtswissenschaften, HU Berlin\\ 
Zweitbetreuer: Prof. Dr. Michael Wildt, Institut für Geschichtswissenschaften, HU Berlin\\
Studiengang: Master of Arts, Geschichtswissenschaften, Schwerpunkt: Digital History\\
Matrikelnr.: 596272\\
E-Mail: sophie.eckenstaler@hu-berlin.de\\
Eberswalde, den 7. Juni 2022
% Layout für den Druck auskommentieren
%\raggedright
%\begin{tabbing}
%\hspace{1.2in}\=\hspace{1in}\=\kill
%Erstbetreuer:\>Prof. Dr. Rüdiger Hohls, Institut für Geschichtswissenschaften, HU Berlin\\ 
%Zweitbetreuer:\>Prof. Dr. Michael Wildt, Institut für Geschichtswissenschaften, HU Berlin\\
%Studiengang:\>Master of Arts, Geschichtswissenschaften, Schwerpunkt: Digital History\\
%Matrikelnr.:\>596272\\
%E-Mail:\>sophie.eckenstaler@hu-berlin.de\\
%\end{tabbing}
}



\maketitle

\singlespacing
\tableofcontents
\restoregeometry

\singlespacing
\chapter{Einleitung}
\onehalfspacing

\section{Motivation und Fragestellung}

\section{Zielsetzung}

\section{Methodisches Vorgehen}

\singlespacing 
\chapter{Ausgangslage und Bedarfsanalyse}
\input{chapters/chapter01}

\singlespacing
\chapter{Rahmenbedingungen und Qualitätseigenschaften}
\onehalfspacing

\section{Rahmenbedingungen}
\subsection{Rechtliche und ethische Rahmenbedingungen}
\subsection{Anwendergruppen und Anwendungsbereich}

\section{Qualitätseigenschaften}
\subsection{Open Data}
\subsection{FAIR Data}
\subsection{CARE Data}
\subsection{Open Methodology}

\singlespacing 
\chapter{Prototypische Lösung}
\onehalfspacing

User Story driven

\section{Lösungskonzepte}
\subsection{Openess-Kriterien}
\subsection{FAIR and CARE Principles}
\subsection{Wikimedia als Open Tech Stack}

\section{User Stories}

Strukturiert an idealtypischen Forschungsprozess

\subsection{User Story 1}

\subsection{User Story 2}

\subsection{User Story 3}

\subsection{User Story 4}

\subsection{User Story 5}

\section{Ergebnisse}

\singlespacing 
\chapter{Fazit und Ausblick}
\input{chapters/conclusion}

\singlespacing 
\appendix
\chapter{Appendix Title}
\input{chapters/appendix}

\end{document}

