\onehalfspacing

\section{Lösungsansatz}
Wikidata als offener Forschungsdatenmanagement-Service
Auf diese Prinzipien beruft sich auch die Wikimedia Foundation, für ihre Produkte und macht diese auch für die Wissenschaft interessant. Mittlerweile gibt es diverse Kooperationen zwischen wissenschaftlichen Einrichtungen und der Wikimedia. So hat die Deutsche Nationalbibliothek ein Projekt gestartet, in dem sie die GND zugänglicher und nachnutzbarer für gestalten will und damit ihre strenge GND-Policy
setzt alle FAIR-Prinzipien um
eigene Wikibase-Instanzen aufsetzen --> technisch aufwändig und Informatik-Kenntnisse, wäre von befragten Historiker*innen nicht umgesetzbar gewesen
ähnliche Infrastrukturen nicht gibt, direkt in Wikidata gearbeitet werden (das was derzeit zur Verfügung steht)
Gleichzeitig in dieser Arbeit: von größtmöglichem Open Tech Stack ausgehen, Einschränungen nach unten offen halten, aber Devise Open Science radikal umgesetzt werden, soll am Ende auch Drawbacks dieser prototypischen Umsetzung diskutiert werden, nicht in Stein gemeißelt ggf. nachjustieren, mutig offeneren Lösungen entgegentreten

Beispiele aus der historischen Forschung:
https://archivfuehrer-kolonialzeit.de/
https://blog.ehri-project.eu/2018/02/12/using-wikidata/


Sichtbarkeit von Daten, Datenkonsistenz und -integrität in Wikidata

Erklären, warum nicht eigene Wikibase-Instanz

Alle anderen Softwarelösungen haben zwei entscheidende Nachteile:

1. fokussieren auf das Veröffentlichen und nachträglich Konsumieren von Forschungsdaten oder nicht so sehr auf die Datenverarbeitung -> kein kollaboratives Arbeiten auf Datenebene mögliche, was vor allem bei Datenmodellierung immens wichtig wäre (auch das OSF) nicht
Kollaboration bereits ab Erhebung der Daten --> Datenmodellierung, Frage, welche Daten sollen erfasst werden
2. es braucht zentrale Infrastrukturen für die Softwarelösungen, die Frage, wo ist das Forschungsdatenemanagement angesiedelt, natürlich hätte in dieser Arbeit eine eigene prototypisch entwickelt werden können, aber die Frage, wo wird diese gehosted eine ganz entscheidende

Auch wenn zwischen den Standards inwzischen Mapping möglich ist, konfliktfrei

Strukturiert an einen idealtypischen Forschungsprozess. Nicht alle möglichen Anwendungsfälle abgedeckt werden. Aber Abdeckung gesamten Forschungsdatenlebenszyklus sicher stellen und für jede Phase herausarbeiten, wan an funktionalen Anforderungen für offenes FDM gebraucht werden
\section{Erhebung}

\begin{quote}
    [...] Dass dieses methodisches Vorgehen auch transparent und nachvollziehbar ist.\footnote{B4\_Transkript, Pos. 67.}
\end{quote}

Datenerhebung im Kontext der historischen Forschung ist ausschließlich mit Quellenanalyse und -verarbeitung verbunden. 

Phase des Sammelns von Daten zu jüdischen Gewerbebetrieben
Die Phase der Datenerhebung ist ausführlich und lückenlos zu dokumentieren.

Für die Kontextualisierung der Forschungsdaten zu jüdischen Gewerbebetrieben werden Metadaten benötigt, die die Daten formal erschließen. 


Es braucht zum einen Informationen zum spezifischen Entstehungskontext der Daten (Datenherkunft). Diese können mit sogenannten deskriptiven Metadaten beschrieben werden. Um eine projekt- und forschungsfeldübergreifende Auffindbarkeitkeit und Nachnutzung der Daten zu garantieren, müssen überall einheitliche Metadaten verwendet werden. Zu diesem Zweck stehen inzwischen generische Metadatenstandards wie \textit{Dublin Core} der \textit{Dublin Core Metadata Initiative}\footnote{URL: \url{https://www.dublincore.org/specifications/dublin-core/dcmi-terms/} (letzter Zugriff am 15.05.2022)} oder \textit{DataCite} des gleichnamigen internationalen Konsortiums\footnote{URL: \url{https://datacite.org/} (letzter Zugriff am 15.05.2022)} zur Verfügung.\footnote{Im wissenschaftlichen Kontext ist allerdings ein Trend hin zu ,,DataCite'' erkennbar. Vgl. forschungsdaten.info, URL: \url{https://www.forschungsdaten.info/themen/beschreiben-und-dokumentieren/metadaten-und-metadatenstandards/} und Julian Schulz, Sonja Kümmet, Stephan Lücke, Martin Spenger, Tobias Weber (2020): Standardisierung eines Standards: Warum und wie ein Best-Practice-Guide für das Metadatenschema DataCite entstand, Version 1 (20.01.2020, 13:49). In: Korpus im Text, Serie A, 42800Absatz 15. URL: \url{http://www.kit.gwi.uni-muenchen.de/?p=42800&v=1\#p:15} (alle letzter Zugriff am 15.05.2022).}

Aufgenommen werden daher strukturierte Informationen zur Datenherkunft, da diese essentiell sind bei der eindeutigen Zuordnung der Daten zu den einzelnen Forschungsprojekten, vor allem wenn das Forschungsdatenmanagement projektübergreifend ist und außerdem an andere Forschungsfelder andockt. An Standards orientieren, versuchen zu mappen

Auswertung der historischen Quellen (Datenerhebung)
historische Grundgesamtheit Teilmenge
Für die Nachvollziehbarkeit werden zum anderen Informationen zur Vorgehensweise der Datenerhebung, also zum methodischen Vorgehen, benötigt. Dafür existieren keine disziplinübergreifenden Metadatenstandards.\footnote{Vgl. forschungsdaten.info, URL: \url{https://www.forschungsdaten.info/themen/beschreiben-und-dokumentieren/metadaten-und-metadatenstandards/} (letzter Zugriff am 15.05.2022).} Das heißt, diese Metadaten sind fachspezifisch. Im naturwissenschaftlichen Bereich und in der Archäologie gibt es mit der \textit{Research Resource Identification Initiative} (RRI)\footnote{} und mit \textit{IANUS}\footnote{URL: \url{https://ianus-fdz.de/}. Der Support war nach Auslaufen der DFG-Projektförderung 2017 allerdings eingeschränkt. So konnten neue Datensammlungen bis 2022 nicht aufgenommen werden, siehe URL: \url{http://datenportal.ianus-fdz.de/pages/information.jsp\#dateneigentuemer} (alle letzter Zugriff 15.05.2022).} bereits zentrale Ansätze, wie Enstehungskontexte und Methodiken anhand von Thesauri oder festen Vokabularen formal beschrieben werden können.\footnote{Siehe zum Beispiel die Thesauri des Deutschen Archäologischen Instituts, URL: \url{http://thesauri.dainst.org/de.html} mit der Kollektion zu den Methoden, URL: \url{http://thesauri.dainst.org/de/collections/\_203bcc05.html} (alle letzter Zugriff am 15.05.2022).} Allerdings sind sie nicht übertragbar auf den geschichtswissenschaftlichen Bereich. Offenes Forschungsdatenmanagement ist hier mit zwei Herausforderungen konfrontiert. Erstens existiert ein fachspezifischer Standard für die Geschichtswissenschaften noch nicht. Zweitens ist fraglich, inwiefern sich die Forschungsdesigns im Forschungsfeld formalisieren lassen. Als essentiell wurden drei Informationen herausgearbeitet: Datenquellen, Erhebungsmethode, Bias. Die Datenquellen lassen sich wie folgt strukturieren:

\begin{enumerate}
    \item Datenquelle: Gedruckte Verzeichnisse und Listen sowie Karteisammlungen, in denen Gewerbebetriebe dezidiert als jüdisch markiert und veröffentlicht wurden\footnote{In München übernahm diese Aufgabe das städtische Gewerbeamt, vgl. Rappl 2000, S. 145f. In Frankfurt am Main war der zentrale Akteur die Industrie- und Handelskammer.} Sie enthalten die wesentlichen Grunddaten der Gewerbebetriebe wie Name, Inhaber, Branche und Adresse.
    \item Datenquelle: Verschiedene zeitgenössische Aktenbestände, die den Vorgang der Verfolgung einzelner Gewerbebetriebe verwaltungsseitig dokumentieren. 
    \item Datenquelle: Eine wichtige Quelle im Forschungsfeld stellen die Wiedergutmachungsakten, insbesondere der Rückerstattungsverfahren, nach 1945 dar, welche seit den 90er Jahren der historischen Forschung zugänglich sind und oft eine Ersatzüberlieferung für die vernichteten und zerstörten zeitgenössischen Quellen darstellen. Der Nachteil is     
\end{enumerate}

Zu den ersten beiden Datenquellen ist generell festzustellen, dass die Überlieferung als disparat und lückenhaft bezeichnet wurde, da viele Bestände teilweise oder überwiegend von den Nationaloszialisten vernichtet wurden, um Spuren zu verwischen, oder in den letzten Kriegstagen unwiederbringlich zerstört wurden. Oft sind nur Überreste und Splitter erhalten, was die Datenerhebung der Studien maßgeblich beeinflusste. Hierbei lassen sich zwei wesentliche Vorgehensweisen unterscheiden:

\begin{enumerate}
    \item Erhebungsmethode: Datenquelle 1 ist überliefert und bildet den Ausgangspunkt, mit der ein Sample von jüdischen Gewerbebetrieben zusammengestellt wurde. Dieses Sample mit den Grunddaten wurde anschließend mit den Datenquellen 2 und 3 abgeglichen und um Daten angereichert, die signifikante Veränderungen des Betriebs zeigten.  
    \item Erhebungsmethode: Datenquelle 1 ist nicht überliefert, weshalb alternative Wege für eine Stichprobenziehung gefunden werden mussten. In Hamburg kamen in erster Linie die Wiedergutmachungsakten sowie Bestände der Devisenstelle zum Einsatz.\footnote{Vgl. Bajohr 1998, S. 21ff.} In Berlin hat man ein gänzlich anderen Ansatz verfolgt. Dort wurden ein Sample anhand der Zentralhandelsregisterbeilage (ZHRB), welche dem Deutschen Reichsanzeiger und Preußischen Staatsanzeiger täglich beilag, erstellt und bis 1945 alle Firmenveränderungen aufgenommen. Damit wurde die ZHRB zwischen 1930 und 1939 einmal komplett digitalisiert. Erst danach wurden nacheinander die Gewerbebetriebe mit überlieferten Quellen und anderen Hinweisen abgeglichen und bei einer klaren Indizienlage als jüdisch identifiziert.\footnote{Der Autor beschreibt dieses eher unkonventionelle Vorgehen im Forschungsfeld sehr detailliert in der Einleitung seiner Studie, vgl. Kreutzmüller 2012, S. 29-38.}
\end{enumerate}

Jede Erhebungsmethode geht mit Verzerrungen einher, die sich aufgrund der Quellensituation vor Ort nicht vermeiden ließen und notgedrungen in Kauf genommen werden mussten. Umso wichtiger ist, diese Fehlstellen oder Lücken zu reflektieren und zu dokumentieren:

\begin{enumerate}
    \item Bias: Die Datenquelle 1 setzen zeitlich überwiegend erst mit den reichsweiten Gesetzen ab 1938 ein. Die frühe Phase der Vernichtung der jüdischen Gewerbetätigkeit bleibt damit oft unterrepräsentiert, weil schlichtweg Daten dazu fehlen.
    \item Bias: Bei der Verwendung von überwiegend Wiedergutmachungsakten, insbesondere aus Rückerstattungsverfahren wie in Hamburg, liegt der Schwerpunkt automatisch auf den größeren Unternehmensverkäufen und den ehemaligen Eigentümern, die den Nationalsozialismus meist durch Emigration überlebt haben. Liquidationen bleiben in diesem Ansatz unterrepräsentiert. 
    \item Bias: In Berlin wiederum ist der Fokus auf den handelsregisterlich eingetragenen Firmen und damit auf mittelständischen Gewerbebetriebe, wodurch vor allem kleinere Unternehmen unterrepräsentiert bleiben.  
\end{enumerate}

Die hier vorgeschlagenen Informationen können den Ausgangspunkt für die weitere Entwicklung eines spezifischen Metadatenstandard im Forschungsfeld bilden. Mangels existierender Standards wird für die prototypische Lösung aber ein pragmatischer Ansatz verfolgt. Im Sinne einer offenen Methodik (Open Methodology) ist das Hauptanliegen, methodische Vorgehensweisen überhaupt erst einmal transparent zu machen. Daher wird auf existierende Open Sciene-Angebote wie das Open Science Framework oder Zenodo verwiesen, mit denen sich die Informationen zum Stichprobendesign in rein textueller oder in einer semistrukturierten dokumentierter Form veröffentlichen lassen und die über DOI oder PID im Forschungsdatenmanagement angesprochen werden können. 





\section{Aufbereitung}
während dieser Phase Kollaboration und Diskursabbildung
Wikidata:WikiProject Destruction of the Economic Existence of the Jews Research bildet Grundlage
solide Datengrundlage für die weitere Auswertung schaffen
\subsection{Problem \textit{Jüdischer} Gewerbebetrieb}
methodisches Problem hier herausgehoben

\begin{quotation}
    Test Test
    
\end{quotation}



Als Untersuchungsgegenstand für die statistische Auswertung sind ,,Jüdische Gewerbebetriebe'' oder ,,Jüdische Unternehmen''. Hieraus ergibt sich eine grundlegende methodische Schwierigkeit im Forschungsfeld. Da die Zugehörigkeit zu einer Konfession bei einem Gewerbebetrieb oder Unternehmen generell keine Rolle spielt, ist schon der Begiff ,,jüdischer Gewerbebetrieb'' unlogisch und ohne Kontext unbrauchbar. Dies wird auch in fast allen Studien reflektiert und klar gestellt, dass es sich um eine antisemitische Zuschreibung und Konstruktion handelte. Diese Kennzeichnung und Diffamierung bildete den Ausgangspunkt für alle weiteren Verfolgungspraktiken. Zur einfacheren Handhabung wurde der Begriff als Quellenbegriff jedoch von allen Studien beibehalten. Hierbei fallen zwei unterschiedliche Verwendungen auf: 

\begin{enumerate}
    \item Der Begriff ,,jüdischer Gewerbebetrieb'' wird ausschließlich auf die jüdischen Besitzer*innnen bezogen und angewandt. Damit wird jedoch das methodische Problem nicht wirklich aufgelöst, sondern verlagert sich nur auf den Begriff ,,jüdische Person'' oder ,,Jude'', bei dem es sich im nationalsozialistischen Kontext ebenfalls um eine rassistische Zuschreibung handelte und nichts mit dem Selbstverständnis der Betroffenen zu tun hatte.\footnote{Das wird in der Studie zu Hamburg auch ausführlicher reflektiert. Vgl. Bajohr 1997, S. 9.} Darüber hinaus werden in dieser Verwendung systematisch Gewerbebetriebe vernachlässigt, deren Besitzer zum Beispiel nichtjüdisch waren, die aber einen hohen Anteil jüdischer Mitarbeiter*innen aufwiesen und daher verfolgt wurden. 
    \item Der Begriff ,,jüdischer Gewerbebetrieb'' wird mit ,,als jüdisch betrachtet/ verfolgt'' gleichgesetzt. Mit dieser Verwendung ist die jüdische Eigentümerschaft eines Gewerbebetriebs zunächst unerheblich, das heißt sie wird nicht vorausgesetzt, sondern es werden alle Gewerbebetriebe gezählt, die im nationalsozialistischen Kontext diffamiert wurden. Damit wird einerseits der Konstruktioncharakter des Begriff hervorgehoben und andererseits dem Umstand Rechnung getragen, dass die rassistischen Zuschreibungen grundsätzlich jeglicher rationalen Begründung entbehrten und aus diesem Grund willkürlich erfolgen konnten. Zudem konnten auch unterschiedliche Verfolgungskontexte erfasst werden, die in der ersten Verwendung ausgeschlossen blieben.
\end{enumerate}

Auch wenn in allen Studien der selbe Untersuchungsgegenstand genannt wird, so zeigt sich erst in der konkreten Verwendung, dass dieser unterschiedlich interpretiert wurde, was jedoch so im Forschungsfeld noch nicht diskutiert wurde. Maßgeblich liegt das daran, dass der Begriff an sich nicht widerspruchsfrei ist. Aus forschungsethischer Perspektive ist es zudem problematisch, dass ein rassistisch konnotierter Begriff in der wissenschaftlichen Forschung beibehalten wird. Umso wichtiger ist eine krititsche (Selbst)Reflexion in der eigenen Forschungsarbeit. Für das Forschungsdatenmanagement wird versucht, den Zuschreibungs- und Konstruktionscharakter abzubilden und auf diese Weise den Begriff ,,jüdischer Gewerbebetrieb'' zu vermeiden. Dafür scheint die Verwendung ,,als jüdisch betrachtet'' ein geeigneter Ansatz zu sein.

\subsection{Zusammenführung der Quellen}
Formale Beschreibung jüdischer Gewerbebetriebe, Relationen, Datensätze zu diesen erstellen --> gibt vor, welche Daten erfasst werden 
\paragraph{Datenmodell}, Modellierunghier konkret zum Datenmodell, jeder sein eigenes Datenmodell, stand mehr oder weniger von Anfang an fest
Bei der Erfassung im Klaren sein, welche Daten ich für Forschungsfrage benötige und welche kassiert werden können
Grunddaten --> Name, Inhaber, Branche, Adresse
weitere ortsbezogene Daten --> Geodaten, mehrere Adressen ermöglichen, wenn es Filialien gab oder bei Umzügen
Eventdaten --> Veränderungen (Prozess der Vernichtung), Namens-Rechtsformveränderungen, Besitzerwechsel, Liquidationen

Herausforderung: Bisher keine festes Vokabular zur Beschreibung, jedes Projekt für sich 

Datenmodell entwickeln, dass für alle Projekte funktioniert, also 
Kompabilität --> Top-Level-Ontologie (stellt Austauschbarkeit sicher)
Inhaltserschließende Metadaten
EntitySchema items, properties, qualifiers und references
\paragraph{Quellennachweise}
bibliografische DatenQuellennachweis für Einzeldaten
Nachweis, der einen Gewerbebtrieb als jüdisch identifiziert hat, aus den Interviews ist auch hervorgegangen, dass das nicht immer so eindeutig ist und es keine feste Kriterien gibt. Dies ergibt sich aus bereits erläuterten der methodischen Schwierigkeit in der Verwendung des Begriffs, die sich auch mit einem Forschungsdatenmanagement nicht vollständig auflösen lässt, es scheint aber eben an dieser Stelle umso wichtiger, transparent und nachvollziehbar für jeden zu machen, warum ein Unternehmen als jüdisch identifiziert wurde, dann ist man zumindest in der Lage Grenzfälle, anders als gegenwärtig, wo man den Autoren einfach glauben muss, zu diskutieren bzw. gemeinsam Kriterien zu entwickeln, falls das überhaupt möglich ist.
Verlinkung zu Digitalisaten um hier gemeinsame Qualitätskontrolle zu erhalten, die es bisher ja noch gar nicht gibt
\paragraph{Verlinkung von Gewerbebetrieben}
Gemeinsame Normdaten vor allem für Personen- und Ortsnamen notwendig
\subsection{Erfassung von jüdischen Gewerbebtrieben}
Linked Open Data Interface von Wikidata --> manuell oder über Open Refine integrieren Bulk Import Funktion
\subsection{Verknüpfung von Sample, Fallbeispielen und Digitalisaten}
Verknüpfung strukturierter und unstrukturierter Daten --> gängige Praxis im Wiki*versum
Textuelle Daten
WikiCommons Möglichkeit Digitalisate zu hinterlegen und mit Wikidata zu verknüpfen

Daneben sind für das Forschungsdatenmanagement nur die Datenquellen relevant, in denen nachweislich Forschungsdaten zu jüdischen Gewerbebetrieben existieren. Das sind erstens vor allem die empirischen Studien, die Teilbereiche wie die Vernichtung der jüdischen Gewerbetätigkeit auf der Basis von Stichproben mit einer (deskriptiven) statistischen Datenanalyse ausgewertet haben. Mit dieser Methode konnten erstmals allgemeinere Aussagen zum Vernichtungsprozess gewonnen werden.\footnote{Daneben gibt es noch die rein qualitativen oder Einzelfall-Studien, die hier aber nicht näher betrachtet werden, da ihr Anteil an Forschungsdaten zu jüdischen Gewerbebetrieben gering ist.}. Zum zweiten sind das Veröffentlichungen in analoger oder digitaler Form, die einen stark dokumentarischen Charakter aufweisen, der sich vorwiegend in einem deskriptiven Zusammentragen von verteilten Informationen zu jüdischen Gewerbebetrieben und jüdischen Unternehmern niedergeschlagen hat.\footnote{Nietzel hebt hier die akribisch recherchierte Textsammlung zu jüdischen Unternehmen in München des Archivars und Historikers Wolfgang Selig aus dem Jahr 2004 hervor, vgl. Nietzel 2009, S. 583.} Hierunter zählen auch jene Veröffentlichungen, die nicht primär auf Daten zu jüdischen Gewerbebetrieben fokussiert sind, sondern wo diese eher als anreichernde Daten verstanden werden können.\footnote{Hier vor allem die zahlreichen Gedenkbücher zu jüdischen Personen, die mittlerweile online zugänglich sind und wo sich Daten zu jüdischen Gewerbebetrieben in den Biogrammen der Personen ,,verstecken''. Siehe zum Beispiel ,,Biografisches Gedenkbuch der Münchner Juden 1933–1945'' der Stadt München, URL: \url{https://gedenkbuch.muenchen.de/} (letzter Zugriff am 12.05.2022). Bei der Biografie von Max Hofman ist unter ,,Weitere Informationen'' vermerkt: ,,Max Hofmann war Inhaber der Fa. Max Hofmann, einem Großhandel und Versand von Manufaktur- und Textilwaren, in der Paul-Heyse-Straße 28/I. Das Gewerbe wurde am 17.10.1938 für den 15.10.1938 abgemeldet.'', URL (stable): \url{https://gedenkbuch.muenchen.de/index.php?id=gedenkbuch_link&gid=5722}.}

Demzufolge existieren zwei Arten von Forschungsdaten zur Vernichtung der jüdischen Gewerbetätigkeit:

\begin{enumerate}
    \item Es handelt es sich um \textbf{quantitative (Massen-)Daten}, die strukturiert, entweder als Rohdaten oder in aggregierter Form, vorliegen. Sie besitzen eine statistische Aussagekraft.
    \item Es handelt es sich überwiegend um \textbf{qualitative Daten}, die in der Regel textuell und damit unstrukturiert oder semistruktiert vorliegen.
\end{enumerate}

Die textuellen Daten waren für eine wissenschaftlich analytische Auswertung bislang zu unsystematisch.\footnote{Ebd.} Umgekehrt fehlt den statistischen Daten ihres Umfang wegens oft die entsprechende Datentiefe und die Einzelschicksale und -geschichten hinter der Statistik sind nicht sichtbar.\footnote{Allein für Berlin hat die Stichprobe einen Umfang von ca. 8.000 jüdischen Gewerbebetrieben. Auch für Frankfurt am Main sind es in der Stichprobe über 2.500 jüdische Gewerbebtriebe. Vgl. Kreutzmüller 2012, URL: \url{https://www2.hu-berlin.de/djgb/www/find} (letzter Zugriff am 07.05.2022) und Nietzel 2012, S. 15.} Das macht diese Daten vor allem außerhalb der wissenschaftlichen Forschung weniger greif- und nutzbar. 

\section{Analyse}
explorativ
\subsection{Gewerbestruktur}
\paragraph{Verteilung nach Branchen}
\paragraph{Verteilung im Stadtraum}
braucht Geodaten
\paragraph{Geschäftsfrauen}
braucht Gender-Angabe
\subsection{Vernichtung}
Anzahl Besitztransfer und Liquidationen (mit Liquidation ab bis Gelöscht) im Vergleich, Entwicklung über die Zeit (Zeitreihen-Analyse)
\paragraph{}
\subsection{Abwehrstrategien}
\paragraph{Namen- und Rechtsformveränderungen}
\paragraph{Umzüge}
\section{Archivierung}
\section{Veröffentlichung und Nachnutzung}

\paragraph{Teamarbeit}
bei der Erfassung und nachträglichen Bearbeitung von Daten (vor allem Anreicherung von Quellendaten)
Sowohl Datenfelder als auch Eingabe

aber auch in Hinblick auch Partizipationsgedanke wurde hier mit aufgegriffen, der in Kapitel 3.2.3 bereits als Kriterium von offenem FDM festgelegt wurde, findet sich auch in den Interviews wieder. Alle grundsätzlich positiv gegenüber Citizen Science eingestellt und sehen es nicht als Behinderung für die wissenschaftliche Forschung 

Strategieentwicklung

\paragraph{Diskussionsforum}
bringt Kollaboration mit sich, dass Diskurs ermöglicht wird, wo Regeln vereinbart werden können, verständigt sich auf Vokabular, Normdaten etc., Weiterentwicklung des Datenmodells
\paragraph{Dynamische Anpassungen}
Flexible und stetige 
Dateneditierebene als auch auf Datenmodellebene
Datenmodell steht nicht von Anfang fest, sondern ist dynamisch, hängt mit den Erhebungsmethoden zusammen
\paragraph{Multiperspektivischer Datenzugang}
Heusler

\paragraph{Datentransfer und Nachnutzung}
Recherche in Datensammlungen
Daten für Erinnerungsinitiativen zur Verfügung stellen, verschiedene Visualisierungmöglichkeiten

\paragraph{Dauerhafte Kuratierung und -pflege}
keine tote Daten produzieren

\paragraph{Test- und Evaluationsphasen}
der Forschungsdatenumgebung, Mitsprache bei neuen Funktionalitäten, Involvierung in den Entwicklungsprozess