\onehalfspacing

\section{Lösungsansatz}

Bei der prototypischen Lösung steht im Zentrum dieser Arbeit die Wissensdatenbank \textit{Wikidata}\footnote{URL: \url{https://www.wikidata.org/wiki/Wikidata:Main_Page} (letzter Zugriff am 20.05.2022).} als offener Forschungsdatenmanagement-Service. Bei Wikidata handelt sich ursprünglich um ein generisches dankenbankbasiertes Angebot von Wikimedia für strukturierte Daten im Wiki*versum, das das Konzept von Linked Open Data umsetzt. Damit ist es flexibel und sprachunabhängig einsetzbar, wodurch es als Modell auch für Forschungsdatenmanagement in der akademischen Wissenschaft interessant wird. Tatsächlich wird dieser Weg im Rahmen von NFDI gegenwärtig bestritten. Das \textit{Open Science Lab} am ,,Leibniz-Informationszentrum Technik und Naturwissenschaften und Universitätsbibliothek''\footnote{URL: \url{https://www.tib.eu/de/} (letzter Zugriff am 20.05.2022).} hat für das Konsortium \textit{NFDI4Culture}\footnote{URL: \url{https://nfdi4culture.de/index.html} (letzter Zugriff am 20.05.2022).} Wikidata und insbesondere die zugrunde liegende Software \textit{Wikibase}\footnote{URL: \url{https://wikibase.consulting/what-is-wikibase/} (letzter Zugriff am 20.05.2022).} auf die Einsetzbarkeit für ein Forschungsdatenmanagement von Kulturdaten hin evaluiert. Erste Ergebnisse wurden im März 2022 auf dem TIB-Blog veröffentlicht.\footnote{Siehe Lozana Rossenova (2022): Examining Wikidata and Wikibase in the context of research data management applications, veröffentlicht am 16.03.2022 auf dem TIB-Blog, URL: \url{https://blogs.tib.eu/wp/tib/2022/03/16/examining-wikidata-and-wikibase-in-the-context-of-research-data-management-applications/}.} Parallel führt das NFDI4Culture-Konsortium selbst die Workshop-Reihe ,,Wikibase'' durch.\footnote{URI: \url{https://nfdi4culture.de/resource/E2261/about.html}.} Die Wissensdatenbank wird vor allem wegen ihrer semantischen Technologien ausdrücklich als ,,FAIR-Plattform'' empfohlen und kommt im naturwissenschaftlichen Bereich schon länger zum Einsatz.\footnote{Vgl. European Commission, Final Report, 2018, S. 42. Siehe T.E Putman, S. Lelong, S. Burgstaller-Muehlbacher, u.a.: WikiGenomes. an open web application for community consumption and curation of gene annotation data in Wikidata. Database (2017) Band 2017: article ID bax025, doi:10.1093/database/bax025.} 

Aber auch im Kontext historischer Forschung wird Wikidata bereits verwendet. Das Online-Portal ,,Archivführer. Deutsche Kolonialgeschichte'' nutzt Wikidata als strukturierte Datenbasis für Forschungsdaten zum Thema ,,Deutsche Kolonien und Schutzgebiete'' stehende Forschungsdaten.\footnote{Das Projekt wurde 2017 an der Fachhochschule Potsdam initiiert und ist vom Auswärtigen Amt gefördert worden, URL: \url{https://archivfuehrer-kolonialzeit.de/} (letzter Zugriff am 20.05.2022).} Das Portal führt lediglich die Wikidata-Daten für die Datenpräsentation zusammen und ermöglicht einen multiperspektivischen Zugang zu den Daten.\footnote{Zum Beispiel Georeferenzierung der Orte anhand historischen Kartenmaterials, URL: \url{https://archivfuehrer-kolonialzeit.de/map} (letzter Zugriff am 20.05.2022).} Die Besonderheit der Datenbereitstellung in Wikidata ist, dass unvollständige sowie neue Daten über die Projektlaufzeit hinaus von jeder/jedem Nutzer*in erweitert und diese in gänzlich anderen Kontexten verwendet werden können. Darüber hinaus verfolgt das Projekt das Ziel, die Daten mit der ,,kolonialen Vergangenheiten anderer Ländern''\footnote{URL: \url{https://archivfuehrer-kolonialzeit.de/about} (letzter Zugriff am 20.05.2022).} zu verknüpfen und auf diese Weise das Forschungsfeld zum Deutschen Kolonialismus anschlussfähig an die Forschung zum Europäischen Kolonialismus zu machen. Die Zusammenarbeit und der kollaborative Austausch dazu erfolgen ebenfalls global in Wikidata in dem ,,Wikidata:WikiProject European Colonialism''.\footnote{URL: \url{Wikidata:WikiProject European Colonialism} (letzter Zugriff am 20.05.2022).} 

Das internationale Projekt ,,European Holocaust Research Infrastructure'' (EHRI), welches im Rahmen der Open Science-Strategie von der Europäischen Kommission seit 2017 gefördert wird\footnote{Im EU-Programm ,,Horizon Europe'', das bis 2027 läuft, URL: \url{https://ec.europa.eu/info/research-and-innovation/funding/funding-opportunities/funding-programmes-and-open-calls/horizon-europe_en}. Projektwebsite von EHRI, URL: \url{https://www.ehri-project.eu/} (alle letzter Zugriff am 20.05.2022).}, nutzt Wikidata als zentrales Verzeichnis zur Erstellung einer Liste von Ghettos aus der Zeit des Holocausts.\footnote{Nancy Cooey (2018): Using Wikidata to build an authority list of Holocaust-era ghettos, veröffentlicht am 12.02.2018 auf dem EHRI Document Blog, URL: \url{https://blog.ehri-project.eu/2018/02/12/using-wikidata/\#Selecting\_Wikidata\_as\_a\_Tool} (letzter Zugriff am 20.05.2022).} Ziel ist, Daten aus verschiedenen Enzyklopädien, die bisher isoliert waren, in Wikidata erstmals zusammenzuführen und zu verknüpfen.\footnote{Vgl. ebd. Zentrale Enzyklopädien sind ,,The Yad Vashem Encyclopedia of the Ghettos During the Holocaust'' von Yad Vashem (Israel) und ,,USHMM Encyclopedia of Camps and Ghettos'' des United States Holocaust Memorial Museum (USA).} 

Für die Umsetzung der in dieser Arbeit verfolgten Strategie der Open Research Data ist Wikidata also ein geeigneter Ansatz. Grundsätzlich ist bei offenem Forschungsdatenmanagement in Wikidata zu beachten, dass das Konzept von Linked (Open) Data umgesetzt wird, bei dem es sich, wie in Kapitel 2.2.2 bereits erläutert wurde, um einen wesentlichen Baustein des \textit{Semantic Web} handelt. Damit erfolgt offenes FDM in der höchsten Open Data-Stufe (= 5 Sterne). Die Daten sind demzufolge interoperabel. Der Vorteil ist, dass die Stärken dieses Konzepts, welche vor allem in der Verknüpfung und Vernetzung von Daten liegen, für das Forschungsdatenmanagement ausgenutzt werden können. Nachteilig ist, dass dieser Ansatz voraussetzungsreicher als andere Lösungen ist, da zum einen Kenntnisse der allgemeinen Technologien des Semantic Web wie RDF (Resource Description Framework), JSON-LD (JavaScript Object Notation for Linked Data) oder URI (Uniform Ressource Identifier) vorhanden sein müssen.\footnote{Im Rahmen dieser Arbeit können diese Technologien nicht detailliert vorgestellt werden, daher wird zur Vertiefung auf die Grundlagenliteratur verwiesen. Siehe zum Beispiel Malte Rehbein: Ontologien, in: Fotis Jannidis, Hubertus Kohle, Malte Rehbein (Hrsg.), Digital Humanities, 2017, doi:10.1007/978-3-476-05446-3\_11; Christian Stein: Linked Open Data – Wie das Web zur Semantik kam, in: Bibliothek Forschung und Praxis (Hrsg.), Band 38, Nr. 3, 2014, S. 447-455, doi:10.1515/bfp-2014-0055; Patrick Danowski, Adrian Pohl: (Open) Linked Data in Bibliotheken, Berlin, Boston, 2013, doi:10.1515/9783110278736; Gradmann, Steffen Hennicke, Marlies Olensky: Linked Data, in: Digitale Dienste für die Wissenschaft (Hrsg.), 2012, S. 18-22, doi:10.18452/6627.} Zum anderen muss sich in das Metadatenschema beziehungsweise in die Ontologie der zugrunde liegenden Wikidata-Software \textit{Wikibase} eingearbeitet werden.\footnote{Siehe Mediawiki (2022): Wikibase/DataModel, URL:\url{https://www.mediawiki.org/wiki/Wikibase/DataModel} (letzter Zugriff am 22.05.2022).} Kurzgefasst ist im Wesentlichen zu beachten, dass jegliche Modellierung von Daten in Wikidata graphenbasiert in sogenannten Tripeln als Subjekt-Prädikat-Ausdrücke erfolgt, was sich grundlegend von der konventionellen tabellenbasierten relationalen Datenmodellierung mit Tupeln unterscheidet.\footnote{In Wikidata werden diese Ausdrücke als Aussagen (Statements) bezeichnet, siehe Wikidata Statements, URL: \url{https://www.wikidata.org/wiki/Help:Statements} (letzter Zugriff am 27.05.2022).}

Für die prototypische Lösung konnten die Forschungsdaten zu Jüdischen Gewerbebetrieben aus Berlin, Mannheim und Krefeld besorgt werden.

\section{Erhebung}

\begin{quote}
    ,,Dass dieses methodisches Vorgehen auch transparent und nachvollziehbar ist.''\footnote{B4\_Transkript, Pos. 67.}

    ,,Das große Problem ist, was ist in Gottes Namen ein jüdisches Unternehmen.''\footnote{B1\_Transkript, Pos. 147.}
\end{quote}

Datenerhebung in der empirischen historischen Forschung geht mit historischer Quellenanalyse und Quellenkritik einher.\footnote{Vgl. W. H. Schröder: Historische Sozialforschung: Forschungsstrategie - Infrastruktur - Auswahlbibliographie.
Historical Social Research, in: Supplement (Hrsg.) 1988, Nr. 1, S. 1-109, hier S. 15ff., URN: \url{https://nbn-resolving.org/urn:nbn:de:0168-ssoar-286038}
} Anders als in der naturwissenschaftlichen Datenerhebung, wo anhand von Experimenten, Beobachtungen, Simulationen oder Messungen, Daten in Echtzeit gewonnen werden und dementsprechend die Erhebungsmethoden an den Forschungsfragen angepasst werden können, ist die Vorgehensweise bei den geschichtswissenschaftlichen Disziplinen maßgeblich von der Überlieferungstruktur und der Quellensituation abhängig.\footnote{Was zu einem ,,Quellenproblem'' führen kann, siehe dazu ebd. S. 19f.} Informationen zur Erhebung sind in beiden Fällen essentiell, um Forschungsdaten im Sinne einer Datenkritik kontextualisieren, verstehen und damit letztlich bewerten zu können. Für die Forschungsdaten zu Jüdischen Gewerbebetrieben sind diese jedoch nicht hinterlegt und es handelt sich im Zusammenhang mit der Erhebung bisher um implizites Wissen, was eine Nachnutzung der Daten erschwert oder sogar unmöglich machen kann. Hinsichtlich der Nachvollziehbarkeit und Transparenz von Forschungsdaten ist daher Ziel von offenem Forschungsdatenmanagement, das Wissen um den Entstehungsrahmen sowie um die geschichtswissenschaftliche Datenerhebungsmethode explizit zu machen. Hierfür werden deskriptive Metadaten und Prozessmetadaten genutzt.\footnote{Vgl. forschungsdaten.info (2022): Welche Metadaten sind zu unterscheiden?, URL: \url{https://www.forschungsdaten.info/themen/beschreiben-und-dokumentieren/metadaten-und-metadatenstandards/} (letzter Zugriff am 04.06.2022).} In diesem Zusammenhang wird auch das grundsätzliche methodische Problem des Begriffs ,,Jüdischer Gewerbebetrieb'' diskutiert.

\subsection{Entstehungsrahmen}

Im Forschungsfeld ist der Großteil der Forschungsdaten zu Jüdischen Gewerbebetrieben in lokalen wissenschaftlichen Forschungsprojekten erhoben worden, daher stellen vor allem sie die relevanten deskriptiven Metadaten dar, welche die Rahmenbedingungen zur Entstehung von Daten beschreiben.\footnote{Vgl. ebd.} Die Frage, wie verschiedene (akademischen) Forschungsaktivitäten zur semantische Anreicherung von Forschungsdaten konzeptionalisiert und formalisiert werden können, scheint gegenwärtig noch nicht Gegenstand des Forschungsdatenmanagements zu sein, denn einen wissenschaftlichen Standard, nach denen diese beschrieben werden können und sollen, konnte nicht ermittelt werden. Zwar gibt es inzwischen generische Metadatenstandards wie \textit{Dublin Core} der \textit{Dublin Core Metadata Initiative}\footnote{URL: \url{https://www.dublincore.org/specifications/dublin-core/dcmi-terms/} (letzter Zugriff am 15.05.2022)} oder \textit{DataCite}\footnote{URL: \url{https://datacite.org/} (letzter Zugriff am 15.05.2022)} des gleichnamigen internationalen Konsortiums. ,,DublinCore'' fokussiert aber in erster Linie auf Informationen zur technischen Umsetzung sowie zur Veröffentlichung von digitalen Ressourcen und ist damit näher an der traditionellen Praxis der Formalerschließung in der Bibliothekskatalogisierung dran. ,,DataCite'' ist umfangreicher und lässt als optionale Elemente auch Angaben zu Fördermittelgebern zu.\footnote{,,Funding references'', siehe Data-Cite-Dokumentation auf GitHub URL: \url{https://github.com/UB-LMU/DataCite\_BestPracticeGuide/blob/master/BestPracticeGuide.md\#fundingreference} (letzter Zugriff am 23.05.2022).} Ein Konzept ,,Forschungsprojekt'' findet sich aber in beiden Standards nicht wieder. Zusammengefasst handelt es sich bei diesen vorwiegend um bibliografische Metadatenstandards.\footnote{Vgl. forschungsdaten.info (2022).} 

In dieser offenen Situation bietet Wikidata einen entscheidenden Vorteil: Zur Verbesserung formaler Beschreibungen von bestimmten Konzepten wie zum Beispiel ,,Mathematik'' oder ,,Astronomie'' können von der Wikidata-Community sogenannte \textit{Wikidata:Wikiprojekte} angelegt werden. Sie bieten die Möglichkeit der kollaborativen Modellierung und des gemeinsamen Austauschs. In den Wikidata-Projekten können kontrollierte Vokabulare (Authority Files) für Konzepte in Wikidata definiert werden, die allerdings nur informellen Charakter haben. Inzwischen gibt es eine Vielzahl an unterschiedlichen Projekten, die zur besseren Auffindbarkeit in Kategorien unterteilt sind.\footnote{Auch die NFDI sowie das Archivportal zum Deutschen Kolonialismus sind mit eigenen Projekten vertreten. Wikidata:WikiProject NFDI, URL: \url{https://www.wikidata.org/wiki/Wikidata:WikiProject_NFDI}.} In der Kategorie \textit{Category:Research WikiProjects} beschäftigt sich eine internationale Wissenschaftler*innengruppe mit der Abbildung des Konzepts ,,Forschung'' in Wikidata.\footnote{URL: \url{https://www.wikidata.org/wiki/Wikidata:WikiProject_Wikidata_for_research}. Darunter ist auch eine deutsche Gruppe, URL: \url{https://www.wikidata.org/wiki/Wikidata:WikiProject_Wikidata_for_research/de}.} Dort integriert ist das Unterprojekt \textit{Wikidata:WikiProject Wikidata for research/Data models/Research projects}, in dem sich ausschließlich mit dem Konzept ,,Forschungsprojekt'' befasst wird.\footnote{URL: \url{https://www.wikidata.org/wiki/Wikidata:WikiProject_Wikidata_for_research/Data_models/Research_projects}.} Hier zeigt sich die Stärke des gemeinschaftlichen Ansatzes von Wikidata, denn die Chance, dass sich in Wikidata mit einem Problem schon befasst wird, ist sehr hoch. Folglich wäre die eigene Modellierung von ,,Forschungsprojekt'' für die lokalen Forschungsprojekte im Forschungsfeld redundant, da diese von dem bestehenden Wikidata-Projekt abgeleitet werden kann.\footnote{Als Orientierung bei der Modellierung diente das Forschungsprojekt ,,Amyloid fibril cytotoxicity: new insights from novel approaches'', URL: \url{https://www.wikidata.org/w/index.php?title=Q52268104&oldid=1528020632}. Die Modellierung befindet sich im Anhang D.5 dieser Arbeit ,,Datenmodell ,Forschungsprojekt' am Bsp. von Berlin in Wikidata''.} Darüber hinaus existieren viele Entitäten wie ,,Humboldt-Universität zu Berlin'', wo das Berliner Forschungsprojekt angesiedelt war, bereits in Wikidata und müssen nicht neu angelegt werden.\footnote{Im Modell (Anhang D.5) die Entitäten mit weißem Hintergrund.} Auch die Verknüpfung von externen Informationen ist möglich. Die Deutsche Forschungsgemeinschaft (DFG) hat mit dem Informationssystem ,,GEPRIS – Geförderte Projekte der DFG'' (GEPRIS)\footnote{URL: \url{https://gepris.dfg.de/gepris/OCTOPUS?task=showAbout} (letzter Zugriff am 21.05.2022).} in Auszügen ihre Daten zu allen gegenwärtigen und vergangenen geförderten Projekten veröffentlicht. Dort ist auch das Forschungsprojekt ,,Geschichte mittlerer und kleiner jüdischer Unternehmen in Frankfurt am Main und Breslau 1929/39 bis 1945'' archiviert.\footnote{URL: \url{https://gepris.dfg.de/gepris/projekt/48308995?context=projekt&task=showDetail&id=48308995&} (letzter Zugriff am 23.05.2022). Hieraus ging u.a. die Lokalstudie zu Frankfurt am Main hervor sowie die im Interview erwähnte Access-Datenbank mit ca. 3.000 Gewerbebetrieben in Frankfurt a.M., Siehe Nietzel 2012 und Interview B2\_Transkript, Pos. 27.} Mit der vorhandenen Wikidata-Property ,,GEPRIS ID (Projekt) (P4870)'', kann demnach das DFG-Projekt durch dessen eindeutiger nummerischer DFG-Kennung ,,48308995'' in Wikidata verknüpft werden.\footnote{Auch die Freie Universität Berlin führt ein zentrales Projektverzeichnis mit detaillierten Informationen zu den einzelnen Projekten, siehe URL: \url{https://research.zuv.fu-berlin.de/projects} (letzter Zugriff am 24.05.2022).}  

Die vielseitige Nutzung der Wikidata bietet also Nachnutzungsmöglichkeiten auch für die historische Forschung. Diese Form der Nachnutzung trägt außerdem zur Qualitätssicherung in Wikidata bei. Zudem können erstmals Informationen zu Projekten aus verteilten externen Datenquellen in Wikidata zusammengeführt und auf diese Weise vernetzt werden, was die Sichtbarkeit der Forschungsprojekte erhöht. Sollten bezüglich der Forschungsprojekte im Forschungsfeld spezifische Informationen benötigt werden, können diese Daten dynamisch in Wikidata ergänzt werden, was wiederum der Vorteil des Linked Data-Konzepts gegenüber einer herkömmlichen relationalen Modellierung in einer SQL-Datenbank ist, wo diese Flexibilität nicht gegeben ist. Die Forschungsprojekte werden als eigene Datenobjekte (Items) in Wikidata angelegt und erhalten damit eine eindeutige Wikidata-ID (Q). Über diese lassen sich die zugehörigen Forschungsdaten eindeutig zuordnen, wodurch der projektbezogene Entstehungsrahmen auf Datenebene erstmals transparent wird. 

\subsection{Erhebungsmethode}

Da die methodischen Vorgehensweisen der verschiedenen Wissenschaftsdisziplinen voneinander abweichen, existieren zu deren formalen Beschreibung keine disziplinübergreifenden Metadatenstandards.\footnote{Vgl. forschungsdaten.info, URL: \url{https://www.forschungsdaten.info/themen/beschreiben-und-dokumentieren/metadaten-und-metadatenstandards/} (letzter Zugriff am 15.05.2022).} Das heißt, diese als Prozessmetadaten bezeichneten Daten sind fachspezifisch. Im naturwissenschaftlichen Bereich und in der Archäologie gibt es mit der \textit{Research Resource Identification Initiative} (RRI)\footnote{URL: \url{https://scicrunch.org/resources} (letzter Zugriff am 03.06.2022).} und mit \textit{IANUS}\footnote{URL: \url{https://ianus-fdz.de/}. Der Support war nach Auslaufen der DFG-Projektförderung 2017 allerdings eingeschränkt. So konnten neue Datensammlungen bis 2022 nicht aufgenommen werden, siehe URL: \url{http://datenportal.ianus-fdz.de/pages/information.jsp\#dateneigentuemer} (alle letzter Zugriff 15.05.2022).} bereits zentrale Ansätze, wie Methodiken schematisch und anhand von Thesauri oder festen Vokabularen formal beschrieben werden können.\footnote{Siehe zum Beispiel die Thesauri des Deutschen Archäologischen Instituts, URL: \url{http://thesauri.dainst.org/de.html} mit der Kollektion zu den Methoden, URL: \url{http://thesauri.dainst.org/de/collections/\_203bcc05.html} (alle letzter Zugriff am 15.05.2022).} Allerdings sind sie nicht übertragbar auf den geschichtswissenschaftlichen Bereich. Offenes Forschungsdatenmanagement ist hier mit zwei Herausforderungen konfrontiert. Erstens gibt es einen fachspezifischen Standard für die Geschichtswissenschaften nicht. Zweitens ist fraglich, wie sich die Erhebungsmethoden im Forschungsfeld formalisieren lassen. Als Einstiegspunkt soll hier der Versuch einer groben Schematisierung der methodischen Vorgehensweise anhand der Lokalstudien, welche systematisch Daten zu Jüdischen Gewerbebetrieben erhoben haben, vorgenommen werden.\footnote{Das sind zuvorderst die Studien zu Hamburg, Berlin, Frankfurt am Main, München, Mannheim und Krefeld.} Zunächst ist festzuhalten, dass die Datenanalyse und -auswertung aller Studien auf Stichprobenziehung beruhte.\footnote{Interessant ist, dass alle Studien mit dem Anspruch gestartet sind, die Gesamtzahl jüdischer Gewerbebetriebe zu erfassen. Dieser war allerdings von keiner Studie einlösbar, da erstens das Ausmaß der Zerstörung unterschätzt wurde und zweitens die Projektlaufzeit für eine Totalerhebung zu kurz war, vgl. Interview B3\_Transkript, Pos. 11 und B2\_Transkript, Pos. 23.} Festzustellen ist weiterhin, dass die Überlieferung überall als disparat und lückenhaft bezeichnet wurde, da viele Bestände teilweise oder überwiegend von den Nationalsozialisten vernichtet wurden, um Spuren zu verwischen, oder in den letzten Kriegstagen unwiederbringlich zerstört wurden. Oft sind nur Überreste und Splitter erhalten. Abbildung \ref{fig:flowchart} zeigt einen idealtypischen Ablauf der Datenerhebung im Forschungsfeld. Demnach wurde eine Hauptquelle (Datenquelle 1) ausgewählt, aus der ein Sample gezogen wurde.\footnote{In München wurde jeder zweite Buchstabe aus der Gewerbekartei mit jüdischen Gewerbebetrieben erfasst, also ca. die Hälfte der Gewerbebetriebe, vgl. Rappl 2000, S. 179 Fußnote 217. In Frankfurt diente ebenfalls der Bestand aus dem Gewerbeamt als Hauptquelle (vgl. Interview B2\_Transkript, Pos. 31 und 45.), während in Mannheim das Verzeichnis jüdischer Gewerbetreibender sowie alle Arisierungsakten ab 1938 erhalten ist, vgl. Interview B3\_Transkript, Pos. 43 und 47 erhalten sind. In Hamburg basierte die Stichprobenziehung im Wesentlichen auf den Wiedergutmachungsakten, vgl. Bajohr 1998, S. 21ff. und Interview B1\_Transkript, Pos. 33.} In den meisten Fällen konnten daraus die wesentlichen Grunddaten (Name, Inhaber, Branche und Adresse) der Gewerbebetriebe entnommen werden. Die Datenquelle 1 bildeten im Idealfall publizierte und unpublizierte Verzeichnisse, Listen oder Karteisammlungen in denen Gewerbebetriebe dezidiert und systematisch mit dem Ziel der Verfolgung als jüdisch markiert und gelistet wurden.\footnote{In München übernahm diese Aufgabe das städtische Gewerbeamt, vgl. Rappl 2000, S. 145f. In Frankfurt am Main war der zentrale Akteur die Industrie- und Handelskammer.} Im nächsten Schritt wurden diese Daten mit weiteren Quellen abgeglichen, die den Vorgang der Verfolgung der einzelnen Gewerbebetriebe verwaltungsseitig dokumentierten. Zu dieser zweiten Datenquelle gehören verschiedene zeitgenössische Aktenbestände.\footnote{Zum Beispiel die Handelsregisterakten, die sogenannten Entjudungsakten oder die Akten der Devisenstellen, aber auch die Wiedergutmachungsakten nach 1945.} Aus diesem Rahmen fällt das Berliner Forschungsprojekt, wo man einen gänzlich anderen Ansatz verfolgt hat. Mangels überlieferter Quellen, wurde ein Sample anhand der Zentralhandelsregisterbeilage (ZHRB) erstellt und aus dieser die Aktivitäten aller handelsregisterlich geführten Unternehmen zwischen 1932 und 1942 erfasst. Man nahm hier folglich eine Gesamtaufnahme des Handelsregisters vor, welches im zweiten Schritt nacheinander mit weiteren Quellen abgeglichen und bei einer eindeutigen Indizienlage Gewerbebetriebe somit nachträglich als jüdisch identifiziert wurden.\footnote{Der Autor beschreibt dieses unkonventionelle Vorgehen im Forschungsfeld sehr detailliert in der Einleitung seiner Studie, vgl. Kreutzmüller 2012, S. 29-38.} Auch wenn mit ca. 8.000 identifizierten Jüdischen Gewerbebetrieben nur etwa 16 Prozent der insgesamt in Berlin ansässigen Jüdischen Gewerbebetriebe erhoben werden konnte, stellt das Sample in Bezug auf das Handelsregister als Grundgesamtheit fast eine Vollerhebung dar.\footnote{Von der Forschung wird geschätzt, dass in Berlin rund die Hälfte der Jüdischen Gewerbebetriebe im Deutschen Reich ansässig war, also rund 50.000. Kreutzmüller geht von ca. 10.000 im Handelsregister eingetragenen Jüdischen Gewerbebetrieben aus, vgl. Kreutzmüller 2012, S. 102f.}

\begin{figure}[h]
    \centering
    \frame{\includegraphics[width=\ScaleIfNeeded]{flowchart_data-collection}}
    \caption[Idealtypische Stichprobenziehung]{Idealtypische Stichprobenziehung von Daten zu Jüdischen Gewerbebetrieben.}
    \label{fig:flowchart}
\end{figure}

Nachteil der vereinfachten, groben Schematisierung ist, dass diese feinen der Datenerhebungen nicht enthalten sind. Darüber hinaus fehlen die mit der Quellenlage einhergehenden Stichproben-Verzerrungen (Bias) der Studien, welche bisher überhaupt nicht kommuniziert werden:

\begin{itemize}
    \item Viele Hauptquellen setzen zeitlich erst mit den reichsweiten Gesetzen ab 1938 ein. Die frühe Phase der Vernichtung der jüdischen Gewerbetätigkeit bleibt somit unterrepräsentiert, weil schlichtweg Daten dazu fehlen.\footnote{Interview B3\_Transkript, Pos. 43.}
    \item Bei der Verwendung von überwiegend Wiedergutmachungsakten als Datenquelle 1, insbesondere aus Rückerstattungsverfahren wie in Hamburg, liegt der Schwerpunkt automatisch auf den größeren Unternehmensverkäufen und den ehemaligen Eigentümern, die den Nationalsozialismus meist durch Emigration überlebt hatten. Liquidationen bleiben in diesem Ansatz unterrepräsentiert sowie der komplette Ostteil Deutschlands, da hier die Wiedergutmachung erst in den 90er Jahren mit dem Ende der DDR teilweise einsetzte.\footnote{Interview B1\_Transkript, Pos. 33.}
    \item Mit der ZHRB als Datenquelle 1 liegt der Fokus auf den handelsregisterlich eingetragenen Firmen und damit auf mittelständischen Gewerbebetrieben, wodurch Kleinunternehmen unterrepräsentiert bleiben. Außerdem liegt der Schwerpunkt auf Liquidationen, da das Handelsregister Besitzübernahmen nicht systematisch abbildet.\footnote{Interview B1\_Transkript, Pos. 37.}
\end{itemize}

\begin{figure}[h]
    \centering
    \frame{\includegraphics[width=\ScaleIfNeeded]{wikidata-data-metadata}}
    \caption[Beziehung von inhaltlichen Daten und Metadaten in Wikidata]{Beziehung von inhaltlichen Daten, deskriptiven Metadaten und Prozessmetadaten in Wikidata}
    \label{fig:datametadata}
\end{figure}

Es wird deutlich, dass geschichtswissenschaftliche Datenerhebungsmethoden aufgrund der historischen Quellengrundlage nicht analog zu den naturwissenschaftlichen Methoden standardisiert werden können. Es ist die Lückenhaftigkeit und es sind die Fehlstellen in der historischen Forschung, die eine adäquate Abbildung auf ein festes Schema zu einer spezifischen Herausforderung im Fach machen. Daher stellt sich insbesondere auch die Frage, welche Notwendigkeit Standardisierung hier besitzt. Es wäre genauer zu untersuchen, was der Mehrwert davon für die historische Forschung wäre oder ob zum Zwecke der methodischen Transparenz und Nachvollziehbarkeit eine rein textuelle Beschreibung oder Dokumentation zum Beispiel in Form einer Readme-Datei ausreicht. Tatsache ist, dass die Ausführungen zur Erhebung in die einzelnen Lokalstudien bisher unterschiedlich ausfallen und wichtige Informationen zum Verständnis der Forschungsdaten fehlen. Auch im Sinne der Nachnutzbarkeit von historischen Forschungsdaten ist also die offene Frage, welche Informationen zur Methodik überhaupt benötigt werden. Die Verknüpfung von inhaltlichen Daten, deskriptiven Metadaten und Prozessmetadaten in Wikidata kann demzufolge in dieser Arbeit nur Vorschlagscharakter haben (Abbildung \ref{fig:datametadata}).

\subsection{Problem \textit{Jüdischer} Gewerbebetrieb}

Untersuchungsgegenstand aller Lokalstudien sind ,,Jüdische Gewerbebetriebe'' oder ,,Jüdische Unternehmen''. Hieraus ergibt sich eine grundlegende methodische Schwierigkeit: Da die Konfessionszugehörigkeit im Zusammenhang mit einem Gewerbebetrieb oder Unternehmen schlichtweg unlogisch ist, ist der Begriff an und für sich unbrauchbar. Dieses Problem wird von den meisten Studien reflektiert und betont, dass es sich um eine antisemitische Zuschreibung und Konstruktion handelte. Diese Kennzeichnung und Diffamierung diente den Nationalsozialisten als Instrument für die weiteren Verfolgungspraktiken. Zur einfacheren Handhabung wurde der Begriff als Quellenbegriff jedoch von allen Studien beibehalten. Hierbei fallen zwei unterschiedliche Verwendungen auf: 

\begin{itemize}
    \item Der Begriff ,,jüdischer Gewerbebetrieb'' wird ausschließlich auf die jüdischen Besitzer*innnen bezogen und angewandt.\footnote{Vgl. Janetzko 2012, S. 18.} Damit wird jedoch das methodische Problem nicht aufgelöst, sondern verlagert sich auf den Begriff ,,jüdische Person'' oder ,,Jude/ Jüdin'', bei dem es sich ebenfalls um eine rassistische Zuschreibung handelte und nichts mit dem Selbstverständnis der Betroffenen zu tun hatte.\footnote{Das wird in der Studie zu Hamburg auch ausführlich reflektiert. Vgl. Bajohr 1997, S. 9.} Darüber hinaus werden in dieser Verwendung weitere Verfolgungskontexte vernachlässigt. So war es in der frühen Phase der Verfolgung durchaus möglich, dass Gewerbebetriebe als jüdisch diffamiert wurden, die einen hohen Anteil jüdischer Mitarbeiter*innen aufwiesen, deren Besitzer aber selbst nach der nationalsozialistischen Ideologie nichtjüdisch waren.\footnote{An diesem Beispiel zeigt sich überdies die in Wechselbeziehung stehenden Teilprozesse der Verdrängung der Juden aus dem Berufsleben und der Vernichtung der jüdischen Gewerbetätigkeit deutlich.}
    \item Der Begriff ,,jüdischer Gewerbebetrieb'' wird mit ,,als jüdisch betrachtet/ verfolgt'' übersetzt. In dieser Verwendung ist die jüdische Eigentümerschaft eines Gewerbebetriebs zunächst unerheblich, das heißt sie wird nicht vorausgesetzt, sondern es werden alle Gewerbebetriebe erfasst, die im nationalsozialistischen Kontext diffamiert wurden. Damit wird einerseits der Konstruktionscharakter des Begriff hervorgehoben und andererseits dem Umstand Rechnung getragen, dass die rassistischen Zuschreibungen grundsätzlich jeglicher rationalen Begründung entbehrten und aus diesem Grund willkürlich erfolgen konnten.
\end{itemize}

Auch wenn in allen Studien der selbe Untersuchungsgegenstand genannt wird, so zeigt sich erst in der konkreten Verwendung, dass dieser unterschiedlich ausgedehnt werden konnte, weil der Begriff an sich nicht widerspruchsfrei ist. Aus forschungsethischer Perspektive ist zudem problematisch, dass ein rassistisch konnotierter Begriff in der wissenschaftlichen Forschung beibehalten wird. Wichtig wäre, sich im Forschungsfeld auf eine einheitliche Verwendung zu einigen, denn bisher werden Jüdische Gewerbebetriebe\footnote{Das Wort ,,jüdisch'' wird im Kontext von Gewerbebetrieb in dieser Arbeit groß geschrieben. Als Orientierung hierfür dient die Selbstbezeichnung ,,Schwarze Menschen'' von People of Color (POC). Amnesty International ordnet den Begriff der diskriminierungssensiblen Sprache zu: ,,Schwarze Menschen ist eine Selbstbezeichnung und beschreibt eine von Rassismus betroffene gesellschaftliche Position. ,Schwarz wird großgeschrieben, um zu verdeutlichen, dass es sich um ein konstruiertes Zuordnungsmuster handelt und keine reelle Eigenschaft, die auf die Farbe der Haut zurückzuführen ist. So bedeutet Schwarz-Sein in diesem Kontext nicht, einer tatsächlichen oder angenommenen ethnischen Gruppe zugeordnet zu werden, sondern ist auch mit der gemeinsamen Rassismuserfahrung verbunden, auf eine bestimmte Art und Weise wahrgenommen zu werden.'' Hervorzuheben ist allerdings, dass es sich auch bei Jüdischer Gewerbebetrieb um keine Selbstbezeichnung handeln kann. Nichtsdestotrotz kann damit insbesondere das ,,konstruierte Zuordnungsmuster'' verdeutlicht werden. Amnestie International (2017): Glossar für diskriminierungssensible Sprache, URL: \url{https://www.amnesty.de/2017/3/1/glossar-fuer-diskriminierungssensible-sprache?gclid=Cj0KCQjwheyUBhD-ARIsAHJNM-MPznwnriOWClM3Qgqhbv6lRQXYHobeGOfVasBx2GV3m574xIcht0caAk57EALw_wcB} und Jamie Schearer, Hadija Haruna: Initiative Schwarze Menschen in Deutschland (ISD), Über Schwarze Menschen in Deutschland berichten, Blogbeitrag, 2013, URL: \url{http://isdonline.de/uber-schwarze-menschen-indeutschland-berichten } (alle letzter Zugriff am 03.06.2022).} unterschiedlich erhoben. Hierzu wird keine abschließende Entscheidung getroffen, da dies in einem Diskurs im Forschungsfeld gemeinsam entschieden werden sollte. Um dafür den Anstoß zu geben und um insbesondere auch die forschungsethischen Implikationen kritisch zu reflektieren, wurde im erstellten Wikidata-Projekt\footnote{Zum Wikidata-Projekt siehe Kapitel 4.3.} der \textit{Wikidata talk} ,,How do we use and model ,Jüdischer Gewerbebetrieb'?'' mit der Disskussionsfunktion angelegt und zwei Vorschläge unterbreitet: 

\begin{itemize}
    \item ,,Jüdischer Gewerbebetrieb'' wird als eigenes Item angelegt und mit Statements angereichert, die die nationalsozialistische Herkunft deutlich machen. Da in Wikidata Items von jedem/ jeder Nutzer*in ohne Einschränkung angelegt werden können, wäre diese Lösung schnell umsetzbar. Bei der Frage mit welcher Eigenschaft (Property) das Item ,,Jüdischer Gewerbebetrieb'' als Value auf einen konkreten Gewerbebetrieb abgebildet werden soll, lohnt abermals ein Blick auf benachbarte Wikidata-Projekte. Im Projekt \textit{Wikidata:WikiProject Victims of National Socialism} wurde 2020 die Verwendung des Begriffs ,,Holocaust-Opfer'' diskutiert.\footnote{Wikidata Talk:Q2763 (2020), Modeling of holocaust victim, URL: \url{https://www.wikidata.org/w/index.php?title=Talk:Q2763&oldid=1392179230}} Da in der Wikidata Konvention ist, Personen so neutral wie möglich zu beschreiben und Zuschreibungen von außen mit entsprechenden Aussagen kenntlich zu machen, hat man sich im Wikidata-Projekt darauf geeinigt, den Begriff nunmehr zusammen mit ,,Subjekt fungiert als (P2868) Opfer des Holocaust (Q5883980)'' zu verwenden und nicht mehr als ,,ist ein(e) (P31) Holocaust-Opfer (Q5883980)''.\footnote{Siehe zum Beispiel Wikidata-Item Anne Frank (Q4583), URL: \url{https://www.wikidata.org/w/index.php?title=Q4583&oldid=1645273699}.} Diese Verwendung kann für Gewerbebetriebe übernommen werden. Zwar geht es hier ausdrücklich nicht um Personen. Da aber die Verwendung ,,ist ein(e) (P31) Jüdischer Gewerbebetrieb (Q...)'' - wie gezeigt wurde - unlogisch wäre, bietet sich ,,Subjekt fungiert als (P2868) Jüdischer Gewerbebetrieb (Q...)'' an.
    \item Statt des Items ,,Jüdischer Gewerbebetrieb'' ist eine eigene Property ,,als jüdisch betrachtet/ verfolgt (P...)'' vorstellbar.\footnote{Dieser Ansatz wurde vom Berliner Forschungsprojekt umgesetzt.} Da diese Eigenschaft bisher noch nicht existiert, wäre diese Umsetzung etwas langwieriger, da Eigenschaften in der Wikidata nicht von jedem/ jeder Nutzer*in erstellt werden dürfen, sondern zunächst vorgeschlagen werden müssen.\footnote{Wikidata:Eigenschaften vorschlagen (2022), URL (stable): \url{https://www.wikidata.org/w/index.php?title=Wikidata:Property_proposal/de&oldid=1624532274}.} Nach einer öffentlichen Debatte entscheidet eine Administratoren-Gruppe der Wikidata, ob die Property neu aufgenommen wird oder ob Alternativ-Eigenschaften zur Verfügung stehen. Mit diesem Verfahren sollen Redundanzen und Widersprüchlichkeiten verhindert werden. Es dient zur Qualitätskontrolle der Wikidata. Daher ist es möglich, dass für das Forschungsfeld notwendige Eigenschaften für die Wikidata insgesamt nicht die Relevanz besitzen und aus diesem Grund abgelehnt werden können. Wie liberal oder konservativ die Wikidata-Politik hier ist, müsste erprobt werden.      
\end{itemize} 

\begin{figure}[h]
    \centering
    \frame{\includegraphics[width=\ScaleIfNeeded]{wikidata-discussion-jued-gewerbebetrieb}}
    \caption[Diskussionsseite zu Jüdischer Gewerbebetrieb in Wikidata]{Diskussionsseite zur Frage, wie "Jüdischer Gewerbebetrieb" verwendet und modelliert werden soll.}
    \label{fig:x cubed graph}
\end{figure}

\section{Aufbereitung}

\begin{quote}
    ,,Also ich denke, die sitzen alle auf irgendwelchen Excellisten oder wenn das ältere Forschungsprojekte sind, Herr Bajohr weiß ich nicht, ob der schon Excel genutzt hat für sein Hamburg-Buch oder ob der noch Karteikarten hatte.''\footnote{Interview B3\_Transkript, Pos. 79.}
\end{quote}

Um eine valide Datengrundlage für die Analyse zu erhalten, werden die erhobenen Rohdaten vorab aufbereitet. Damit erfolgt erstmals eine Verarbeitung der Daten, denn der Operationalisierung der Forschungsfragen entsprechend werden die Daten ausgewählt, strukturiert erfasst und bereinigt. In der historischen Forschung liegt die Situation vor, dass die Rohdaten im Quellenmaterial bereits vorliegen, sich aber mitunter über viele Quellen verteilen. Daher muss festgelegt werden, erstens welche Informationen aus den Quellen extrahiert werden sowie zweitens, mit welchem Werkzeug sie organisiert werden sollen. Dieser Prozess der Forschungsdaten-Genese ist bisher im Forschungsfeld weitestgehend unsichtbar und findet lediglich in den Studien zu Berlin und Frankfurt am Main nachträglich in den Publikationen Erwähnung.\footnote{Vgl. Kreutzmüller 2012, S. 38f., Nietzel 2012, S. 17.}. In beiden Projekten kamen ,,Datenbanken'' zum Einsatz, die anhand der Interviews als Microsoft Access-Datenbanken der Version 2007 spezifiziert werden konnten.\footnote{Vgl. Interview B2\_Transkript, Pos. 27.} Da es sich hierbei um eine Anwendung handelt, deren Datenorganisation auf relationalen Tabellen beruht, braucht es als Basis vorab ein Datenmodell, visualisiert zum Beispiel anhand eines Entity-Relationship-Diagramms (ERD) mit einer Beschreibung der darin verwendeten Elemente. Dieses ist für beide Studien allerdings nicht verfügbar. Damit ist eine Beurteilung der Daten hinsichtlich ihrer Verarbeitung bisher nicht möglich. Ziel von offenem Forschungsdatenmanagement ist es, die bisher unsichtbare Phase der Aufbereitung durch kollaborative Zusammenarbeit im Forschungsfeld transparenter zu machen. 

Zu diesem Zweck wurde in Wikidata das öffentliche Projekt \textit{Wikidata:WikiProject Destruction of the Economic Existence of the Jews Research} erstellt (Abbildung \ref{fig:wikidataprojecttabs}).\footnote{URL: \url{https://www.wikidata.org/wiki/Wikidata:WikiProject_Destruction_of_the_Economic_Existence_of_the_Jews_Research}.} Dieses besitzt grob drei Funktionen: Erstens können beliebig viele Seiten mithilfe von standardisierten Templates hierarchisch im Projekt angelegt werden (Pages und Subpages).\footnote{Siehe URL: \url{https://www.mediawiki.org/wiki/Help:Templates} (letzter Zugriff am 24.05.2022).} Diese bieten die Möglichkeit, die in Kapitel 3 methodisch aufgegriffene Taxonomie und damit die unterschiedlichen Zugänge im Forschungsfeld funktional umzusetzen. Auf der Hauptseite (Home) wurden bereits Hintergrundinformationen zum Projekt sowie zu dessen Zielen hinzugefügt. Dort ist auch erwähnt, dass diese Arbeit nur den Ausgangspunkt bildet und von hier aus sukzessive die angrenzenden Untersuchungsbereiche integriert werden können. Außerdem findet sich hier die nicht unwichtige Information, dass die Taxonomie im Forschungsfeld dem Systematisierungsversuch von Nietzel aus dem Jahr 2009 entlehnt ist.\footnote{Siehe Kapitel 3.2.1.}

\begin{figure}[h]
    \centering
    \frame{\includegraphics[width=\ScaleIfNeeded]{wikidata-project_tabs}}
    \caption[Wikidata:WikiProject Destruction of the Economic Existence of the Jews Research]{Wikidata:WikiProject Destruction of the Economic Existence of the Jews Research mit den in Tabs angelegten Subpages.}
    \label{fig:wikidataprojecttabs}
\end{figure}

Die bisherige Implementierung versteht sich explizit als Vorschlag, um eine Ausgangsbasis zu haben, von der aus Anpassungen und Weiterentwicklungen möglich werden. Um später in den gemeinsamen Austausch zu treten und Änderungen vorzunehmen, kann hierfür die zweite grundlegende Funktion der Diskussionseiten genutzt werden. Schließlich gibt es mit der Versionierung (,,Versionsgeschichte'') eine Kontrollfunktion, mit der sich alle Bearbeitungen zurückverfolgen und gegebenenfalls auf einen früheren Stand zurücksetzen lassen.\footnote{URL: \url{https://www.wikidata.org/w/index.php?title=Wikidata:WikiProject_Destruction_of_the_Economic_Existence_of_the_Jews_Research&action=history} (letzter Zugriff am 24.05.2022).} Insgesamt bietet das Wikidata-Projekt damit die Möglichkeit des kollaborativen Austauschs und der gemeinsamen Strategieentwicklung im Forschungsfeld. Erstmals können Methodiken und Konzepte im Forschungsfeld diskutiert sowie in Bezug auf die in der Arbeit betrachteten Forschungsdaten ein allgemeingültiger Leitfaden zur Erfassung Jüdischer Gewerbebetriebe entwickelt werden. Thematisch ist das Wikidata-Projekt in die Kategorien \textit{History WikiProjects} und \textit{Research WikiProjects} eingeordnet.\footnote{Siehe Wikidata:WikiProjekte, URL: \url{https://www.wikidata.org/wiki/Wikidata:WikiProjects/de} (letzter Zugriff am 24.05.2022).} Hier zeigt sich darüber hinaus, dass benachbarte Forschungsfelder zum Nationalsozialismus und zum Holocaust bereits mit eigenen Projekten vertreten sind, wodurch sich Anknüpfungspunkte über das Forschungsfeld hinaus ergeben.\footnote{Siehe WikiProject WWII, URL: \url{https://www.wikidata.org/wiki/Wikidata:WikiProject\_WWII}; WikiProject NS Perpetrator Research, URL: \url{https://www.wikidata.org/wiki/Wikidata:WikiProject\_NS\_Perpetrator\_Research}; WikiProject Victims of National Socialism, URL: \url{https://www.wikidata.org/wiki/Wikidata:WikiProject\_Victims\_of\_National\_Socialism}; WikiProject NS-Täterforschung, URL: \url{https://www.wikidata.org/wiki/Wikidata:WikiProject\_NS-Täterforschung}; Wikidata:WikiProject Nuremberg Trials, URL: \url{https://www.wikidata.org/wiki/Wikidata:WikiProject\_Nuremberg\_Trials} (alle letzter Zugriff am 24.05.2022).} 

\subsection{Zusammenführen der Quellen}

\paragraph{Datenmodell} Aus den Interviews ging hervor, dass beim Zusammenführen der Quellen die ausgewählten verteilten Informationen als strukturierte Daten in Excel oder Access erfasst wurden. Auch wenn es von keinem Befragten bewusst formuliert wurde, so haben alle zur ,,Handhabbarmachung der Informationen''\footnote{Kreutzmüller 2012, S. 38.} eine \textit{Modellierung} von den zu erfassenden Daten vorgenommen. Bei diesem Vorgang wird ein fest definierter realer Ausschnitt auf ein Modell mit Entitäten, zugehörigen Attributen und Beziehungen abgebildet. Aus den Interviews geht außerdem hervor, dass ein Datenmodell vorab nicht fest fixiert war, sondern dieses parallel zur Datenerfassung entstand und erweitert wurde.\footnote{Interview B1\_Transkript, Pos. 3, B2\_Transkript, Pos. 31 und Interview B1\_Transkript, Pos. 75.} Daraus ergeben sich zwei Anforderungen an offenes Forschungsdatenmanagement: Kollaborative Zusammenarbeit zwischen den Studien kann nur funktionieren, wenn man sich auf ein Modell mit einer einheitlichen Terminologie einigt. Es müssen folglich erstens die vielen unterschiedlichen Modelle und Begriffe der einzelnen Studien für eine gemeinsame Nutzung kompatibel gemacht werden. Da aufgrund der disparaten Überlieferungsstruktur ein statisches Modell vorab nicht immer feststehen kann, muss dieses zweitens dynamisch und skalierbar sein.

Anhand der für die Arbeit zu Verfügung gestellten Daten aus Berlin, Mannheim und Krefeld sowie mithilfe der Interviews wurde zunächst versucht, eine begriffliche Kontrolle im Untersuchungsfeld zur Vernichtung der jüdischen Gewerbetätigkeit im NS zu erhalten. Hierbei wurde sich der Methodik der Dokumentbeschreibungssprachen aus den Bibliotheks-, Dokumentations- und Informationswissenschaften bedient, mit der Fachgebiete mittels Thesauri oder Klassifikationen hierarchisch geordnet und inhaltlich erschlossen werden (Sacherschließung).\footnote{Siehe Gernot Wersig: Thesaurus-Leitfaden. Eine Einführung in das Thesaurus-Prinzip in Theorie und Praxis, Berlin, Boston 2016, doi:10.1515/9783111412719.} In diesem Sinne wird das Untersuchungsfeld als eigenes Begriffssystem verstanden, mittels dessen es sich inhaltlich erschließen lässt.\footnote{Der erstellte Thesaurus ist im Anhang D.5 beigefügt.} Auf diese Weise konnte nicht nur eine Übersicht über die wesentlichen historischen Informationen im Untersuchungsfeld erstellt werden, sondern es zeigte sich mit dieser Methode auch, dass es zum einen Mehrdeutigkeiten bei der Bezeichnung von Sachverhalten gibt (Synonymproblem) und zum anderen Unklarheiten bestehen, wie Begriffe angewandt werden sollen.\footnote{Ebd., S. 47-51.} Für das Synonymproblem können Äquivalenzklassen vorgeschlagen werden.\footnote{Im Modell in den einzelnen Kästchen fett hervorgehoben} Die unklaren Begriffe müssen in dieser Arbeit offen bleiben, da abschließend deren globale Relevanz für das Forschungsfeld nicht bestimmt (z.B. Insolvenz)\footnote{Die Geschäftsauflösung bzw. Insolvenz wurde nur in der Krefelder Studie untersucht.} oder ihre Ambiguität (z.B. Geschäftsaufgabe) nicht aufgelöst werden konnte. 

Das feine Begriffssystem\footnote{Im Modell grau hinterlegt} wurde grob abstrahiert, sodass die generischen Begriffe auf der ersten Ebene eine Top-Level-Ontologie ergeben, an die Studien-unabhängig alle Forschungsdaten im Forschungsfeld andocken können.\footnote{Siehe zu Top-Level-Ontologie Rehbein, Ontologien, 2017, S. 162-174.} Auf diese Weise kann das Datenmodell kompatibel und interoperabel gehalten werden, in der Konsequenz also zukünftig auch an andere Forschungsfelder anschließen.

Das generische strukturelle Metadatenschema wurde im nächsten Schritt in das Wikidata-Projekt integriert, welches somit eine Strukturierung der Daten vorgibt (Abbildung \ref{fig:wikidatagenericmodel}). 

\begin{figure}[h]
    \centering
    \frame{\includegraphics[scale=0.6]{wikidata-data-model-generic}}
    \caption{Integration des strukturellen Metadatenschemas in Wikidata}
    \label{fig:wikidatagenericmodel}
\end{figure}

Am Beispiel des Berliner Gewerbebetriebs ,,Gorbatschow Liköre F. Kramer \& Co'', welches 1938 vom Eigentümer Josef Kramer verkauft werden musste, wurde ein erster Entwurf für das präzise Datenmodell in Wikidata erstellt.\footnote{Das Modell ist als Anhang ... beigefügt.} Analog zur Modellierung der Forschungsprojekte wurden vorhandene Items und Properties nachgenutzt. Wo dies nicht möglich war, sind die Entities farblich markiert. Der Entwurf wurde anschließend im Wikidata-Projekt angelegt (Abbildung \ref{fig:wikidatadatamodel}).\footnote{Siehe auch Wikidata-Projekt, URL (stable): \url{https://www.wikidata.org/w/index.php?title=Wikidata:WikiProject\_Destruction\_of\_the\_Economic\_Existence\_of\_the\_Jews\_Research/Vernichtung\_der\_jüdischen\_Gewerbetätigkeit&oldid=1648462059}.}   

\begin{figure}[h]
    \centering
    \frame{\includegraphics[width=\ScaleIfNeeded]{wikidata-data-model}}
    \caption{Beschreibung der Statements zu den einzelnen Entities}
    \label{fig:wikidatadatamodel}
\end{figure}

Im öffentlichen Wikidata-Projekt kann das Datenmodell zur Beschreibung Jüdischer Gewerbebetriebe kollaborativ angepasst und weiterentwickelt werden. In der Tabelle in Abbildung \ref{fig:wikidatadatamodel} stellt jede Zeile eine Aussage zu einer Entität dar (im Bild Gewerbebetrieb und Branche). In dieser können neben der Statements außerdem Verwendungsregeln und detaillierte Beschreibungen dokumentiert werden. Das aktuelle prototypische Datenmodell versteht sich lediglich als Vorschlag und soll in erster Linie den Anstoß für weitere Diskussionen geben. Denn insbesondere die Frage der Modellierung von Forschungsdaten wird im Forschungsfeld bisher nicht systematisch bearbeitet. Aber schon in dieser frühen Phase ergeben sich Pfadabhängigkeiten, die Einfluss auf die anschließende Datenanalyse haben. Dies kann an einem Beispiel veranschaulicht werden: Wenn zu einem Gewerbebetrieb nur eine Adresse strukturiert erfasst werden kann (1:1 Kardinalität), können (überregionale) Umzüge später nicht mehr untersucht werden. In Berlin gab es in der Access-Datenbank nur Felder für eine Adresse pro Gewerbebetrieb. Weitere Adressen wurden unstrukturiert in sogenannten Freitextfeldern erfasst. Damit war und ist es nur schwer möglich, sich der Untersuchung von Ausweichsbewegungen - was in Berlin nur auf qualitativer Ebene geschah - quantitativ zu nähern.\footnote{siehe Kreutzmüller 2012, S. 310-310 (Kap. Umzug).} Das Wikidata-Datenmodell mit dem dahinter stehenden Linked Data-Konzept bietet demgegenüber den Vorteil, dass ausschließlich strukturierte Daten in Subjekt-Prädikat-Objekt-Ausdrücken erfasst sowie neue Properties und Items dynamisch ergänzt werden können. Eine aufwändige Anpassung des Datenmodells entfällt dadurch. Die Einschränkung ist jedoch, dass erfasste Daten zu Jüdischen Gewerbebetrieben in dieser Form nicht gegen das Modell geprüft werden können, da - wie oben beschrieben - das Modell im Wikidata-Projekt nur informellen Charakter hat. Das bedeutet, dass Daten auf der technischen Ebene auch dann gültig wären, wenn diese vollkommen anders erfasst würden. Damit ist eine Kontrolle über die Gültigkeit von Daten zu Jüdischen Gewerbebetrieben zum jetzigen Zeitpunkt nicht gegeben. Wikidata bietet aber mit dem Ziel der weiteren Qualitätssicherung die Erstellung von \textit{EntitySchemas} an (Abbildung \ref{fig:wikidataentityschema}).\footnote{Siehe Wikidata Schemas, URL: \url{https://www.wikidata.org/wiki/Wikidata:Schemas}. Siehe zum Beispiel das Entity Schema zu Mensch (E10), URL: \url{https://www.wikidata.org/wiki/EntitySchema:E10} (alle letzter Zugriff am 27.05.2022).} Damit ließe sich ein verbindliches Schema zur Erfassung von Jüdischen Gewerbebetrieben definieren. Dies ist jedoch erst dann sinnvoll, wenn ein gemeinsamer Grundstamm an Aussagen im Forschungsfeld feststeht.

\begin{figure}[h]
    \centering
    \frame{\includegraphics[scale=0.7]{wikidata-entity-schema}}
    \caption[EntitySchemas in Wikidata]{Mithilfe von EntitySchemas können Daten gegen ein fest definiertes Schema auf ihre Gültigkeit hin geprüft werden.}
    \label{fig:wikidataentityschema}
\end{figure}

\paragraph{Personenbezogene Daten}

Auch wenn die Daten zu Jüdischen Gewerbebetrieben größtenteils als ethisch unbedenklich eingestuft wurden\footnote{Vgl. Kapitel 3.5.}, gibt es mit den Unternehmenseigentümern personenbezogene Daten, die besondere forschungsethische Fragen aufwerfen, wenn sie in Open Data verfügbar sind. Zu beachten ist, dass es sich in der Regel nicht um Personen des öffentlichen Interesses handelt, was eine detaillierte Veröffentlichung bibliografischer Daten rechtfertigen würde. Das bedeutet, dass der Eigentümer Josef Kramer von Gorbatschow Liköre F. Kramer \& Co nicht mit Anne Frank\footnote{Wikidata-Item Anne Frank (Q4583), URL: \url{https://www.wikidata.org/wiki/Q4583}.} oder der Holocaust-Überlebenden und Aktivistin Margot Friedländer\footnote{Wikidata-Item Margot Friedländer (Q1895371), URL: \url{https://www.wikidata.org/wiki/Q1895371}.} gleichgesetzt werden kann. Gerade auch die Fälle, wo ehemalige Inhaber den Holocaust durch Emigration überlebt haben und nach 1945 einen Antrag auf Rückerstattung stellten, können rechtliche Einwände gegen eine Veröffentlichung von detaillierten personenbezogenen Daten sprechen.\footnote{Vgl. Kapitel 3.5.} Anders als bisher im Forschungsfeld braucht offenes Forschungsdatenmanagement in Wikidata hier ein gemeinsames Vorgehen sowie eine klare und nachvollziehbare Strategie, die den verantwortungsvollen Umgang mit diesen sensiblen Daten sicherstellt. 

Hierzu wird folgende Empfehlung gemacht: Generell sollte das grundlose Sammeln personenbezogener Daten vermieden werden. Das bedeutet, auch wenn sie in den Quellen vorhanden sind, aber nicht der Bearbeitung von Forschungsfragen direkt dienen, werden sie nicht in Wikidata aufgenommen. Der Grundsatz ist hier, so wenig wie möglich Daten und so viel wie nötig zu erfassen. Sofern es also keine personenbezogenen Forschungsfragen gibt, werden lediglich Daten erfasst, die im Zusammenhang mit der unternehmerischen Tätigkeit stehen. Dies wurde am Beispiel der Gorbatschow Liköre F. Kramer \& Co für den Eigentümer Josef Kramer in Wikidata umgesetzt.\footnote{Wikidata-Item Josef Kramer (Q112135768), URL: \url{https://www.wikidata.org/wiki/Q112135768}.} Wenn wie in einigen Lokalstudien das Schicksal der Eigentümer nach der Besitzübernahme oder Liquidation statistisch untersucht wird\footnote{Siehe Bajohr 1998, S. 388 und Nietzel 2012, S. 121ff.}, werden nur die wesentlichen Informationen zu Emigration oder Deportation aufgenommen. Bei der Beschreibung des Verfolgungskontextes wird auf das bereits erwähnte WikiData-Projekt ,,Wikidata:WikiProject Victims of National Socialism'' zurückgegriffen. Demnach werden die Eigentümer*innen mit ,,Subjekt fungiert als (P2868) Holocaustüberlebender (Q12409870)'' bzw. ,,Subjekt fungiert als (P2868) Opfer des Holocaust (Q5883980)'' beschrieben. Für deren Schicksal werden die Aussagen ,,Schlüsselereignis (P793) ist ein(e) (P31) Holocaust-Gefangenentransport (Q61927259)'' bzw. ,,Schlüsselereignis (P793) ist ein(e) (P31) Auswanderung (Q187668)'' verwendet. Für den Fall, dass es weitere Informationen zu den Eigentümer*innen in externen öffentlichen Datenbanken gibt, die aber für die eigene Forschung nicht relevant sind, kann zur Datenvernetzung die eindeutige externe Personenkennung als Wikidata-Identifikator hinzugefügt werden (Abbildung \ref{fig:wikidataidentificator}).

\begin{figure}[h]
    \centering
    \frame{\includegraphics[width=\ScaleIfNeeded]{wikidata-idenficator}}
    \caption[Eindeutige Identifikatoren in Wikidata]{VIAF-Kennung der Holocaustüberlebender und Aktivistin Eva Schloss als Wikidata-Identifikator im zugehörigen Wikidata-Item \url{https://www.wikidata.org/wiki/Q90250}.}
    \label{fig:wikidataidentificator}
\end{figure}

Aus forschungsethischer Sicht kann das in dieser Arbeit angelegte Wikidata-Projekt ein Forum sein, wo die Handhabung personenbezogener Daten diskutiert werden kann und wo allgemeingültige Grundsätze festgehalten werden können.\footnote{Der Vorschlag aus dieser Arbeit wurde auf der Diskussionsseite im Wikidata-Projekt dokumentiert.} Damit wäre es über die Datenmodellierung hinaus eine Plattform, die wichtige Orientierung im Umgang mit sensiblen Daten im Forschungsfeld gibt vor allem auch für Forscher*innen, die sich gänzlich neu mit dem Thema befassen.

\paragraph{Quellennachweise}

Die Information, woher die Daten zu Jüdischen Gewerbebetrieben stammen, stellt das vielleicht wichtigste Qualitätskriterium von offenem Forschungsdatenmanagement im Forschungsfeld dar.\footnote{Vgl. Interview B1\_Transkript, Pos. 139 und Interview B3\_Transkript, Pos. 73.} Insbesondere weil der Untersuchungsgegenstand ,,Jüdischer Gewerbebetrieb'', wie gezeigt worden ist, methodische Schwierigkeiten mit sich bringt, braucht es Nachweise, die diesen in den Quellen eindeutig belegen. Wikidata ist für diese Anforderung funktional besonders gut geeignet. Denn die globale Wissensdatenbank versteht sich ausdrücklich als sekundäre Datenbank und nicht als Primärquelle.\footnote{Vgl. Wikidata Hilfe:Belege, URL: \url{https://www.wikidata.org/wiki/Help:Sources/de} und Wikidata:Nachprüfbarkeit, URL: \url{https://www.wikidata.org/wiki/Wikidata:Verifiability/de} (alle letzter Zugriff am 28.05.2022).} Das bedeutet, dass jede Aussage in Wikidata grundsätzlich als Behauptung aufgefasst wird, die erst dann als valide gewertet wird, sobald sie durch Quellen- und Literaturangaben belegt ist. Um den hohen Anspruch der Überprüfbarkeit erfüllen zu können, enthält das allgemeine Datenmodell der Wikidata neben der Items, Properties außerdem noch sogenannte Qualifier und References (Fundstellen), die jedem Aussagenwert (Value) eines Items beliebig oft hinzugefügt werden können.\footnote{Siehe Wikidata Help:Qualifikatoren, URL: \url{https://www.wikidata.org/wiki/Help:Qualifiers/de} und Wikidata:Tours/References, URL: \url{https://www.wikidata.org/w/index.php?title=Wikidata:Tours/References&oldid=1619471790} (alle letzter Zugriff am 28.05.2022).} Der Funktionsumfang der Wikidata geht hier also über das einfache Linked Data-Modell hinaus. Bei der Zitation und Erstellung von bibliografischen Items, orientiert sich Wikidata zudem an bewährten bibliografischen Metadatenstandards wie \textit{Functional Requirements for Bibliographic Records} (FRBR) und verweist auf entsprechende Wikidata-Projekte, die sich auf die Modellierung bestimmter Quellengattungen spezialisiert haben.\footnote{Vgl. Wikidata Hilfe:Belege, ebd. Zu FRBR siehe IFLA Study Group on the Functional Requirements for Bibliographic Records, Susanne Oehlschläger: Funktionelle Anforderungen an bibliografische Datensätze. Abschlussbericht (2006), in: Deutsche Nationalbibliothek (Hrsg.), IFLA Series on Bibliographic Control (Translation of Vol. 19), 2006, URL (stable): \url{https://repository.ifla.org/handle/123456789/817}. Beispiel für Wikidata-Prjekt siehe Wikidata:WikiProject Periodicals, URL (stable): \url{https://www.wikidata.org/w/index.php?title=Wikidata:WikiProject_Periodicals&oldid=1609366270}.} 

Für das Forschungsfeld eröffnet sich dadurch die Möglichkeit, detailliert erstens Informationen zu Jüdischen Gewerbebetrieben mit einer oder mehreren Belegstelle zu versehen und zweitens Angaben zu deren Gültigkeit mittels Qualifikatoren zu machen (Abbildung \ref{fig:wikidatareference}).

\begin{figure}[h]
    \centering
    \frame{\includegraphics[width=\ScaleIfNeeded]{wikidata-reference}}
    \caption[Qualifikatoren und Fundstellen in Wikidata]{Qualifikator und Fundstellen zu der Namensform ,,Gorbatschow Liköre F. Kramer \& Co'' des gleichnamigen Gewerbebetriebs.}
    \label{fig:wikidatareference}
\end{figure}

Wie in der Abbildung \ref{fig:wikidatareference} an der zweiten Fundstelle außerdem zu sehen ist, kann ein permanenter Link zum gegebenenfalls im Web vorhandenen Quellendigitalisat hinterlegt werden. Falls dieses in einer Open Data-Lizenz veröffentlicht ist, bietet sich darüber hinaus an, es direkt in das Schwesternprojekt und in die öffentliche Mediensammlung \textit{Wikimedia Commons}\footnote{URL: \url{https://commons.wikimedia.org/wiki/Hauptseite} (letzter Zugriff am 28.05.2022).} hochzuladen. Dort gibt es bereits Bildmaterial zu Jüdischen Gewerbebetrieben vor allem in Zusammenhang mit der Reichspogromnacht 1938 sowie mit annoncierten Besitzübernahmen. Die Commons-Ressourcen können direkt im zugehörigen Wikidata-Item eingebunden werden (Abbildung \ref{fig:wikidatacommons}).\footnote{Siehe Abfrage zu ,,Arisierung'' in Commons, URL: \url{https://commons.wikimedia.org/w/index.php?search=Arisierung&title=Special:MediaSearch&go=Go&type=image} (letzter Zugriff am 28.05.2022).}

\begin{figure}[h]
    \centering
    \frame{\includegraphics[width=\ScaleIfNeeded]{wikidata-commons}}
    \caption[Verknüpfungen in Wikidata]{Direkte Verknüpfung eines Fotodigitalisats aus Wikimedia Commons in Wikidata.}
    \label{fig:wikidatacommons}
\end{figure}

Daraus ergibt sich erstmals eine direkte Verknüpfung von Forschungsdaten und historischen Quellen, die eine bisher nie dagewesene Datenüberprüfung und -verifizierung ermöglicht und in der Konsequenz die Glaubwürdigkeit von Forschungsdaten im Forschungsfeld enorm steigern kann.\footnote{Auch in den Interviews wurde eine mögliche Verknüpfung als Funktionalität von offenem Forschungsdatenmanagement herausgehoben, vgl. Interview B3\_Transkript, Pos. 77.} 

Das Wikidata-Projekt kann daneben zur methodischen Führung sowie zur Entwicklung von Kriterien, welche Quellen sich als Belege für Jüdische Gewerbebetriebe eignen, genutzt und eine qualifizierte Quellensammlung im Forschungsfeld sukzessive aufgebaut werden.

\subsection{Erfassung von Jüdischen Gewerbebetrieben}

Für die Erfassung der Daten zu Jüdischen Gewerbebetrieben kamen, wie bereits erwähnt worden ist, herkömmliche Microsoft-Produkte wie Excel oder Access zum Einsatz.\footnote{Vgl. Interview B3\_Transkript, Pos. 11 und Interview B2\_Transkript, Pos. 27.} Es wurde folglich in erster Linie proprietäre, also kostenpflichtige, Software genutzt, die in der Regel nicht plattformunabhängig ist. Dies erschwert generell eine kollaborative Arbeit auf den Daten, denn die MS-Access-Anwendung zum Beispiel steht für Unix-basierte Betriebssysteme (Linux, Apple) gar nicht oder nur eingeschränkt zur Verfügung. Das heißt, dass grundlegende Open Source-Kriterien von diesen Produkten nicht erfüllt werden. 

Im Zusammenhang mit der Datenerfassung ist daher die wohl größte Herausforderung und aufwändigste Arbeit, ein User-Interface (UI) zu gestalten, das die bestmögliche User Experience und Usability (UX) bietet. Hier hält Wikidata nicht die perfekte Lösung bereit, aber zumindest Auswege aus möglichen anwendungsbedingten Einschränkungen und Zwängen, indem es nicht nur eine sondern mehrere Möglichkeiten der Erfassung von Daten gibt.\footnote{Siehe Wikidata:Datenspende, URL: \url{https://www.wikidata.org/wiki/Wikidata:Data\_donation/de\#Online-Tools\_=} (letzter Zugriff am 29.05.2022).} Von diesen werden drei nachfolgend vorgestellt, die sich an den bisherigen Kenntnisständen und Erfahrungen mit digitalen Werkzeugen im Forschungsfeld orientieren. 

Naheliegend ist die Eingabe der Daten mit dem Linked Open Data-Interface direkt auf der Website von Wikidata, wo per Mouseclick eines neues einzelnes Datenobjekt erstellt und erfasst werden kann (Abbildung \ref{fig:wikidatainterface}). 

\begin{figure}[h]
    \centering
    \frame{\includegraphics[width=\ScaleIfNeeded]{wikidata-linked-data-interface}}
    \caption[Erstellung eines neuen Datenobjekts in Wikidata]{Links im Bild wird das Datenobjekt ,,Max Kann'' angelegt. Rechts ist das Item erstellt worden und es können weitere Aussagen erfasst werden.}
    \label{fig:wikidatainterface}
\end{figure}

Diese Möglichkeit eignet sich besonders gut, wenn nur wenige Jüdische Gewerbebetriebe erfasst werden sollen. Der Vorteil ist auch, dass ein Team gleichzeitig an der Eingabe von Daten arbeiten kann, was in den älteren Excel- oder Access-Desktopversionen nicht möglich war.\footnote{Seit den Webversionen der Office-Sammlung von Microsoft kann allerdings auch in diesen kollaborativ gearbeitet werden. Siehe Microsoft Support (2022): Gleichzeitiges Bearbeiten von Excel-Arbeitsmappen mit der gemeinsamen Dokumenterstellung, URL: \url{https://support.microsoft.com/de-de/office/gleichzeitiges-bearbeiten-von-excel-arbeitsmappen-mit-der-gemeinsamen-dokumenterstellung-7152aa8b-b791-414c-a3bb-3024e46fb104}.} Mit steigender Zahl kann die Eingabe im Wikidata-Interface jedoch an Grenzen stoßen. Für Berlin, Frankfurt a.M. sowie Mannheim wurden jeweils Daten im 1.000er-Bereich erhoben.\footnote{In Berlin ca. 8.000, Frankfurt a.M. ca. 3.000 und Mannheim ca. 1.200.} Diese alle manuell und einzeln einzugeben, ist extrem zeitaufwändig, zumal diese bereits in Tabellenform vorliegen. In diesem Fall bietet sich die Stapel-Importfunktion ,,QuickStatements'' (batch import) an, bei der Daten, die als Tabstopp- oder Komma-separierte strukturierte Daten vorliegen, in Wikidata importiert werden.\footnote{URL: \url{https://quickstatements.toolforge.org/\#/batch} (letzer Zugriff am 29.05.2022).} Bevor der eigentliche Import jedoch erfolgen kann, bedarf es einer Datenvorbereitung und -bereinigung. Zuerst müssen proprietäre Formate in das offene CSV-Format transformiert werden, was zumindest für Excel-Dateien unproblematisch mit der Exportfunktion erfolgen kann (Abbildung \ref{fig:excelcsv}).

\begin{figure}[h]
    \centering
    \frame{\includegraphics[width=\ScaleIfNeeded]{excel-csv}}
    \caption[Export in das offene CSV-Format]{Export der Excel-Tabelle mit Daten zu Jüdischen Gewerbebetrieben aus Mannheim in das offene CSV-Format}
    \label{fig:excelcsv}
\end{figure}

Bei den Access-Datenbanken ist diese Transformation aufwändiger, da hier das Problem hinzu kommt, dass es sich um veraltete Software-Versionen von 2007 handelt, die sich mit neueren Versionen nicht mehr so einfach öffnen lassen. Für Berlin wurde kürzlich in einem eigenen Projekt diese Transformation durchgeführt.\footnote{Im Zuge dieses Transformationsprozesses wurde eine eigene Online-Datenbank für die Berliner Forschungsdaten entwickelt, siehe URL: \url{https://dbjg.geschichte.hu-berlin.de/} (letzter Zugriff am 06.06.2022).} Im nächsten Schritt muss die ursprüngliche Datenstrukturierung in den exportierten CSV-Dateien in das Datenmodell des Wikidata-Projekts transformiert werden, wofür Wikidata eine ausführliche Dokumentation bereitstellt.\footnote{Siehe Wikidata Help:QuickStatements, URL: \url{https://www.wikidata.org/wiki/Help:QuickStatements} (letzter Zugriff am 29.05.2022).} Dies wurde am Beispiel des Gewerbebetriebs \textit{Franz Mettner GmbH} aus Mannheim\footnote{URL (stable): \url{https://www.wikidata.org/w/index.php?title=Q112163392\&oldid=1649916700}.} getestet (Abbildung \ref{fig:wikidatacleanup}).

\begin{figure}[h]
    \centering
    \frame{\includegraphics[width=\ScaleIfNeeded]{wikidata-bereinigung}}
    \caption[QuickStatements: Datenvorbereitung und -bereinigung]{Datenvorbereitung und -bereinigung für den Import mit ,,QuickStatements''}
    \label{fig:wikidatacleanup}
\end{figure}

Grau hinterlegt sind die Komma-separierten Werte, welche mit QuickStatements importiert wurden. Farblich markiert sind die ursprünglichen Felder aus der Excel-Tabelle, welche dem Wikidata-Datenmodell zugeordnet werden konnten (gelb) oder Schwierigkeiten bei der Zuordnung bereitet haben (orange). So scheint die Einordnung von ,,Einzelhandelsgeschäft'' unter Branche nicht treffend zu sein. Zudem sind die Quellenangaben ,,siehe 7627-7632'' nicht überprüfbar. Eventuell beziehen sich die Nummern auf ein projektinterne Verzeichnis, das aber nicht verfügbar ist. Das bedeutet, dass eine Verifizierung des Jüdischen Gewerbebetriebs anhand der Excel-Tabelle nicht möglich ist. Hier müssten demnach die exakten Quellenangaben noch ergänzt werden. Der Import selbst in QuickStatements ist, sofern das Schema in der CSV-Datei valide ist, schnell ausgeführt (Abbildung \ref{fig:wikidataquickstatements}).

\begin{figure}[h]
    \centering
    \frame{\includegraphics[width=\ScaleIfNeeded]{wikidata-quickstatements}}
    \caption{Erfolgreicher Import in QuickStatements}
    \label{fig:wikidataquickstatements}
\end{figure}

Während des Test-Imports zeigte sich, dass vor allem die Vorbereitung und Bereinigung der Daten zeitintensive Aufgaben sind. Hier tauchen schließlich auch Probleme auf, die nicht immer vorhersehbar sind und für die eine Lösung gefunden werden muss. Dazu gehören Freitextfelder, die von allen Studien verwendet und in denen unterschiedlichste, teils sehr ausführliche Informationen festgehalten wurden, wie zum Beispiel: ,,1910 verlegte er sein Geschäft nach Mannheim (4-5 Arbeiter, die Ehefrau und der Sohn haben auch dort gearbeitet); 1937: wegen Hehlerei zu 1 Jahr, 4 Monaten und 2 Wochen Haft verurteilt Verbot zur Weiterführung des Geschäfts für 3 Jahre; nach USA emigriert''. Es ist nicht klar, welche Rolle diese Felder später bei der Auswertung spielten. Statistisch lässt sich damit jedenfalls nicht arbeiten. Daher wäre in diesem Fall eine Entscheidung notwendig, welche Daten kassiert werden können, weil die Informationen für die Beantwortung der Forschungsfragen letztlich nicht relevant sind.  

\begin{figure}[h]
    \centering
    \frame{\includegraphics[width=\ScaleIfNeeded]{wikidata-pipeline}}
    \caption{Wikidata-Pipeline in Open Refine}
    \label{fig:wikidatapipeline}
\end{figure}

Die dritte und letzte Option der Datenerfassung, die in dieser Arbeit vorgestellt wird, verdeutlicht, wie vorhandene Pipelines genutzt werden können, um die Datenerfassung zu optimieren. Der Nachteil von QuickStatements ist, dass die Daten aus den CSV-Dateien manuell in die Webanwendung kopiert werden müssen. Außerdem können die Daten in der Anwendung selbst nicht weiter überprüft werden. Hierfür ist das externe Open Source-Tool ,,Open Refine''\footnote{URL: \url{https://openrefine.org/} (letzter Zugriff am 29.05.2022).} besser geeignet. Die mächtige Anwendung, die auf die Bereinigung und Anreicherung von Massendaten spezialisiert ist, ermöglicht den Abgleich von Tabellendaten mit externen Wissensdatenbanken und darüber hinaus den Import direkt aus der Anwendung nach Wikidata.\footnote{Siehe Wikidata:Tools/OpenRefine, URL (stable:) \url{https://www.wikidata.org/w/index.php?title=Wikidata:Tools/OpenRefine\&oldid=1620901604}, Open Refine (2022): Overview of Wikibase support. Editing Wikidata with OpenRefine, URL: \url{https://docs.openrefine.org/manual/wikibase/overview\#editing-wikidata-with-openrefine} (letzter Zugriff am 29.05.2022).}

Die Kernfunktionen der Datenbereinigung werden hier nicht weiter erläutert, sondern auf die Wikidata-Upload-Pipeline fokussiert. In Abbildung \ref{fig:wikidatapipeline} sind die Daten zum Jüdischen Gewerbebetrieb \textit{Otto Simon Straus} aus Mannheim\footnote{URL (stable): \url{https://www.wikidata.org/w/index.php?title=Q112166241\&oldid=1650023676}.} bereits von einer CSV-Datei in ein neu erstelltes Open Refine-Projekt hochgeladen worden. Die grünen Balken unter jeder Titelspalte zeigen den Status des Datenabgleichs mit Wikidata an, welcher in Open Refine als ,,Reconciliation process'' bezeichnet wird. Dieser muss einmal für jede Titelspalte manuell durchgeführt werden. Die dunkelgrünen Balken stehen für eindeutige Treffer (Abbildung \ref{fig:wikidatareconciliation} am Beispiel ,,Liquidation''), die hellgrünen für neue Werte und die grauen Balken für die noch abzugleichenden Daten. 

\begin{figure}[h]
    \centering
    \frame{\includegraphics[width=\ScaleIfNeeded]{wikidata-reconciliation}}
    \caption{Eindeutiger Abgleich mit Wikidata in Open Refine}
    \label{fig:wikidatareconciliation}
\end{figure}

Im zweiten Schritt (in der Abbildung \ref{fig:wikidatapipeline} in der Mitte links) erfolgt die Prüfung der abgeglichenen Daten gegen ein Wikidata-Schema. Dies kann entweder direkt in Open Refine erstellt oder ein bestehendes als JSON-File importiert werden. Wenn für die Jüdischen Gewerbebetriebe demnach ein grundlegendes Datenmodell feststeht, kann dieses als JSON im Wikidata-Projekt zur Verfügung gestellt und in Open Refine von jedem wiederverwendet werden.\footnote{Zum Test wurde das in Open Refine erstellte Schema im Wikidata-Projekt hochgeladen, siehe URL: \url{https://www.wikidata.org/wiki/Wikidata:WikiProject\_Destruction\_of\_the\_Economic\_Existence\_of\_the\_Jews\_Research/Vernichtung\_der\_jüdischen\_Gewerbetätigkeit/Schema}.} Auf diese Weise lässt sich eine Datenkontrolle bei der Dateneingabe im Forschungsfeld forcieren. Daneben ist es eine Arbeitserleichterung und gibt methodische Orientierung, wenn Schemata nachgenutzt und nicht für jede Studie immer wieder neu erstellt werden müssen, was zur besseren Datenqualität insgesamt beiträgt. Im dritten und letzten Schritt der Pipeline (Abbildung \ref{fig:wikidatapipeline} in der Mitte rechts sowie unten) lassen sich die Daten in einer Vorschau in Open Refine nochmals überprüfen, bevor sie in Wikidata importiert werden.\footnote{Permalink zum lokalen Projekt (localhost) URL: \url{http://127.0.0.1:3333/project?project=2437124036317\&ui=\%7B\%22facets\%22\%3A\%5B\%5D\%7D}.} Der Nachteil von Open Refine ist, dass die Möglichkeiten der kollaborativen Arbeit an einem Projekt noch begrenzt sind. Bisher können diese nur manuell zusammengeführt werden.\footnote{Siehe Consortium Historicum (2018): Ergänzen eines OpenRefine-Projekts mit einem anderen, Blogbeitrag auf histHub am 26.02.2018, URL: \url{https://histhub.ch/ergaenzen-eines-openrefine-projekts-mit-einem-anderen/} (letzter Zugriff am 30.05.2022).} Sowohl für QuickStatements als auch Open Refine findet sich nach dem Import in den betreffenden Items ein entsprechender Eintrag in der Versionsgeschichte, womit nachvollziehbar ist, wie die Daten in Wikidata gelangt sind (Abbildung \ref{fig:wikidataversions}).

\begin{figure}[h]
    \centering
    \frame{\includegraphics[width=\ScaleIfNeeded]{wikidata-import-versions}}
    \caption[Import-Eintrag in der Versionsgeschichte in Wikidata]{Eintrag des Imports mit QuickStatements/ Open Refine in der Item-Versionsgeschichte.}
    \label{fig:wikidataversions}
\end{figure}

Auch wenn die größere Auswahl beim Datenimport in Wikidata zunächst überfordern kann\footnote{Neben den drei vorgestellten Tools gibt es auch noch die REST-Api von Wikimedia sowie die Möglichkeit der Verwendung von Bots. Auch Wikimedia Cloud Services-Projekte mit weiteren Werkzeugen befinden im Aufbau, URL: \url{https://wikitech.wikimedia.org/wiki/Help:Cloud_Services_introduction} (letzter Zugriff am 30.05.2022).}, ist der Vorteil insgesamt, dass durch diese Vielseitigkeit die Datenerfassung an jeweilige Use Cases und an Nutzungsgewohnheiten optimal angepasst werden kann. Auch vor dem Hintergrund, dass immer mehr historische Quellen selbst digitalisiert vorliegen, was perspektivisch auch teil- und vollautomatisierte Datenimporte ermöglicht, wird zunehmend ein breiteres Angebotsspektrum für die Datenerfassung benötigt.\footnote{Das NFDI-Konsortium nfdi4Culture organisiert Ende Juni einen Workshop, der sich explizit mit der Wikibase-Upload-Pipeline in Open Refine beschäftigt, siehe URL: \url{https://nfdi4culture.de/news-events/events/jcdl-workshop-open-refine-to-wikibase-a-new-data-upload-pipeline.html} (letzter Zugriff am 29.05.2022).} 

\subsection{Verknüpfung von Sample und Fallbeispielen}

Mehrmals wurde in den Interviews betont, dass die rein quantitative Arbeit im Forschungsfeld lediglich einen Teil der Forschung zur Vernichtung der jüdischen Gewerbetätigkeit ausmacht.\footnote{Vgl. Interviews B2\_Transkript, Pos. 53, 63 und B3\_Transkript, Pos. 83.} Den anderen Teil bilden Fallbeispiele, die vor allem zeigen, dass der Prozess der Verfolgung und Vernichtung von zahlreichen Einzelfaktoren abhing und auf der individuellen Ebene daher sehr unterschiedlich verlaufen konnte. Neben diesen Einzelfallstudien gibt es außerdem die Gedenkbücher in analoger oder digitaler Form, die einen stark dokumentarischen Charakter haben, der sich vorwiegend in einem deskriptiven Zusammentragen von verteilten Informationen zu Jüdischen Gewerbebetrieben zeigt.\footnote{Nietzel hebt hier die akribisch recherchierte Textsammlung zu jüdischen Unternehmen in München des Archivars und Historikers Wolfgang Selig aus dem Jahr 2004 hervor, vgl. Nietzel 2009, S. 583.} Hierunter zählen auch jene Veröffentlichungen, die nicht primär auf Daten zu Jüdischen Gewerbebetrieben fokussiert sind, sondern wo diese von anreichernder Bedeutung sind.\footnote{Hier vor allem die zahlreichen Gedenkbücher zu jüdischen Personen, die mittlerweile online zugänglich sind und wo sich Daten zu jüdischen Gewerbebetrieben in den Biogrammen der Personen ,,verstecken''. Siehe zum Beispiel ,,Biografisches Gedenkbuch der Münchner Juden 1933–1945'' der Stadt München, URL: \url{https://gedenkbuch.muenchen.de/} (letzter Zugriff am 12.05.2022). Bei der Biografie von Max Hofman ist unter ,,Weitere Informationen'' vermerkt: ,,Max Hofmann war Inhaber der Fa. Max Hofmann, einem Großhandel und Versand von Manufaktur- und Textilwaren, in der Paul-Heyse-Straße 28/I. Das Gewerbe wurde am 17.10.1938 für den 15.10.1938 abgemeldet.'', URL (stable): \url{https://gedenkbuch.muenchen.de/index.php?id=gedenkbuch_link&gid=5722}.} Diese Datenvielfalt im Forschungsfeld lässt sich wie folgt zusammenfassen: 

\begin{enumerate}
    \item Es gibt \textbf{quantitative (Massen-)Daten}, die strukturiert, entweder als Rohdaten oder in aggregierter Form, vorliegen. Sie besitzen eine statistische Aussagekraft.
    \item Es gibt \textbf{qualitative Daten}, die in der Regel textuell und damit unstrukturiert oder semistruktiert vorliegen.
\end{enumerate}

Bereits Nietzel beklagte in seinem Forschungsbericht aus dem Jahr 2009, dass die qualitativen Daten insbesondere aus der Gedenk- und Erinnerungskultur für eine wissenschaftliche analytische Auswertung bislang zu unsystematisch seien.\footnote{Nietzel 2009, S. 583.} Umgekehrt fehlt den statistischen Massendaten ihres Umfang wegen oft die entsprechende Datentiefe und die Einzelschicksale und -geschichten hinter der Statistik bleiben unsichtbar.\footnote{Allein für Berlin hat die Stichprobe einen Umfang von ca. 8.000 jüdischen Gewerbebetrieben. Auch für Frankfurt am Main sind es in der Stichprobe über 2.500 jüdische Gewerbebetriebe. Vgl. Kreutzmüller 2012, URL: \url{https://www2.hu-berlin.de/djgb/www/find} (letzter Zugriff am 07.05.2022) und Nietzel 2012, S. 15.} Das macht diese Daten vor allem außerhalb der wissenschaftlichen Forschung weniger greif- und nutzbar. 

\begin{figure}[h]
    \centering
    \frame{\includegraphics[scale=0.4]{wikidata-wikipedia}}
    \caption[Wikidata und Wikipedia]{In Wikipedia werden Wikidata-Daten üblicherweise für kompakte Infoboxen genutzt, hier am Beispiel des Artikels zur Berliner Stadtbahn (links). In Wikidata wiederum lässt sich dem Item Berliner Stadtbahn (Q694223) der Wikipedia-Artikel aller Sprachversionen eindeutig zuordnen (rechts).}
    \label{fig:wikidatawikipedia}
\end{figure}

Festzuhalten ist, dass es bisher im Forschungsfeld noch nicht gelungen ist, quantitative und qualitative Forschungsdaten zu verknüpfen. Es ist aber eben diese Verknüpfung von verteilter Datenvielfalt, die im Wiki*versum gängige Praxis ist. Dies wurde bereits anhand der Quellendigitalisate in ,,Wikimedia Commons'' und deren Integration in Wikidata deutlich.\footnote{Vgl. Kapitel 4.3.2 Quellennachweise.} Gleiches lässt sich auch auf der Textebene mit der Enzyklopädie \textit{Wikipedia} realisieren. Analog zu Wikidata-Projekten gibt es in der Wikipedia Themenportale, die sich auf das Schreiben von Wikipedia-Artikeln zu einer bestimmten Thematik spezialisiert haben.\footnote{Portal:Wikipedia nach Themen, URL: \url{https://de.wikipedia.org/wiki/Portal:Wikipedia_nach_Themen} (letzter Zugriff am 30.05.2022).} Unter den Rubriken ,,Geschichte'' oder ,,Wissenschaft'' gibt es inhaltlich dem Forschungsfeld nahestehende Portale wie das ,,Portal:Geschichte des 20. Jahrhunderts''\footnote{URL (stable): \url{https://de.wikipedia.org/w/index.php?title=Portal:Geschichte\_des\_20.\_Jahrhunderts\&oldid=216577544}.} oder ,,Portal:Geschichte''\footnote{URL (stable): \url{https://de.wikipedia.org/w/index.php?title=Portal:Geschichte\&oldid=215435556}.} Es kann jedoch auch ein neues Themenportal angelegt werden. Im Wikipedia-Artikel können die strukturierten Daten aus Wikidata üblicherweise in einer kompakten Infobox hinzugefügt werden, während das zugehörige Wikidata-Item mit dem Wikipedia-Artikel verknüpft wird (Abbildung \ref{fig:wikidatawikipedia}).\footnote{Siehe zur Umsetzung der Verknüpfungen die Dokumentationsseite ,,Wikidata:Wie man Daten in Wikimedia-Projekten nutzt'', URL: \url{https://www.wikidata.org/wiki/Wikidata:How_to_use_data_on_Wikimedia_projects/de} (letzter Zugriff am 30.05.2022).}

Im Rahmen dieser Arbeit liegt der Schwerpunkt auf den strukturierten (Massen)Daten und damit auf Wikidata. Somit bleiben die Möglichkeiten der Verbindung zu Wikipedia hier nur angedeutet. Sie zeigen aber bereits die Potentiale, die sich über Wikidata hinaus im Wiki*versum für das Forschungsfeld ergeben. So können kollaborativ ,,Geschichten'' zu Jüdischen Gewerbebetrieben gesammelt und diese in Wikipedia-Artikeln veröffentlicht werden, die für alle zugänglich und nachnutzbar sind. Gleichzeitig können aus den Artikeln Daten extrahiert, strukturiert in Wikidata erfasst und dort ebenfalls nachgenutzt werden.

\section{Analyse}

\begin{quote}
    ,,Ich hatte da auch bestimmte Ideen, dass man auch so auf städtischen Karten mal einzeichnen könnte, wo die ganzen Unternehmen lagen, wo die sich gehäuft haben. Das fände ich super spannend, aber das ist super viel Arbeit und ich kann das selber gar nicht machen.''\footnote{Interview B2\_Transkript, Pos. 67.}
\end{quote}

Für die Datenanalyse kam im Forschungsfeld einfache deskriptive Statistik zur Anwendung. Es ging zuvorderst darum, die Daten zu Jüdischen Gewerbebetrieben den Forschungsfragen entsprechend zu ordnen und übersichtlich darzustellen. Dies geschah überwiegend in Tabellenform. Nur im Fall von Berlin wurden die Daten auch mit statistischen Schaubildern wie Karten, Balken- und Liniendiagrammen graphisch präsentiert. In dieser aggregierten Form sind sie in den Publikationen der Lokalstudien zugänglich. Welche Werkzeuge für die Datenanalyse in den einzelnen Studien verwendet wurden, ist nicht bekannt. Da aber mit einfachen statistischen Verfahren gearbeitet wurde, sind hierfür mehr oder weniger komplexe Datenabfragen (Queries) ausreichend. 

Wikidata bietet neben der Speicherung von Daten auch deren Abfrage mit dem eigenen ,,Wikidata Query Service'' (WDQS) an.\footnote{Wikidata Query Service, URL: \url{https://query.wikidata.org/} (letzter Zugriff am 30.05.2022).} Dies erfolgt mit der Linked Open Data- und RDF-Abfragesprache \textit{SPARQL} (SPARQL Protocol And RDF Query Language), welche seit 2013 vom ,,World Wide Web Consortium'' (W3C) als offizielle Spezifikation veröffentlicht und folglich zum Standard erklärt wurde.\footnote{Vgl. W3C (2013): SPARQL 1.1 Overview. W3C Recommendation 21 March 2013, URL: \url{http://www.w3.org/TR/2013/REC-sparql11-overview-20130321/} (letzter Zugriff am 30.05.2022).} Ein grundlegender Unterschied zur konventionellen SQL-Datenabfragesprache (Structured Query Language) in relationalen Datenbanken besteht darin, dass mit SPARQL unter der Verwendung von ,,Namespaces'' über Datenquellen hinweg Daten abgefragt werden können, während mit SQL nur auf der eigenen Datenbasis gearbeitet werden kann.\footnote{Der Namensraum von Wikidata lautet <https://www.wikidata.org/>, der von DBpedia <https://dbpedia.org/ontology/>.} Gerade hier liegt eine der Stärken von Linked Open Data und des Semantic Webs, nämlich verteilte Informationen, die im RDF-Format gespeichert sind, zu beschaffen und  weiter zu verarbeiten. 

In der Benutzeroberfläche des WDQS werden die SPARQL-Abfragen geschrieben und können dort direkt ausgeführt werden. Standardmäßig wird das Ergebnis in Tabellenform ausgegeben. Doch hat Wikidata zahlreiche weitere Tools vor allem für die Darstellung und Visualisierung von Daten im Angebot.\footnote{Eine Übersicht über die Werkzeuge für Wikidata siehe Wikidata:Tools, URL: \url{https://www.wikidata.org/wiki/Wikidata:Tools}. Siehe auch Wikidata:SPARQL query service/Wikidata Query Help/Result Views/de, URL: \url{https://www.wikidata.org/wiki/Wikidata:SPARQL_query_service/Wikidata_Query_Help/Result_Views/de} (alle letzter Zugriff am 30.05.2022).} Neben der reinen Präsentation von Daten können sie auch als Methode für eine (visuelle) Datenexploration aufgegriffen werden, die neue Perspektiven auf die Daten eröffnet und mit der schrittweise ein detailliertes Verständnis von den Daten entwickelt werden kann.\footnote{Vgl. H. Degen: Graphische Datenexploration, in: C. Wolf, H. Best (Hrsg.), Handbuch der sozialwissenschaftlichen Datenanalyse, Wiesbaden 2010, S. 91ff., doi:10.1007/978-3-531-92038-2\_5.} 

In den nachfolgenden Kapiteln soll es darum gehen, exemplarisch die Möglichkeiten der graphische Datenexploration in Wikidata für das Forschungsfeld aufzuzeigen, da es hier auch - wie das einleitende Zitat zeigt - Bedarf gibt. Aber auch sich neu ergebende Forschungsfragen sollen antizipiert sowie Datenqualität beurteilt werden. Zu diesem Zweck wurden in Wikidata nachfolgende drei Beispieldatensätze angelegt:

\begin{itemize}
    \item Gorbatschow Liköre F. Kramer \& Co (Q112127138), Berlin\footnote{URL: \url{https://www.wikidata.org/w/index.php?title=Q112127138\&oldid=1651194448}.}
    \item Otto Simon Straus (Q112166241), Mannheim\footnote{URL: \url{https://www.wikidata.org/w/index.php?title=Q112166241\&oldid=1651188294}.}
    \item Franz Mettner GmbH (Q112163392), Mannheim\footnote{URL: \url{https://www.wikidata.org/w/index.php?title=Q112163392\&oldid=1651187976}.}
\end{itemize}

\subsection{Gewerbestruktur}

\paragraph{Branchen} Die Branchenverteilung der Jüdischen Gewerbebetriebe wurde von allen Lokalstudien untersucht, denn damit konnten zum einen Aussagen über deren Anteil und Bedeutung für die lokale Wirtschaft getroffen werden. Zum anderen wurde herausgearbeitet, welche Branchen die Verfolgung und Vernichtung zuerst und besonders stark trafen beziehungsweise ob es Branchen gab, die relativ verschont blieben. SPARQL-Queries zur Branchenverteilung können unter der Voraussetzung erstellt werden, dass einheitliche Branchen und Branchennamen verwendet werden, was aber für die Lokalstudien insgesamt nicht zutrifft. Für die Beispieldatensätze wurden daher die Branchen unter Nachnutzung der ,,Branchensystematikstelle des Pressearchiv 20. Jahrhundert'' vereinheitlicht.\footnote{Siehe Wikidata:WikiProject 20th Century Press Archives, URL: \url{https://www.wikidata.org/w/index.php?title=Wikidata:WikiProject\_20th\_Century\_Press\_Archives\&oldid=1562096427}.} Eine Abfrage über deren Datenobjekte zeigt jedoch, dass eine Reihe von Branchen, welche für die Jüdischen Gewerbebetriebe verwendet werden, nicht vorhanden sind (Abbildung \ref{fig:wikidatasectors}).\footnote{Short-URL der Abfrage: \url{https://w.wiki/5DpB}.} 

\begin{figure}[h]
    \centering
    \frame{\includegraphics[width=\ScaleIfNeeded]{wikidata-sectors-pressearchiv}}
    \caption[Branchensystematik]{Branchensystematik des Pressearchiv 20. Jahrhundert (links) und die (nicht)zuordenbaren Branchen aus Berlin (grün/ rot unterstrichen, rechts).}
    \label{fig:wikidatasectors}
\end{figure}

\begin{figure}[h]
    \centering
    \frame{\includegraphics[width=\ScaleIfNeeded]{wikidata-mulitbar-chart}}
    \caption[Anzahl der Jüdischen Gewerbebetriebe nach Branche und Stadt]{Anzahl der Jüdischen Gewerbebetriebe nach Branche und Stadt; links im Wikidata Query Service, rechts als Template im Wikidata-Projekt.}
    \label{fig:wikidatacharts}
\end{figure}

In diesem Fall kann eine ergänzende Systematik entwickelt und in Wikidata hinzugefügt werden. Hierfür wurde im Wikidata-Projekt ein erster Vorschlag für das Forschungsfeld auf Basis der Branchenliste aus der Berliner Studie unterbreitet.\footnote{URL: \url{https://www.wikidata.org/w/index.php?title=Wikidata_talk:WikiProject\_Destruction\_of\_the\_Economic\_Existence\_of\_the\_Jews\_Research/Vernichtung\_der\_jüdischen_Gewerbetätigkeit\&oldid=1651252043}.} Die Abfrageergebnisse lassen sich direkt im WDQS als Diagramme visualisieren. Es besteht auch die Möglichkeit, das externe ,,Wikidata Visualization''-Tool zu verwenden, welches mehr Auswahl bei der Darstellung hat.\footnote{Wikidata Visualization, URL: \url{https://dataviz.toolforge.org/} (letzter Zugriff am 31.05.2022).} Gibt es eine gemeinsame Branchensystematik für das Forschungsfeld, ließe sich damit erstmals insgesamt und im Städtevergleich die Branchenstruktur untersuchen, was sich zum Beispiel durch ein Multi-Säulendiagramm gut explorieren ließe (Abbildung \ref{fig:wikidatacharts}).\footnote{Siehe hierzu auch die Wikipedia-Dokumentation ,,Graph:Stacked'', URL: \url{https://de.wikipedia.org/w/index.php?title=Vorlage:Graph:Stacked\&oldid=198988739}.}

\paragraph{Verteilung im Stadtraum}

In den Interviews wurde explizit auch die Möglichkeit der topografischen Untersuchung von Jüdischen Gewerbebetrieben erwähnt, um deren Verteilung im Stadtraum und etwaige Ballungszentren zu untersuchen. Hierfür braucht es allerdings die Koordinatenpunkte der Gewerbebetriebe, die dessen topografische Lage eindeutig bestimmen. Diese Daten werden als Geodaten bezeichnet und stellen einen eigenen Datentyp dar.\footnote{Vgl. Wikidata-Property geographische Koordinaten (P625), URL: \url{https://www.wikidata.org/wiki/Property:P625}.} Ohne selbst dafür eine Anwendung aufwändig programmieren zu müssen, wird in Wikidata automatisch der Standort eines Datenobjekts direkt in einem Kartenausschnitt ausgegeben, wenn geographische Koordinatenpunkte als Property hinterlegt sind. Darüber hinaus lassen sich Geodaten mit SPARQL abfragen und auf einer Karte visualisieren. Die größte Hürde in Bezug auf das Forschungsfeld stellt daher nicht die Kartenvisualisierung an sich dar. Es sind die fehlenden geographischen Daten, die bisher von keiner Lokalstudie erfasst wurden und die demzufolge nachträglich ergänzt werden müssten. Erst mit diesen kann eine Verteilung von Jüdischen Gewerbebetrieben im Stadtraum sowie erstmals auch deutschlandweit visuell untersucht werden (Abbildung \ref{fig:wikidatamap}).\footnote{Short-URL zur Abfrage: \url{https://w.wiki/5Dsz}.}

\begin{figure}[h]
    \centering
    \frame{\includegraphics[width=\ScaleIfNeeded]{wikidata-map}}
    \caption{Kartenvisualisierungen in Wikidata}
    \label{fig:wikidatamap}
\end{figure}

\paragraph{Geschäftsfrauen}
Bislang spielte es in den Lokalstudien noch gar keine Rolle, ob es sich bei den Eigentümern von Jüdischen Gewerbebetrieben um Frauen oder Männer handelte. Da nur die Vor- und Nachnamen erfasst wurden, sind geschlechterspezifische Fragestellungen bisher statistisch auch nicht greifbar. Dabei wären mögliche Forschungsfragen, welchen Anteil Frauen am Gewerbeleben hatten, in welchen Branchen sie vorwiegend selbstständig tätig waren und ob sie andere Abwehrstrategien verfolgten als männliche Eigentümer, durchaus spannend. Hierfür ist eine Gender-Angabe bei den Eigentümer*innen notwendig, die in Wikidata als Property schon vorhanden ist und nachgenutzt werden kann. Damit ließen sich perspektivisch Datenabfragen zum Geschlechterverhältnis entwickeln (Abbildung \ref{fig:wikidatagender}). 

\begin{figure}[h]
    \centering
    \frame{\includegraphics[width=\ScaleIfNeeded]{wikidata-sparql-gender}}
    \caption{Geschlechterverhältnis der Eigentümer Jüdischer Gewerbebetriebe}
    \label{fig:wikidatagender}
\end{figure}

\subsection{Vernichtung}

Den größten Teil bei der statistischen Auswertung nahm der Prozess der Vernichtung der jüdischen Gewerbetätigkeit ein. Dieser bestand, wie in Kapitel 3.1 erläutert wurde, wiederum aus den beiden Teilprozessen Besitzübernahme und/ oder Liquidation. Statistisch lässt sich die Prozesshaftigkeit der Vernichtung schwer greifen, daher wurden für die Studien zu Berlin und Frankfurt a.M. signifikante punktuelle Daten als Analyseeinheiten herausgearbeitet, mit denen sich der Prozess annähernd untersuchen ließ. Diese orientieren sich an standardisierte bürokratische Verfahren und sind zusammengefasst:

\begin{itemize}
    \item Datum der gewerblichen Abmeldung.
    \item Datum der Einleitung des Liquidationsvorgangs (durch einbestellten Liquidator oder von Amts wegen).
    \item Datum der Löschung.
\end{itemize}

Oft sind allerdings nur Jahresangaben zu den beiden Prozessen vorhanden, womit nicht eindeutig ist, auf welches Ereignis diese sich beziehen (Abbildung \ref{fig:datavernichtung}).   

\begin{figure}[h]
    \centering
    \frame{\includegraphics[angle=90, scale=0.9]{date-vernichtung}}
    \caption{Jahresangaben zu Liquidationen/ Besitzübernahmen}
    \label{fig:datavernichtung}
\end{figure}

Wikidata hat für diese Situation eine Lösung, denn zum Konzept ,,Zeit/ Datum'' gibt es mehrere Optionen. An diesen orientiert, ist der Vorschlag, die reinen Jahresangaben als Intervall zu interpretieren und folglich die Eigenschaft ,,betroffener Zeitraum'' (P1264) im Datenobjekt als Qualifikator zu verwenden. Dadurch würde die Prozesshaftigkeit von Besitzübernahme und Liquidation deutlicher werden. Sofern es konkrete Ereignisse mit Datum wie oben gibt, können sie mit der Eigenschaft ,,zum Zeitpunkt/ Stand'' (P585) ergänzt werden. Auf diese Weise ließen sich die unterschiedlichen Datumsangaben in den Forschungsdaten vereinheitlichen und deren Aussagegehalt durch Wikidata noch verfeinern. Der Vorteil von vollständigen Datumsangaben ist, dass sich damit Zeitreihen-Analysen in Wikidata durchführen lassen, die bei reinen Jahresangaben verfälscht würden, da hier automatisch der ,,1. Januar'' als Startzeitpunkt gesetzt wird.\footnote{Zeitreihen-Analysen lassen sich direkt im Query Service ausgeben oder mit dem externen Tools wie ,,Wikidata Timeline'' erstellen, URL: \url{https://wikidata-timeline.toolforge.org/} (letzter Zugriff am 01.06.2022).} 


\subsection{Abwehrstrategien}

Christoph Kreutzmüller resümierte 2020 in seinem Forschungsbericht zur Vernichtung der jüdischen Gewerbetätigkeit:

\begin{quote}
    ,,Das in vielen Lokalstudien gezeichnete Bild der sich bis 1937/38 vollziehenden weitgehenden Vernichtung der jüdischen Gewerbetätigkeit ist demzufolge ergänzungsbedürftig. Denn dieser Prozess ist wohl teilweise als eine innerdeutsche Ausweichbewegung und damit als Teil der Behauptungsstrategien jüdischer Gewerbetreibender zu sehen.''\footnote{Christoph Kreutzmüller: Vernichtung der jüdischen Gewerbetätigkeit im Nationalsozialismus. Abläufe, Blickwinkel und Begrifflichkeiten, Version: 2.0, in: Docupedia-Zeitgeschichte, 12.03.2020, S.14, doi:10.14765/zzf.dok-1736.}
\end{quote}

Es zeigen sich hier die Grenzen der als Lokalstudien angelegten Forschung im Forschungsfeld. Deren Erkenntnisse beziehen sich hauptsächlich auf topografisch fest abgesteckte Räume. Diese Räume wurden manchmal vergleichend gegenübergestellt aber überwiegend getrennt voneinander betrachtet.\footnote{Die Historikern Maren Janetzko hat 2012 ein Studie zu einem interregionalen Vergleich in Bayern veröffentlicht. Siehe Janetzko, Die „Arisierung“ Mittelständischer jüdischer Unternehmen in Bayern 1933-1939. Ein interregionaler Vergleich, Ansbach 2012.} Die von Kreutzmüller erwähnten ,,innerdeutschen Ausweichsbewegungen'' bleiben in dieser Herangehensweise unsichtbar. Ein neuer Ansatz im Forschungsfeld wäre zum Beispiel das Forschungskonzept der ,,Translokalität/ Transnationalität'' der Global und International Studies, mit dem sich verstärkt Wechselbeziehungen, Verflechtungen und Netzwerke untersuchen lassen.\footnote{Siehe zu Translokalität Ulrike Freitag: Translokalität als ein Zugang zur Geschichte globaler Verflechtungen, in: Connections. A Journal for Historians and Area Specialists, 10.06.2005, URL: \url{www.connections.clio-online.net/debate/id/diskussionen-632}.} 

Mit Wikidata kann auf der Datenebene eine solche Analyse vorbereitet werden, wie das Beispiel ,,Umzug'' zeigt. Innerstädtische Umzüge sind über die Jüdischen Gewerbebetriebe leicht zu greifen, wenn diese in den Quellen erfasst sind.\footnote{Vgl. Interview B1\_Transkript, Pos. 115.} Da sich außer des Standorts an der bürokratischen Verfasstheit des Betriebs nichts änderte, können alle Adressen dem Jüdischen Gewerbebetrieb zugeordnet werden, wie an der Firma ,,Gorbatschow Liköre F. Kramer \& Co'' exemplarisch zu sehen ist. (Abbildung \ref{fig:wikidataaddress}).

\begin{figure}[h]
    \centering
    \frame{\includegraphics[scale=0.7]{wikidata-address}}
    \caption[Innerstädtische Umzüge]{Zur Firma ,,Gorbatschow Liköre F. Kramer \& Co'' sind zwei Adressen in Berlin erfasst.}
    \label{fig:wiwikidataaddress}
\end{figure}

Bei Umzügen, wo Personen aus der Stadt verzogen sind, muss eine andere Methode gefunden werden, denn in diesem Fall wurden die Betriebe, sofern sie gewerblich gemeldet waren, beim Gewerbeamt am alten Standort ab- und am neuen Standort angemeldet. Während dieses Vorgangs konnte sich auch die Namens- und die Rechtsform ändern. Es sind diese Fälle, die bisher von der Forschung nicht als Ausweichbewegung erkannt wurden, sondern die Abmeldung als Endpunkt der Vernichtung eines Jüdischen Gewerbebetriebs interpretiert wurde. Die Verbindung zwischen den innerdeutschen Standorten von Jüdischen Gewerbebetrieben kann aber über die Eigentümer*innen erschlossen werden. Die Voraussetzung dafür ist, dass sie eindeutig über die Lokalstudien hinweg identifizierbar sind. Wikidata kann an dieser Stelle als eine gemeinsame Normdatei für Personen (Authority File) im Forschungsfeld genutzt werden. Mit diesen Normdaten ist der eindeutige Verweis auf einen Eigentümer möglich, auch wenn unterschiedliche Schreibweisen zum Namen in den Quellen existieren. \\ \\
Abschließend kann für die Datenanalyse in Wikidata festgehalten werden, dass das Hauptinstrument die mächtige Abfragesprache SPARQL ist. Fast alle Analysen lassen sich nur auf Basis von selbst geschriebenen Queries durchführen. Entsprechende visuelle Interfaces sind noch rar und die, die es gibt, lassen bisher nur simple Abfragen zu.\footnote{Siehe zum Beispiel den Abfragegenerator ,,Wikidata Query Builder''. In diesem lassen sich sehr schnell alle Jüdischen Gewerbebetrieben ausgeben ohne SPARQL verwenden zu müssen. Komplexere Anfragen wie zum Beispiel die Branchenverteilung lassen damit allerdings nicht realisieren, URL: \url{https://query.wikidata.org/querybuilder/?uselang=de} (letzter Zugriff am 01.06.2022).} Auch wenn es mit dem Wikidata Query Service eine komfortable Benutzeroberfläche gibt, welche das Schreiben der Queries, deren Ausführung und Visualisierung zusammenführt, muss SPARQL beherrscht werden. Damit ist die Datenanalyse in Wikidata für alle ohne Vorkenntnisse sehr voraussetzungsreich und könnte von keinem der Befragten aus den Interviews ohne Unterstützung umgesetzt werden. Daneben reicht die Qualität der Forschungsdaten für bestimmte Analyseszenarien bisher noch nicht aus. Hier bedarf es in erster Linie einer Datennachbearbeitung.

Die Empfehlung ist daher, da sich bestimmte Analysen für jede Lokalstudie wiederholen, entsprechende SPARQL-Abfragen im Wikidata-Projekt zu sammeln, sodass sie kollaborativ angepasst und nachgenutzt werden können. Dadurch müsste nicht jedes Projekt immer wieder neu die Abfragen entwickeln. Darüber hinaus ließe sich auf diese Weise auch kontrollieren, dass im Forschungsfeld identische Queries verwendet und weiterentwickelt werden, was das Risiko der Datenverfälschung durch fehlerhafte Queries minimiert. Nichtsdestotrotz wäre es wünschenswert, wenn Wikidata in Zukunft gleichwertige Alternativen zum Query Service im Angebot hätte.
  
\section{Veröffentlichung und Nachnutzung}

\begin{quote}
    ,,Ja, das wichtigste ist natürlich, dass sie [die Forschungsdaten, Anm. S.E.] nicht zu einem bestimmten Zweck nur gedacht sind, also sozusagen von einem Projekt, und sie eigentlich nur für das gleiche Projekt nochmal benutzt werden können, sondern dass sie halt offen sind für alle möglichen Anwendungen. Das wäre eigentlich das Beste. Also auch für alle möglich Fragestellungen, die man noch gar nicht antizipiert hat, als man die Daten selber gesammelt hat.''\footnote{Interview B2\_Transkript, Pos. 51.}
\end{quote}

Alle Daten in Wikidata sind in der offenen Creative Common-Lizenz CCO, also in Public Domain, veröffentlicht. Der Hinweis ist unter jedem Datenobjekt dokumentiert und die Nutzungsbedingungen damit eindeutig. Bei dieser gibt es keinerlei Einschränkung. Es ist folglich jede Nutzung ohne Namensnennung erlaubt.

Diese Verwendung ist insbesondere im akademischen Bereich fremd, wo Zitierhäufigkeiten (mit Namensnennung) als Indikatoren wissenschaftlicher Reputation gelten. Hierfür wäre eine offene CC-BY-SA-Lizenz also geeigneter, die sich aber in Wikidata individuell nicht umsetzen lässt. Zwar ist das offene Forschungsdatenmanagement so konzipiert, dass die Forschungsdaten mit den Forschungsprojekten und -studien verknüpft sind, aber es ist fraglich, ob diese semantische Anreicherung akzeptiert wird. Andererseits ist zumindest für das Forschungsfeld zu konstatieren, dass bisher im Zusammenhang mit den Forschungsdaten mehrheitlich noch gar keine Reputation verbunden ist, da diese schlichtweg überhaupt nicht zur Verfügung stehen.\footnote{Ausnahme sind die Forschungsdaten aus Berlin, die auf einer Website veröffentlicht sind, allerdings ohne Lizenzangabe, womit die Nutzungsbedingungen nicht klar sind.}

Bei genauer Betrachtung kann die offene Lizenz für das Forschungsfeld einen Mehrwert darstellen. Sie ermöglicht, dass die Breitenerschließung der empirischen Studien nachträglich durch eine Tiefenerschließung ergänzt werden kann. Da es sich um Stichproben handelt, sind zudem längst noch nicht alle Jüdischen Gewerbebetriebe vor allem zu den größeren Städten erfasst. Das heißt, dass sich akademische Wissenschaft, zivilgesellschaftliche Initiativen aus der Erinnerungs- und Gedenkkultur sowie Amateurforscher*innen in Wikidata ohne Zugangshürden zusammenschließen können, um gemeinsam fehlende Informationen zu Jüdischen Gewerbebetrieben zusammenzutragen. Mit Wikidata kann die Basis dafür geschaffen werden, sukzessive zum einen vorhandene Forschungsdaten zu vervollständigen und nachzubearbeiten, wie das vorherige Kapitel gezeigt hat. Zum anderen können neue Daten zu noch fehlenden Orten aufgenommen werden. Das Wikidata-Projekt ,,Destruction of the Economic Existence of the Jews Research/Vernichtung der jüdischen Gewerbetätigkeit'' kann hierfür ein Leitfaden auch für den außerwissenschaftlichen Bereich sein. Sind zudem die Stichprobendesigns der einzelnen Forschungsstudien untereinander bekannt, können Methoden übertragen werden. Dies kann an zwei Beispiel veranschaulicht werden: Zum Jüdischen Gewerbebetrieb ,,Rosenbaum \& Kahn'' aus Mannheim gibt es bislang mit ,,Liquidation: vermutlich Februar 1937'' keine gesicherte Angabe zu dessen Verschwindens. Vermutlich haben die in Mannheim verwendeten Quellen ein Datum nicht hergegeben. Da bei der Branche ,,Herrenkleiderfabrik'' von einem größeren Betrieb ausgegangen werden kann, ist eine Eintragung im Handelsregister wahrscheinlich. Dementsprechend sollte die Firma in der publizierten ZHRB recherchierbar sein. Tatsächlich konnte im Rahmen dieser Arbeit ein Eintrag ,,Rosenbaum \& Kahn, Mannheim'' in der 6. Ausgabe vom 8. Januar 1938 gefunden werden (Abbildung \ref{fig:zhrbmannheim}).\footnote{Deutscher Reichs-Anzeiger und Königlich Preußischer Staats-Anzeiger / Deutscher Reichsanzeiger und Preußischer Staatsanzeiger 1938, Nr. 6 vom 8. Januar 1938, S. 8, URL: \url{https://digi.bib.uni-mannheim.de/viewer/reichsanzeiger/film/021-8462/0067.jp2} (letzter Zugriff am 01.06.2022).}

\begin{figure}[h]
    \centering
    \frame{\includegraphics[width=\ScaleIfNeeded]{zhrb-mannheim}}
    \caption[ZHRB-Eintrag zu ,,Rosenbaum \& Kahn, Mannheim'']{Eintrag zu ,,Rosenbaum \& Kahn, Mannheim'' in der ZHRB vom 08.01.1938.}
    \label{fig:zhrbmannheim}
\end{figure}

Dort findet sich der Hinweis, dass die Firma gelöscht wurde und zusätzlich, dass eine Frau Charlotte Rosenbaum Prokura hatte. Da Mannheim keine mit Berlin vergleichbare Metropole ist, ist die Wahrscheinlichkeit hoch, dass es sich um dieselbe Firma handelt. Definitiv gibt es aber neue Informationen (Charlotte Rosenbaum, Löschung 08.01.1938), die neue Anhaltspunkte für weitere Recherchen liefern und die im Falle eines positiven Abgleichs überprüfbar erfasst werden könnten. 

Im Fall des Beispieldatensatzes ,,Franz Mettner GmbH'' aus Mannheim findet sich eine Firma unter dem selben Namen auch in der Berliner Datenbank. Dort ist intern in einem Freitextfeld notiert, dass der Hauptsitz im Jahr 1936 aus Gelsenkirchen nach Berlin verlegt worden ist, es sich um ein Herrenkonfektionsgeschäft handelte und dieses 1939 aus dem Handelsregister gelöscht wurde. In der Excel-Liste mit Jüdischen Gewerbebetrieben aus Mannheim ist bei der Firma ein Einzelhandelsgeschäft sowie die Liquidation im Sommer 1939 erfasst. Eine Verbindung zwischen beiden Firmen wurde bisher nicht hergestellt, aber die Vermutung liegt nahe, dass es sich in Mannheim um eine Zweigniederlassung gehandelt haben könnte. Diese wird von einer im Rahmen dieser Arbeit vorgenommenen Recherche gestützt. Da seit 2017 auf den digitalisierten Ausgaben des Reichsanzeigers/ ZHRB eine Volltextsuche möglich ist, ließen sich alle zwischen 1819 und 1945 nach der Firma ohne größeren Aufwand durchsuchen.\footnote{Siehe Wiki auf GitHub, URL: \url{https://github.com/UB-Mannheim/Reichsanzeiger/wiki\#2017-08-02} (letzter Zugriff am 02.06.2022).} Der früheste Eintrag zur Firma mit Bezug zu Mannheim konnte in der Ausgabe vom 29. November 1929 gefunden werden: 

\begin{quote}
    ,,Franz Mettner Gesellshaft mit be= \\
    schränkter Haftung Zweigniederlassung \\
    Mannheim in Mannheim als Zweig- \\
    niederlassung der Firma Franz Mettner \\
    Gesellschaft mit beschränkter Haftung in \\
    Gelsenkirchen: Gustav Kaatz ist nicht mehr Geschäftsführer.''\footnote{Deutscher Reichs-Anzeiger und Königlich Preußischer Staats-Anzeiger / Deutscher Reichsanzeiger und Preußischer Staatsanzeiger 1929, Nr. 256 vom 1. November 1929, S. 8, URL: \url{https://digi.bib.uni-mannheim.de/viewer/reichsanzeiger/film/076-9036/0095.jp2} (letzter Zugriff am 02.06.2022).}
\end{quote}

Der letzte diesbezügliche Eintrag ist in den Ausgaben vom 2. sowie vom 23. November 1936 unter Mannheim beziehungsweise Gelsenkirchen zu finden. Dort ist jeweils notiert, dass der Hauptsitz der Firma nach Berlin verlegt wurde und daraufhin alle Zweigniederlassung, darunter Mannheim, aufgehoben worden sind.\footnote{Vgl. Ebd. 1936, Nr. 273 vom 23. November 1936, S. 4, URL: \url{https://digi.bib.uni-mannheim.de/viewer/reichsanzeiger/film/017-8458/0200.jp2} und Nr. 256 vom 2. November 1936, S. 2, URL: \url{https://digi.bib.uni-mannheim.de/viewer/reichsanzeiger/film/017-8458/0010.jp2} (alle letzter Zugriff am 02.06.2022).} Die Wahrscheinlichkeit, dass es zwischen dem Mannheimer und Berliner Gewerbebetrieb eine Verbindung gab, ist auch hier hoch. Dennoch ergäbe erst ein Abgleich mit den historischen Quellen aus Mannheim hundertprozentige Sicherheit, die jedoch zumindest aus der Excel-Liste heraus nicht bekannt sind und damit eine weitere Recherche nicht möglich war. Auffällig ist, dass es Einträge in der ZHRB zur Firma Franz Mettner GmbH in Mannheim nach 1936 nicht mehr gegeben hat. Lediglich für Berlin ist im Jahr 1939 eingetragen, dass die Firma erloschen ist.\footnote{Ebd. 1939, Nr. 244 vom 18. Oktober 1939, S. 1.} Im Falle eines positiven Abgleich, stellt sich insbesondere die Frage, in welcher Form die Zweigniederlassung in Mannheim nach 1936 weitergeführt wurde und wer bis zur Liquidation 1939 Eigentümer*in des Betriebs gewesen war. Wie das bestehende Datenobjekt zur Firma in Wikidata um die Vielfalt an neuen Informationen nachträglich angereichert werden könnte, ist in Abbildung \ref{fig:wikidatamettner} veranschaulicht. 

\begin{figure}[h]
    \centering
    \frame{\includegraphics[scale=0.6]{wikidata-mettner}}
    \caption[Nachträgliche Datenanreicherung in Wikidata]{Nachträgliche Datenanreicherung zum Sitz der Firma ,,Franz Mettner GmbH'' in Mannheim in Wikidata.}
    \label{fig:wikidatamettner}
\end{figure}

An beiden Beispielen ist zu sehen, dass sich die Methodik, welche für Berlin angewandt wurde, in der Nachnutzung auch für andere Städte zur Tiefenerschließung eignet. Vor allem wenn die Arbeit überwiegend manuell - wie bisher im Forschungsfeld - erfolgt, kann die Zusammenarbeit der beteiligten Stakeholder einen grenzerweiternden Effekt auf die Erfassung Jüdischer Gewerbebetriebe haben. Im zweiten Beispiel deutet sich zudem an, dass sich die ZHRB als Quelle dafür eignen könnte, Querverbindungen und Beziehungen zwischen Jüdischen Gewerbebetrieben zu untersuchen. Dafür müssen allerdings, wie deutlich geworden ist, die Daten der Studien verfügbar sein.

Neben der Datennachnutzung innerhalb von Wikidata können die Forschungsdaten auch außerhalb der Wissensdatenbank weiter verwendet werden. Ergebnisse von jeder belieben SPARQL-Abfrage lassen sich in diversen offenen Formaten wie HTML, CSV/TSV, JSON und SVG exportieren (Abbildung \ref{fig:wikidataexport}). Außerdem werden Code-Snippets in verschiedenen Programmiersprachen zur Verfügung gestellt, die für API-Abfragen eingebunden werden können. Auch Kurz-URL's sowie die Einbettung in externe Websites mittels HTML-iframe werden angeboten. Damit gibt es eine Palette an Exportfunktionen, die diverse Nachnutzungsszenarien ermöglichen.

\begin{figure}[h]
    \centering
    \frame{\includegraphics[width=\ScaleIfNeeded]{wikidata-export}}
    \caption{Exportfunktionen in Wikidata}
    \label{fig:wikidataexport}
\end{figure}

Für das Forschungsfeld wäre insgesamt eine Nachnutzung der Forschungsdaten analog zum vorgestellten Projekt „Archivführer. Deutsche Kolonialgeschichte'' vorstellbar. Es könnte demzufolge ein gemeinsames Portal zur Datenpräsentation aufgebaut werden, das die Daten zu den Jüdischen Gewerbebetrieben über SPARQL-Endpoints von Wikidata bezieht. Eine eingebaute Suchfunktion könnte die Recherche in den Daten erleichtern und würde die Zugänglichkeit gegenüber Wikidata verbessern, wo der Schwerpunkt auf SPARQL-Abfragen liegt. Auch im Kontext der Erinnerungs- und Gedenkkultur wären Nachnutzungen denkbar. So lassen sich die Daten in digitale historische Stadtspaziergänge oder in Gedenkbücher zu verfolgten Personen einspeisen. Ferner werden in der offenen Lizenz die Forschungsdaten zu Jüdischen Gewerbebetrieben für zukünftige Fragestellungen bereitgestellt, die bisher noch gar nicht antizipiert wurden.

\section{Archivierung}

\begin{quote}
    ,,Das ist doch Wissenschaftlichkeit, dass man begründete Behauptungen aufstellt und die begründet man durch nachvollziehbares Wissen.''\footnote{Interview B1\_Transkript, Pos. 133.}
\end{quote}

Die offene Lizenz in Wikidata hat im wissenschaftlichen Kontext bei allem Mehrwert nichtsdestotrotz einen Nachteil. Dass Daten zu jeder Zeit und von jeder/ jedem Nutzer*in editiert werden können, heißt auch, dass diese sich permanent verändern. Es stellt sich hier insbesondere für die empirischen Forschungsstudien die Frage, wie sich aus den unterschiedlichen Veränderunghistorien der einzelnen Datenobjekte zu Jüdischen Gewerbebetrieben das ursprüngliche Samples (wieder)herstellen lässt, auf die sich die Erkenntnisse stützen. Wie zu Beginn dieser Arbeit ausgeführt wurde, ist gerade Nachvollziehbarkeit eines der wichtigsten Kriterien von Open Science. Zwar besitzt Wikidata ein internes Versionskontrollsystem, mit dem sich jede Änderung auf Datenebene nachvollziehen lässt\footnote{Siehe zum Beispiel die Versionsgeschichte „Gorbatschow Liköre F. Kramer \& Co“ (Q112127138), URL: \url{https://www.wikidata.org/w/index.php?title=Q112127138&action=history} (letzter Zugriff am 02.06.2022).}, aber anders als in öffentlichen Diensten zur expliziten Versionsverwaltung wie GitHub oder GitLab können die Versionen von einzelnen Nutzer*innen nicht gesteuert werden, sondern laufen vollautomatisch im Hintergrund. Versionen mit Tags oder Commit-Nachrichten zu versehen, ist nicht möglich. Diese strikte Handhabung der Versionierung scheint vor dem Hintergrund, Kompromittierung von öffentlichen Daten zu verhindern, wichtig und ein verlässliches Verfahren. Allerdings ist damit im Fall der empirischen Forschungsstudien die eindeutige Kennung des ursprünglichen Samples ausgeschlossen. Hinzu kommt, dass einzelne Versionen mit SPARQL (noch) nicht abgefragt werden können. Kurz gesagt wäre eine Replikation der Forschungsergebnisse in Wikidata wohl nicht möglich, da deren Archivierung nicht ausreichend sichergestellt ist. Die Frage, inwiefern Wikidata sich hier in der Zukunft funktional noch stärker nach der Überprüfbarkeit von wissenschaftlichen Erkenntnissen ausrichten wird, muss in dieser Arbeit offen bleiben. Festzuhalten ist, dass in dieser Frage derzeit das primäre Partizipationsziel von Wikidata noch mit wissenschaftlicher Integrität kollidiert. Die Empfehlung ist daher, in diesem Fall zu einer anderen Lösung zu greifen und parallel zu Wikidata die Forschungsdaten, welche Ausgangspunkt von Forschungsergebnissen sind, in einem wissenschaftlichen Repositorium zu archivieren, das öffentlich zugänglich ist. Hierfür käme das in Kapitel 2.1.3 vorgestellte Repositorium \textit{Zenodo} in Frage. Die Daten sollten in einem offenen Format wie CSV oder JSON hochgeladen werden. Naheliegend ist, an dieser Stelle, die in Kapitel 4.2.2 erwähnte Dokumentation der Erhebungsmethode in Form einer ReadMe-Datei mit hochzuladen (Abbildung \ref{fig:zenodo}).

Zenodo vergibt automatisch für jeden Upload einen \textit{Digital Object Identifier} (DOI).\footnote{Siehe zum Beispiel Sophie Eckenstaler, Christoph Kreutzmüller (2022): Wikidata-Datenobjekt Jüdischer Gewerbebetrieb "Gorbatschow Liköre F. Kramer \& Co". (1.0.0) [Data set]. Zenodo. doi:10.5281/zenodo.6607805.} Diese eindeutige Kennung kann verwendet werden, um das Zenodo-Archiv wiederum mit dem zugehörigen Forschungsprojekt in Wikidata zentral zu verknüpfen.\footnote{Gleiches Szenario ist auch mit dem \textit{Open Science Framework} realisierbar, das ebenfalls DOI's vergibt.}


\begin{figure}[h]
    \centering
    \frame{\includegraphics[width=\ScaleIfNeeded]{zenodo}}
    \caption[Archivierung in Zenodo]{Archivierung eines Wikidata-Datenobjekts mit Dokumentation (ReadMe) in Zenodo.}
    \label{fig:zenodo}
\end{figure}