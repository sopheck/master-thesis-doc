\onehalfspacing

\section{Wikidata:WikiProject Destruction of the Economic Existence of the Jews Research}
bildet Grundlage

\begin{itemize}
    \item Was ist Untersuchungsgegenstand?
    \item Welche Ergebungsmethode wurde verwendet?
    \item Wie wurden die Daten erfasst?
    \item Was für Daten wurden erfasst?
    \item Wie wurden die Daten ausgewertet?
    \item In welcher Form sind die Daten veröffentlicht?
    \item Stehen die Daten (langfristig) zur Verfügung?
    \item In welcher Form können die Daten nachgenutzt werden?
    \end{itemize}
    
    Auffällig ist sofort, dass die Studien über einen Zeitraum von 20 Jahren verteilt entstanden sind. Es ist klar, dass sich die Forschungsarbeit mit zunehmender Digitalisierung gerade innerhalb dieser Zeit enorm gewandelt hat. Auch wenn die Kriterien für gute wissenschaftliche Praxis im digitalen Zeitalter natürlich für alle Forschungsdaten gleich gelten, so ist zu berücksichtigen, dass die technischen Möglichkeiten der digitalen Datenverarbeitung im Jahr 1997 begrenzter als im Jahr 2015 waren. Insbesondere für die frühen Lokalstudien (Nr. 1 und 2) stellt sich die Frage, ob hier überhaupt schon Daten in digitaler Form generiert wurden.  
    
    \begin{table}
    \caption{Lokalstudien mit statistischem Methodenteil}
    \label{tab:lokalstudientabelle}
    \begin{tabular} { L{0.5cm}|L{8cm}|L{2.5cm}|L{1cm}|L{1cm} }
    Nr. & Titel & Ort & Jahr & Sample \\
    \hline 
    1 & Arisierung in Hamburg. Die Verdrängung der jüdischen Unternehmer 1933-1945 & Hamburg & 1997 & 300 \\
    \hline 
    2 & „Arisierungen“ in München. Die Verdrängung der jüdischen Gewerbetreibenden aus dem Wirtschaftsleben der Stadt 1933-1939 & München & 2000 & 720 \\
    \hline 
    5 & Ausverkauf. Die Vernichtung der jüdischen Gewerbetätigkeit in Berlin 1930-1945 & Berlin & 2012 & 8.012 \\
    \hline
    6 & Handeln und Überleben. Jüdische Unternehmer aus Frankfurt am Main 1924-1964 & Frankfurt am Main & 2012 & 2.600 \\
    \hline
    7 & Die „Arisierung“ Mittelständischer jüdischer Unternehmen in Bayern 1933-1939. Ein interregionaler Vergleich & Mittelfranken & 2012 & 529 \\
    \hline
    8 & Ausgeplündert, zurückerstattet und entschädigt. Arisierung und Wiedergutmachung in Mannheim & Mannheim & 2013 & 1.234 \\
    \hline
    9 & „… doch nicht bei uns in Krefeld!“ Arisierung, Enteignung, Wiedergutmachung in der Samt- und Seidenstadt 1933 bis 1963 & Krefeld & 2015 & 135 \\
    \end{tabular}
    \end{table}

Strukturiert an einen idealtypischen Forschungsprozess. Nicht alle möglichen Anwendungsfälle abgedeckt werden. Aber Abdeckung gesamten Forschungsdatenlebenszyklus sicher stellen

\section{Metadaten: Die Forschungsprojekte als Wikidata-Items}
Metadatenschema
\section{Exkurs: Erhebungsmethoden}

\paragraph{Erhebungsmethode} Im Forschungsfeld lassen sich grob drei Datenquellen gruppieren:

\begin{enumerate}
    \item Gedruckte Verzeichnisse und Listen sowie Karteisammlungen, in denen Gewerbebetriebe dezidiert als jüdisch markiert und veröffentlicht wurden\footnote{In München übernahm diese Aufgabe das städtische Gewerbeamt, vgl. Rappl 2000, S. 145f. In Frankfurt am Main war der zentrale Akteur die Industrie- und Handelskammer.} Sie enthalten die wesentlichen Grunddaten der Gewerbebetriebe wie Name, Inhaber oder Branche.
    \item Verschiedene zeitgenössische Aktenbestände, die den Vorgang der Verfolgung verwaltungsseitig dokumentieren. 
    \item Eine wichtige Quelle im Forschungsfeld stellen die Wiedergutmachungsakten nach 1945 dar, welche seit den 90er Jahren der historischen Forschung zugänglich sind und oft eine Ersatzüberlieferung für die vernichteten und zerstörten zeitgenössischen Quellen darstellen.     
\end{enumerate}

Zu den ersten beiden Datenquellen ist generell festzustellen, dass die Überlieferung als disparat und lückenhaft bezeichnet wurde, da viele Bestände teilweise oder überwiegend von den Nationaloszialisten vernichtet wurden, um Spuren zu verwischen, oder in den letzten Kriegstagen unwiederbringlich zerstört wurden. Oft sind nur Überreste und Splitter erhalten, was die Datenerhebung der Studien maßgeblich beeinflusste. Hierbei lassen sich zwei Methoden unterscheiden:

\begin{enumerate}
    \item Datenquelle 1 ist überliefert und bildet den Ausgangspunkt, mit der ein Sample von jüdischen Gewerbebetrieben erstellt wurde. Dieses wurde anschließend mit verschieden Quellen aus der 2. Gruppe abgeglichen und um weitere relevante Daten ergänzt. Dieser Ansatz wurde von den meisten Studien umgesetzt.  
    \item Datenquelle 1 ist nicht überlieferrt 
\end{enumerate}
 


Generell verlief die Erhebung im Forschungsfeld zweischrittig. Im ersten Schritt ging es darum, das Sample mit jüdischen Gewerbebetrieben zu erstellen. Ausgehend davon wurden im zweiten Schritt weitere Daten erfasst, die zur Klärung der Forschungsfrage benötigt wurden. Diese zweischrittige Vorgehensweise ist auf die Quellenlage im Forschungsfeld zurückzuführen, wobei sich hier zwei Methoden herauskristallisieren:

Drei Quellenarten abhängig: 

\begin{enumerate}
    \item 
    
    Wiedergutmachungsakten
    
    Die Überlieferung zur Vernichtung der jüdischen Gewerbetätigkeit ist folglich sehr lückenhaft, was für alle Studien eine Herausforderung bei der Datenerhebung darstellte. Mehrheitlich musste eine Vielzahl an Quellenarten und Aktensplitter erschlossen werden, um eine einigermaßen repräsentative Stichprobe zu erhalten. Folglich sind insbesondere wegen der Quellensituation der quantitativen Forschung im Forschungsfeld klar Grenzen gesetzt. 
\end{enumerate}





, die im Forschungsfeld als sehr disparat bezeichnet wurde. Zentrale Bestände wurden entweder von den Nationalsozialisten gezielt vernichtet, um Spuren zu verwischen, oder in den letzten Kriegstagen unwiederbringlich zerstört. Daneben ist die Erhebungsmethode auch von der jeweiligen Forschungsfrage und dem Erkenntnisinteresse ab. 


Während bei der Identifizierung jüdischer Gewerbebetriebe erhaltene Verzeichnisse oder Listen , mussten zur Untersuchung des Vorgangs mehrere Datenquellen erschlossen werden.





mittelständische Unternehmen
Jüdische Unternehmen von der Verfolgung aus betrachtet erfassen, das hießt


zerstört wurden Die Besonderheit im Forschungsfeld ist zudem




, was insbesondere die Ausgangslage für die Datenerhebung vor eine Herausforderung stellte. 
Es lassen sich für den statistischen Untersuchungsteil zwei Forschungsdesigns unterscheiden.

\begin{table}
    \caption{Forschungsdesigns der Lokalstudien}
    \label{tab:lokalstudientabelle}
    \begin{tabular} { L{0.5cm}|L{8cm}|L{2.5cm}|L{1cm}|L{1cm} }
    Nr. & Ort & Foschungsschwerpunkt & Untersuchungszeitraum & Sample \\
    \hline 
    1 & Arisierung in Hamburg. Die Verdrängung der jüdischen Unternehmer 1933-1945 & Hamburg & 1997 & 300 \\
    \hline 
    2 & „Arisierungen“ in München. Die Verdrängung der jüdischen Gewerbetreibenden aus dem Wirtschaftsleben der Stadt 1933-1939 & München & 2000 & 720 \\
    \hline 
    5 & Ausverkauf. Die Vernichtung der jüdischen Gewerbetätigkeit in Berlin 1930-1945 & Berlin & 2012 & 8.012 \\
    \hline
    6 & Handeln und Überleben. Jüdische Unternehmer aus Frankfurt am Main 1924-1964 & Frankfurt am Main & 2012 & 2.600 \\
    \hline
    7 & Die „Arisierung“ Mittelständischer jüdischer Unternehmen in Bayern 1933-1939. Ein interregionaler Vergleich & Mittelfranken & 2012 & 529 \\
    \hline
    8 & Ausgeplündert, zurückerstattet und entschädigt. Arisierung und Wiedergutmachung in Mannheim & Mannheim & 2013 & 1.234 \\
    \hline
    9 & „… doch nicht bei uns in Krefeld!“ Arisierung, Enteignung, Wiedergutmachung in der Samt- und Seidenstadt 1933 bis 1963 & Krefeld & 2015 & 135 \\
    \end{tabular}
\end{table}
\section{Modellierung mit den Wikidata-Entities}

\paragraph{Untersuchungsgegenstand}

Als Untersuchungsgegenstand für die statistische Auswertung wird in allen neun Lokalstudien ,,Jüdische Gewerbebetriebe'' oder ,,Jüdische Unternehmen'' genannt. Hieraus ergibt sich eine grundlegende methodische Schwierigkeit im Forschungsfeld. Da die Zugehörigkeit zu einer Konfession bei einem Gewerbebetrieb oder Unternehmen generell keine Rolle spielt, ist schon der Begiff ,,jüdischer Gewerbebetrieb'' unlogisch und ohne Kontext unbrauchbar. Dies wird auch in fast allen Studien reflektiert und klar gestellt, dass es sich um eine antisemitische Zuschreibung und Konstruktion handelte. Diese Kennzeichnung und Diffamierung bildete den Ausgangspunkt für alle weiteren Verfolgungspraktiken. Zur einfacheren Handhabung wurde der Begriff als Quellenbegriff jedoch von allen Studien beibehalten. Hierbei fallen zwei unterschiedliche Verwendungen auf: 

\begin{enumerate}
    \item Der Begriff ,,jüdischer Gewerbebetrieb'' wird ausschließlich auf die jüdischen Besitzer*innnen bezogen und angewandt. Damit wird jedoch das methodische Problem nicht wirklich aufgelöst, sondern verlagert sich nur auf den Begriff ,,jüdische Person'' oder ,,Jude'', bei dem es sich im nationalsozialistischen Kontext ebenfalls um eine rassistische Zuschreibung handelte und nichts mit dem Selbstverständnis der Betroffenen zu tun hatte.\footnote{Das wird in der Studie zu Hamburg auch ausführlicher reflektiert. Vgl. Bajohr 1997, S. 9.} Darüber hinaus werden in dieser Verwendung systematisch Gewerbebetriebe vernachlässigt, deren Besitzer zum Beispiel nichtjüdisch waren, die aber einen hohen Anteil jüdischer Mitarbeiter*innen aufwiesen und daher verfolgt wurden. 
    \item Der Begriff ,,jüdischer Gewerbebetrieb'' wird mit ,,als jüdisch betrachtet/ verfolgt'' gleichgesetzt. Mit dieser Verwendung ist die jüdische Eigentümerschaft eines Gewerbebetriebs zunächst unerheblich, das heißt sie wird nicht vorausgesetzt, sondern es werden alle Gewerbebetriebe gezählt, die im nationalsozialistischen Kontext diffamiert wurden. Damit wird einerseits der Konstruktioncharakter des Begriff hervorgehoben und andererseits dem Umstand Rechnung getragen, dass die rassistischen Zuschreibungen grundsätzlich jeglicher rationalen Begründung entbehrten und aus diesem Grund willkürlich erfolgen konnten. Zudem konnten auch unterschiedliche Verfolgungskontexte erfasst werden, die in der ersten Verwendung ausgeschlossen blieben.
\end{enumerate}

Auch wenn in allen Studien der selbe Untersuchungsgegenstand genannt wird, so zeigt sich erst in der konkreten Verwendung, dass dieser unterschiedlich interpretiert wurde, was jedoch so im Forschungsfeld noch nicht diskutiert wurde. Maßgeblich liegt das daran, dass der Begriff an sich nicht widerspruchsfrei ist. Aus forschungsethischer Perspektive ist es zudem problematisch, dass ein rassistisch konnotierter Begriff in der wissenschaftlichen Forschung beibehalten wird. Umso wichtiger ist eine krititsche (Selbst)Reflexion in der eigenen Forschungsarbeit. Für das Forschungsdatenmanagement wird versucht, den Zuschreibungs- und Konstruktionscharakter abzubilden und auf diese Weise den Begriff ,,jüdischer Gewerbebetrieb'' zu vermeiden. Dafür scheint die Verwendung ,,als jüdisch betrachtet'' ein geeigneter Ansatz zu sein.
Formale Beschreibung jüdischer Gewerbebetriebe Datenmodell, 
EntitySchema items, properties, qualifiers und references
\section{Erfassung im ,,Linked Data Interface''}

\section{Wikidata-Schnittstellen zur Datennutzung}
Datenanalyse und -visualisierung
Abfrage, Auswertung, Visualisierung, Zitation, Nachnutzung

Sparql query Service
Maps
stable URI's
Sparql Endpoints
