\onehalfspacing

Use Case driven

\section{Lösungskonzept}
\subsection{Wikidata und Wikibase als offener Forschungsdatenmanagement-Service}
setzt alle FAIR-Prinzipien um
eigene Wikibase-Instanzen aufsetzen --> sehr aufwändig und Informatik-Kenntnisse, wäre von befragten Historiker*innen nicht umgesetzbar gewesen
ähnliche Infrastrukturen nicht gibt, direkt in Wikidata gearbeitet werden (das was derzeit zur Verfügung steht)
Gleichzeitig in dieser Arbeit: von größtmöglichem Open Tech Stack ausgehen, Einschränungen nach unten offen halten, aber Devise Open Science radikal umgesetzt werden, soll am Ende auch Drawbacks dieser prototypischen Umsetzung diskutiert werden, nicht in Stein gemeißelt ggf. nachjustieren, mutig offeneren Lösungen entgegentreten

Beispiele aus der historischen Forschung:
https://archivfuehrer-kolonialzeit.de/
https://blog.ehri-project.eu/2018/02/12/using-wikidata/


\subsection{Wikidata:WikiProject Destruction of the Economic Existence of the Jews Research}
bildet Grundlage

Sichtbarkeit von Daten, Datenkonsistenz und -integrität in Wikidata

Erklären, warum nicht eigene Wikibase-Instanz

\section{Implementierung}

Strukturiert an einen idealtypischen Forschungsprozess. Nicht alle möglichen Anwendungsfälle abgedeckt werden. Aber Abdeckung gesamten Forschungsdatenlebenszyklus sicher stellen

\subsection{Berliner Forschungsprojekt als Wikidata-Datensatz}

\subsection{Datenmodel zur Beschreibung jüdischer Gewerbebetriebe}

\subsection{Erfassung jüdischer Gewerbebetriebe mit dem Wikidata-Interface}

\subsection{Exkurs: Datenkonsistenz und -integrität in Wikidata}

\subsection{Möglichkeiten der Datenanalyse und -visualisierung in Wikidata}

\subsection{Wikidata-Schnittstellen zur Daten(nach)nutzung}

Abfrage, Auswertung, Visualisierung, Zitation, Nachnutzung

Sparql query Service
Maps
stable URI's
Sparql Endpoints

\section{Ergebnisse}

Drawbacks