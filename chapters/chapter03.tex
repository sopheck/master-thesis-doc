\onehalfspacing

Use Case driven

\section{Lösungskonzept}
\subsection{Wikidata und Wikibase als offener Forschungsdatenmanagement-Service}
setzt alle FAIR-Prinzipien um
eigene Wikibase-Instanzen aufsetzen --> sehr aufwändig und Informatik-Kenntnisse, wäre von befragten Historiker*innen nicht umgesetzbar gewesen
ähnliche Infrastrukturen nicht gibt, direkt in Wikidata gearbeitet werden (das was derzeit zur Verfügung steht)
Gleichzeitig in dieser Arbeit: von größtmöglichem Open Tech Stack ausgehen, Einschränungen nach unten offen halten, aber Devise Open Science radikal umgesetzt werden, soll am Ende auch Drawbacks dieser prototypischen Umsetzung diskutiert werden, nicht in Stein gemeißelt ggf. nachjustieren, mutig offeneren Lösungen entgegentreten

Beispiele in der historischen Forschung:
https://archivfuehrer-kolonialzeit.de/
https://blog.ehri-project.eu/2018/02/12/using-wikidata/


\subsection{WikiProjekt Jewish Owned Businesses}
bildet Grundlage

\section{Use Cases}

Strukturiert an einen idealtypischen Forschungsprozess. Nicht alle möglichen Anwendungsfälle abgedeckt werden. Aber Abdeckung gesamten Forschungsdatenlebenszyklus sicher stellen

\subsection{Datenmodellierung}

\subsection{Datenerfassung und -speicherung}

\subsection{Datenauswertung}

\subsection{Datenveröffentlichung}

\subsection{Daten(nach)nutzung}

\section{Ergebnisse}

Drawbacks