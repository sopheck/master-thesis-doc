\onehalfspacing



Drei wesentliche Erkenntnisse:

1. Interviews haben gezeigt, dass Bedarfe adhoc nicht eindeutig formuliert werden (können), was aber nicht bedeutet, dass diese nicht vorhanden sind oder das Wissenschaftler diese nicht sehen (es braucht Übersetzungszeit) sondern unterschiedliche Sprachwelten aufeinander prallen, Kommunikation essenziell --> hier braucht es Übersetzer, wie es mit dieser Masterarbeit unternommen wurde. Erkenntnis, die persönlich aus dieser Arbeit mitgenommen wird, dass Fragen teilweise viel zu technisch gestellt waren, würde heute anders gestellt werden, d.h. Versuch der Übersetzungen, Bedarfsformulierung scheitert nicht an mangelnden Bedarfen sondern wegen der Kommunikation

2. Gerade für verteilte projektbasierte Forschungsvorhaben zu einem Themenkomplex wie der Vernichtung der wirtschaftlichen Existenz sind zentralere Services notwendig, Projekte institutionell unterschiedlich angebunden, welche jeweils ihre eigenen Dienste und Infrastrukturen haben. Für das Forschungsfeld kann es konzeptionell ein immenser Fortschritt bedeuten, projektübergreifend kollaborativ zu arbeiten. Reichen Datenrepositorien Fortschritt, wären aber für das Forschungsfeld nicht ausreichend, braucht auf der Ebene der Datenmodellierung Infrastrukturen

3. Forschung ist nicht auf die akademische Wissenschaft allein beschränkt wie das hier betrachtete Forschungsfeld besondern deutlich macht, die Frage ist, wie bekommt man die unterschiedlichen Akteure zusammen bzw. welche Akteure werden einbezogen und welche ausgeschlossen

4. Wenn entsprechende Infrastrukturen vorhanden und genutzt werden, das zeigt die prototypische Implementierung, Open Science erweitert Erkenntnismöglichkeiten, welche im Ergebnis zu einem Erkenntnisfortschritt führen können. Verschiebt Erkenntnisgrenzen


Denn was Open Science am Ende ist, ist - wenn man der Open-Bewegung konsequent folgt - keine Frage von einzelnen Akteuren, sondern ein andauernder demokratischer Aushandlungsprozess vor allem aber nicht ausschließlich auf der wissenschaftlichen Ebene