\onehalfspacing

Forschungsdatenmanagement in Verbindung

\section{Ausgangspunkt}

Berliner Forschungsdaten zu jüdischen Gewerbebetrieben, Transformation von Access-DB in Online-DB

\section{Fragestellung und Zielsetzung}

Was kann Open Science für die geschichtswissenschaftliche Forschung bringen. 
Zeigen, was hinsichtlich FDM und Open Science heute möglich ist.

Implementierbarkeit von offenem FDM exemplarisch untersuchen, indem prototypische Lösung implementiert wird und Möglichkeiten sowie Grenzen dieser Implementierung herausgearbeitet werden.   

FDM offen bezüglich:
- projektübergreifend funktioniert und nicht auf eine Projektinstanz begrenzt ist, Möglichkeit für Historiker*innen, die in separaten Projekten aber im selben Forschungsfeld arbeiten, ein einheitliches FDM zu praktizieren
- inhaltlich offen, d.h. nicht auf jüdische Gewerbebetriebe beschränkt ist, sondern das gesamte Forschungsfeld umfassen kann
- technisch offen ist, das heißt das offene Technologien verwendete

Versuch unternommen werden Open Science auf Forschungsdatenmanagement anzuwenden. über den gesamten Research Data Lifecycle hinweg, die Forschungsdaten offen sind. Am beispiel des Forschungsfeld untersuchen, welchen Mehrgewinn das insbesondere für die historische Forschung bringen kann. 

\section{Methodisches Vorgehen}

hier erwähnen, dass Open Science Framework verwendet wurde --> dort sind auch alle Materialien enthalten (public)