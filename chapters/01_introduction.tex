\onehalfspacing

Forschungsdatenmanagement in Verbindung

Die große Menge an Open Science Initiativen Anwendern von Open Science Angeboten zeigt, dass Open Science in der Wissenschaft angekommen und in Begriff ist, sich dort zu etablieren. 
Aufschwung erlebte Open Science zuletzt im Zusammenhang mit der COVID-19-Pandemie, wo der als Mangel und damit letzten Endes Leben zu retten angesehen Den Bedarf von Open Science verstärkt. 



 Im Kern geht es auch darum, die Integrität von wissenschaftlicher Forschung zu wahren, sie gerade im sogenannten postfaktischen Zeitalter zu stärken, das heißt sie weniger anfällig für Betrug und Fälschung in einer digitalen Welt zu machen. 
  

 

Auch bei wissenschafts- wie gesellschaftspolitischen Entscheidungen gewinnt Open Science auf Bundes- sowie auf EU-Ebene an Relevanz, wobei zu konstatieren ist, dass der Schwerpunkt zumindest in Deutschland 


Wenn auch noch nicht die volle Bandbreite von Open Science, so unterstützt die Deutsche Forschungsgemeinschaft (DFG) immerhin offiziell gezielt Open Access-Publikationen finanziell. 

Die Europäische Union hat Open Science zu einem von insgesamt drei Grundsatzzielen für die Forschungsarbeit in Europa erklärt  und die  Deutsche UNESCO-Kommission betont in ihrer Empfehlung für Open Science:

,,Darüber hinaus besteht mit Open Science eine Chance auf die praktische Umsetzung von seit Langem bestehenden politischen Forderungen: Mit Open Science kann Teilhabe an und Zugang zu wissenschaftlichen Erkenntnissen als Gemeingut und Menschenrecht praktisch umgesetzt werden, wie es bereits seit Ende des Zweiten Weltkriegs in der Allgemeinen Erklärung der Menschenrechte gefordert war.''
Und auch auf der EU-Ebene 

\section{Ausgangspunkt}

Berliner Forschungsdaten zu jüdischen Gewerbebetrieben, Transformation von Access-DB in Online-DB

\section{Fragestellung und Zielsetzung}

Was kann Open Science für die geschichtswissenschaftliche Forschung bringen. 
Zeigen, was hinsichtlich FDM und Open Science heute möglich ist.

Implementierbarkeit von offenem FDM exemplarisch untersuchen, indem prototypische Lösung implementiert wird und Möglichkeiten sowie Grenzen dieser Implementierung herausgearbeitet werden.   

FDM offen bezüglich:
- projektübergreifend funktioniert und nicht auf eine Projektinstanz begrenzt ist, Möglichkeit für Historiker*innen, die in separaten Projekten aber im selben Forschungsfeld arbeiten, ein einheitliches FDM zu praktizieren
- inhaltlich offen, d.h. nicht auf jüdische Gewerbebetriebe beschränkt ist, sondern das gesamte Forschungsfeld umfassen kann
- technisch offen ist, das heißt das offene Technologien verwendete

Versuch unternommen werden Open Science auf Forschungsdatenmanagement anzuwenden. über den gesamten Research Data Lifecycle hinweg, die Forschungsdaten offen sind. Am beispiel des Forschungsfeld untersuchen, welchen Mehrgewinn das insbesondere für die historische Forschung bringen kann. 

\section{Methodisches Vorgehen}

hier erwähnen, dass Open Science Framework verwendet wurde --> dort sind auch alle Materialien enthalten (public)