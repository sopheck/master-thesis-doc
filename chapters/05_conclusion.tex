\onehalfspacing

\section{Zusammenfassung}

Zusammengefasst hat Open Science am Beispiel von Forschungsdaten zu Jüdischen Gewerbebetrieben Auswirkungen auf mehreren Ebenen:

\paragraph{Wissenschaftliche Qualitätsstandards} Da, wie gezeigt worden ist, ein statistisch- quantifizierender Methodenteil im Forschungsfeld zur Vernichtung der wirtschaftlichen Existenz der Juden im Nationalsozialismus unerlässlich bei der Erkenntnisgenerierung ist, lassen sich an diesen die Fragen nach Qualität und Reliabilität aus den Replikationsstudien genauso stellen. Auch wenn die Geschichtswissenschaften von der Replikationskrise nicht direkt betroffen waren, haben die Forschungsdaten zu Jüdischen Gewerbebetrieben gezeigt, dass Nachvollziehbarkeit und Überprüfbarkeit von Forschungsergebnissen auch in der historischen Forschung nicht zufriedenstellend sind. Der Hauptgrund dafür ist, dass die Daten bisher mehrheitlich nicht verfügbar sind. Inbesondere die Open Science-Grundsätze der Transparenz, offenen Kommunikation und Kollaboration können daher auch für die Geschichtswissenschaften ein Handlungsrahmen sein, um grundsätzlich gute wissenschaftliche Praxis in der historischen Forschung zu sichern und zu verbessern - gerade in der digitalen Transformationsphase, die tiefgreifende Veränderungen auch für das Fach mit sich bringt. Die prototypische Lösung mit Wikidata als offener Forschungsdatenmanagement-Service hat hier allerdings gezeigt, dass die existierenden Open Science-Infrastrukturen für sich allein nicht alle notwendigen Anwendungsfälle im Forschungsprozess abdecken, was in der Konsequenz auf die Kombination verschiedener vorhandender Lösungsansätze hinausläuft.

\paragraph{Gesellschaftliche und historische Verantwortung} Deutlich geworden ist, dass mit den Forschungsdaten zu Jüdischen Gewerbebetrieben nicht allein ein wissenschaftlicher sondern auch ein dokumentarischer Auftrag verfolgt wird. Dementsprechend wird das Forschungsfeld von diversen Akteure innerhalb und außerhalb des akademischen Bereichs bearbeitet. Gerade hier erweisen sich die Open Science-Technologien als Schnittstelle zwischen Wissenschaft und Öffentlichkeit als gewinnbringend. Zum einen kann durch ein konsequent offenes Forschungsdatenmanagent Informationszugang und -transfer zwischen den Akteuren erstmals ungehindert ermöglicht und dadurch die Teilhabe an den Daten zu Jüdischen Gewerbebetrieben im Sinne der einleitend zitierten UNESCO-Empfehlung praktisch umgesetzt sowie mit Open Science ein wichtiger erinnerungskultureller Beitrag geleistet werden. Zum anderen ergibt sich ein Mehrwert auch für die historische Forschung, denn vor allem aufgrund des schwierig zu bewältigenden disparaten Quellenmaterials ist es bisher keiner empirischen Studie gelungen, eine Gesamterhebung Jüdischer Gewerbebetriebe vorzunehmen. Stattdessen wurde auf der Basis von Stichproben gearbeitet. Die quantitative Dimension des Vernichtungsprozesses basiert folglich nicht auf absoluten Zahlen. Die Forschungsdaten für die gesamte Community im Forschungfeld offen zur Verfügung zu stellen, ermöglicht die kollaborative sukzessive Ergänzung und Erweiterung dieser Daten. Dies scheint vor dem Hintergrund, sich im Forschungsfeld einer Gesamtzahl verschwundener Jüdischer Gewerbebetriebe in Deutschland annähern zu wollen, ein vielversprechender und zielführender Ansatz zu sein. Der Wikidata-Lösungsansatz hat gezeigt, dass das Einspeisen wissenschaftlicher Erkenntnisse in die Wissensdatenbank deren Daten- und Informationsqualität enorm steigern kann, wovon nicht allein das Forschungsfeld sondern die gesamte Öffentlichkeit profitiert. Im Gegenzug kann Wikidata Sichtbarkeit von wissenschaftlicher Forschung generell über die traditionelle Publikation hinaus erhöhen und dadurch reputationelle Anreize für Wissenschaftler*innen schaffen.

\paragraph{Methodische und konzeptionelle Weiterentwicklung} Für das Forschungsfeld insgesamt ist zu konstatieren, dass dieses inhaltlich mit steigender Anzahl von Lokalstudien in den letzten 20 Jahren enorm voranschritt, aber im Vergleich auf konzeptueller Ebene die Weiterentwicklung überraschend stagnierte. Wenn mehrheitlich in den Studien der Begriff ,,Arisierung'' (oder ,,Entjudung'') hinterfragt wird, in der Konsequenz aber nicht aus der wissenschaftlichen Arbeit verbannt, sondern entgegen der eigenen Argumentation als Untersuchungsbegriff beibehalten wird, dann herrscht ein offensichtlicher Mangel an einer breiteren konzeptionellen und methodischen Auseinandersetzung im Forschungsfeld. Dafür spricht auch, dass es bis heute keine einheitliche Definition des Begriffs gibt. Einerseits wird darunter speziell der Transfer von jüdischem Eigentum, insbesondere Firmeneigentum, in nicht-jüdischen Besitz und andererseits generisch der gesamte Prozess der wirtschaftlichen Existenzvernichtung der Juden gefasst, wobei dieser unterschiedlich ausgedehnt wird. Einen allgemeingültigen wissenschaftlichen Konsens scheint es auf der methodischen Ebene im Forschungsfeld nicht zu geben. Gerade hier hat die prototypische Lösung in Wikidata gezeigt, dass die Implementierung des offenen Forschungsdatenmanagements mit dem Linked Open Data-Konzept vorwiegend bei der Datenmodellierung zu einer semantischen Auseinandersetzung mit diesen methodischen Schwächen zwingt. Die kollaborative Austausch- und Arbeitsumgebung von Wikidata ermöglicht diesbezüglich erstmals im Forschungsfeld, einheitliche Lösungswege zu finden, zu diskutieren und schließlich zu implementieren. Damit kann sich offenes Forschungsdatenmanagement normsetzend auf das Forschungsfeld auswirken und dadurch die überfällige methodische und konzeptionelle Weiterentwicklung vorantreiben. 

\paragraph{Analytische Grenzverschiebungen} Durch die zahlreichen separierten Lokalstudien ist das Forschungsfeld eigentlich komparatistisch angelegt. Denn nur im Vergleich lässt sich beurteilen, inwiefern lokale oder regionale Entwicklungen Regel oder Abweichung waren. Zudem ist eine Synthese der vielen einzelnen Forschungsergebnisse für ein Gesamtbild der Vernichtung der jüdischen Gewerbetätigkeit in Deutschland bisher noch ausgeblieben. Bei der prototypischen Lösung in Wikidata ist deutlich geworden, dass dafür die Qualität der Forschungsdaten zu Jüdischen Gewerbebetrieben insgesamt bisher noch nicht ausreicht, da in den Studien zu unterschiedlich gearbeitet wurde. Mit einem offenen Forschungsdatenmanagement im Forschungsfeld würde sich gleichzeitig der Herausforderung des lokalgeschichtlichen Ansatzes gestellt werden, die Vergleichbarkeit der Studien auf Datenebene herzustellen. Im Zuge dessen könnten zur studienübergreifenden eindeutigen Beschreibung und Identifizierung von Personen, Orten, Gewerbebetrieben, etc. erstmals ein kontrolliertes Vokabular für Jüdische Gewerbebetriebe entstehen und Wikidata als Authority File (Normdatei) im Forschungsfeld verwendet werden. Dies würde im Endeffekt zu mehr Standardisierung im Forschungsfeld führen, was ein allgemeingültiges Vorgehen bezüglich der Forschungsdaten und somit deren Vergleichbarkeit forciert. Dadurch könnten perspektivisch auch inhaltlich eng verknüpfte Forschungsdaten außerhalb der Vernichtung der jüdischen Gewerbetätigkeit integriert und gänzlich neue Forschungsfragen initiiert werden. Insgesamt hat die prototypische Lösung mit Wikidata als offenem Forschungsdatenmanagement-Service gezeigt, dass durch Open Science analytische Grenzen der Erkenntnismöglichkeiten im Forschungsfeld verschoben werden, was im Ergebnis zu einem Erkenntnisfortschritt führen kann.

\section{Zukünftige Arbeiten}

Wie einleitend in der Arbeit erläutert, handelt es sich bei dieser Konzeption nur um den ersten Schritt bei der Implementierung des offenen Forschungsdatenmanagements. Daher wäre der logische nächste Schritt,