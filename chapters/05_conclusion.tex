\onehalfspacing

\section{Zusammenfassung}

Offenes Forschungsdatenmanagement mit Open Science-Bezug am Beispiel von Forschungsdaten zu Jüdischen Gewerbebetrieben hat Auswirkung auf mehreren Ebenen:

\paragraph{Wissenschaftliche Qualitätsstandards} Da, wie gezeigt worden ist, ein statistisch- quantifizierender Methodenteil im Forschungsfeld zur Vernichtung der wirtschaftlichen Existenz der Juden im Nationalsozialismus unerlässlich bei der Erkenntnisgenerierung ist, lassen sich an diesen die Fragen nach Qualität und Reliabilität aus den Replikationsstudien genauso stellen. Auch wenn die Geschichtswissenschaften von der Replikationskrise nicht direkt betroffen waren, haben die Forschungsdaten zu Jüdischen Gewerbebetrieben gezeigt, dass Nachvollziehbarkeit und Überprüfbarkeit von Forschungsergebnissen auch in der historischen Forschung nicht zufriedenstellend sind. Der Hauptgrund dafür ist, dass die Daten mehrheitlich nicht zur Verfügung stehen. Die Open Science-Grundsätze der Transparenz, offenen Kommunikation und Kollaboration können daher für das Forschungsfeld ein Handlungsrahmen sein, um grundsätzlich gute wissenschaftliche Praxis zu sichern und zu verbessern. Der Blick auf die existierenden Open Science-Infrastrukturen zeigt, dass für diese sich allein nicht alle Anwendungsfälle im Forschungsprozess abdecken, was in der Folge auf die Kombination mehrerer vorhandener Lösungen hinausläuft.

\paragraph{Gesellschaftliche und historische Verantwortung} Deutlich geworden ist, dass mit den Forschungsdaten zu Jüdischen Gewerbebetrieben nicht allein ein wissenschaftlicher sondern auch ein dokumentarischer Auftrag verfolgt wird. Dementsprechend wird das Forschungsfeld von diversen Akteure innerhalb und außerhalb des akademischen Bereichs bearbeitet. Gerade hier erweisen sich die Open Science-Technologien als gewinnbringende Schnittstelle zwischen Wissenschaft und Öffentlichkeit. Zum einen kann durch ein konsequent offenes Forschungsdatenmanagement Informationszugang und -transfer zwischen den Akteuren erstmals ungehindert ermöglicht und dadurch die Teilhabe an den Daten zu Jüdischen Gewerbebetrieben im Sinne der einleitend zitierten UNESCO-Empfehlung praktisch umgesetzt sowie mit Open Science ein wichtiger erinnerungskultureller Beitrag geleistet werden. Zum anderen ergibt sich ein Mehrwert auch für die historische Forschung, denn vor allem aufgrund des schwierig zu bewältigenden disparaten Quellenmaterials ist es bisher keiner empirischen Studie gelungen, eine Gesamterhebung Jüdischer Gewerbebetriebe vorzunehmen. Stattdessen wurde auf der Basis von Stichproben gearbeitet. Die quantitative Dimension des Vernichtungsprozesses basiert folglich nicht auf absoluten Zahlen. Die Forschungsdaten für die gesamte Community im Forschungsfeld offen zur Verfügung zu stellen, ermöglicht die kollaborative sukzessive Ergänzung und Erweiterung dieser Daten. Dies scheint vor dem Hintergrund, sich im Forschungsfeld einer Gesamtzahl verschwundener Jüdischer Gewerbebetriebe in Deutschland annähern zu wollen, ein vielversprechender und zielführender Ansatz zu sein. Der Wikidata-Lösungsansatz macht deutlich, dass das Einspeisen wissenschaftlicher Erkenntnisse in die Wissensdatenbank deren Informationsgehalt steigert, wovon nicht allein das Forschungsfeld sondern die gesamte Öffentlichkeit profitiert. Im Gegenzug kann Wikidata Sichtbarkeit von wissenschaftlicher Forschung generell über die traditionelle Publikation hinaus erhöhen und dadurch reputationelle Anreize für Wissenschaftler*innen schaffen.

\paragraph{Methodisch-konzeptionelle Weiterentwicklung} Für das Forschungsfeld insgesamt ist zu konstatieren, dass dieses inhaltlich mit steigender Anzahl von Lokalstudien in den letzten 20 Jahren enorm voranschritt, aber im Vergleich auf konzeptueller Ebene die Weiterentwicklung überraschend stagnierte. Wenn mehrheitlich in den Studien der Begriff ,,Arisierung'' (oder ,,Entjudung'') hinterfragt wird, in der Konsequenz aber nicht aus der wissenschaftlichen Arbeit verbannt, sondern entgegen der eigenen Argumentation als Untersuchungsbegriff beibehalten wird, dann herrscht ein offensichtlicher Mangel an einer breiteren konzeptionellen und methodischen Auseinandersetzung im Forschungsfeld. Dafür spricht auch, dass es bis heute keine einheitliche Definition des Begriffs gibt. Einerseits wird darunter speziell der Transfer von jüdischem Eigentum, insbesondere Firmeneigentum, in nicht-jüdischen Besitz und andererseits generisch der gesamte Prozess der wirtschaftlichen Existenzvernichtung der Juden gefasst, wobei dieser unterschiedlich ausgedehnt wird. Einen allgemeingültigen wissenschaftlichen Konsens scheint es auf der methodischen Ebene im Forschungsfeld nicht zu geben. Gerade hier hat die prototypische Lösung in Wikidata gezeigt, dass die Implementierung des offenen Forschungsdatenmanagements mit dem Linked Open Data-Konzept vorwiegend bei der Datenmodellierung zu einer semantischen Auseinandersetzung mit diesen methodischen Schwächen zwingt. Die kollaborative Austausch- und Arbeitsumgebung von Wikidata ermöglicht diesbezüglich erstmals im Forschungsfeld, einheitliche Lösungswege zu finden, zu diskutieren und schließlich zu implementieren. Damit kann sich offenes Forschungsdatenmanagement normsetzend auf das Forschungsfeld auswirken und dadurch dessen überfällige methodisch-konzeptionelle Weiterentwicklung vorantreiben. 

\paragraph{Analytische Grenzverschiebungen} Durch die zahlreichen separierten Lokalstudien ist das Forschungsfeld komparatistisch angelegt. Denn nur im Vergleich lässt sich beurteilen, inwiefern lokale oder regionale Entwicklungen Regel oder Abweichung waren. Zudem ist eine Synthese der vielen einzelnen Forschungsergebnisse für ein Gesamtbild der Vernichtung der jüdischen Gewerbetätigkeit in Deutschland bisher noch ausgeblieben. Bei der prototypischen Lösung in Wikidata ist deutlich geworden, dass dafür die Qualität der Forschungsdaten zu Jüdischen Gewerbebetrieben insgesamt bisher noch nicht ausreicht, da in den Studien zu unterschiedlich gearbeitet wurde. Mit einem offenen Forschungsdatenmanagement im Forschungsfeld würde sich gleichzeitig der Herausforderung des lokalgeschichtlichen Ansatzes gestellt werden, die Vergleichbarkeit der Studien auf Datenebene herzustellen. Im Zuge dessen könnten zur studienübergreifenden eindeutigen Beschreibung und Identifizierung von Personen, Orten, Gewerbebetrieben, etc. erstmals ein kontrolliertes Vokabular für Jüdische Gewerbebetriebe entstehen und Wikidata als Authority File (Normdatei) im Forschungsfeld verwendet werden. Dies würde im Endeffekt zu mehr Standardisierung im Forschungsfeld führen, was ein allgemeingültiges Vorgehen bezüglich der Forschungsdaten und somit deren Vergleichbarkeit forciert. Dadurch könnten perspektivisch auch inhaltlich eng verknüpfte Forschungsdaten außerhalb der Vernichtung der jüdischen Gewerbetätigkeit integriert und gänzlich neue Forschungsfragen initiiert werden. Insgesamt hat die prototypische Lösung gezeigt, dass durch Open Science analytische Grenzen der Erkenntnismöglichkeiten im Forschungsfeld verschoben werden, was im Ergebnis zu einem Erkenntnisfortschritt führen kann.

\section{Zukünftige Arbeiten}

Wie einleitend in dieser Arbeit erläutert bildet diese Konzeption nur den ersten Schritt in einem längeren Anforderungsanalyse-Prozess. Daher ist der logische nächste Schritt, das offene Forschungsdatenmanagement in der Praxis zu erproben. Da im Forschungsfeld noch eine Reihe von Lokalstudien fehlen, wäre ein mögliches Szenario, Daten zu Jüdischen Gewerbebetrieben in zwei verschiedenen Orten zu erheben und kollaborativ in Wikidata zu erfassen, aufzubereiten und auszuwerten. Mit dem Fellow-Programm ,,Freies Wissen'' von Wikimedia gäbe es dafür einen möglichen Rahmen, in dem die praktische Erprobung erfolgen und wissenschaftlich begleitet werden könnte. In der Praxis kann der Wikidata-Lösungsansatz schließlich weiter angepasst und verfeinert werden, da sich unweigerlich neue Herausforderungen ergeben werden, die von einer rein prototypischen Lösung nicht alle vorhersehbar sind. 

Ferner stellt sich die Frage, wie im Rahmen eines Forschungsdatenmanagements mit den alten Forschungsdaten, die in letzten 20 Jahren produziert wurden und die meist in veralteten Softwareversionen sowie proprietären Formaten auf Privatrechnern vorliegen, umgegangen werden soll. Die Interviews haben gezeigt, dass den Besitzer*innen mitunter nicht bewusst ist, dass es sich hierbei um Forschungsdaten handelt. Diese haben aber nach wie vor, wie am Beispiel der Daten zu Jüdischen Gewerbebetrieben deutlich wird, einen epistemologischen Wert für die historische Forschung, auch wenn sie seit Jahren von den Besitzer*innen selbst nicht mehr genutzt werden. Die Empfehlung ist, die weitere Entwicklung des offenen Forschungsdatenmanagements mit zentralen niedrigschwelligen Aktionen zu koppeln, bei denen dazu aufgerufen wird, ,,alte'' Forschungsdaten zu spenden.

Abschließend liegt mit dem Wikidata-Lösungsansatz im Kern ein semantisches Forschungsdatenmanagement vor, dessen Potentiale in dieser Arbeit nur angedeutet bleiben konnten. Thema einer nächsten Arbeit könnte demzufolge sein, geschichtswissenschaftliches Forschungsdatenmanagement mit Linked (Open) Data-Ansatz vertiefend zu untersuchen und nutzbar zu machen.
 