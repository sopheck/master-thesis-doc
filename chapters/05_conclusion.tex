\onehalfspacing



Drei wesentliche Erkenntnisse:

1. Interviews haben gezeigt, dass Bedarfe adhoc nicht eindeutig formuliert werden (können), was aber nicht bedeutet, dass diese nicht vorhanden sind oder das Wissenschaftler diese nicht sehen (es braucht Übersetzungszeit) sondern unterschiedliche Sprachwelten aufeinander prallen, Kommunikation essenziell --> hier braucht es Übersetzer, wie es mit dieser Masterarbeit unternommen wurde. Erkenntnis, die persönlich aus dieser Arbeit mitgenommen wird, dass Fragen teilweise viel zu technisch gestellt waren, würde heute anders gestellt werden, d.h. Versuch der Übersetzungen, Bedarfsformulierung scheitert nicht an mangelnden Bedarfen sondern wegen der Kommunikation

2. Gerade für verteilte projektbasierte Forschungsvorhaben zu einem Themenkomplex wie der Vernichtung der wirtschaftlichen Existenz sind zentralere Services notwendig, Projekte institutionell unterschiedlich angebunden, welche jeweils ihre eigenen Dienste und Infrastrukturen haben. Für das Forschungsfeld kann es konzeptionell ein immenser Fortschritt bedeuten, projektübergreifend kollaborativ zu arbeiten. Reichen Datenrepositorien Fortschritt, wären aber für das Forschungsfeld nicht ausreichend, braucht auf der Ebene der Datenmodellierung Infrastrukturen

3. Forschung ist nicht auf die akademische Wissenschaft allein beschränkt wie das hier betrachtete Forschungsfeld besondern deutlich macht, die Frage ist, wie bekommt man die unterschiedlichen Akteure zusammen bzw. welche Akteure werden einbezogen und welche ausgeschlossen

4. Wenn entsprechende Infrastrukturen vorhanden und genutzt werden, das zeigt die prototypische Implementierung, Open Science erweitert Erkenntnismöglichkeiten, welche im Ergebnis zu einem Erkenntnisfortschritt führen können. Verschiebt Erkenntnisgrenzen


Denn was Open Science am Ende ist, ist - wenn man der Open-Bewegung konsequent folgt - keine Frage von einzelnen Akteuren, sondern ein andauernder demokratischer Aushandlungsprozess vor allem aber nicht ausschließlich auf der wissenschaftlichen Ebene


Sichtbarkeit der Forschungsarbeit erhöhen, nicht nur auf Publikationsebene, Projekt und Forschungsdatenebene

Wie kann Schnittstelle zwischen Wissenschaft und öffentlichem Wissen/ Öffentlichkeit funktionieren (Fellow-Programm Wikimedia)

hier auf Desiderate aus den Interviews eingehen

\paragraph{Benefits}

\paragraph{Drawbacks}

Datenqualität in Wikidata nicht perfekt, aber bei Christoph auch nicht
Datenkonsistenz und -integrität

\paragraph{Sideeffects}

Datenqualität der Wikidata verbessern und Informationen auf der Wikipedia nachweislich stärker kontextualisieren als bisher
--> am Beispiel von \url{https://de.wikipedia.org/w/index.php?title=Wodka_Gorbatschow&oldid=222273519} und Q2587685

Abschließend zur Forschungsfeldbetrachtung ist festzustellen, dass das dieses inhaltlich mit steigender Anzahl von Lokalstudien in den letzten 20 Jahren enorm voranschritt, aber im Vergleich auf konzeptueller Ebene die Weiterentwicklung überraschend stagnierte. Wenn mehrheitlich in den Studien der Begriff ,,Arisierung'' (oder ,,Entjudung'') kritisch und problemorientiert hinterfragt wird, in der Konsequenz aber nicht aus der wissenschaftlichen Arbeit verbannt, sondern entgegen der eigenen Argumentation als Untersuchungsbegriff beibehalten wird, dann herrscht ein offensichtlicher Mangel an einer breiteren konzeptionellen und methodischen Auseinandersetzung im Forschungsfeld. Dafür spricht auch, dass es bis heute keine einheitliche Definition des Begriffs gibt.\footnote{Und die es auch in der Geschichte des Begriffs nie gegeben hat.\textbf{Vgl. Nietzel und Kreutzmüller}} Einerseits wird darunter speziell der Transfer von jüdischem Eigentum, insbesondere Firmeneigentum, in nicht-jüdischen Besitz und andererseits generisch der gesamte Prozess der wirtschaftlichen Existenzvernichtung der Juden gefasst, wobei dieser unterschiedlich ausgedehnt wurde\footnote{Nachweis} Einen allgemeingültigen wissenschaftlichen Konsens scheint es auf der methodischen Ebene im Forschungsfeld nicht zu geben. Unklar ist, warum nach den eindeutig nachvollziehbaren Gegeneinwänden und alternativen Vorschlägen aus dem Forschungsfeld selbst sich diese methodische Schwäche bis heute hartnäckig hält.


Die Herausforderung besteht darin, zentrale sowie einheitliche Infrastrukturen zu schaffen, die von den überwiegend einzelgeförderten Forschungsprojekten - bei der DFG immerhin mehr als ein Drittel im Jahr 2020 projektbezogener Einzelförderung – nicht allen Forschungsvorhaben ein nachhaltiges Forschungsdatenmanagement inhärent ist. Da es entsprechende Forschungsgebiete in der Vergangenheit schlichtweg noch nicht gab, war der Umgang mit Forschungsdaten mehr von individuellen digitalen Kenntnissen und Kompetenzen des oder der Wissenschaftler*in abhängig als von allgemeingültigen wissenschaftlichen Kriterien sowie technischen Standards. Zeitökonomisch betrachtet bedeutet der wissenschaftliche Umgang mit digitalen Forschungsdaten zudem Arbeitsaufwand, der zu den routinierten Abläufen hinzukommt. Erst recht, wenn sich ganz neu mit dieser Thematik auseinandergesetzt werden muss. Das wirft die berechtigte Frage nach dem Kosten-Nutzen-Verhältnis für die eigene Forschungsarbeit auf.

Eine Synthese dieser bisher nebeneinander existierenden Forschungsergebnisse gibt es noch nicht.\footnote{Vgl. Nietzel S.}

