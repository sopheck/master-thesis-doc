
\section{Hinweise}

In diesem Anhang befinden sich auf den Folgeseiten alle relevanten Forschungsdaten, die der Erkenntnisgewinnung dieser Arbeit dienten. Diese sind:

\begin{itemize}
  \item Abschnitt D.1 enthält den Fragebogen, der den Befragten vorab geschickt wurde und der als Leitfaden für die Interviews diente.
  \item Abschnitt D.2 enthält die Transkripte der Interviews mit Positionsangaben, welche in der Arbeit zitiert sind. Die Interviews wurden mit der Textanalyse-Software \textit{MAXQDATA} transkribiert.
  \item Abschnitt D.3 enthält das Codesystem der qualititativen Inhaltsanalyse der Interviews, das mit der Textanalyse-Software \textit{MAXQDATA} entwickelt wurde.
  \item Abschnitt D.4 enthält die Datenmodelle für die Konzepte ,,Forschungsprojekt'' und ,,Jüdischer Gewerbebetrieb'' sowie die Begriffsklassifikation (Thesaurus), welche der Sacherschließung diente.
\end{itemize} 

Die meisten Daten sind außerdem im OSF-Projekt \textit{Master thesis: Open Science in History? Conception of open research data management using the example of research data on Jewish commercial enterprises under National Socialism} unter einer CC-BY-SA-Lizenz veröffentlicht.\footnote{Eckenstaler, Sophie (2022, June 4). Master thesis: Open Science in History? (1.0). Open Science Framework (OSF). \url{https:\\osf.io/sc9yf}.} Dort sind auch jene Daten verfügbar, die im Rahmen der schriftlichen Arbeit nicht beigefügt werden konnten. Darunter zählen die aus \textit{MAXQDATA} exportierte Liste der codierten Segmente und die Versionierung der schriftlichen Arbeit mit \textit{Git} und \textit{GitHub}. Aus datenschutzrechtlichen Gründen ist das gesamte \textit{MAXQDATA}-Projekt ,,interviews-transcriptions-evaluation.mx22'' mit den (Roh)Transkripten sowie den Audiodateien der Interviews von der Veröffentlichung in der offenen Lizenz ausgenommen. Das Projekt wurde in einer nichtoffenen Lizenz in Zenodo hochgeladen, mit Metadaten angereichert und mit dem OSF-Projekt verknüpft.\footnote{Eckenstaler, Sophie (2022, June 4). Qualitative Interviews zu offenem Forschungsdatenmanagement am Beispiel von Forschungsdaten zu Jüdischen Gewerbebetrieben im Nationalsozialismus (Masterarbeit) (1.0.0). Zenodo. \url{https://doi.org/10.5281/zenodo.6613413}.}

\includepdf[pages=1,scale=.85,pagecommand=\section{Fragebogen}]{pdf/Fragebogen.pdf}
\includepdf[pages=2-,scale=.85,pagecommand={}]{pdf/Fragebogen.pdf}

\includepdf[pages=1,scale=.85,pagecommand={},landscape]{pdf/Codesystem.pdf}
\includepdf[pages=2-,scale=.85,pagecommand={},landscape]{pdf/Codesystem.pdf}

\includepdf[pages=1,scale=.85,pagecommand=\section{Transkripte}]{pdf/B1-4_Transkripte.pdf}
\includepdf[pages=2-,scale=.85,pagecommand={}]{pdf/B1-4_Transkripte.pdf}

\includepdf[pages=1,scale=.85,pagecommand=\section{Modelle},landscape]{pdf/wikidata-models.pdf}
\includepdf[pages=2,scale=.85,pagecommand={}]{pdf/wikidata-models.pdf}
\includepdf[pages=3,scale=.85,pagecommand={},landscape]{pdf/wikidata-models.pdf}

