
In diesem Kapitel befinden sich auf den Folgeseiten alle relevanten Forschungsdaten, die der Erkenntnisgewinnung dieser Arbeit dienten. Diese sind:

\begin{itemize}
  \item Abschnitt D.1 enthält den Fragebogen, der den Befragten vorab geschickt wurde und der als Leitfaden für die Interviews diente.
  \item Abschnitt D.2 enthält die Transkripte der Interviews mit Positionsangaben, welche in der Arbeit zitiert sind.
  \item Abschnitt D.3 enthält das Codesystem der qualititativen Inhaltsanalyse der Interviews in der Software \textit{MaxQData}.
  \item Abschnitt D.4 enthält die Datenmodelle für die Konzepte ,,Forschungsprojekt'' und ,,Jüdischer Gewerbebetrieb'' sowie die Begriffsklassifikation (Thesaurus), welche der Sacherschließung diente.
\end{itemize} 

Alle Daten dieser Arbeit sind außerdem im OSF-Projekt \textit{Master thesis: Open Science in History? Conception of open research data management using the example of research data on Jewish commercial enterprises under National Socialism} unter einer CC-BY-SA-Lizenz veröffentlicht.\footnote{URL (stable): \url{https://osf.io/sc9yf/}.} Dort sind auch jene Daten verfügbar, die im Rahmen der schriftlichen Arbeit nicht beigefügt werden konnten, wie das \textit{MaxQData}-Projekt ,,interviews-transcriptions-evaluation.mx22'', dessen Codesegmente, sowie die Versionierung der Arbeit mit \textit{Git} und \textit{GitHub}.



\includepdf[pages=1,scale=.8,pagecommand=\section{Fragebogen}]{Fragebogen.pdf}
\includepdf[pages=2-,scale=.8,pagecommand={}]{Fragebogen.pdf}



\section{Transkripte}
\section{Codesystem}
\section{Modelle}