\onehalfspacing

In diesem Kapitel geht es darum, sich einen grundlegenden Überblick über die Ausgangslage zu beschaffen sowie den Bedarf von offenem Forschungsdatenmanagement (FDM) zu ermitteln, der diese Arbeit begründet. Beides bildet die Basis, auf die offenes FDM aufsetzt und von der sich die funktionale Anforderungen an das offene FDM ableiten. 

Im zweiten Schritt folgt eine datenkritische Auseinandersetzung mit dem Forschungsfeld. Hier geht es darum, die Forschungsdaten, die Ergebnis eines Genese- und Verarbeitungsprozesses sind, zu historisieren um sie verstehen und bewerten zu können. Dafür werden Studien, in denen Forschungsdaten während des Forschungsprozesses angefallen sind, systematisch untersucht. Ziel ist erstens, im Forschungsfeld den Zweck von Forschungsdaten und die daraus folgende Handhabung dieser Daten zu ermitteln. Darauf aufbauend soll zweitens der Erkenntnis- sowie Nutzwert von Forschungsdaten und damit deren Bedeutung im Forschungsfeld bestimmt werden. Dabei ist das Vorgehen problemorientiert, es werden auch Grenzen und Defizite diskutiert, an die offenes Forschungsdatenmanagement lösungsorientiert ansetzen soll. Vor diesem Hintergrund werden in einem dritten Schritt zusammenfassend außerdem weitergehende Probleme sowie Desiderate im Forschungsfeld mit in den Blick genommen.

Auswertung der Interviews und der Literaturanalyse
Benötigter Funktionsumfang ist mit FAIR, CARE und Open Data Modellen bereits vorgegeben, sie geben Anforderungen vor
Es geht um diesen Funktionsumfang des offenen FDM an Forschungsfeld anzupassen, d.h. weitere Anforderungen zu erarbeiten und damit die Anschlussfähigkeit des offenen FDM sicher zu stellen.



\section{Wirtschaftliche Existenzvernichtung der Juden im Nationalsozialismus}

Die ersten grundlegenden, wissenschaftlichen Auseinandersetzungen mit der wirtschaftlichen Verfolgung, Verdrängung und Vernichtung der Juden im Nationalsozialismus erfolgten zwar schon früh in der BRD im Nachkriegsdeutschland.\footnote{Im Jahr 1966 erschien die Pionierstudie von Helmut Genschel. Erst 20 Jahre später folgte die nächste grundlegende Studie des israelischen Historikers Avraham Barkai, der an Gentschels Ergebnisse anknüpfte. Vgl. Benno Nietzel: Die Vernichtung der wirtschaftlichen Existenz der deutschen Juden 1933-1945. Ein Literatur und Forschungsbericht, in: Friedrich-Ebert-Stiftung (Hg.), Archiv für Sozialgeschichte, Band 49, Bonn 2009, S. 561-613.} Allerdings blieben diese vereinzelt und ohne größere Resonanz. 

Erst Ende der 1990er Jahren trat in Deutschland eine längere Forschungswelle zum Thema auf, die eine Bandbreite an Studien hervorgebracht hat und in deren Folge sich ein eigenes Forschungsfeld zur wirtschaftlichen Existenzvernichtung der Juden im Nationalsozialismus etablierte.\footnote{Als wegweisend wird regelmäßig die Lokalstudie zu Arisierung in Hamburg des Historikers Frank Bajohr aus dem Jahr 1997/98 gewertet. Siehe zum Beispiel Nietzel 2009, S. 561 oder Christiane Fritsche: Ausgeplündert, zurückerstattet und entschädigt. Arisierung und Wiedergutmachung in Mannheim, 2. Aufl., Ubstadt-Weiher, Heidelberg, Neustadt a. d. W., Basel 2013, S. 21. Frank Bajohr: ,,Arisierung'' in Hamburg. Die Verdrängung der jüdischen Unternehmer 1933-1945, 2. Aufl., Hamburg 1998 (zuerst 1997). Auf Ursachen des Forschungsbooms kann im Rahmen dieser Arbeit nicht eingegangen werden. Siehe dazu auch Christoph Kreutzmüller, Vernichtung der jüdischen Gewerbetätigkeit im Nationalsozialismus. Abläufe, Blickwinkel und Begrifflichkeiten, Version: 2.0, in: Docupedia-Zeitgeschichte, 12.3.2020, URL: \url{http://docupedia.de/zg/Kreutzmueller_vernichtung_der_juedischen_Gewerbetaetigkeit_v2_de_2020}} Es lieferte innerhalb der NS-Forschung weitere Erklärungsansätze zur antisemitischen Verfolgungs- und Vernichtungspolitik, deren Antriebskräfte in der Vergangenheit unterschiedlich interpretiert wurden.\footnote{Siehe zu den unterschiedlichen Deutungen und Perspektiven (insbesondere Intentionalismus vs. Strukturalismus) Bajohr 1998, S. 10-14} Hierbei waren lange nationalsozialistische Akteure, kommunale Verwaltungsinstanzen und nicht-jüdische Nutznießer sowie deren Strategien, Verhalten und Handlungsoptionen Schwerpunkt der Forschung. Diese Fokussierung wurde in zunehmendem Maß als zu einseitig kritisiert, da insbesondere die jüdischen Betroffenen ganz ausgeblendet oder sie ausschließlich als passive Opfer gezeigt worden seien. Zudem entwickelte sich langsam ein wissenschaftlicher Diskurs über die Anwendung historischer Begrifflichkeiten in der Forschung.\footnote{Vgl. Ludolf Herbst, Christoph Kreutzmüller, Ingo Loose u.a., Einleitung, in: Ludolf Herbst, Christoph Kreutzmüller, Thomas Weihe (Hg.): Die Commerzbank und die Juden 1933-1945, München 2004, S. 10-13. Diese Selbstkritik war ohne Zweifel richtig und auch notwendig, da sie grundlegende konzeptionelle Probleme im Forschungsfeld aufdeckte. Dennoch ist die einseitige Perspektive auf Täter, Mittäter und Mitwisser vor dem Hintergrund des jahrzehntelangen Verdrängens in der deutschen Nachkriegs- und Tätergesellschaft bis hin zu Geschichtsrevisionismus und Opfer-Umkehrung ein verständliches Anliegen gewesen. Letztlich leistete die Geschichtswissenschaft damit zwar einen späten aber nicht weniger wichtigen Beitrag zur historischen Aufarbeitung der NS-Verbrechen.} Im Zentrum stand hierbei die Kritik, dass die meisten Studien die Bandbreite und Komplexität des Forschungsthemas unter dem diffusen Begriff ,,Arisierung'' untersuchten und diesen dabei unterschiedlich ausdehnten.\footnote{Vgl. Nietzel 2009, S. 562-565. Mitunter wird der Begriff bis in die Zwangsarbeit hinein ausgeweitet. Siehe Britta Bopf: ,,Arisierung'' in Köln. Die wirtschaftliche Existenzvernichtung der Juden 1933-1945, Köln 2004, S. 11.} Häufig lag der Schwerpunkt der Untersuchung jedoch auf jüdischen Unternehmern und der Übernahme deren Eigentums\footnote{Siehe zum Beispiel Barbara Händler-Lachmann/Thomas Werther: Vergessene Geschäfte, verlorene Geschichte. Jüdisches Wirtschaftsleben in Marburg und seine Vernichtung im Nationalsozialismus, Marburg 1992; Alex Bruns-Wüstefeld: Lohnende Geschäfte. Die ,,Entjudung'' der Wirtschaft am Beispiel Göttingens, Hannover 1997; Bajohr 1997/98, Einleitung, S. 9f.; Marian Rappl: ,,Arisierung'' in München. Die Verdrängung der jüdischen Gewerbetreibenden aus dem Wirtschaftsleben der Stadt 1933-1939, in: Kommission für bayerische Landesgeschichte bei der Bayerischen Akademie der Wissenschaften in Verbindung mit der Gesellschaft für fränkische Geschichte und der Schwäbischen Forschungsgemeinschaft (Hrsg.), Zeitschrift für bayerische Landesgeschichte, Bd. 63, Heft 1, München 2000, S. 82-123, hier S. 125; Heinz-Jürgen Priamus (Hrsg.): Was die Nationalsozialisten ,,Arisierung'' nannten. Wirtschaftsverbrechen in Gelsenkirchen während des ,,Dritten Reiches'', Essen 2007, S. 11ff.}, wodurch die historische Forschung zuweilen Schlagseite erlitt, da andere Aspekte der wirtschaftlichen Existenzvernichtung wie zum Beispiel die Verdrängung von Juden aus ihren Berufen unterbelichtet blieben.\footnote{Vgl. Nietzel 2009, S. 565.} Zusammengefasst war der Einwand, dass die bisher verwendeten Untersuchungsbegriffe ,,engführend''\footnote{Kreutzmüller 2016/2020,  URL: \url{http://docupedia.de/zg/Kreutzmueller_vernichtung_der_juedischen_Gewerbetaetigkeit_v2_de_2020.}} dahingehend seien, das Geschehene nur einseitig zu rekonstruieren, zu dessen gesamtheitlicher Analyse folglich nicht taugen.\footnote{Vgl. Nietzel 2009, S. 564 und Herbst/Weihe, Commerzbank, 2004, S. 10ff..}

Ab Mitte der 2000er Jahre lässt sich daraufhin eine Weiterentwicklung beobachten, die vor allem von größeren universitären Forschungsprojekten vorangetrieben wurde und die mit der Verschiebung in der Forschungsperspektive sowie der begrifflichen Ausdifferenzierung einher ging.\footnote{Pionierarbeit leistet hier u.a. das Forschungsprojekt ,,Geschichte der Commerzbank von 1870 bis 1958'' am Lehrstuhl für Zeitgeschichte an der Humboldt-Universität zu Berlin unter Leitung von Prof. Dr. Ludolf Herbst sowie das Forschungsprojekt zur Vernichtung der jüdischen Gewerbetätigkeit im Nationalsozialismus in den drei Großstädten Berlin, Breslau, Frankfurt am Main, ebendort. Siehe Ludolf Herbst/Thomas Weihe (Hg.), Die Commerzbank und die Juden 1933-1945, München 2004; Christoph Kreutzmüller, Ausverkauf. Die Vernichtung der jüdischen Gewerbetätigkeit in Berlin 1930-45, Berlin 2012; Benno Nietzel, Handeln und Überleben: jüdische Unternehmer aus Frankfurt am Main 1924-1964, Göttingen 2012} Die neueren Studien unterschieden sich im Wesentlichen dadurch, dass sie die jüdischen Betroffenen als handelnde Akteure begriffen und deren \textit{agency} in den Blick nahmen. Außerdem versuchten sie erstmals mit den Begriffen ,,Arisierung'' oder ,,Entjudung'' zu brechen\footnote{Unwissenschaftlich insofern, als dass es sich um rassistisch konnotierte Begriffe handelt, die selbst eigentlich zu historisieren wären, anstatt diese in die Wissenschaftssprache aufzunehmen. Vgl. Nietzel 2009, S. 563.} und Phänomene des Forschungsthema durch eine wissenschaftliche Terminologie zu benennen. Dabei wurde ein prozessorientierter Zugang gewählt, der an die Holocaust-Forschung des US-amerikanischen Historikers Raul Hilberg anknüpfte. Hilberg analysierte den Massenmord an den Juden wegweisend als einen Prozess, der über Definition, Kennzeichnung, Enteignung, Konzentration und Mord mehrstufig verlief.\footnote{Raul Hilberg: Die Vernichtung der europäischen Juden, Band 1, Frankfurt am Main 1990 (zuerst englisch 1961), S. 85-163. Eine wichtige Ergänzung zu Hilbergs Thesen war, dass die wirtschaftliche Existenzvernichtung der Juden der Teilprozess, war, der ,,am längsten – nämlich über den Tod der Opfer hinaus – dauerte und demzufolge in alle anderen Prozesse hineinreichte''. Kreutzmüller 2012, S. 378.} Als integraler Bestandteil dieses Prozesses wurde die Vernichtung der wirtschaftlichen Existenz der Juden im Nationalsozialismus als ein mehrschichtiger Gesamtprozess analysiert, der sich aus den abgrenzbaren, aber überlagernden und in Wechselbeziehung stehenden Teilprozessen Verdrängung, Besitztransfer, Liquidation und Vermögensentzug zusammensetzte. Diese schlossen folglich die Verdrängung der Juden aus dem Berufsleben, die Vernichtung der jüdischen Gewerbetätigkeit durch Besitzübernahme oder Liquidation sowie die Entziehung des Vermögens der Juden ein.\footnote{Exemplarisch wurden erstmals alle Teilprozesse systematisch im Rahmen der Erforschung der Geschichte der Commerzbank betrachtet. Siehe Herbst/Weihe, Commerzbank, 2004..}

Mit diesem Forschungsansatz konnte zum einen anhand der drei deutschen Großstädte Berlin, Frankfurt am Main und Breslau empirisch  gezeigt werden, dass die als jüdisch verfolgten Unternehmen nicht - wie bisher durch die Schwerpunktsetzung der historischen Forschung suggeriert - größtenteils in den Besitz nichtjüdischer Erwerber*innen übergingen, sondern schlichtweg liquidiert wurden.\footnote{Vgl. Kreutzmüller 2016/2020.} Diesbezüglich lag der Erkenntnisfortschritt in der Freilegung des Teilprozess der Vernichtung der jüdischen Gewerbetätigkeit als ein ,,großangelegtes Liquidationsprogramm'', das bisher kaum als solches von der historischen Forschung reflektiert worden war.\footnote{Vgl. Nietzel 2012, S. 164 und Kreutzmüller 2012, S. 250.} Des Weiteren wurde durch den Wechsel der Forschungsperspektive systematisch herausgearbeitet, dass sich die jüdischen Betroffenen gegen ihre Entrechtung wehrten und dazu verschiedenen institutionelle wie individuelle Strategien nutzten.\footnote{Systematisch untersucht von Kreutzmüller, Ausverkauf, 2012, Kapitel IV. Abwehrstrategien jüdischer Gewerbetreibender, S. 257-357; Nietzel, Handeln und Überleben, 2012, Kapitel II.2 Erwartungen, Anpassung und Selbstbehauptung, S. 99-150..}

An diesen Forschungsstand anknüpfend, unternahm zuletzt der Historiker Benno Nietzel im Jahr 2009 den Versuch, die zahlreichen Forschungsstudien zur Vernichtung der wirtschaftlichen Existenz der Juden im Nationalsozialismus zu ordnen, indem er die bisherigen Forschungsfragen, Untersuchungsgegenstände sowie Forschungsergebnisse zusammenfasste und strukturierte.\footnote{Auch Nietzel sprach von "analaytischer Hilflosigkeit angesichts der Vielschichtigkeit und Komplexität des Prozesses [der wirtschaftlichen Existenzvernichtung der Juden, Anm. S.E.]", ebd. S. 564..}. Sein Ziel war es, die wirtschaftliche Existenzvernichtung der Juden als ein abgrenzbares Forschungsfeld abzustecken, einheitlich zu definieren und damit einer einer systematischeren Bearbeitung zuzuführen. Dafür definierte er fünf Teilbereiche des Forschungsfelds:
\begin{itemize}
\item Verdrängung der Juden aus dem Berufsleben (Angestellte, Beamte, Selbstständige wie Rechtsanwälte, Ärzte oder Wissenschaftler)
\item Vernichtung der jüdischen Gewerbetätigkeit (Besitztransfer und Liquidation)
\item staatliche Enteignung des jüdischen Vermögens (Privatbesitz, Firmenvermögen, Immobilienvermögen aus Grundbesitz) 
\item Entgrenzung (transnationale Perspektiven)
\item Wiedergutmachung nach 1945 in der BRD
\end{itemize}

Zwar betonte er deren überschneidende Beziehungen und Verhältnisse zueinander, nahm aber in erster Linie eine separierte Betrachtung zum Zwecke der inhaltlichen Erschließung und zur Herausarbeitung von Spezifika des Forschungsthemas vor.\footnote{Nietzel 2009, S. 562. Nietzel greift außerdem die Beteiligung von nichtjüdischen Unternehmen mit auf aber explizit nicht als eine eigene Kategorie sondern als Querschnittaspekt, weshalb dieser hier nicht berücksichtigt wird, da er strenggenommen zum Forschungsfeld der Unternehmensgeschichte gehört. Siehe zu Unternehmensgeschichte Ralf Ahrens, Unternehmensgeschichte, Version: 1.0, in: Docupedia-Zeitgeschichte, 1.11.2010, URL: \url{http://docupedia.de/zg/Ahrens_unternehmensgeschichte_v1_de_2010.}.} 

Neben den bereits erläuterten Teilprozessen ordnete Nietzel dem Forschungsfeld außerdem die historisch untrennbare materielle Wiedergutmachung nach 1945 in der BRD zu, welche zum einen die Restitution/ Rückerstattung und zum anderen die Entschädigung meint. Hiervon ausgenommen ist die Entziehung und die Restitution von Kulturgütern, die Nietzel dem eigenen Forschungsfeld der Provenienzforschung zuordnete.\footnote{Vgl. ebd. S. 273.} Im Falle der Entgrenzung vor allem nach Kriegsbeginn geht um die europaweite Perspektive der wirtschaftlichen Existenzvernichtung. Im Sinne des transnationalen Forschungsansatzes stehen dabei der Transfer von Erfahrungswissen und der Export von Verfolgungspraktiken sowie deren Weiterentwicklung in den besetzten Gebieten im Fokus. Auch Kollaboration und die Rolle von deutschen Unternehmen bei der Ausplünderung der europäischen Juden werden in den Blick genommen.\footnote{Vgl. ebd. S. 602-608.}

Nietzels Systematisierungsversuch wurde bisher auffallend wenig von der historischen Forschung rezipiert.\footnote{Aus Literaturrecherche und Interviews ging nicht hervor, dass Nietzels Systematik nachträglich kontrovers diskutiert oder weiterentwickelt wurde.} Lediglich der Historiker Christoph Kreutzmüller nahm 2016 darauf Bezug und ergänzte den neuesten Forschungsstand zur Vernichtung der jüdischen Gewerbetätigkeit.\footnote{Siehe Kreutzmüller 2016/2020,  URL: \url{http://docupedia.de/zg/Kreutzmueller_vernichtung_der_juedischen_Gewerbetaetigkeit_v2_de_2020.}} Auch wenn dieser eine deutliche Professionalisierung darstellt, weil erstmals unter Einbeziehung aller relevanten Forschungsstudien konzeptionell mit dem komplexen Forschungsthema auseinandergesetzt wurde, so bleibt festzuhalten, dass der Begriff ,,Arisierung'' als Untersuchungsbegriff in der historischen Forschung nach wie vor zur Anwendung kommt.\footnote{Siehe Maren Janetzko: Die ,,Arisierung'' mittelständischer jüdischer Unternehmen in Bayern 1933-1939. Ein interregionaler Vergleich, Ansbach 2012, S. 17f; Claudia Flümann: ,,... doch nicht bei uns in Krefeld!". Arisierung, Enteignung, Wiedergutmachung in der Samt- und Seidenstadt 1933-1963, Krefeld 2015, S. 13 oder jüngst bei Monika Juliane Gibas: ,,Arisierung'' der Wirtschaft in Thüringen: Das Beispiel Arnstadt, in: Schlossmuseum Arnstadt (Hrsg.): Jüdische Familien aus Arnstadt und Plaue. Katalog zur Sonderausstellung im Schlossmuseum Arnstadt, Arnstadt 2021, S. 108-148..}  

Charakteristisch für das Forschungsfeld ist zudem, dass lokal- bzw. regionalgeschichtliche Studien dominieren. Zwar wurde das Thema auch in Form von Überblicks- oder Gesamtdarstellungen zum Deutschen Reich (in den Grenzen von 1937) abgehandelt, dies jedoch nur vereinzelt und vor allem in den Anfangsjahren der wissenschaftlichen Auseinandersetzung mit dem Thema.\footnote{Siehe zum Beispiel die bereits erwähnten grundlegenden Studien von Genschel 1966 und Barkai 1987. Danach erschienen sind noch: Günter
Plum, Wirtschaft und Erwerbsleben, in: Wolfgang Benz (Hrsg.), Die Juden in Deutschland 1933–
1945. Leben unter nationalsozialistischer Herrschaft, München 1988, S. 268–313. Dieter Ziegler, Die wirtschaftliche
Verfolgung der Juden im »Dritten Reich«, in: Heinz-Jürgen Priamus (Hrsg.), Was die
Nationalsozialisten ,,Arisierung'' nannten. Wirtschaftsverbrechen in Gelsenkirchen während des
»Dritten Reiches«, Essen 2007, S. 17–40. Für die Literaturanalyse wurden vier Überblicks- bzw. Gesamtdarstellungen und fünfzehn Lokalstudien erfasst. Es ist natürlich nicht auszuschließen, dass es mehr Darstellungen zum Deutschen Reich oder zu Europa gibt, aber eine Tendenz im Forschungsfeld hin zu lokalhistorischen Studien ist nichtsdestotrotz deutlich erkennbar.} In den letzten fünfzehn Jahren sind überwiegend Untersuchungen zu Klein- und Großstädten erschienen, deren Ergebnisse ebenfalls vereinzelt in Form von Sammelbänden zusammengefasst wurden.\footnote{Siehe zum Beispiel Christiane Fritsche u.a (Hrsg.), ,,Arisierung'' und ,,Wiedergutmachung'' in deutschen Städten, Köln 2014. Allerdings handelt es sich dabei um einen ,,partikularistischen Zugriff'' auf das Thema, dessen Stärken vor allem in der zusammenfassenden Darstellung der aktuellen Forschungsergebnisse liegt als im Generieren neuer Erkenntnisse. Siehe Rezension dazu: Jan Schleusener: Rezension zu: Fritsche, Christiane; Paulmann, Johannes (Hrsg.), ,,Arisierung'' und ,,Wiedergutmachung'' in deutschen Städten, Köln  2014. ISBN 978-3-412-22160-7, In: H-Soz-Kult, 10.12.2014, \url{www.hsozkult.de/publicationreview/id/reb-21747.}..} Diese Entwicklung hat zwei Gründe:

Da sich die historische Forschung zum Thema, wie oben erläutert, früh auf die Vernichtung der jüdischen Gewerbetätigkeit in Deutschland konzentriert hat, ist sie wissenschaftlich begründet. Denn jene erfolgte erst ab 1938 mit der Einführung reichsweiter Gesetze und Regelungen.\footnote{Darunter fiel auch die antisemitische Definition, was unter einem "jüdischen Gewerbebetrieb" verstanden werden sollte.} Das heißt, dass die jüdische Gewerbetätigkeit für die nationalsozialistische Wirtschaftspolitik erst spät auf dem Plan stand.\footnote{Vgl. Nietzel 2009, S. 562, 565 und 576.} Anders sah es hingegen in der politischen Peripherie aus, wo bereits ab 1933 mit den Aprilboykotten jüdische Gewerbebetriebe gezielt verfolgt wurden und in deren Folge jüdische Gewerbebetriebe verschwanden. Es waren insbesondere also lokale Akteure gewesen, die den Vernichtungsprozess vorangetrieben hatten. Auch nach 1938 waren sie es, die die reichsweiten Gesetze und Bestimmungen umsetzten. Es ist daher wenig überraschend, dass die Wissenschaft überwiegend den lokalhistorischen Zugang gewählt hat, da in einer Überblicksdarstellung für Deutschland die Vernichtung der jüdischen Gewerbetätigkeit unmöglich in der notwendigen Dichte beschrieben und rekonstruiert werden kann.\footnote{Programmatisch war hier wieder die Lokalstudie zu Hamburg von Frank Bajohr Ende der neunziger Jahre. Siehe Bajohr 1997/98..} 

Neben der wissenschaftlichen Begründung, die von fast allen Studien vorgetragen wird\footnote{\textbf{hier Studien}}, wird in diesen seltener reflektiert, dass viele Forschungsprojekte dem Bereich der lokalen, insbesondere der städtischen Erinnerungskultur entsprungen sind, was zur lokalgeschichtlichen Dominanz sicherlich mit beigetragen hat.\footnote{\textbf{hier Projekte aufzählen}}. Als Erklärungsansatz für diese besondere Entwicklung scheinen die gesellschaftlichen Auf- und Umbruchszeiten der 1980er Jahre plausibel. In der Tradition der basisdemokratischen und dezentralen Graswurzelbegewegung (,,Grabe, wo du stehst'')\footnote{\textbf{Programmatisch war hier ???}} mit der Etablierung zahlreicher lokaler Geschichtswerkstätten ab Anfang der 1980er Jahre in der BRD war die Motivation verbunden, die nationalsozialistische Geschichte des eigenen Ortes kritisch aufzuarbeiten.\footnote{Siehe zur Geschichte und zum Einfluss der Bewegung: Jenny Wüstenberg, Zivilgesellschaft und Erinnerungspolitik in Deutschland seit 1945, Berlin Münster 2020, Kapitel 4 Grabe, wo stehst: Die Geschichtsbewegung und die Graswurzel-Erinnerungskultur S. 147-200 und Kapitel 5 Memorialästhetik und die Erinnerungsbewegungen der 1980er, S. 201-230.} Ab Mitte der 80er Jahre rückten zunehmend die jüdischen Opfer ins Bewusstsein und es stand ein angemessenes, innovatives Gedenken sowie die Schaffung von Gedenkorten im Fokus.\footnote{Das bekannteste Projekt ist wahrscheinlich das Stolperstein-Projekt des Künstlers Gunther Demnig. Vgl. Wüstenberg 2020, S. 209. Die erste Verlegung in Berlin-Kreuzberg im Jahr 1996 war von den Behörden noch nicht genehmigt worden und wurde erst später legalisiert. Siehe Projektwebsite, URl: \url{http://www.stolpersteine.eu/start/} (Letzter Zugriff am 26.01.2022).} Alles in allem waren die Akteure dieser Bewegung von einem emanzipatorischen (,,Geschichte von unten''), einem aufklärerischem (Lernen aus der Geschichte) sowie einem moralischen (Vergangenheit nicht vergessen) Antrieb geleitet. Sie wollten die etablierte Geschichtsforschung und Erinnerungspolitik durch Demokratisierung von unten und Partizipation von Grund auf verändern.\footnote{Das diese Ideale in der Praxis nicht vollkommen widerspruchs- und konfliktfrei blieben, zeigt sehr anschaulich der historische Abriss von Jenny Wüstenberg. Vgl. Wüstenberg 2020, S. 166f. und 182ff.} Diese Entwicklung hatte Rückkopplungseffekte auf die akademische Geschichtswissenschaft, die sich von einer sozialhistorischen Ausrichtung hin zu einer \textit{Alltagsgeschichte.} als neuer Forschungsansatz weiterentwickelte. 

Schnittmenge zu Open Bewegung und deren Partizipation Paradigma

Abschließend deutlich geworden ist, dass die Forschungsdaten historiographisch im  Kontext des Forschungsfelds zur Vernichtung der wirtschaftlichen Existenz der Juden im Nationalsozialismus entstanden sind, welches Teil der umfassenden NS-Forschung ist und insbesondere an die Holocaust-Forschung angeknüpft. Das Forschungsfeld wird seit circa 20 Jahren Jahren systematisch bearbeitet. Hierbei dominieren lokalgeschichtliche Zugänge. Sofern es also Forschungsdaten gibt, dann wurden diese in der Vergangenheit vorwiegend im Rahmen von Studien generiert, die sich bei ihrer wissenschaftlichen Analyse geografisch begrenzt haben. Dementsprechend sind die zugehörigen Forschungsdaten räumlich von begrenzter Aussage, da sie jeweils lediglich einen Ort oder eine Region abbilden. Damit handelt es sich bei diesen Lokalstudien gleichzeitig um Fallstudien, die genau genommen erst in ihrer Summe eine Gesamtdarstellung für das Deutsche Reich in den Grenzen von 1937 ergeben. Eine Synthese dieser bisher nebeneinander existierenden Forschungsergebnisse gibt es noch nicht.\footnote{Vgl. Nietzel S.} Die Herausforderung besteht darin, die in einem Zeitraum von über zwanzig Jahren publizierten, verschiedenen Lokalstudien in Bezug auf ihre Forschungsdaten erstmals zusammenzuführen und in ein (projekt)übergreifendes FDM zu überführen. Zu beachten ist hierbei, dass das Forschungsfeld nicht ausschließlich im akademischen Umfeld bearbeitet wurde und wird, sondern unterschiedlichste zivilgesellschaftliche Initiativen oder Einzelpersonen ebenfalls ein wesentlicher Treiber der Forschung waren und sind. Eine strikte Trennung in akademisch einerseits und nichtakademisch andererseits erscheint nicht sinnvoll, da sich beide Bereiche in der Vergangenheit gegenseitig bedingten und befruchteten.\footnote{Zum Verhältnis von akademischer und nichtakademischer historischer Forschung vgl. Wüstenberg 2020, S. 163ff. Überschneidungen gab es vor allem bei beim Organisieren auf personeller Ebene.} Das bedeutet, dass potentielle Anwender*innen von offenem FDM im Forschungsfeld sowie deren Nutzungsmotive und Nutzungserwartungen äußerst heterogen sind. Die sich daraus ableitenden Zielgruppen und Stakeholder von offenem FDM werden in Kapitel 3.1. separat definiert und beschrieben.

Auffällig ist, dass das Forschungsfeld inhaltlich in den letzten 20 Jahren enorm voranschritt, aber im Vergleich auf konzeptueller Ebene die Weiterentwicklung stagnierte. Wenn in ausnahmslos jeder Studie der Begriff ,,Arisierung'' (oder ,,Entjudung'') kritisch und problemorientiert hinterfragt wird, in der Konsequenz aber nicht aus der wissenschaftlichen Arbeit verbannt, sondern entgegen der eigenen Argumentation als Untersuchungsbegriff beibehalten wird, dann herrscht ein offensichtlicher Mangel an einer breiteren konzeptionellen und methodischen Auseinandersetzung im Forschungsfeld. Dafür spricht auch, dass es bis heute keine einheitliche Definition des Begriffs gibt.\footnote{Und die es auch in der Geschichte des Begriffs nie gegeben hat.\textbf{Vgl. Nietzel und Kreutzmüller}} Einerseits wird darunter speziell der Transfer von jüdischem Eigentum, insbesondere Firmeneigentum, in nicht-jüdischen Besitz und andererseits generisch der gesamte Prozess der wirtschaftlichen Existenzvernichtung der Juden gefasst, wobei dieser unterschiedlich ausgedehnt wurde\footnote{•} Einen allgemeingültigen wissenschaftlichen Konsens scheint es auf der methodischen Ebene im Forschungsfeld nicht zu geben. Unklar ist, warum nach den eindeutig nachvollziehbaren Gegeneinwänden und alternativen Vorschlägen aus dem Forschungsfeld selbst sich diese methodische Schwäche bis heute hartnäckig hält. 
Im Umkehrschluss stellt sich damit die Ausgangslage für das offene Forschungsdatenmanagement als nicht absolut eindeutig dar, was insofern problematisch ist, als dass das offenen FDM, sofern es digital laufen soll, aus entwicklungstechnischer Sicht widerspruchsfrei, in der Regel in Form eines Datenmodells, beschrieben werden muss. Die Weiterverwendung unwissenschaftlicher Begrifflichkeiten scheint an dieser Stelle erst recht nicht geeignet, da sie in keiner Weise zur Präzision beitragen. Als derzeit einzige Möglichkeit, sich im Forschungsfeld zwischen den unterschiedlichen Studien zu orientieren, bietet sich der in Kapitel 2.1. erläuterte Systematisierungsversuch des Historikers Nietzel an. Er wird in dieser Arbeit methodisch als Taxonomie aufgegriffen, die es ermöglicht, erstens die Berliner Forschungsdaten sowie die in Kapitel 2.2. weiter betrachteten Forschungsdaten zur Vernichtung der jüdischen Gewerbetätigkeit zu klassifizieren. Von hier aus wird deutlich sichtbar, dass diese inhaltlich lediglich einen kleinen Ausschnitt aus dem Gesamtkomplex der wirtschaftlichen Existenzvernichtung der Juden im NS abbilden, diesen also nur teilweise repräsentieren und darüber hinaus inhaltlich in den größeren Prozess der Verfolgung und Vernichtung der Juden in Deutschland eingebettet sind. Auch wenn im Rahmen dieser Arbeit der Schwerpunkt auf der Vernichtung der jüdischen Gewerbetätigkeit liegt, wird das FDM offen konzipiert, das bedeutet, dass es inhaltlich anschlussfähig erstens an die weiteren Unterkategorien des Forschungsfelds ist und zweitens in der Entwicklungsperspektive auch an benachbarte Forschungsfelder der Verfolgung und Vernichtung im Nationalsozialismus andocken kann. Damit läuft die Konzeption auf eine prototypische Lösung von offenem FDM hinaus, die übertragbar auch auf andere zeitgeschichtliche Forschungsfelder ist.

\section{Datenkritische Auseinandersetzung im Forschungsfeld}

Der Schwerpunkt dieser Arbeit


Daten zu jüdischen Gewerbebetrieben wurden, wie im vorausgegangenen Kapitel erläutert, im akademischen wie nichtakademischen Umfeld gesammelt. Für beide Bereiche stellt sich die Frage, 

Unabhängig davon stellt sich für beide Bereiche die Frage, zum welchem Zweck diese generiert wurden und wie die Handhabung erfolgte. Davon lässt sich der Erkenntnis- und Nutzwert dieser Daten feststellen und 


Siehe Forschungsdaten zu verfolgten Ärztinnen und Ärzte des Berliner Städtischen Gesundheitswesens (1933-1945)
\url{https://geschichte.charite.de/verfolgte-aerzte/index.html}


Allerdings lässt sich in den letzten Jahren ein Abflauen

Zum anderen

Die Frage, 

 dominierten seit Ende der 1990er Jahre wissenschaftliche Untersuchungen, die sehr präzise die vergangenen Vorgänge der zu rekonstruieren versuchten.      


Im Forschungsfeld zur Vernichtung der jüdischen Gewerbetätigkeit lässt in den letzten Jahren einen Abflauen des zu Beginn der 2000er Jahre sehr regen Publizierens zum Thema beobachten.\footnote{Die letzte Studie zum Thema erschien}


Dazu wird dieses systematisch nach folgenden Fragen untersucht:

\begin{itemize}
\item Wo im Forschungsfeld existieren über Berlin hinaus weitere Forschungsdaten zu jüdischen Gewerbebetrieben.
\item Wann wurden die Forschungsdaten generiert (Zeitpunkt).
\item Welche Ziele sollten mit den Forschungsdaten erreicht werden.
\item Wann wurden die Forschungsdaten generiert und wie wurden sie erhoben (Methode). 
\item Welche Daten wurden erfasst und wie erfolgte die Datenerfassung.
\item Wie wurden die Daten ausgewertet (Datenauswertung und -interpretation).
\item Was für Datenarten liegen vor (Charakteristik).
\item Welche Daten sind in welcher Form veröffentlicht (Publikation).
\end{itemize}


\begin{table}
\caption{Metaangaben zu den ausgewählten Lokalstudien}
\label{tab:lokalstudientabelle}
\begin{tabular} { L{0.5cm}|L{3cm}|L{4cm}|L{0.7cm}|L{4cm} }
Nr. & Titel & Untertitel & Ort & Entstehungskontext \\
\hline 
1 & Arisierung in Hamburg & Die Verdrängung der jüdischen Unternehmer 1933-1945 & 1997 & Dissertation \\
\hline 
2 & ,,Arisierungen'' in München & Die Verdrängung der jüdischen Gewerbetreibenden aus dem Wirtschaftsleben der Stadt 1933-1939 & 2000 & n.n.\\
\hline 
3 & ,,Arisierung'' in Köln & Die wirtschaftliche Existenzvernichtung der Juden 1933-1945 & 2004 & Dissertation\\
\hline
4 & ,,Arisierung'' in München & Die Vernichtung jüdischer Existenz 1937-1939 & 2004 & n.n.\\
\hline
5 & Ausverkauf & Die Vernichtung der jüdischen Gewerbetätigkeit in Berlin 1930-1945 & 2012 & n.n.\\
\hline
6 & Handeln und Überleben & Jüdische Unternehmer aus Frankfurt am Main 1924-1964 & 2012 & n.n.\\
\hline
7 & Die ,,Arisierung'' Mittelständischer jüdischer Unternehmen in Bayern 1933-1939 & Ein interregionaler Vergleich & 2012 & n.n.\\
\hline
8 & Ausgeplündert, zurückerstattet und entschädigt & Arisierung und Wiedergutmachung in Mannheim & 2013 & Dissertation\\
\hline
9 & ,,… doch nicht bei uns in Krefeld!'' & Arisierung, Enteignung, Wiedergutmachung in der Samt- und Seidenstadt 1933 bis 1963 & 2015 & n.n\\
\end{tabular}
\end{table}
\paragraph{Forschungsziele und Forschungsdesign}
\paragraph{Datenerfassung und Datensicherung}
\paragraph{Datenauswertung}
\paragraph{Datenveröffentlichung}



\section{Auswertung der Interviews}
{Kollaboratives Arbeiten auf den Daten
Tools zur Datenauswertung
Nutzung projektübergreifender Datensammlungen
\paragraph{Forschungsdesiderate im Forschungsfeld}Allgemeinaussagen zum Vernichtungsprozess halten nur auf einer gesicherten empirischen Basis stand, daher ist ein quantitativer Methodenteil in dem Forschungsfeld unerlässlich. Vor allem aufgrund der disparaten Quellenmaterials, die innerhalb eines Projektzeitraums nicht vollständig zu bewältigen waren, ist es bisher keiner Studie gelungen, alle jüdischen Gewerbebetriebe zu erfassen. Stattdessen wurde auf der Basis von Stichprobenziehung (sampling) gearbeitet. Die quantitative Dimension des Vernichtungsprozesses basiert folglich nicht auf absoluten Zahlen.
Weiterhin ist auffallend, dass in fast allen Lokalstudien die Zahlen von Genschel und Barkai herangezogen und diskutiert wurden, um Aussagen dazu zu verifizieren. Einige Autor*innen wiesen jedoch zurecht daraufhin, dass es sich bei Genschel und Barkai um grobe Schätzungen und zwar für Gesamtdeutschland handelte, die sich so nicht zwingend auf lokaler Ebene wiederfinden müssen, dennoch aber korrekt sein können. Daraus ist zu schießen, dass erst die Summe hinreichender Lokalstudien ein Gesamtbild dazu ergeben werden können.
Anknüpfungspunkte für offenes FDM

\paragraph{Rechtliche und forschungsethische Rahmenbedingungen}

\section{Zwischenergebnisse}
\subsection{Stakeholder}
Recherche
Stakeholderanalyse
- Aufzeigen und Zusammenführen der Beteiligten in einem Projekt
- Systematisches Bewerten und Ordnen der Beteiligten nach Interessen, Macht und Rolle
- Zielbestimmung entsprechend der Interessenbeteiligten

\paragraph{Wissenschaftler}

akademisch an Universitäten, nicht-akademisch z.B. an Erinnerungseinrichtungen wie Archive, aber auch Historiker*innen, die nicht universitär angebunden sind (Claudia Fürmann)


\paragraph{Akteure der Erinnerungskultur}

z.B. Stolperstein-Initiativen sehr von moralischen Impetus geleitet, ehrenamtliche

\paragraph{Familienangehörige}

\subsection{Funktionsmodell}

siehe \url{https://en.wikipedia.org/wiki/Feature_model}

\subsection{Open Science Grad}