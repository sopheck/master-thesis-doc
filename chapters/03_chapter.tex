\onehalfspacing

\section{Kontextualisierung der Forschungsdaten}

Anknüpfungspunkte und Potentiale von offenem Forschungsdatenmanagement erkennbar sind.


\subsection{Inhaltliche Einordnung und Charakteristika} 

Inhaltlich sind die hier exemplarisch betrachteten Forschungsdaten zur Vernichtung der jüdischen Gewerbetätigkeit in den größeren Themenkomplex der wirtschaftlichen Verfolgung, Verdrängung und Vernichtung der Juden im Nationalsozialismus eingebettet. Die ersten grundlegenden, wissenschaftlichen Auseinandersetzungen dazu erfolgten zwar schon früh in der BRD im Nachkriegsdeutschland.\footnote{Im Jahr 1966 erschien die Pionierstudie von Helmut Genschel. Erst 20 Jahre später folgte die nächste grundlegende Studie des israelischen Historikers Avraham Barkai, der an Gentschels Ergebnisse anknüpfte. Vgl. Benno Nietzel: Die Vernichtung der wirtschaftlichen Existenz der deutschen Juden 1933-1945. Ein Literatur und Forschungsbericht, in: Friedrich-Ebert-Stiftung (Hg.), Archiv für Sozialgeschichte, Band 49, Bonn 2009, S. 561-613.} Allerdings blieben diese vereinzelt und ohne größere Resonanz. Erst Ende der 1990er Jahren trat in Deutschland eine längere Forschungswelle zum Thema auf, die eine Bandbreite an Studien hervorgebracht hat und in deren Folge sich ein eigenes Forschungsfeld zur wirtschaftlichen Existenzvernichtung der Juden im Nationalsozialismus etablierte.\footnote{Als wegweisend wird regelmäßig die Lokalstudie zu Arisierung in Hamburg des Historikers Frank Bajohr aus dem Jahr 1997/98 gewertet. Siehe zum Beispiel Nietzel 2009, S. 561 oder Christiane Fritsche: Ausgeplündert, zurückerstattet und entschädigt. Arisierung und Wiedergutmachung in Mannheim, 2. Aufl., Ubstadt-Weiher, Heidelberg, Neustadt a. d. W., Basel 2013, S. 21. Frank Bajohr: ,,Arisierung'' in Hamburg. Die Verdrängung der jüdischen Unternehmer 1933-1945, 2. Aufl., Hamburg 1998 (zuerst 1997). Auf Ursachen des Forschungsbooms kann im Rahmen dieser Arbeit nicht eingegangen werden. Siehe dazu auch Christoph Kreutzmüller, Vernichtung der jüdischen Gewerbetätigkeit im Nationalsozialismus. Abläufe, Blickwinkel und Begrifflichkeiten, Version: 2.0, in: Docupedia-Zeitgeschichte, 12.3.2020, URL: \url{http://docupedia.de/zg/Kreutzmueller_vernichtung_der_juedischen_Gewerbetaetigkeit_v2_de_2020}} Es lieferte innerhalb der NS-Forschung weitere Erklärungsansätze zur antisemitischen Verfolgungs- und Vernichtungspolitik, deren Antriebskräfte in der Vergangenheit unterschiedlich interpretiert wurden.\footnote{Siehe zu den unterschiedlichen Deutungen und Perspektiven (insbesondere Intentionalismus vs. Strukturalismus) Bajohr 1998, S. 10-14} Hierbei waren lange nationalsozialistische Akteure, kommunale Verwaltungsinstanzen und nicht-jüdische Nutznießer sowie deren Strategien, Verhalten und Handlungsoptionen Schwerpunkt der Forschung. Diese Fokussierung wurde in zunehmendem Maß als zu einseitig kritisiert, da insbesondere die jüdischen Betroffenen ganz ausgeblendet oder sie ausschließlich als passive Opfer gezeigt worden seien. Zudem entwickelte sich langsam ein wissenschaftlicher Diskurs über die Anwendung historischer Begrifflichkeiten in der Forschung.\footnote{Vgl. Ludolf Herbst, Christoph Kreutzmüller, Ingo Loose u.a., Einleitung, in: Ludolf Herbst, Christoph Kreutzmüller, Thomas Weihe (Hg.): Die Commerzbank und die Juden 1933-1945, München 2004, S. 10-13. Diese Selbstkritik war ohne Zweifel richtig und auch notwendig, da sie grundlegende konzeptionelle Probleme im Forschungsfeld aufdeckte. Dennoch ist die einseitige Perspektive auf Täter, Mittäter und Mitwisser vor dem Hintergrund des jahrzehntelangen Verdrängens in der deutschen Nachkriegs- und Tätergesellschaft bis hin zu Geschichtsrevisionismus und Opfer-Umkehrung ein verständliches Anliegen gewesen. Letztlich leistete die Geschichtswissenschaft damit zwar einen späten aber nicht weniger wichtigen Beitrag zur historischen Aufarbeitung der NS-Verbrechen.} Im Zentrum stand hierbei die Kritik, dass die meisten Studien die Bandbreite und Komplexität des Forschungsthemas unter dem diffusen Begriff ,,Arisierung'' untersuchten und diesen dabei unterschiedlich ausdehnten.\footnote{Vgl. Nietzel 2009, S. 562-565. Mitunter wird der Begriff bis in die Zwangsarbeit hinein ausgeweitet. Siehe Britta Bopf: ,,Arisierung'' in Köln. Die wirtschaftliche Existenzvernichtung der Juden 1933-1945, Köln 2004, S. 11.} Häufig lag der Schwerpunkt der Untersuchung jedoch auf jüdischen Unternehmern und der Übernahme deren Eigentums\footnote{Siehe zum Beispiel Barbara Händler-Lachmann/Thomas Werther: Vergessene Geschäfte, verlorene Geschichte. Jüdisches Wirtschaftsleben in Marburg und seine Vernichtung im Nationalsozialismus, Marburg 1992; Alex Bruns-Wüstefeld: Lohnende Geschäfte. Die ,,Entjudung'' der Wirtschaft am Beispiel Göttingens, Hannover 1997; Bajohr 1997/98, Einleitung, S. 9f.; Marian Rappl: ,,Arisierung'' in München. Die Verdrängung der jüdischen Gewerbetreibenden aus dem Wirtschaftsleben der Stadt 1933-1939, in: Kommission für bayerische Landesgeschichte bei der Bayerischen Akademie der Wissenschaften in Verbindung mit der Gesellschaft für fränkische Geschichte und der Schwäbischen Forschungsgemeinschaft (Hrsg.), Zeitschrift für bayerische Landesgeschichte, Bd. 63, Heft 1, München 2000, S. 82-123, hier S. 125; Heinz-Jürgen Priamus (Hrsg.): Was die Nationalsozialisten ,,Arisierung'' nannten. Wirtschaftsverbrechen in Gelsenkirchen während des ,,Dritten Reiches'', Essen 2007, S. 11ff.}, wodurch die historische Forschung zuweilen Schlagseite erlitt, da andere Aspekte der wirtschaftlichen Existenzvernichtung wie zum Beispiel die Verdrängung von Juden aus ihren Berufen unterbelichtet blieben.\footnote{Vgl. Nietzel 2009, S. 565.} Zusammengefasst war der Einwand, dass die bisher verwendeten Untersuchungsbegriffe ,,engführend''\footnote{Kreutzmüller 2016/2020,  URL: \url{http://docupedia.de/zg/Kreutzmueller_vernichtung_der_juedischen_Gewerbetaetigkeit_v2_de_2020.}} dahingehend seien, das Geschehene nur einseitig zu rekonstruieren, zu dessen gesamtheitlicher Analyse folglich nicht taugen.\footnote{Vgl. Nietzel 2009, S. 564 und Herbst/Weihe, Commerzbank, 2004, S. 10ff..}

Ab Mitte der 2000er Jahre lässt sich daraufhin eine Weiterentwicklung beobachten, die vor allem von größeren universitären Forschungsprojekten vorangetrieben wurde und die mit der Verschiebung in der Forschungsperspektive sowie der begrifflichen Ausdifferenzierung einher ging.\footnote{Pionierarbeit leistet hier u.a. das Forschungsprojekt ,,Geschichte der Commerzbank von 1870 bis 1958'' am Lehrstuhl für Zeitgeschichte an der Humboldt-Universität zu Berlin unter Leitung von Prof. Dr. Ludolf Herbst sowie das Forschungsprojekt zur Vernichtung der jüdischen Gewerbetätigkeit im Nationalsozialismus in den drei Großstädten Berlin, Breslau, Frankfurt am Main, ebendort. Siehe Ludolf Herbst/Thomas Weihe (Hg.), Die Commerzbank und die Juden 1933-1945, München 2004; Christoph Kreutzmüller, Ausverkauf. Die Vernichtung der jüdischen Gewerbetätigkeit in Berlin 1930-45, Berlin 2012; Benno Nietzel, Handeln und Überleben: jüdische Unternehmer aus Frankfurt am Main 1924-1964, Göttingen 2012} Die neueren Studien unterschieden sich im Wesentlichen dadurch, dass sie die jüdischen Betroffenen als handelnde Akteure begriffen und deren \textit{agency} in den Blick nahmen. Außerdem versuchten sie erstmals mit den Begriffen ,,Arisierung'' oder ,,Entjudung'' zu brechen\footnote{Unwissenschaftlich insofern, als dass es sich um rassistisch konnotierte Begriffe handelt, die selbst eigentlich zu historisieren wären, anstatt diese in die Wissenschaftssprache aufzunehmen. Vgl. Nietzel 2009, S. 563.} und Phänomene des Forschungsthema durch eine wissenschaftliche Terminologie zu benennen. Dabei wurde ein prozessorientierter Zugang gewählt, der an die Holocaust-Forschung des US-amerikanischen Historikers Raul Hilberg anknüpfte. Hilberg analysierte den Massenmord an den Juden wegweisend als einen Prozess, der über Definition, Kennzeichnung, Enteignung, Konzentration und Mord mehrstufig verlief.\footnote{Raul Hilberg: Die Vernichtung der europäischen Juden, Band 1, Frankfurt am Main 1990 (zuerst englisch 1961), S. 85-163. Eine wichtige Ergänzung zu Hilbergs Thesen war, dass die wirtschaftliche Existenzvernichtung der Juden der Teilprozess, war, der ,,am längsten – nämlich über den Tod der Opfer hinaus – dauerte und demzufolge in alle anderen Prozesse hineinreichte''. Kreutzmüller 2012, S. 378.} Als integraler Bestandteil dieses Prozesses wurde die Vernichtung der wirtschaftlichen Existenz der Juden im Nationalsozialismus als ein mehrschichtiger Gesamtprozess analysiert, der sich aus den abgrenzbaren, aber überlagernden und in Wechselbeziehung stehenden Teilprozessen Verdrängung, Besitztransfer, Liquidation und Vermögensentzug zusammensetzte. Diese schlossen folglich die Verdrängung der Juden aus dem Berufsleben, die Vernichtung der jüdischen Gewerbetätigkeit durch Besitzübernahme oder Liquidation sowie die Entziehung des Vermögens der Juden ein.\footnote{Exemplarisch wurden erstmals alle Teilprozesse systematisch im Rahmen der Erforschung der Geschichte der Commerzbank betrachtet. Siehe Herbst/Weihe, Commerzbank, 2004.}

Mit diesem Forschungsansatz konnte zum einen anhand der drei deutschen Großstädte Berlin, Frankfurt am Main und Breslau empirisch  gezeigt werden, dass die als jüdisch verfolgten Unternehmen nicht - wie bisher durch die Schwerpunktsetzung der historischen Forschung suggeriert - größtenteils in den Besitz nichtjüdischer Erwerber*innen übergingen, sondern schlichtweg liquidiert wurden.\footnote{Vgl. Kreutzmüller 2016/2020.} Diesbezüglich lag der Erkenntnisfortschritt in der Freilegung des Teilprozess der Vernichtung der jüdischen Gewerbetätigkeit als ein ,,großangelegtes Liquidationsprogramm'', das bisher kaum als solches von der historischen Forschung reflektiert worden war.\footnote{Vgl. Nietzel 2012, S. 164 und Kreutzmüller 2012, S. 250.} Des Weiteren wurde durch den Wechsel der Forschungsperspektive systematisch herausgearbeitet, dass sich die jüdischen Betroffenen gegen ihre Entrechtung wehrten und dazu verschiedenen institutionelle wie individuelle Strategien nutzten.\footnote{Systematisch untersucht von Kreutzmüller, Ausverkauf, 2012, Kapitel IV. Abwehrstrategien jüdischer Gewerbetreibender, S. 257-357; Nietzel, Handeln und Überleben, 2012, Kapitel II.2 Erwartungen, Anpassung und Selbstbehauptung, S. 99-150.}

An diesen Forschungsstand anknüpfend, unternahm zuletzt der Historiker Benno Nietzel im Jahr 2009 den Versuch, die zahlreichen Forschungsstudien zur Vernichtung der wirtschaftlichen Existenz der Juden im Nationalsozialismus zu ordnen, indem er die bisherigen Forschungsfragen, Untersuchungsgegenstände sowie Forschungsergebnisse zusammenfasste und strukturierte.\footnote{Auch Nietzel identifizierte im Forschungsfeld ,,analaytische Hilflosigkeit angesichts der Vielschichtigkeit und Komplexität des Prozesses [der wirtschaftlichen Existenzvernichtung der Juden, Anm. S.E.]'', ebd. S. 564.}. Nietzels Bericht ist für diese Arbeit insofern relevant, als dass sich hieraus drei wesentliche Merkmale ableiten lassen, mit denen die Forschungsdaten zur Vernichtung der jüdischen Gewerbetätigkeit für das offenen Forschungsdatenmanagement charakterisiert werden können:

\paragraph{Feldzugang}Wenn die wirtschaftliche Existenzvernichtung der Juden als ein abgrenzbares Forschungsfeld definiert ist, dann lässt es sich folglich für eine differenzierte Unterschung abstecken. Nach Nietzel kann dies in fünf Teilbereichen erfolgen:
\begin{itemize}
\item Verdrängung der Juden aus dem Berufsleben (Angestellte, Beamte, Selbstständige wie Rechtsanwälte, Ärzte oder Wissenschaftler)
\item Vernichtung der jüdischen Gewerbetätigkeit (Besitztransfer und Liquidation)
\item staatliche Enteignung des jüdischen Vermögens (Privatbesitz, Firmenvermögen, Immobilienvermögen aus Grundbesitz) 
\item Entgrenzung (transnationale Perspektiven)
\item Wiedergutmachung nach 1945 in der BRD
\end{itemize}

Zwar betonte Nietzel deren überschneidende Beziehungen und Verhältnisse zueinander, nahm aber in erster Linie eine separierte Betrachtung zum Zwecke der inhaltlichen Erschließung und zur Herausarbeitung von Spezifika des Forschungsthemas vor.\footnote{Nietzel 2009, S. 562. Nietzel greift außerdem die Beteiligung von nichtjüdischen Unternehmen mit auf aber explizit nicht als eine eigene Kategorie sondern als Querschnittaspekt, weshalb dieser hier nicht berücksichtigt wird, da er strenggenommen zum Forschungsfeld der Unternehmensgeschichte gehört. Siehe zu Unternehmensgeschichte Ralf Ahrens, Unternehmensgeschichte, Version: 1.0, in: Docupedia-Zeitgeschichte, 1.11.2010, URL: \url{http://docupedia.de/zg/Ahrens_unternehmensgeschichte_v1_de_2010.}.} 

Neben den bereits erläuterten Teilprozessen ordnete Nietzel dem Forschungsfeld außerdem die historisch untrennbare materielle Wiedergutmachung nach 1945 in der BRD zu, welche zum einen die Restitution/ Rückerstattung und zum anderen die Entschädigung meint. Hiervon ausgenommen ist die Entziehung und die Restitution von Kulturgütern, die Nietzel dem eigenen Forschungsfeld der Provenienzforschung zuordnete.\footnote{Vgl. ebd. S. 273.} Im Falle der Entgrenzung vor allem nach Kriegsbeginn geht um die europaweite Perspektive der wirtschaftlichen Existenzvernichtung. Im Sinne des transnationalen Forschungsansatzes stehen dabei der Transfer von Erfahrungswissen und der Export von Verfolgungspraktiken sowie deren Weiterentwicklung in den besetzten Gebieten im Fokus. Auch Kollaboration und die Rolle von deutschen Unternehmen bei der Ausplünderung der europäischen Juden werden in den Blick genommen.\footnote{Vgl. ebd. S. 602-608.}

Nietzels Systematisierungsversuch wurde bisher auffallend wenig von der historischen Forschung rezipiert.\footnote{Aus Literaturrecherche und Interviews ging nicht hervor, dass Nietzels Systematik nachträglich kontrovers diskutiert oder weiterentwickelt wurde.} Lediglich der Historiker Christoph Kreutzmüller nahm 2016 darauf Bezug und ergänzte den neuesten Forschungsstand zur Vernichtung der jüdischen Gewerbetätigkeit.\footnote{Siehe Kreutzmüller 2016/2020,  URL: \url{http://docupedia.de/zg/Kreutzmueller_vernichtung_der_juedischen_Gewerbetaetigkeit_v2_de_2020.}} Auch wenn dieser eine deutliche Professionalisierung darstellt, weil erstmals unter Einbeziehung aller relevanten Forschungsstudien konzeptionell mit dem komplexen Forschungsthema auseinandergesetzt wurde, so bleibt festzuhalten, dass der Begriff ,,Arisierung'' als Untersuchungsbegriff in der historischen Forschung nach wie vor zur Anwendung kommt.\footnote{Siehe Maren Janetzko: Die ,,Arisierung'' mittelständischer jüdischer Unternehmen in Bayern 1933-1939. Ein interregionaler Vergleich, Ansbach 2012, S. 17f; Claudia Flümann: ,,... doch nicht bei uns in Krefeld!". Arisierung, Enteignung, Wiedergutmachung in der Samt- und Seidenstadt 1933-1963, Krefeld 2015, S. 13 oder jüngst bei Monika Juliane Gibas: ,,Arisierung'' der Wirtschaft in Thüringen: Das Beispiel Arnstadt, in: Schlossmuseum Arnstadt (Hrsg.): Jüdische Familien aus Arnstadt und Plaue. Katalog zur Sonderausstellung im Schlossmuseum Arnstadt, Arnstadt 2021, S. 108-148..} 

Diese Situation ist für das offene Forschungsdatenmanagement insofern problematisch, als dass sich mit ,,Arisierung'' (oder auch ,,Entjudung'') auf der technischen Ebene nicht arbeiten lässt, da eine widerspruchsfreie Abbildung und Beschreibung des unpräzisen Begriffs in Form eines Datenmodells nicht möglich ist. Eine kritische Reflexion reicht, wie es in den meisten Studien gehandhabt wird, hier nicht aus, da die technische Implementierung an sich zur Differenzierung zwingt. Als derzeit einzige Möglichkeit bietet sich an dieser Stelle der Systematisierungsversuch des Historikers Nietzel an, der in dieser Arbeit methodisch als \textbf{Taxonomie} aufgegriffen wird. Sichtbar wird damit auch, dass die Forschungsdaten zur Vernichtung der jüdischen Gewerbetätigkeit inhaltlich lediglich einen kleinen Ausschnitt aus dem Gesamtkomplex der wirtschaftlichen Existenzvernichtung der Juden im NS abbilden, diesen also nur teilweise repräsentieren. Auch wenn im Rahmen dieser Arbeit der Schwerpunkt auf der Vernichtung der jüdischen Gewerbetätigkeit liegt, wird das FDM inhaltlich offen konzipiert, damit es anschlussfähig erstens an die anderen verflochtenden Teilbereiche ist und zweitens in der Entwicklungsperspektive auch an benachbarte Forschungsfelder der Verfolgung und Vernichtung im Nationalsozialismus andocken kann. 

\paragraph{Studientyp}Ausgehend von den methodischen Herangehensweisen lassen sich zwei Typen von Studien im Forschungsfeld unterscheiden. Auf der einen Seite stehen die empirischen Studien, die Teilbereiche wie die Vernichtung der jüdischen Gewerbetätigkeit auf der Basis von Stichproben mit einer (deskriptiven) statistischen Datenanalyse ausgewertet haben. Mit dieser Methode konnten erstmals allgemeinere Aussagen zum Themenkomplex gewonnen werden. Der zweite Studientyp weist einen stark dokumentarischen Charakter auf, der sich vorwiegend in einem deskriptiven Zusammentragen von verteilten Informationen zu jüdischen Gewerbebetrieben und jüdischen Unternehmern in Form von Gedenkbüchern oder Geschichtensammlungen niedergeschlagen hat.\footnote{Nietzel hebt hier die akribisch recherchierte Sammlung zu jüdischen Unternehmen in München des Archivars und Historikers Wolfgang Selig aus dem Jahr 2004 hervor, vgl. Nietzel 2009, S. 583.}

Demzufolge existieren zwei Arten von Forschungsdaten zur Vernichtung der jüdischen Gewerbetätigkeit:

\begin{enumerate}
    \item Im ersten Studientyp handelt es sich um \textbf{quantitative Massendaten}, die strukturiert, entweder als Rohdaten oder in aggregierter Form, vorliegen. Sie besitzen eine statistische Aussagekraft.
    \item Im zweiten Studientyp handelt es sich um \textbf{qualitative Daten}, die in der Regel meist textuell und damit unstrukturiert vorliegen.
\end{enumerate}

Der Schwerpunkt in dieser Arbeit liegt auf den mehrheitlich quantitativen Daten. Dennoch wird versucht, auch die textuellen Daten mit in das Forschungsdatenmanagement einzubeziehen. Denn bisher ist es der Forschung nicht gelungen, strukturierte und unstruktierte Daten zu verknüpfen. Die textuellen Daten waren für eine wissenschaftlich analytische Auswertung bislang zu unsystematisch.\footnote{Ebd.} Umgekehrt fehlt den statistischen Daten ihres Umfang wegens oft die Bedeutung, das heißt die Einzelschicksale und -geschichten hinter den Daten bleiben unsichtbar.\footnote{Allein für Berlin hat die Stichprobe einen Umfang von ca. 8.000 jüdischen Gewerbebetrieben. Auch für Frankfurt am Main sind es in der Stichprobe über 2.500 jüdische Gewerbebtriebe. Vgl. Kreutzmüller 2012, URL: \url{https://www2.hu-berlin.de/djgb/www/find} (letzter Zugriff am 07.05.2022) und Nietzel 2012, S. 15.} Sie sind damit vor allem außerhalb der wissenschaftlichen Forschung weniger greifbar.

\paragraph{Ausrichtung}Im Forschungsfeld dominieren lokal- bzw. regionalgeschichtliche Studien. Zwar wurde das Thema auch in Form von Überblicks- oder Gesamtdarstellungen zum Deutschen Reich (in den Grenzen von 1937) abgehandelt, dies jedoch nur vereinzelt und vor allem in den Anfangsjahren der wissenschaftlichen Auseinandersetzung mit dem Thema.\footnote{Siehe zum Beispiel die bereits erwähnten grundlegenden Studien von Genschel 1966 und Barkai 1987. Danach erschienen sind noch: Günter
Plum, Wirtschaft und Erwerbsleben, in: Wolfgang Benz (Hrsg.), Die Juden in Deutschland 1933–
1945. Leben unter nationalsozialistischer Herrschaft, München 1988, S. 268–313. Dieter Ziegler, Die wirtschaftliche
Verfolgung der Juden im »Dritten Reich«, in: Heinz-Jürgen Priamus (Hrsg.), Was die
Nationalsozialisten ,,Arisierung'' nannten. Wirtschaftsverbrechen in Gelsenkirchen während des
»Dritten Reiches«, Essen 2007, S. 17–40. Für die Literaturanalyse wurden vier Überblicks- bzw. Gesamtdarstellungen und fünfzehn Lokalstudien erfasst. Es ist natürlich nicht auszuschließen, dass es mehr Darstellungen zum Deutschen Reich oder zu Europa gibt, aber eine Tendenz im Forschungsfeld hin zu lokalhistorischen Studien ist nichtsdestotrotz deutlich erkennbar.} In den letzten fünfzehn Jahren sind überwiegend Untersuchungen zu Klein- und Großstädten erschienen, deren Ergebnisse ebenfalls vereinzelt in Form von Sammelbänden zusammengefasst wurden.\footnote{Siehe zum Beispiel Christiane Fritsche u.a (Hrsg.), ,,Arisierung'' und ,,Wiedergutmachung'' in deutschen Städten, Köln 2014. Allerdings handelt es sich dabei um einen ,,partikularistischen Zugriff'' auf das Thema, dessen Stärken vor allem in der zusammenfassenden Darstellung der aktuellen Forschungsergebnisse liegt als im Generieren neuer Erkenntnisse. Siehe Rezension dazu: Jan Schleusener: Rezension zu: Fritsche, Christiane; Paulmann, Johannes (Hrsg.), ,,Arisierung'' und ,,Wiedergutmachung'' in deutschen Städten, Köln  2014. ISBN 978-3-412-22160-7, In: H-Soz-Kult, 10.12.2014, \url{www.hsozkult.de/publicationreview/id/reb-21747.}..} Diese Entwicklung hat zwei Gründe:

Da sich die historische Forschung zum Thema, wie oben erläutert, früh auf die Vernichtung der jüdischen Gewerbetätigkeit in Deutschland konzentriert hat, ist sie wissenschaftlich begründet. Denn jene erfolgte erst ab 1938 mit der Einführung reichsweiter Gesetze und Regelungen.\footnote{Darunter fiel auch die antisemitische Definition, was unter einem "jüdischen Gewerbebetrieb" verstanden werden sollte.} Das heißt, dass die jüdische Gewerbetätigkeit für die nationalsozialistische Wirtschaftspolitik erst spät auf dem Plan stand.\footnote{Vgl. Nietzel 2009, S. 562, 565 und 576.} Anders sah es hingegen in der politischen Peripherie aus, wo bereits ab 1933 mit den Aprilboykotten jüdische Gewerbebetriebe gezielt verfolgt wurden und in deren Folge jüdische Gewerbebetriebe verschwanden. Es waren insbesondere also lokale Akteure gewesen, die den Vernichtungsprozess vorangetrieben hatten. Auch nach 1938 waren sie es, die die reichsweiten Gesetze und Bestimmungen umsetzten. Es ist daher wenig überraschend, dass die Wissenschaft überwiegend den lokalhistorischen Zugang gewählt hat, da in einer Überblicksdarstellung für Deutschland die Vernichtung der jüdischen Gewerbetätigkeit unmöglich in der notwendigen Dichte beschrieben und rekonstruiert werden kann.\footnote{Programmatisch war hier wieder die Lokalstudie zu Hamburg von Frank Bajohr Ende der neunziger Jahre. Siehe Bajohr 1997/98..} 

Neben der wissenschaftlichen Begründung\footnote{Siehe Bajohr 1997, S. 12f., Rappl 2000, S. 123f., Nietzel 2009, S. 17}, wird im Forschungsfeld seltener reflektiert, dass viele Forschungsprojekte dem Bereich der lokalen, insbesondere der städtischen Gedenk- und Erinnerungskultur entsprungen sind, was zur lokalgeschichtlichen Dominanz im Forschungsfeld beigetragen hat.\footnote{Siehe zum Beispiel das Netzwerk ,,Jüdisches Leben Erfurt'', Informationen zu jüdischen Unternehmen in Erfurt zusammenträgt, URL: \url{https://juedisches-leben.erfurt.de/jl/de/19jh/jgemeinde/junternehmen/index.html}. Bisher erschienen ist daraus die Miniatur von Christoph Kreutzmüller, Eckart Schörle (Hg.): Stadtluft macht frei? Jüdische Gewerbebetriebe in Erfurt 1919 bis 1939, Berlin 2013. Das Jüdische Museum Berlin (JMB) hat im Jahr 2020 die Citizen Science Plattform ,,Jewish Places'' online geschalten, auf der Orte zu jüdischem Leben europaweit kollaborativ gesammelt werden können, darunter auch Gewerbe, URL: \url{https://www.jewish-places.de/map?term=&filter[type][0]=facility&filter[facility_category_facet][0]=Gewerbe~Geschäft&filter[location][center]=52.829120842815996,13.830385954234998&rows=100000}. (alle letzter Zugriff am 07.05.2022). Oft sind Informationen zu jüdischen Gewerbebetrieben und Unternehmern in Form von Gedenkbüchern gesammelt erschienen, siehe zum Beispiel: Wolfram Selig: ,,Arisierung'' in München. Die Vernichtung jüdischer Existenz 1937-1939, München 2004.} Als Erklärungsansatz für diese besondere Entwicklung sind die gesellschaftlichen Auf- und Umbruchszeiten der 1980er Jahre plausibel. In der Tradition der basisdemokratischen und dezentralen Graswurzelbegewegung (,,Grabe, wo du stehst'')\footnote{Programmatisch war das gleichnamige Handbuch des schwedischen Literaturhistorikers Sven Lindqvist aus dem Jahr 1978, deutsch 1989: Grabe wo du stehst. Handbuch zur Erforschung der eigenen Geschichte, Bonn 1989.} mit der Etablierung zahlreicher lokaler Geschichtswerkstätten ab Anfang der 1980er Jahre in der BRD war die Motivation verbunden, die nationalsozialistische Geschichte des eigenen Ortes kritisch aufzuarbeiten.\footnote{Siehe zur Geschichte und zum Einfluss der Bewegung: Jenny Wüstenberg, Zivilgesellschaft und Erinnerungspolitik in Deutschland seit 1945, Berlin Münster 2020, Kapitel 4 Grabe, wo stehst: Die Geschichtsbewegung und die Graswurzel-Erinnerungskultur S. 147-200 und Kapitel 5 Memorialästhetik und die Erinnerungsbewegungen der 1980er, S. 201-230.} Ab Mitte der 80er Jahre rückten zunehmend die jüdischen Opfer ins Bewusstsein und es stand ein angemessenes, innovatives Gedenken sowie die Schaffung von Gedenkorten im Fokus.\footnote{Das bekannteste Projekt ist wahrscheinlich das Stolperstein-Projekt des Künstlers Gunther Demnig. Vgl. Wüstenberg 2020, S. 209. Die erste Verlegung in Berlin-Kreuzberg im Jahr 1996 war von den Behörden noch nicht genehmigt worden und wurde erst später legalisiert. Siehe Projektwebsite, URl: \url{http://www.stolpersteine.eu/start/} (Letzter Zugriff am 26.01.2022).} Die Historiker Thomas Lindenberger und Michael Wildt, beide zum damaligen Zeitpunkt sowohl akademisch tätig als auch in Geschichtswerkstätten aktiv, haben bereits im Jahr 1989 die Bedeutung der von den Geschichtswerkstätten praktizierten ,,lokalen Feldforschung'' zur Freilegung von Spuren und Zeugnissen jüdischen Lebens als mikrohistorischen Zugriff auf die Vergangenheit für die historische Forschung herausgearbeitet.\footnote{Thomas Lindenberger, Michael Wildt: Radikale Pluralität. Geschichtswerkstätten als praktische Wissenschaftskritik, in: Friedrich-Ebert-Stiftung (Hrsg.), Archiv für Sozialgeschichte, Band 29, Bonn 1989, S. 393-411 (hier S. 395), URL (stable): \url{http://library.fes.de/jportal/receive/jportal_jparticle_00013422}.} Es waren und sind also vor allem auch diese zivilgesellschaftlichen Akteure, die akribisch Informationen zu jüdischen Personen, Geschäften und anderen Orten aus unterschiedlichen Quellen zusammengetragen und veröffentlicht haben.

In Bezug auf die hier betrachteten Forschungsdaten sowie das Forschungsdatenmanagement kristallisieren sich abschließend zwei Feststellungen heraus:

Erstens sind die Forschungsdaten zur jüdischen Gewerbetätigkeit und darüber hinaus nicht ausschließlich im akademischen Umfeld entstanden, sondern gleichermaßen abseits der traditionellen Wissenschaft aus unterschiedlichsten öffentlichen Aktivitäten hervorgegangen. Es waren die Akteure der Basisbewegungen, die von einem emanzipatorischen (,,Geschichte von unten''), einem aufklärerischem (Lernen aus der Geschichte) sowie einem moralischen (Vergangenheit nicht vergessen) Antrieb geleitet waren und die etablierte Geschichtsforschung und Erinnerungspolitik durch Demokratisierung von unten und Pluralismus von Grund auf verändern wollten.\footnote{Diese Entwicklung hatte natürlich auch Auswirkung auf die akademische Geschichtswissenschaft, die sich von einer sozialhistorischen Ausrichtung hin zu einer \textit{Alltagsgeschichte}, als neuen Forschungsansatz, weiterentwickelte. Siehe dazu Lindenberg/ Wildt 1989, S. 393f., 405-409.} Lindenberg und Wildt sprechen in Bezug auf die Praxis der Geschichtswerkstätten schon 1989 von ,,öffentlicher Wissenschaft''\footnote{Lindenberg/ Wildt 1989, S. 394.} und zitieren jene mit:

\begin{quote}
    Wir beanspruchen, unsere Projekte für jede/n - ob ,wissenschaftlich' ausgebildet oder nicht - offen zu halten. Das Interesse am Gegenstand, an der gemeinsamen Auseinandersetzung mit der Vergangenheit im jeweiligen Projekt, sind entscheidend.\footnote{Ebd.}
\end{quote}

Damit wird sehr deutlich, dass der historischen Forschung die von der Open Science-Bewegung eingeforderte Offenheit im Sinne der Partizipation an Wissenschaft keinesfalls fremd ist, sondern im Gegenteil bereits über Jahrzehnte praktiziert wird. Gerade in dieser Überschneidung sind die Anknüpfungspunkte von offenem Forschungsdatenmanagement an das Forschungsfeld offenkundig.

Zweitens konstatierte Nietzel in seinem Literaturbericht von 2009, dass die einzelnen Lokalstudien gegenseitig kaum Kenntnis voneinander genommen haben und bisher mehrheitlich nebeneinander stehen als sich aufeinander zu beziehen.\footnote{Die einzige vergleichend angelegte Studie, allerdings nur auf regionaler Ebene, stammt aus dem Jahr 2012 von der Historikerin Maren Janetzko, erschien also nach Nietzels Literaturbericht. Vgl. Nietzel 2009, S. 562. Janetzko, Die ,,Arisierung'' Mittelständischer jüdischer Unternehmen in Bayern 1933-1939. Ein interregionaler Vergleich 2012.} Wenn man also im Forschungsfeld von geografisch abgekapselten Studien sprechen kann, dann gilt dies auch für die zugehörigen Forschungsdaten. Damit bleiben Aussagen zum Vernichtungsprozess über lokale/ regionale Grenzen hinaus bisher also ebenfalls noch begrenzt. Ein projektübergreifendes Forschungsdatenmanagement könnte hier erstmals die Möglichkeit eröffnen, zumindest auf der Datenebene die Lokalstudien für komparastische Studien oder für die in den Interviews vorwiegend genannte aber noch ausstehende Synthese zusammenzuführen.

\subsection{Datenkritik}
Systematische und globale Aussagen zur Vernichtung der jüdischen Gewerbetätigkeit halten, wie Nietzel 2009 für das Forschungsfeld betonte\footnote{Nietzel 2009, S. 585.} nur auf einer validen empirischen Basis stand und können nicht anhand von Einzelfällen getroffen werden.\footnote{Nietzel 2012, S. 154.} Das bedeuetet, dass ein statistisch-quantifizierender Methodenteil im Forschungsfeld unerlässlich bei der Erkenntnisgenerierung ist.\footnote{Die Betonung liegt hier auf \textit{Teil}, da - wie Nietzel und andere auch reflektieren, einer ausschließlich statistischen Herangehensweise im Forschungsfeld klar Grenzen gesetzt sind. Siehe Nietzel 2012, S. 154 und Nietzel 2009, S. 582.}, an den sich die Fragen nach Qualität und Reliabilität von den in Kapitel 2 erläuterten Replikationsstudien ebenso stellen lassen. Explizit soll hiermit keine Replikationsstudie entstehen. Aber mit den Open Science-Grundsätzen im Allgemeinen sowie mit FAIR Data/ Open Data im Besonderen gibt es ein Best Practice-Instrumentarium, mit dem sich der bisherige Umgang mit Forschungsdaten im Forschungsfeld untersuchen und bewerten lässt, woraus sich wiederum Anforderungen für das Forschungsdatenmanagement ergeben. Mit dieser Ausrichtung wird das Forschungsfeld systematisch nach folgenden Fragestellungen untersucht:

\begin{itemize}
\item Wie sehen die quantitativen Forschungsdesigns der Studien aus?
\item Wie wurden die Daten erfasst?
\item Welche Daten wurden erfasst?
\item Wie wurden die Daten ausgewertet und ggf. visualisiert?
\item In welcher Form sind die Daten veröffentlicht?
\item Stehen die Daten langfristig zur Verfügung?
\item In welcher Form können die Daten nachgenutzt werden?
\end{itemize}



\begin{table}
\caption{Metaangaben zu den ausgewählten Lokalstudien}
\label{tab:lokalstudientabelle}
\begin{tabular} { L{0.5cm}|L{2cm}|L{3cm}|L{1.5cm}|L{5cm}|L{2cm} }
Nr. & Ort & Erscheinungsjahr & Sample & Hauptquelle & Untersuchungszeitraum \\
\hline 
1 & Hamburg & 1997 & 300 & Wiedergutmachungsakten & 1933-45 \\
\hline 
2 & München & 2000 & 720 & Gewerbekartei & 1934-39 \\
\hline 
5 & Berlin & 2012 & 8.012 & Handelsregister & 1930-45 \\
\hline
6 & Frankfurt am Main & 2012 & 2.600 & Handelsregister & 1924-64 \\
\hline
7 & 4 Städte in Bayern & 2012 & insg. 529 & n.n. & 1933-39 \\
\hline
8 & Mannheim & 2013 & 1.234 & n.n & n.n. \\
\hline
9 & Krefeld & 2015 & 135 & n.n & 1933-63 \\
\end{tabular}
\end{table}
\paragraph{Forschungsziele und Forschungsdesign}
\paragraph{Datenerfassung und Datensicherung}
\paragraph{Datenauswertung}
\paragraph{Datenveröffentlichung}




\section{Auswertung der Interviews}
Kollaboratives Arbeiten auf den Daten
Tools zur Datenauswertung
Nutzung projektübergreifender Datensammlungen 

\subsection{Stakeholder}
Recherche
Stakeholderanalyse
- Aufzeigen und Zusammenführen der Beteiligten in einem Projekt
- Systematisches Bewerten und Ordnen der Beteiligten nach Interessen, Macht und Rolle
- Zielbestimmung entsprechend der Interessenbeteiligten
Eine strikte Trennung in akademisch einerseits und nichtakademisch andererseits ist nicht sinnvoll, da sich beide Bereiche in der Vergangenheit gegenseitig bedingten und befruchteten.\footnote{Zum Verhältnis von akademischer und nichtakademischer historischer Forschung vgl. Wüstenberg 2020, S. 163ff. Überschneidungen gab es vor allem bei beim Organisieren auf personeller Ebene.} Das bedeutet, dass offenes FDM potentiell multiple Anwendergruppen ansprechen wird, deren Nutzungsmotive und Nutzungserwartungen sich erheblich unterscheiden können. Es ist unmöglich, alle Anwender*innen in dieser Arbeit gleichermaßen zu berücksichtigen
\paragraph{Wissenschaftler}

akademisch an Universitäten, nicht-akademisch z.B. an Erinnerungseinrichtungen wie Archive, aber auch Historiker*innen, die nicht universitär angebunden sind (Claudia Fürmann)


\paragraph{Akteure der Erinnerungskultur}

z.B. Stolperstein-Initiativen sehr von moralischen Impetus geleitet, ehrenamtliche

\paragraph{Familienangehörige}

\subsection{Funktionsmodell}

siehe \url{https://en.wikipedia.org/wiki/Feature_model}


\subsection{Rechtliche und ethische Implikationen}


\subsection{Open Science Grad}
