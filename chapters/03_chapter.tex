\onehalfspacing

\section{Einordnung der Forschungsdaten} 

Inhaltlich sind die hier exemplarisch betrachteten Forschungsdaten zur Vernichtung der jüdischen Gewerbetätigkeit in den größeren Themenkomplex der wirtschaftlichen Verfolgung, Verdrängung und Vernichtung der Juden im Nationalsozialismus eingebettet. Die ersten grundlegenden, wissenschaftlichen Auseinandersetzungen dazu erfolgten zwar schon früh in der BRD im Nachkriegsdeutschland.\footnote{Im Jahr 1966 erschien die Pionierstudie von Helmut Genschel. Erst 20 Jahre später folgte die nächste grundlegende Studie des israelischen Historikers Avraham Barkai, der an Gentschels Ergebnisse anknüpfte. Vgl. Benno Nietzel: Die Vernichtung der wirtschaftlichen Existenz der deutschen Juden 1933-1945. Ein Literatur und Forschungsbericht, in: Friedrich-Ebert-Stiftung (Hg.), Archiv für Sozialgeschichte, Band 49, Bonn 2009, S. 561-613.} Allerdings blieben diese vereinzelt und ohne größere Resonanz. Erst Ende der 1990er Jahren trat in Deutschland eine längere Forschungswelle zum Thema auf, die eine Bandbreite an Studien hervorgebracht hat. In deren Folge etablierte sich ein eigenes Forschungsfeld zur wirtschaftlichen Existenzvernichtung der Juden im Nationalsozialismus, in dem vor allem lokal- und regionalgeschichtliche Zugänge dominieren.\footnote{Als wegweisend wird regelmäßig die Lokalstudie zu Arisierung in Hamburg des Historikers Frank Bajohr aus dem Jahr 1997/98 gewertet. Siehe zum Beispiel Nietzel 2009, S. 561 oder Christiane Fritsche: Ausgeplündert, zurückerstattet und entschädigt. Arisierung und Wiedergutmachung in Mannheim, 2. Aufl., Ubstadt-Weiher, Heidelberg, Neustadt a. d. W., Basel 2013, S. 21. Frank Bajohr: ,,Arisierung'' in Hamburg. Die Verdrängung der jüdischen Unternehmer 1933-1945, 2. Aufl., Hamburg 1998 (zuerst 1997). Auf Ursachen des Forschungsbooms kann im Rahmen dieser Arbeit nicht eingegangen werden. Siehe dazu auch Christoph Kreutzmüller, Vernichtung der jüdischen Gewerbetätigkeit im Nationalsozialismus. Abläufe, Blickwinkel und Begrifflichkeiten, Version: 2.0, in: Docupedia-Zeitgeschichte, 12.3.2020, URL: \url{http://docupedia.de/zg/Kreutzmueller_vernichtung_der_juedischen_Gewerbetaetigkeit_v2_de_2020}} Es lieferte innerhalb der NS-Forschung weitere Erklärungsansätze zur antisemitischen Verfolgungs- und Vernichtungspolitik, deren Antriebskräfte in der Vergangenheit unterschiedlich interpretiert wurden.\footnote{Siehe zu den unterschiedlichen Deutungen und Perspektiven (insbesondere Intentionalismus vs. Strukturalismus) Bajohr 1998, S. 10-14} Hierbei waren lange nationalsozialistische Akteure, kommunale Verwaltungsinstanzen und nicht-jüdische Nutznießer sowie deren Strategien, Verhalten und Handlungsoptionen Schwerpunkt der Forschung. Diese Fokussierung wurde in zunehmendem Maß als zu einseitig kritisiert, da insbesondere die jüdischen Betroffenen ganz ausgeblendet oder sie ausschließlich als passive Opfer gezeigt worden seien. Zudem entwickelte sich langsam ein wissenschaftlicher Diskurs über die Anwendung historischer Begrifflichkeiten in der Forschung.\footnote{Vgl. Ludolf Herbst, Christoph Kreutzmüller, Ingo Loose u.a., Einleitung, in: Ludolf Herbst, Christoph Kreutzmüller, Thomas Weihe (Hg.): Die Commerzbank und die Juden 1933-1945, München 2004, S. 10-13. Diese Selbstkritik war ohne Zweifel richtig und auch notwendig, da sie grundlegende konzeptionelle Probleme im Forschungsfeld aufdeckte. Dennoch ist die einseitige Perspektive auf Täter, Mittäter und Mitwisser vor dem Hintergrund des jahrzehntelangen Verdrängens in der deutschen Nachkriegs- und Tätergesellschaft bis hin zu Geschichtsrevisionismus und Opfer-Umkehrung ein verständliches Anliegen gewesen. Letztlich leistete die Geschichtswissenschaft damit zwar einen späten aber nicht weniger wichtigen Beitrag zur historischen Aufarbeitung der NS-Verbrechen.} Im Zentrum stand hierbei die Kritik, dass die meisten Studien die Bandbreite und Komplexität des Forschungsthemas unter dem diffusen Begriff ,,Arisierung'' untersuchten und diesen dabei unterschiedlich ausdehnten.\footnote{Vgl. Nietzel 2009, S. 562-565. Mitunter wird der Begriff bis in die Zwangsarbeit hinein ausgeweitet. Siehe Britta Bopf: ,,Arisierung'' in Köln. Die wirtschaftliche Existenzvernichtung der Juden 1933-1945, Köln 2004, S. 11.} Häufig lag der Schwerpunkt der Untersuchung jedoch auf jüdischen Unternehmern und der Übernahme deren Eigentums\footnote{Siehe zum Beispiel Barbara Händler-Lachmann/Thomas Werther: Vergessene Geschäfte, verlorene Geschichte. Jüdisches Wirtschaftsleben in Marburg und seine Vernichtung im Nationalsozialismus, Marburg 1992; Alex Bruns-Wüstefeld: Lohnende Geschäfte. Die ,,Entjudung'' der Wirtschaft am Beispiel Göttingens, Hannover 1997; Bajohr 1997/98, Einleitung, S. 9f.; Marian Rappl: ,,Arisierung'' in München. Die Verdrängung der jüdischen Gewerbetreibenden aus dem Wirtschaftsleben der Stadt 1933-1939, in: Kommission für bayerische Landesgeschichte bei der Bayerischen Akademie der Wissenschaften in Verbindung mit der Gesellschaft für fränkische Geschichte und der Schwäbischen Forschungsgemeinschaft (Hrsg.), Zeitschrift für bayerische Landesgeschichte, Bd. 63, Heft 1, München 2000, S. 82-123, hier S. 125; Heinz-Jürgen Priamus (Hrsg.): Was die Nationalsozialisten ,,Arisierung'' nannten. Wirtschaftsverbrechen in Gelsenkirchen während des ,,Dritten Reiches'', Essen 2007, S. 11ff.}, wodurch die historische Forschung zuweilen Schlagseite erlitt, da andere Aspekte der wirtschaftlichen Existenzvernichtung wie zum Beispiel die Verdrängung von Juden aus ihren Berufen unterbelichtet blieben.\footnote{Vgl. Nietzel 2009, S. 565.} Zusammengefasst war der Einwand, dass die bisher verwendeten Untersuchungsbegriffe ,,engführend''\footnote{Kreutzmüller 2016/2020,  URL: \url{http://docupedia.de/zg/Kreutzmueller_vernichtung_der_juedischen_Gewerbetaetigkeit_v2_de_2020.}} dahingehend seien, das Geschehene nur einseitig zu rekonstruieren, zu dessen gesamtheitlicher Analyse folglich nicht taugen.\footnote{Vgl. Nietzel 2009, S. 564 und Herbst/Weihe, Commerzbank, 2004, S. 10ff..}

Ab Mitte der 2000er Jahre lässt sich daraufhin eine Weiterentwicklung beobachten, die vor allem von größeren universitären Forschungsprojekten vorangetrieben wurde und die mit der Verschiebung in der Forschungsperspektive sowie der begrifflichen Ausdifferenzierung einher ging.\footnote{Pionierarbeit leistet hier u.a. das Forschungsprojekt ,,Geschichte der Commerzbank von 1870 bis 1958'' am Lehrstuhl für Zeitgeschichte an der Humboldt-Universität zu Berlin unter Leitung von Prof. Dr. Ludolf Herbst sowie das Forschungsprojekt zur Vernichtung der jüdischen Gewerbetätigkeit im Nationalsozialismus in den drei Großstädten Berlin, Breslau, Frankfurt am Main, ebendort. Siehe Ludolf Herbst/Thomas Weihe (Hg.), Die Commerzbank und die Juden 1933-1945, München 2004; Christoph Kreutzmüller, Ausverkauf. Die Vernichtung der jüdischen Gewerbetätigkeit in Berlin 1930-45, Berlin 2012; Benno Nietzel, Handeln und Überleben: jüdische Unternehmer aus Frankfurt am Main 1924-1964, Göttingen 2012} Die neueren Studien unterschieden sich im Wesentlichen dadurch, dass sie die jüdischen Betroffenen als handelnde Akteure begriffen und deren \textit{agency} in den Blick nahmen. Außerdem versuchten sie erstmals mit den Begriffen ,,Arisierung'' oder ,,Entjudung'' zu brechen\footnote{Unwissenschaftlich insofern, als dass es sich um rassistisch konnotierte Begriffe handelt, die selbst eigentlich zu historisieren wären, anstatt diese in die Wissenschaftssprache aufzunehmen. Vgl. Nietzel 2009, S. 563.} und Phänomene des Forschungsthema durch eine wissenschaftliche Terminologie zu benennen. Dabei wurde ein prozessorientierter Zugang gewählt, der an die Holocaust-Forschung des US-amerikanischen Historikers Raul Hilberg anknüpfte. Hilberg analysierte den Massenmord an den Juden wegweisend als einen Prozess, der über Definition, Kennzeichnung, Enteignung, Konzentration und Mord mehrstufig verlief.\footnote{Raul Hilberg: Die Vernichtung der europäischen Juden, Band 1, Frankfurt am Main 1990 (zuerst englisch 1961), S. 85-163. Eine wichtige Ergänzung zu Hilbergs Thesen war, dass die wirtschaftliche Existenzvernichtung der Juden der Teilprozess, war, der ,,am längsten – nämlich über den Tod der Opfer hinaus – dauerte und demzufolge in alle anderen Prozesse hineinreichte''. Kreutzmüller 2012, S. 378.} Als integraler Bestandteil dieses Prozesses wurde die Vernichtung der wirtschaftlichen Existenz der Juden im Nationalsozialismus als ein mehrschichtiger Gesamtprozess analysiert, der sich aus den abgrenzbaren, aber überlagernden und in Wechselbeziehung stehenden Teilprozessen Verdrängung, Besitztransfer, Liquidation und Vermögensentzug zusammensetzte. Diese schlossen folglich die Verdrängung der Juden aus dem Berufsleben, die Vernichtung der jüdischen Gewerbetätigkeit durch Besitzübernahme oder Liquidation sowie die Entziehung des Vermögens der Juden ein.\footnote{Exemplarisch wurden erstmals alle Teilprozesse systematisch im Rahmen der Erforschung der Geschichte der Commerzbank betrachtet. Siehe Herbst/Weihe, Commerzbank, 2004.}

Mit diesem Forschungsansatz konnte zum einen anhand der drei deutschen Großstädte Berlin, Frankfurt am Main und Breslau empirisch  gezeigt werden, dass die als jüdisch verfolgten Unternehmen nicht - wie bisher durch die Schwerpunktsetzung der historischen Forschung suggeriert - größtenteils in den Besitz nichtjüdischer Erwerber*innen übergingen, sondern schlichtweg liquidiert wurden.\footnote{Vgl. Kreutzmüller 2016/2020.} Diesbezüglich lag der Erkenntnisfortschritt in der Freilegung des Teilprozess der Vernichtung der jüdischen Gewerbetätigkeit als ein ,,großangelegtes Liquidationsprogramm'', das bisher kaum als solches von der historischen Forschung reflektiert worden war.\footnote{Vgl. Nietzel 2012, S. 164 und Kreutzmüller 2012, S. 250.} Des Weiteren wurde durch den Wechsel der Forschungsperspektive systematisch herausgearbeitet, dass sich die jüdischen Betroffenen gegen ihre Entrechtung wehrten und dazu verschiedenen institutionelle wie individuelle Strategien nutzten.\footnote{Systematisch untersucht von Kreutzmüller, Ausverkauf, 2012, Kapitel IV. Abwehrstrategien jüdischer Gewerbetreibender, S. 257-357; Nietzel, Handeln und Überleben, 2012, Kapitel II.2 Erwartungen, Anpassung und Selbstbehauptung, S. 99-150.}

An diesen Forschungsstand anknüpfend unternahm zuletzt der Historiker Benno Nietzel im Jahr 2009 den Versuch, die zahlreichen Forschungsstudien zur Vernichtung der wirtschaftlichen Existenz der Juden im Nationalsozialismus zu ordnen, indem er die bisherigen Forschungsfragen, Untersuchungsgegenstände sowie Forschungsergebnisse zusammenfasste und strukturierte. Er diagnostizierte dem Forschungsfeld im Großen und Ganzen weiterhin methodisch-konzeptionelle Probleme aufgrund undifferenzierter Zugänge\footnote{Vgl. ebd. S. 562-565.} und folglich eine ,,analytische Hilflosigkeit angesichts der Vielschichtigkeit und Komplexität des Prozesses [der wirtschaftlichen Existenzvernichtung der Juden, Anm. S.E.]'', die Erkenntnisfortschritt im Forschungsfeld hemmen.\footnote{Ebd. S. 564.}

\section{Kriterien des offenen Forschungsdatenmanagements}

Nachdem der historiographische Kontext der Forschungsdaten zu jüdischen Gewerbebetrieben klar ist, können darauf aufbauend die drei Kriterien ,,anschlussfähig'', ,,projektübergreifend'' und ,,partizipativ'' entwickelt werden, welche die Anknüpfungspunkte für Open Science-Ansätze darstellen und das \textit{offene} Forschungsdatenmanagement im Forschungsfeld spezifizieren.

\subsection{Anschlussfähig}

Wenn die wirtschaftliche Existenzvernichtung der Juden als ein abgrenzbares Forschungsfeld definiert ist, dann lässt es sich folglich für eine differenzierte Unterschung abstecken. Nach Nietzel kann dies in fünf Teilbereichen erfolgen:\footnote{Vgl. Nietzel 2009, S. 562.}
\begin{itemize}
\item Verdrängung der Juden aus dem Berufsleben (Angestellte, Beamte, Selbstständige wie Rechtsanwälte, Ärzte oder Wissenschaftler)
\item Vernichtung der jüdischen Gewerbetätigkeit (Besitztransfer und Liquidation)
\item staatliche Enteignung des jüdischen Vermögens (Privatbesitz, Firmenvermögen, Immobilienvermögen aus Grundbesitz) 
\item Entgrenzung (transnationale Perspektiven)
\item Wiedergutmachung nach 1945 in der BRD
\end{itemize}

Zwar betonte Nietzel deren überschneidende Beziehungen und Verhältnisse zueinander, nahm aber in erster Linie eine separierte Betrachtung zum Zwecke der inhaltlichen Erschließung und zur Herausarbeitung von Spezifika des Forschungsthemas vor.\footnote{Nietzel 2009, S. 562. Nietzel greift außerdem die Beteiligung von nichtjüdischen Unternehmen mit auf aber explizit nicht als eine eigene Kategorie sondern als Querschnittaspekt, weshalb dieser hier nicht berücksichtigt wird, da er strenggenommen zum Forschungsfeld der Unternehmensgeschichte gehört. Siehe zu Unternehmensgeschichte Ralf Ahrens, Unternehmensgeschichte, Version: 1.0, in: Docupedia-Zeitgeschichte, 1.11.2010, URL: \url{http://docupedia.de/zg/Ahrens_unternehmensgeschichte_v1_de_2010.}.} 

Neben den bereits erläuterten Teilprozessen ordnete Nietzel dem Forschungsfeld außerdem die historisch untrennbare materielle Wiedergutmachung nach 1945 in der BRD zu, welche zum einen die Restitution/ Rückerstattung und zum anderen die Entschädigung meint. Hiervon ausgenommen ist die Entziehung und die Restitution von Kulturgütern, die Nietzel dem eigenen Forschungsfeld der Provenienzforschung zuordnete.\footnote{Vgl. ebd. S. 273.} Im Falle der Entgrenzung vor allem nach Kriegsbeginn geht um die europaweite Perspektive der wirtschaftlichen Existenzvernichtung. Im Sinne des transnationalen Forschungsansatzes stehen dabei der Transfer von Erfahrungswissen und der Export von Verfolgungspraktiken sowie deren Weiterentwicklung in den besetzten Gebieten im Fokus. Auch Kollaboration und die Rolle von deutschen Unternehmen bei der Ausplünderung der europäischen Juden werden in den Blick genommen.\footnote{Vgl. ebd. S. 602-608.}

Nietzels Systematisierungsversuch wurde bisher auffallend wenig von der historischen Forschung rezipiert.\footnote{Aus Literaturrecherche und Interviews ging nicht hervor, dass Nietzels Systematik nachträglich kontrovers diskutiert oder weiterentwickelt wurde.} Lediglich der Historiker Christoph Kreutzmüller nahm 2016 darauf Bezug und ergänzte den neuesten Forschungsstand zur Vernichtung der jüdischen Gewerbetätigkeit.\footnote{Siehe Kreutzmüller 2016/2020,  URL: \url{http://docupedia.de/zg/Kreutzmueller_vernichtung_der_juedischen_Gewerbetaetigkeit_v2_de_2020.}} Auch wenn dieser eine deutliche Professionalisierung darstellt, weil erstmals unter Einbeziehung aller relevanten Forschungsstudien konzeptionell mit dem komplexen Forschungsthema auseinandergesetzt wurde, so bleibt festzuhalten, dass der Begriff ,,Arisierung'' als Untersuchungsbegriff in der historischen Forschung nach wie vor zur Anwendung kommt.\footnote{Siehe Maren Janetzko: Die ,,Arisierung'' mittelständischer jüdischer Unternehmen in Bayern 1933-1939. Ein interregionaler Vergleich, Ansbach 2012, S. 17f; Claudia Flümann: ,,... doch nicht bei uns in Krefeld!". Arisierung, Enteignung, Wiedergutmachung in der Samt- und Seidenstadt 1933-1963, Krefeld 2015, S. 13 oder jüngst bei Monika Juliane Gibas: ,,Arisierung'' der Wirtschaft in Thüringen: Das Beispiel Arnstadt, in: Schlossmuseum Arnstadt (Hrsg.): Jüdische Familien aus Arnstadt und Plaue. Katalog zur Sonderausstellung im Schlossmuseum Arnstadt, Arnstadt 2021, S. 108-148..} 

Diese Situation ist für das offene Forschungsdatenmanagement insofern problematisch, als dass sich mit ,,Arisierung'' (oder auch ,,Entjudung'') auf der technischen Ebene nicht arbeiten lässt, da eine widerspruchsfreie Abbildung und Beschreibung des unpräzisen Begriffs in Form eines Datenmodells nicht möglich ist. Eine kritische Reflexion reicht, wie es in den meisten Studien gehandhabt wird, hier nicht aus, da die technische Implementierung an sich zur Differenzierung zwingt. Als derzeit einzige Möglichkeit bietet sich an dieser Stelle der Systematisierungsversuch des Historikers Nietzel an, der in dieser Arbeit methodisch als Taxonomie aufgegriffen wird. Sichtbar wird damit auch, dass die Forschungsdaten zu den jüdischen Gewerbebetrieben inhaltlich lediglich einen kleinen Ausschnitt aus dem Gesamtkomplex der wirtschaftlichen Existenzvernichtung der Juden im NS abbilden, diesen also nur teilweise repräsentieren. Zudem ist das zugehörige Forschungsfeld Teil der umfassenden NS-Forschung und knüpft insbesondere an die Holocaust-Forschung an.

Das Forschungsdatenmanagement ist folglich inhaltlich offen, das heißt es muss neben der Vernichtung der jüdischen Gewerbetätigkeit anschlussfähig erstens an alle angrenzenden Untersuchungbereiche im Forschungsfeld sein und muss zweitens in der Entwicklungsperspektive auch an benachbarte Forschungsfelder der Verfolgung und Vernichtung im Nationalsozialismus andocken können.\footnote{} 

\subsection{Projektübergreifend}

Im Forschungsfeld dominieren lokal- bzw. regionalgeschichtliche Studien.\footnote{Zwar wurde das Thema auch in Form von Überblicks- oder Gesamtdarstellungen zum Deutschen Reich (in den Grenzen von 1937) abgehandelt, dies jedoch nur vereinzelt und vor allem in den Anfangsjahren der wissenschaftlichen Auseinandersetzung mit dem Thema. Siehe zum Beispiel die bereits erwähnten grundlegenden Studien von Genschel 1966 und Barkai 1987. Danach erschienen sind noch: Günter
Plum, Wirtschaft und Erwerbsleben, in: Wolfgang Benz (Hrsg.), Die Juden in Deutschland 1933–
1945. Leben unter nationalsozialistischer Herrschaft, München 1988, S. 268–313. Dieter Ziegler, Die wirtschaftliche
Verfolgung der Juden im »Dritten Reich«, in: Heinz-Jürgen Priamus (Hrsg.), Was die
Nationalsozialisten ,,Arisierung'' nannten. Wirtschaftsverbrechen in Gelsenkirchen während des
»Dritten Reiches«, Essen 2007, S. 17–40. Für die Literaturanalyse wurden vier Überblicks- bzw. Gesamtdarstellungen und fünfzehn Lokalstudien erfasst. Es ist natürlich nicht auszuschließen, dass es mehr Darstellungen zum Deutschen Reich oder zu Europa gibt, aber eine Tendenz im Forschungsfeld hin zu lokalhistorischen Studien ist nichtsdestotrotz deutlich erkennbar.} Da sich die historische Forschung, wie oben erläutert, früh auf die Vernichtung der jüdischen Gewerbetätigkeit in Deutschland konzentriert hat, ist diese Entwicklung wissenschaftlich begründet. Denn die systematische Vernichtung erfolgte erst ab 1938 mit der Einführung reichsweiter Gesetze und Regelungen.\footnote{Darunter fiel auch die antisemitische Definition, was unter einem "jüdischen Gewerbebetrieb" verstanden werden sollte.} Das heißt, dass die jüdische Gewerbetätigkeit für die nationalsozialistische Wirtschaftspolitik erst spät auf dem Plan stand.\footnote{Vgl. Nietzel 2009, S. 562, 565 und 576.} Anders sah es hingegen in der politischen Peripherie aus, wo bereits ab 1933 mit den Aprilboykotten jüdische Gewerbebetriebe gezielt verfolgt wurden und in deren Folge jüdische Gewerbebetriebe verschwanden. Es waren insbesondere also lokale Akteure gewesen, die den Vernichtungsprozess vorangetrieben hatten. Auch nach 1938 waren sie es, die die reichsweiten Gesetze und Bestimmungen umsetzten. Es ist daher wenig überraschend, dass die Wissenschaft überwiegend den lokalhistorischen Zugang gewählt hat, da in einer Überblicksdarstellung für Deutschland die Vernichtung der jüdischen Gewerbetätigkeit unmöglich in der notwendigen Dichte beschrieben und rekonstruiert werden kann.\footnote{Programmatisch war hier wieder die Lokalstudie zu Hamburg von Frank Bajohr Ende der neunziger Jahre. Siehe Bajohr 1997/98.} In den letzten fünfzehn Jahren sind in diversen einzelnen lokalen Forschungsprojekten, Publikationen zu Klein- und Großstädten erschienen und erstmals auch systematisch Daten zu jüdischen Gewerbebetrieben erfasst worden. 

Aus den Interviews sowie aus Nietzels Bericht von 2009 geht jedoch hervor, dass die einzelnen Lokalstudien gegenseitig kaum Kenntnis voneinander genommen haben und bisher mehrheitlich nebeneinander stehen als sich aufeinander zu beziehen.\footnote{Die einzige vergleichend angelegte Studie, allerdings nur auf regionaler Ebene, stammt aus dem Jahr 2012 von der Historikerin Maren Janetzko, erschien also nach Nietzels Literaturbericht. Vgl. Nietzel 2009, S. 562. Janetzko, Die ,,Arisierung'' Mittelständischer jüdischer Unternehmen in Bayern 1933-1939. Ein interregionaler Vergleich 2012. Vgl. Interview B3\_Transkript: ,,[...] dass esviele Einzelstudien zur verschiedenen Städten gibt, zu Hamburg, zu München, zu Berlin ansatzweise - ist natürlich eine ganz andere Dimension in Berlin. Zu Göttingen, dann eben zu Mannheim, aber das sind ja alles so einzelne Bausteine.''.} Wenn man also im Forschungsfeld von geografisch geschlossenen Studien sprechen kann, dann gilt dies auch für die zugehörigen Forschungsdaten, welche sich deshalb als Datensilos charakterisieren lassen. Damit bleiben Aussagen zum Vernichtungsprozess über lokale/ regionale Grenzen auf der Datenebene bisher noch begrenzt. 

Um diese Isolation der Daten aufzubrechen und Datenvernetzung zu ermöglichen, muss das Forschungsdatenmanagement demnach projektübergreifend funktionieren.\footnote{Vgl. zu den Datensilos Interview B4\_Transkript: ,,[...] dass diese Vernetzungsansätze nicht nur punktuell stattfinden, weil sie dann auch wieder nur Fragment bleiben, sondern dass sie tatsächlich auch übergreifend funktionieren [...]''.}

\subsection{Partizipativ} 

Neben der wissenschaftlichen Begründung des lokalgeschichtlichen Zugangs\footnote{Siehe Bajohr 1997, S. 12f., Rappl 2000, S. 123f., Nietzel 2009, S. 17}, wird seltener reflektiret, dass viele Forschungsprojekte dem Bereich der lokalen, insbesondere der städtischen Gedenk- und Erinnerungskultur entsprungen sind, was zur lokalgeschichtlichen Dominanz im Forschungsfeld beigetragen hat.\footnote{Siehe zum Beispiel das Netzwerk ,,Jüdisches Leben Erfurt'', Informationen zu jüdischen Unternehmen in Erfurt zusammenträgt, URL: \url{https://juedisches-leben.erfurt.de/jl/de/19jh/jgemeinde/junternehmen/index.html}. Bisher erschienen ist daraus die Miniatur von Christoph Kreutzmüller, Eckart Schörle (Hg.): Stadtluft macht frei? Jüdische Gewerbebetriebe in Erfurt 1919 bis 1939, Berlin 2013. Das Jüdische Museum Berlin (JMB) hat im Jahr 2020 die Citizen Science Plattform ,,Jewish Places'' online geschalten, auf der Orte zu jüdischem Leben europaweit kollaborativ gesammelt werden können, darunter auch Gewerbe, URL: \url{https://www.jewish-places.de/map?term=&filter[type][0]=facility&filter[facility_category_facet][0]=Gewerbe~Geschäft&filter[location][center]=52.829120842815996,13.830385954234998&rows=100000}. (alle letzter Zugriff am 07.05.2022). Oft sind Informationen zu jüdischen Gewerbebetrieben und Unternehmern in Form von Gedenkbüchern gesammelt erschienen, siehe zum Beispiel: Wolfram Selig: ,,Arisierung'' in München. Die Vernichtung jüdischer Existenz 1937-1939, München 2004.} Als Erklärungsansatz für diese besondere Entwicklung sind die gesellschaftlichen Auf- und Umbruchszeiten der 1980er Jahre plausibel. In der Tradition der basisdemokratischen und dezentralen Graswurzelbegewegung (,,Grabe, wo du stehst'')\footnote{Programmatisch war das gleichnamige Handbuch des schwedischen Literaturhistorikers Sven Lindqvist aus dem Jahr 1978, deutsch 1989: Grabe wo du stehst. Handbuch zur Erforschung der eigenen Geschichte, Bonn 1989.} mit der Etablierung zahlreicher lokaler Geschichtswerkstätten ab Anfang der 1980er Jahre in der BRD war die Motivation verbunden, die nationalsozialistische Geschichte des eigenen Ortes kritisch aufzuarbeiten.\footnote{Siehe zur Geschichte und zum Einfluss der Bewegung: Jenny Wüstenberg, Zivilgesellschaft und Erinnerungspolitik in Deutschland seit 1945, Berlin Münster 2020, Kapitel 4 Grabe, wo stehst: Die Geschichtsbewegung und die Graswurzel-Erinnerungskultur S. 147-200 und Kapitel 5 Memorialästhetik und die Erinnerungsbewegungen der 1980er, S. 201-230.} Ab Mitte der 80er Jahre rückten zunehmend die jüdischen Opfer ins Bewusstsein und es stand ein angemessenes, innovatives Gedenken sowie die Schaffung von Gedenkorten im Fokus.\footnote{Das bekannteste Projekt ist wahrscheinlich das Stolperstein-Projekt des Künstlers Gunther Demnig. Vgl. Wüstenberg 2020, S. 209. Die erste Verlegung in Berlin-Kreuzberg im Jahr 1996 war von den Behörden noch nicht genehmigt worden und wurde erst später legalisiert. Siehe Projektwebsite, URl: \url{http://www.stolpersteine.eu/start/} (Letzter Zugriff am 26.01.2022).} Die Historiker Thomas Lindenberger und Michael Wildt, beide zum damaligen Zeitpunkt sowohl akademisch tätig als auch in Geschichtswerkstätten aktiv, haben bereits im Jahr 1989 die Bedeutung der von den Geschichtswerkstätten praktizierten ,,lokalen Feldforschung'' zur Freilegung von Spuren und Zeugnissen jüdischen Lebens als mikrohistorischen Zugriff auf die Vergangenheit für die historische Forschung herausgearbeitet.\footnote{Thomas Lindenberger, Michael Wildt: Radikale Pluralität. Geschichtswerkstätten als praktische Wissenschaftskritik, in: Friedrich-Ebert-Stiftung (Hrsg.), Archiv für Sozialgeschichte, Band 29, Bonn 1989, S. 393-411 (hier S. 395), URL (stable): \url{http://library.fes.de/jportal/receive/jportal_jparticle_00013422}.} Es waren und sind also vor allem auch diese zivilgesellschaftlichen Akteure, die akribisch Informationen zu jüdischen Personen, Geschäften und anderen Orten aus unterschiedlichen Quellen zusammengetragen und veröffentlicht haben.

Das bedeutet für das Forschungsdatenmanagement, dass die Forschungsdaten zur jüdischen Gewerbetätigkeit und darüber hinaus nicht ausschließlich im akademischen Umfeld entstanden, sondern gleichermaßen abseits der traditionellen Wissenschaft aus unterschiedlichsten öffentlichen Aktivitäten hervorgegangen sind. Es waren die Akteure der Basisbewegungen, die von einem emanzipatorischen (,,Geschichte von unten''), einem aufklärerischem (Lernen aus der Geschichte) sowie einem moralischen (Vergangenheit nicht vergessen) Antrieb geleitet waren und die etablierte Geschichtsforschung und Erinnerungspolitik durch Demokratisierung von unten und Pluralismus von Grund auf verändern wollten.\footnote{Diese Entwicklung hatte natürlich auch Auswirkung auf die akademische Geschichtswissenschaft, die sich von einer sozialhistorischen Ausrichtung hin zu einer \textit{Alltagsgeschichte}, als neuen Forschungsansatz, weiterentwickelte. Siehe dazu Lindenberg/ Wildt 1989, S. 393f., 405-409.} Lindenberg und Wildt sprechen in Bezug auf die Praxis der Geschichtswerkstätten schon 1989 von ,,öffentlicher Wissenschaft''\footnote{Lindenberg/ Wildt 1989, S. 394.} und zitieren jene mit:

\begin{quote}
    Wir beanspruchen, unsere Projekte für jede/n - ob ,wissenschaftlich' ausgebildet oder nicht - offen zu halten. Das Interesse am Gegenstand, an der gemeinsamen Auseinandersetzung mit der Vergangenheit im jeweiligen Projekt, sind entscheidend.\footnote{Ebd.}
\end{quote}

Damit wird sehr deutlich, dass der historischen Forschung im Forschungsfeld die von der Open Science-Bewegung eingeforderte Offenheit im Sinne der Partizipation an Wissenschaft keinesfalls fremd ist, sondern im Gegenteil bereits über Jahrzehnte praktiziert wird. In der Konsequenz muss auch das Forschungsdatenmanagement partizipativ angelegt sein.

\section{Stakeholder}
 
Im vorausgegangenem Kapitel haben sich bereits diverse potentielle Nutzer*innen von offenem Forschungsdatenmanagement im Forschungsfeld herauskristallisiert. Wenn dieses, wie oben zum Ziel erklärt, konsequent partizipativ sein will, müssen demnach alle Anspruchsgruppen (Stakeholder) berücksichtigt werden, die ein berechtigtes Interesse an offenem Forschungsdatenmanagement haben und selbst, wie gezeigt worden ist, einen Beitrag zur (historischen) Forschung leisten. Nachfolgend werden deshalb die Beteiligten an offenem Forschungsdatenmanagement noch einmal aufgeschlüsselt. Freilich sind die Grenzen durchlässig, da sich die Akteure nicht in feste Kategorien pressen lassen, sondern sich fluide hin und her bewegen. Dennoch bietet die Einteilung die Möglichkeit, unterschiedliche Interessen und Ziele aufzuzeigen, die unberücksichtigt bleiben würden, wenn von vornherein eine Zielgruppe festgelegt wäre. Dies scheint insbesondere im Zusammenhang mit den sich im Aufbau befindlichen Infrastrukturen von Bedeutung. Aus der aktuellen statistischen Übersicht der DFG zu den Antragseingägen für NFDI geht hervor, dass mit 60 Prozent die Universitäten als antragstellende Einrichtungen klar in der Mehrheit sind und notwendige Infrastrukturen demzufolge vorwiegend aus dem Wissenschaftssystem heraus entstehen.\footnote{DFG 2021, S. 13.} Es steht die Frage im Raum, inwieweit diese ausschließlich auf die zugehörigen Akteure ausgerichtet hin entwickelt werden. Wie die Forschungsdaten zu den jüdischen Gewerbetrieben bereits gezeigt haben, wäre es unzureichend, außerhalb liegende Interessengruppen lediglich nachträglich als reine Konsumenten von Forschungsdaten zu begreifen. Vielmehr sind sie (Mit-)Produzenten von Forschungsdaten, für die ein gleichberechtigter Zugang zu entsprechenden Infrastrukturen von Anfang an mitgedacht werden sollte. Im Fall der hier betrachteten Forschungsdaten liefe man andernfalls Gefahr, essentielle Gruppen im Forschungsfeld auszuschließen.

\subsection{Akademische Wissenschaft}

Die größte Interessengruppe stellt die akademische Wissenschaft dar, denn sie hat systematisch und in Bezug auf die Vernichtung der jüdischen Gewerbetätigkeit  bisher den Großteil der Forschungsdaten produziert. Dies geschah überwiegend im Rahmen von Dissertations- oder akademischen Forschungsprojekten.\footnote{Dissertationen: Hamburg (Bajohr 1998), Köln (Bopf 2004), Mittelfranken (Janetzko 2012), Mannheim (Fritsche 2013); Akademische Forschungsprojekte: Berlin (Kreutzmüller 2012), Frankfurt am Main (Nietzel 2012), Breslau (2012).} Zur Gruppe gehören demnach Wissenschaftler*innen, die in der Regel aber nicht ausschließlich an Universitäten angebunden sind und folglich innerhalb des Wissenschaftssystems agieren. Abgrenzungskriterium ist, dass in dieser Gruppe kritische Methodenreflexion, Konzeptentwicklungen und analytische Durchdringung mit dem klaren Ziel des Erkenntnisfortschritts im Zentrum stehen. 

\subsection{Gedenk- und Erinnerungskultur}

Eine weitere große Interessengruppe stellen die Akteure aus der Gedenk- und Erinnerunskultur dar. Hier stehen die Daten zu jüdischen Gewerbebetrieben meist im Kontext von Ausstellungen, Stadtführern, Gedenkbüchern und anderen öffentlichen, oft städtischen, Aktionen.\footnote{Nürnberg und Fürth (Matthias Henkel u.a.: Entrechtet, entwürdigt, beraubt. Die Arisierung in Nürnberg und Fürth, hrsg. für d. Museen d. Stadt Nürnberg, 2012/2013), Erfurt (Christoph Kreutzmüller, Eckart Schörle: Stadtluft macht frei? Jüdische Gewerbebetriebe in Erfurt 1919 bis 1939, Leipzig 2013), jüngst Arnstadt (Schlossmuseum Arnstadt (Hrsg.): Jüdische Gewerbeansiedlungen in Arnstadt von 1874 bis 1929 und ,,Arisierung'' der Wirtschaft in Thüringen: Das Beispiel Arnstadt, in: Jüdische Familien in Arnstadt und Plaue, Begleitband zur Ausstellung, Arnstadt 2021)
).} Die Akteure sind vorwiegend zivilgesellschaftliche Initiativen, aber auch Gedächtniseinrichtungen wie kleinere städtische Museen und Archive, die nicht primär wissenschaftliche Institutionen sind, werden zu dieser Gruppe gezählt. Die gemeinsame Klammer bei sämtlichen Aktivitäten ist die Bewahrung und Vermittlung von vergangener Wirklichkeit sowie ein sensibles, sinnstiftendes Gedenken und Erinnern.\footnote{Vgl. Interview B4\_Transkript: ,,[...] und da habe ich vielleicht einen anderen Zugang, als ein reiner Wissenschaftler - mir geht es auch immer um die erinnerungskulturelle Bedeutung oder die erinnerungskulturelle Sinnstiftung hier in diesem Gemeinwesen München, die steht für mich - nicht an erster Stelle, aber sie steht für mich sehr prominent weit vorne [...]''.}

\subsection{Einzelpersonen}

In der dritten Interessengruppe werden all die Akteure zusammengefasst, die weder institutionell noch an sonstige Infrastrukturen angebunden sind. Hierbei handelt es sich vorwiegend um Einzelpersonen, deren intrinsische Interessen und Motive sehr voneinander abweichen können. Es ist selbst für ein offenes Forschungsdatenmanagement, das sich als partizipativ versteht, unmöglich, alle Einzelinteressen gleichermaßen zu berücksichtigen. Hervorzuheben sind allerdings zwei Gruppen. Erstens sind das die sogenannten Amateur- oder Hobbyforscher sowie selbstständige Historiker*innen. Sie haben einerseits ebenfalls systematisch Daten zu jüdischen Gewerbebetrieben gesammelt und analysiert.\footnote{Für Krefeld immerhin 135 jüdische Gewerbebetriebe, vgl. Flümann 2015. Die Autorin hat der Verfasserin dieser Arbeit dankenswerterweise ihre Daten zur Verfügung gestellt.} Andererseits fordern inbesondere diese Akteure den Zugang zu Forschungsdaten ein.\footnote{Vgl. Interview B2\_Transkript: ,,[...] weil ich immer wieder Anfragen bekomme und weiß, dass Leute sich mit all möglichen Unternehmensschicksalen oder Schicksalen jüdischer Bürger in ihrer Stadt, in ihrem Viertel auseinandersetzen und dazu auch Informationen suchen.''.}

Die zweite wichtige Gruppe, die mit Forschungsdatenmanagement nicht unbedingt assoziiert wird, sind die Nachkommen der Opfer des Nationalsozialismus. Sie leben heute aufgrund von Flucht und Vertreibung ihrer Vorfahren aus Deutschland häufig über den gesamten Globus verstreut. Oft sprechen sie nicht mehr die deutsche Sprache. Wegen dieser geografischen und sprachlichen Barrieren ist für sie die Aufarbeitung der eigenen Familiengeschichte vor Ort in Deutschland in städtischen Archiven besonders schwierig. Deshalb sollten gerade die Angehörigen der Opfer Zugang zu den Forschungsdaten haben, die Auskunft geben über das Leben der vertriebenen oder ermordeten Verwandten.\footnote{Vgl. Interview B1\_Transkript: ,,Und das ist auch wirklich erstaunlich, dass ich auch nach wie vor immer noch Anfragen von Nachkommen erhalte, die mich fragen, was ich noch mehr zu ihren Vorfahren rausfinden kann.'', Pos. 39.}

\section{Bereitschaft zu Open Science im Forschungsfeld}

Damit offenes Forschungsdatenmanagement im Forschungsfeld am Ende funktioniert, braucht es neben der Erfüllung technischer Voraussetzungen die grundätzliche Bereitschaft von den diversen Stakeholdern, Open Science in die eigene Forschungsarbeit zu integrieren. Die für diese Arbeit geführten Experteninterviews stellen keine repräsentive Umfrage dazu dar, allein weil sie das Akteursspektrum nicht widerspiegeln, aber sie vermitteln ein Stimmungsbild. Festzuhalten ist zunächst, dass von insgesamt acht Interviewanfragen\footnote{Ausgewählt für die Interviews wurden insgesamt 14 Personen, von denen acht erreichbar waren.}, zwei Personen ein Gespräch mit der Begründung ablehnten, mit den Themen der Arbeit nicht vertraut zu sein und daher nicht in der Lage seien, umfassende und fundierte Auskunft zu erteilen. Ohne diese Selbsteinschätzungen im Einzelnen beurteilen zu können, deuten sie darauf hin, dass es Berührungsängste mit der Thematik gibt.

Bei den befragten Personen ist Bereitschaft vor allem in Bezug auf die universellen Open Science-Grundsätze vorhanden. Schlagwörter wie Verfügbarkeit, Teilen, Austausch, Vernetzung oder Nachvollziehbarkeit sind mehrheitlich gefallen. Es wird sogar hervorgehoben, dass sie gerade im Kontext des Forschungsfelds wichtig seien.\footnote{Vgl. Interview B3\_Transkript, Pos. 67.} Die konkrete Realisierung wurde allerdings an Bedingungen geknüpft, die wie folgt zusammengefasst werden können:
\begin{itemize}
    \item Es muss ersichtlich sein, was offenes Forschungsdatenmanagement bezwecken will. Offenes Forschungsdatenmanagement ist, jedenfalls in der gegenwärtigen Phase, noch kein Selbstzweck, sondern braucht eine klare Zielformulierung, die die Benefits deutlich heraushebt.\footnote{Vgl. Interview B2\_Transkript, Pos. 47.}
    \item Offenes Forschungsdatenmanagement im Forschungsfeld kann nicht rein wissenschaftlich ausgerichtet sein, sondern braucht eine Kopplung zum erinnerungskulturellen Teil des Forschungsfelds.\footnote{Vgl. Interview B4\_Transkript, Pos. 61.}
    \item Um ein offenes Forschungsdatenmanagement steuern und kontrollieren zu können, bedarf es gemeinsamer Regeln und Strategieentwicklung sowie methodischer Führung.\footnote{Vgl. Interview B3\_Transkript, Pos. 83.}
    \item Es bedarf der Reflektion forschungsethischer Implikationen und der Umsetzung entsprechender Richtlinien.\footnote{Vgl. Interview B4\_Transkript, Pos. 19.}
    \item Offenes Forschungsdatenmanagement muss Diskurse im Forschungsfeld abbilden können.\footnote{Vgl. Interview B4\_Transkript, Pos. 87.} 
    \item Offenes Forschungsdatenmanagement braucht langfristige Betreuung und Pflege. Es muss sich stetig an neue Bedarfe im Forschungsfeld anpassen lassen können.\footnote{Vgl. Interview B2\_Transkript, Pos. 47.}
\end{itemize}

\section{Rechtliche und ethische Rahmenbedingungen}

Die rechtlichen und ethischen Rahmenbedingungen entscheiden maßgeblich darüber, ob die Forschungsdaten zu jüdischen Gewerbebetrieben in einer Open Data-Lizenz publiziert werden können. In Bezug auf nutzungsrechtliche Fragen gingen aus den Interviews keine gesichterten Antworten hervor.\footnote{Vgl. Interviews B2\_Transkript, Pos. 35 und B3\_Transkript, Pos. 51.} Daher können pauschal für das Forschungsfeld keine Aussagen gemacht werden. Eine ansatzweise fundierte Auskunft ist aber auf der Grundlage der vorliegenden Forschungsdaten zu Berlin möglich. Hier wurden vier relevante Datenquellen identifiziert. Die erste Datenquelle, aus der Grunddaten zu Name, Rechtsform, Adresse, Inhaber und Bilanzen entnommen wurden, stammen aus der Zentralhandelsregisterbeilage (ZHRB), welche dem Deutschen Reichsanzeiger und Preußischen Staatsanzeiger täglich beilag.\footnote{Heute Bundesanzeiger. Die ZHRB liegt inzwischen als Scan vollständig  digitalisiert vor, URL: \url{https://digi.bib.uni-mannheim.de/periodika/reichsanzeiger/} (letzter Zugriff am 18.05.2022). Siehe zur Geschichte des Deutschen Reichsanzeigers und Preußischen Staatsanzeigers Christoph Kling: ,,Deutscher Reichsanzeiger und Preußischer Staatsanzeiger. Einleitung zur Veröffentlichung der Digitalausgabe'', Mannheim, 2016.} Bei diesen Daten handelt es sich um Informationen aus dem Handelsregister, zu deren Offenlegung Unternehmer nach dem Handelsgesetzbuch (HGB) verpflichtet waren.\footnote{Die Veröffentlichungs-, Offenlegungs- und Bekanntmachungspflichten bestehen bis heute. Siehe Bundesamt für Justiz, URL: \url{https://www.bundesjustizamt.de/DE/Themen/Ordnungs_Bussgeld_Vollstreckung/Jahresabschluesse/Offenlegung/Offenlegungspflichten/Offenlegungspflichten_node.html}. Das Handelsregister kann jedoch heute online eingesehen werden, URL: \url{https://www.handelsregister.de/rp_web/welcome.xhtml} (alle Zugriff am 18.05.2022).} Es handelt sich folglich um amtliche, öffentliche Informationen, die keiner rechtlichen Einschränkung unterliegen. Das gilt generell für publiziertes historisches Material.\footnote{Für Berlin zum Beispiel Zeitschriften wie die ,,Jüdische Rundschau'' oder ,,Der Stürmer'' sowie öffentliche Vereinsmitgliederverzeichnisse, Jüd. Gemeindeblätter, Jüd. Adressbücher, etc. Informationen basieren auf einer SQL-Datenbankabfrage vom 18.05.2022.} Die zweite Datenquelle bildet eine Grauzone. Hierbei geht es um Daten, die aus externen Online-Datenbanken kommen und wo die Nachnutzung nicht eindeutig  ist. Dies ist zum Beispiel bei dem ,,Gedenkbuch
Opfer der Verfolgung der Juden unter der nationalsozialistischen Gewaltherrschaft in Deutschland 1933 - 1945''\footnote{URL:\url{https://www.bundesarchiv.de/gedenkbuch/}.} des Bundesarchivs der Fall. Dort ist ein Copyright ,,© Bundesarchiv'' für die gesamte Website zwar vermerkt, aber das Gedenkbuch erlaubt durch Datenexporte (CSV und PDF) theoretisch, Daten nachzunutzen. Im Datensatz selbst sowie in den Dateien findet sich jedoch keinerlei Hinweis darauf, wie die Daten nachgenutzt werden dürfen.\footnote{Siehe am Beispiel des Datensates de1086146, URL: \url{https://www.bundesarchiv.de/gedenkbuch/de1086146}.} Hier zeigt sich, dass im Sinne der Creative Commons-Philosophie eine klare Kommunikation seitens der Datenprovider notwendig ist.\footnote{Das gleiche gilt im Übrigen auch für die ,,Zentrale Datenbank der Namen der Holocaustopfer'' der Gedenkstätte Yad Vashem. Siehe Datensatz 11536340 zu selben Person wie oben, URL: \url{https://yvng.yadvashem.org/index.html?language=de&s_id=&s_lastName=Kann&s_firstName=Marion&s_place=Berlin&s_dateOfBirth=&cluster=true} (letzter Zugriff am 18.05.2022).} Die dritte Datenquellen stellen alle in Archiven vorliegenden, aber nicht veröffentlichten Quellen dar.\footnote{Dazu gehören sogenannte Arisierungslisten, Entjudungsakten, Handelsregisterakten, etc.}. Auch wenn die darin enthaltenden Daten selbst keinen Schutzfristen mehr unterliegen, verfügt das Archiv als Besitzer über die Vergabe Nutzungsrechte. Rechtlich brisant sind die Wiedergutmachungsakten, da sie sich auf natürliche Personen beziehen und daher besonderen Schutzfristen unterliegen. Sie werden deshalb hier als vierte Datenquelle extra gezählt. Das betrifft nicht nur Daten zu Überlebenden, sondern auch die zu den nichtjüdischen Erwerber*innen von jüdischem Eigentum.\footnote{Hier gilt mitunter noch die Einschränkung nach dem Bundesarchivgesetz § 11 Abs. 2, dass nach Ablauf der allgemeinen Schutzfrist (für die Wiedergutmachungsakten in den 90er Jahren), personenbezogene Akten entweder mit Erlaubnis der betroffenen Personen oder frühestens 10 Jahre nach Tod der Person benutzt werden dürfen. Vgl. Bundesarchivgesetz vom 10. März 2017, URL: \url{https://www.bundesarchiv.de/DE/Navigation/Meta/Ueber-uns/Rechtsgrundlagen/Bundesarchivgesetz/bundesarchivgesetz.html} (letzter Zugriff am 18.05.2022).} Für das offene FDM mit Open Research Data wird eine offene Lizenz angestrebt. Wichtig wäre also, dass für die Datenquellen, bei denen die Nachnutzung nicht sicher ist, im Vorfeld eine entsprechende Veröffentlichung mit den Archiven abgeklärt wird. Das macht deutlich, dass Open Science im Forschungsfeld auch von der Bereitschaft anderer Institutionen wie Archiven abhängt. Unabhängig davon ist generell wichtig für das offene FDM, Nutzungsrechte zum Beispiel mit einer Creative Commons-Lizenz transparent zu machen.

Aus ethischer Perspektive scheinen die Forschungsdaten auf den ersten Blick unbedenklich, da es sich vorwiegend um amtliche, öffentliche Massendaten handelt. Allerdings gibt es im Forschungsfeld sowie in der Holocaust-Forschung allgemein eine Auseinandersetzung zum Missverhältnis in der Veröffentlichung von Daten von Holocaust-Opfern gegenüber deutschen Täter*innen und Mittäter*innen. Dass heute Daten über jüdischen Personen überhaupt in dieser Breite und Tiefe publiziert werden dürfen, beruht einzig auf der Tatsache, dass diese Menschen vor 80 Jahren ermordet wurden. Zudem waren sie zu Lebzeiten bereits einer vollständigen Erfassung und Markierung ausgesetzt, die die systematische bürokratische Verfolgung erst ermöglichte.\footnote{Vgl. Götz Aly, Karl Heinz Roth: Die restlose Erfassung. Volkszählen, Identifizieren, Aussondern im Nationalsozialismus, Berlin 1984, S. 67-105.} Das Recht auf Anonymität existierte für sie zu Lebzeiten nicht. Im Gegenzug unterliegen personenbezogene Daten zu deutschen Täter*innen und Mittäter*innen gesetzlichen Schutzfristen über den Tod hinaus, weil diese Menschen noch leben oder bis vor Kurzem noch gelebt haben.\footnote{Bajohr spricht sogar von ,,umfassenden Täterschutz'', Bajohr 1998, S. 24.} Dieses ethische Dilemma kann offenes Forschungsdatenmanagament nicht auflösen. Festhalten ist jedoch, dass es sich hierbei um eine genuin deutsche Debatte handelt.\footnote{Sie hat sich auch in den Interviews widergespiegelt, vgl. Interview B1\_Transkript, Pos. 123, 125, 127, 129.} Das internationale Holocaust-Museum \textit{Yad Vashem} in Israel wiederum sieht in der Online-Veröffentlichung seiner Daten von über 3 Millionen Personen die Chance, fehlende Informationen von der Öffentlichkeit zu erhalten, die die Sammlung der Namen der Ermordeten sukzessive erweitern können\footnote{The Central Database of Shoah Victims' Names, URL: \url{https://yvng.yadvashem.org/} (letzter Zugriff am 18.05.2022).}

Letztendlich muss immer abgewogen werden, ob ethische Grenzen dem öffentliches Interesse an diesen Daten überwiegen. Die Forschungsdaten zu den jüdischen Gewerbebetrieben werden an dieser Stelle im Großen und Ganzen als unproblematisch eingestuft, weil es in erster Linie Verwaltungsdaten sind. Nichtsdestotrotz hat offenes Forschungsdatenmanagament aufgrund des sensiblen Forschungsthemas forschungsethische Implikationen, die parallel zur prototypischen Implementierung im nächsten Kapitel diskutiert werden.




