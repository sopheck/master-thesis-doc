\onehalfspacing

In diesem Kapitel geht es darum, sich einen grundlegenden Überblick über die Ausgangslage zu beschaffen sowie den Bedarf von offenem Forschungsdatenmanagement (FDM) zu ermitteln, der diese Arbeit begründet. Beides bildet die Basis, auf die offenes FDM aufsetzt und von der sich die funktionale Anforderungen an das offene FDM ableiten. 

Im zweiten Schritt folgt eine datenkritische Auseinandersetzung mit dem Forschungsfeld. Hier geht es darum, die Forschungsdaten, die Ergebnis eines Genese- und Verarbeitungsprozesses sind, zu historisieren um sie verstehen und bewerten zu können. Dafür werden Studien, in denen Forschungsdaten während des Forschungsprozesses angefallen sind, systematisch untersucht. Ziel ist erstens, im Forschungsfeld den Zweck von Forschungsdaten und die daraus folgende Handhabung dieser Daten zu ermitteln. Darauf aufbauend soll zweitens der Erkenntnis- sowie Nutzwert von Forschungsdaten und damit deren Bedeutung im Forschungsfeld bestimmt werden. Dabei ist das Vorgehen problemorientiert, es werden auch Grenzen und Defizite diskutiert, an die offenes Forschungsdatenmanagement lösungsorientiert ansetzen soll. Vor diesem Hintergrund werden in einem dritten Schritt zusammenfassend außerdem weitergehende Probleme sowie Desiderate im Forschungsfeld mit in den Blick genommen.

Auswertung der Interviews und der Literaturanalyse
Benötigter Funktionsumfang ist mit FAIR, CARE und Open Data Modellen bereits vorgegeben, sie geben Anforderungen vor
Es geht um diesen Funktionsumfang des offenen FDM an Forschungsfeld anzupassen, d.h. weitere Anforderungen zu erarbeiten und damit die Anschlussfähigkeit des offenen FDM sicher zu stellen.

\section{Datenkritische Auseinandersetzung}

Der Schwerpunkt dieser Arbeit


Daten zu jüdischen Gewerbebetrieben wurden, wie im vorausgegangenen Kapitel erläutert, im akademischen wie nichtakademischen Umfeld gesammelt. Für beide Bereiche stellt sich die Frage, 

Unabhängig davon stellt sich für beide Bereiche die Frage, zum welchem Zweck diese generiert wurden und wie die Handhabung erfolgte. Davon lässt sich der Erkenntnis- und Nutzwert dieser Daten feststellen und 


Siehe Forschungsdaten zu verfolgten Ärztinnen und Ärzte des Berliner Städtischen Gesundheitswesens (1933-1945)
https://geschichte.charite.de/verfolgte-aerzte/index.html


Allerdings lässt sich in den letzten Jahren ein Abflauen

Zum anderen

Die Frage, 

 dominierten seit Ende der 1990er Jahre wissenschaftliche Untersuchungen, die sehr präzise die vergangenen Vorgänge der zu rekonstruieren versuchten.      


Im Forschungsfeld zur Vernichtung der jüdischen Gewerbetätigkeit lässt in den letzten Jahren einen Abflauen des zu Beginn der 2000er Jahre sehr regen Publizierens zum Thema beobachten.\footnote{Die letzte Studie zum Thema erschien}


Dazu wird dieses systematisch nach folgenden Fragen untersucht:

\begin{itemize}
\item Wo im Forschungsfeld existieren über Berlin hinaus weitere Forschungsdaten zu jüdischen Gewerbebetrieben.
\item Wann wurden die Forschungsdaten generiert (Zeitpunkt).
\item Welche Ziele sollten mit den Forschungsdaten erreicht werden.
\item Wann wurden die Forschungsdaten generiert und wie wurden sie erhoben (Methode). 
\item Welche Daten wurden erfasst und wie erfolgte die Datenerfassung.
\item Wie wurden die Daten ausgewertet (Datenauswertung und -interpretation).
\item Was für Datenarten liegen vor (Charakteristik).
\item Welche Daten sind in welcher Form veröffentlicht (Publikation).
\end{itemize}


\begin{table}
\caption{Metaangaben zu den ausgewählten Lokalstudien}
\label{tab:lokalstudientabelle}
\begin{tabular} { L{0.5cm}|L{3cm}|L{4cm}|L{0.7cm}|L{4cm} }
Nr. & Titel & Untertitel & Ort & Entstehungskontext \\
\hline 
1 & Arisierung in Hamburg & Die Verdrängung der jüdischen Unternehmer 1933-1945 & 1997 & Dissertation \\
\hline 
2 & ,,Arisierungen'' in München & Die Verdrängung der jüdischen Gewerbetreibenden aus dem Wirtschaftsleben der Stadt 1933-1939 & 2000 & n.n.\\
\hline 
3 & ,,Arisierung'' in Köln & Die wirtschaftliche Existenzvernichtung der Juden 1933-1945 & 2004 & Dissertation\\
\hline
4 & ,,Arisierung'' in München & Die Vernichtung jüdischer Existenz 1937-1939 & 2004 & n.n.\\
\hline
5 & Ausverkauf & Die Vernichtung der jüdischen Gewerbetätigkeit in Berlin 1930-1945 & 2012 & n.n.\\
\hline
6 & Handeln und Überleben & Jüdische Unternehmer aus Frankfurt am Main 1924-1964 & 2012 & n.n.\\
\hline
7 & Die ,,Arisierung'' Mittelständischer jüdischer Unternehmen in Bayern 1933-1939 & Ein interregionaler Vergleich & 2012 & n.n.\\
\hline
8 & Ausgeplündert, zurückerstattet und entschädigt & Arisierung und Wiedergutmachung in Mannheim & 2013 & Dissertation\\
\hline
9 & ,,… doch nicht bei uns in Krefeld!'' & Arisierung, Enteignung, Wiedergutmachung in der Samt- und Seidenstadt 1933 bis 1963 & 2015 & n.n\\
\end{tabular}
\end{table}
\paragraph{Forschungsziele und Forschungsdesign}
\paragraph{Datenerfassung und Datensicherung}
\paragraph{Datenauswertung}
\paragraph{Datenveröffentlichung}




\subsection{...}
\subsection{...}

\section{Auswertung der Interviews}
\subsection{Kollaboratives Arbeiten auf den Daten}
\subsection{Tools zur Datenauswertung}
\subsection{Nutzung projektübergreifender Datensammlungen}
Recherche

\section{Rechtliche Aspekte und forschungsethische Implikationen}
\section{Zwischenergebnisse}

Stakeholderanalyse
- Aufzeigen und Zusammenführen der Beteiligten in einem Projekt
- Systematisches Bewerten und Ordnen der Beteiligten nach Interessen, Macht und Rolle
- Zielbestimmung entsprechend der Interessenbeteiligten

\paragraph{Wissenschaftler}

akademisch an Universitäten, nicht-akademisch z.B. an Erinnerungseinrichtungen wie Archive, aber auch Historiker*innen, die nicht universitär angebunden sind (Claudia Fürmann)


\paragraph{Akteure der Erinnerungskultur}

z.B. Stolperstein-Initiativen sehr von moralischen Impetus geleitet, ehrenamtliche

\paragraph{Familienangehörige}

\paragraph{Forschungsdesiderate im Forschungsfeld}Allgemeinaussagen zum Vernichtungsprozess halten nur auf einer gesicherten empirischen Basis stand, daher ist ein quantitativer Methodenteil in dem Forschungsfeld unerlässlich. Vor allem aufgrund der disparaten Quellenmaterials, die innerhalb eines Projektzeitraums nicht vollständig zu bewältigen waren, ist es bisher keiner Studie gelungen, alle jüdischen Gewerbebetriebe zu erfassen. Stattdessen wurde auf der Basis von Stichprobenziehung (sampling) gearbeitet. Die quantitative Dimension des Vernichtungsprozesses basiert folglich nicht auf absoluten Zahlen.
Weiterhin ist auffallend, dass in fast allen Lokalstudien die Zahlen von Genschel und Barkai herangezogen und diskutiert wurden, um Aussagen dazu zu verifizieren. Einige Autor*innen wiesen jedoch zurecht daraufhin, dass es sich bei Genschel und Barkai um grobe Schätzungen und zwar für Gesamtdeutschland handelte, die sich so nicht zwingend auf lokaler Ebene wiederfinden müssen, dennoch aber korrekt sein können. Daraus ist zu schießen, dass erst die Summe hinreichender Lokalstudien ein Gesamtbild dazu ergeben werden können.
Anknüpfungspunkte für offenes FDM