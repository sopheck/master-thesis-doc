\onehalfspacing

Open Science in Zusammenhang mit Forschungsdaten und Forschungsdatenmanagement praktiziert werden kann. Nach einem allgemeinen Überblick über Ursachen, Ziele und Konzepte von Open Science, wird anschließend auf Open Data fokussiert und 
Hier wird im Weiteren untersucht in welchem Verhältnis Open Data zum noch relativ jungen Forschungsfeld des Forschungsdatenmanagements (FDM) steht, das bezüglich digitaler Forschungsdaten in den letzten Jahren an Bedeutung in der Wissenschaft gewonnen hat. Dazu wird das Forschungsfeld selbst kurz überblickt. Zum Schluss bleibt das dritte Foschungsfeld der Vernichtung der wirtschaftlichen Existenz der Juden im Nationalsozialismus, an dem exemplarisch anhand von Foschungsdaten zur Vernichtung der jüdischen Gewerbetätigkeit der Bedarf sowie Anforderungen an offenes Forschungsdatenmanagement, welches Open Science-Ansätze integriert, analysiert werden.


Auch wenn von der Replikationskrise nicht direkt betroffen, so stellt sich die Frage nach Qualitätssicherung von wissenschaftlicher Forschung und Arbeit ebenso für die Geschichtswissenschaften - gerade in der digitalen Transformationphase, die tiefgreifende Veränderungen in der Forschungspraxis mit sich bringt. Die selbstkritische Haltung der Open Science-Bewegung gegenüber der eigenen Disziplin wird in dieser Arbeit aufgegriffen und deren abgeleitete Lösungsstrategien aufgenommen.



über den historiographischen Kontext Klarheit verschafft zur Einordnung der hier betrachteten historischen Forschungsdaten

Hinsichtlich der Verfahren und Strategien nicht alle der gesamten Open Science Taxonomie erläutern, sondern in Bezug auf Thema der Arbeit auf Forschungsdaten beschränken. Open Data und FAIR-Data, wo letztere seinen Ursprung aus dem Feld des FDM kommt  

Um ein solches offenes FDM in dieser Arbeit im Sinne der Praktikabilität und Überschaubarkeit im Griff zu behalten, muss es allerdings auch klar abgegrenzt sein. Hier bietet sich die Fokussierung auf einen handhabbaren thematischen Rahmen an, dessen Grenzen also zwischen wissenschaftlicher Schlüssigkeit/ Nachvollziehbarkeit und pragmatischer Machbarkeit ausloten. In einem ersten Schritt wird daher zur thematischen Einordnung der größere  historiographische Kontext dargestellt, in dem sich das offene FDM bewegt. Darauf folgt die thematische Eingrenzung auf Forschungsfeld zur Vernichtung der jüdischen Gewerbetätigkeit im Nationalsozialismus. Dieses bildet den Kontextrahmen, in dem das offene FDM zukünftig arbeiten wird und dessen Merkmale in einem Überblick vorgestellt werden.

\section{Open Science}

In diesem Überblick soll geklärt werden, wie der inzwischen zum Buzzword gewordene Begriff ,,Open Science'' definiert ist. Dazu werden der historische Ursprung von Open Science umrissen und die Akteure sowie deren Aktivitäten beispielhaft vorgestellt. Zudem wird ein Ist-Stand erarbeitet, wo sich Open Science heute befindet, das heißt was bisheriger Output ist und welchen Status Open Science bei wissenschaftspolitischen Entscheidungen auf europäischer und bundesdeutscher Ebene gegenwärtig hat. 

\subsection{,,Replikationskrise'' und Open-Bewegung}

Die Frage der Enstehung von Open Science betreffend können zwei Entwicklungslinien verfolgt werden. Zum einen geht Open Science auf ein konkretes Ereignis innerhalb der Wissenschaft zurück, nämlich auf die sogenannte Replikationskrise. Hier bezieht sich Open Science explizit auf die Transformation wissenschaftlicher Forschungsmethoden und -praktiken, um Forschung noch robuster zu machen. Zum anderen ist Open Science Teil der breiteren sozialen Open-Bewegung, welche von der Do-it-yourself-Bewegung, der Hacker-Bewegung der 1960/ 70er sowie der Freie-Software-Bewegung der 1980er Jahre
(Vorgänger der Open-Source-Bewegung) stark beeinflusst ist.\footnote{Vgl. ayway media (Hrsg.): Das digitale Handbuch,
Kapitel C.15 Die „Open-Bewegung”, Vettelschloss 2016, S. 252 }.

\paragraph{Replikationskrise} Ab Mitte der 2010er Jahre erhielten in der Wissenschaft, vordergründig in der Psychologie sowie in den Lebens- und Naturwissenschaften, zunehmend Replikationsstudien Aufmerksamkeit. Diese konnten in sogenannten Replikationsversuchen eine statistisch signifikante Anzahl publizierter empirischer Forschungsergebnisse entweder falsifizieren oder nicht replizieren, weil die Daten nicht zur Verfügung standen.\footnote{Als erste Replikationsstudie dieser Art wird jene des Medizinwissenschaftlers und Statistiskers John Ioannidis aus dem Jahr 2005 gezählt, mit der er erstmals systematisch versuchte, veröffentlichte Untersuchungsergebnisse nachträglich zu replizieren/ reproduzieren. Siehe John P.A. Ioannidis: Why Most Published Research Findings Are False, PLoS Med 2(8): e124, 2005, doi:10.1371/journal.pmed.0020124. Es folgten eine Reihe weiterer Replikationsstudien auch in anderen Fächern wie den Sozialwissenschaften. Siehe zum Beispiel Marjan Bakker, Annette van Dijk, Jelte M. Wicherts: The Rules of the Game Called Psychological Science, in: Perspectives on Psychological Science, 7(6), 2012, S. 543-554, doi:10.1177/1745691612459060; Thomas Herndon, Michael Ash, Robert Pollin: Does high public debt consistently stifle economic growth? A critique of Reinhart and Rogoff, in: Cambridge Political Economy Society (Hrsg.), Cambridge Journal of Economics, Band 38, 2. Ausgabe, Oxford 2014, S. 257-279, URL (stable): \url{https://www.jstor.org/stable/24694929}; Jeremy Freese, David Peterson: Replication in Social Science, in: Annual Reviews (Hrsg), Annual Review of Sociology, Band 43, San Mateo 2017, S. 147-165, doi:10.1146/annurev-soc-060116-053450} Das löste die vielfach diskutierte ,,Replikationskrise'' in den betroffenen Fächern aus, da eine wesentliche Eigenschaft von robuster evidenzbasierter Forschung, nämlich die Nachvollziehbarkeit ihrer Ergebnisse durch Replikation (als Bestandteil von Qualitätssicherung), in einem hohen Maße nicht mehr gegeben war und damit in der Konsequenz auch ein gesellschaftlicher Bedeutungsverlust von Wissenschaft bei der Wissensproduktion zu befürchten war. Folglich ging es nachträglich um Ursachenforschung, die sich auf Defizite insbesondere bei den Forschungsmethoden und in der Publikationspraxis wissenschaftlicher Journals fokussierte.\footnote{Diskutiert wurden insbesondere, wie das Institut für Psychologie an der Humboldt-Universität zu Berlin konzis berichtete, "p-hacking, selektives Berichten von (abhängigen) Variablen, Hypothesizing After the Results are Known (HARKING), nur signifikante Ergebnisse berichten, mehr Daten sammeln nachdem die bestehenden Daten keine positiven Ergebnisse hervorgebracht haben, Publikations Bias". Methodengruppe Berlin (Autorengruppe): Die Replikationskrise und Open Science, Blog Post, Humboldt-Universität zu Berlin, Lebenswissenschaftliche Fakultät Institut für Psychologie, Lehrstuhl für Psychologische Methodenlehre (Hrsg), URL: \url{http://methods-berlin.com/de/replikationskrise_open_science/} (letzter Zugriff am 21.04.2022). Siehe auch Klaus Fiedler, Norbert Schwarz: Questionable Research Practices Revisited, in: SAGE Publishing (Hrsg.), Social Psychological and Personality Science, Band 7, 1. Ausgabe, 2016, S. 45-52, doi:10.1177/1948550615612150; Annie Franco, Neil Malhotra, Gabor Simonovits: Publication bias in the social sciences. Unlocking the file drawer, in: American Association for the Advancement of Science (Hrsg.), Science, Band 345, Ausgabe 6203, Washington 2014, S. 1502-1505, doi:10.1126/science.1255484} Aber auch die Replikationsstudien selbst wurden kritisch betrachtet.\footnote{Vgl. Deutsche Forschungsgemeinschaft (Hrsg.): Replizierbarkeit von Forschungsergebnissen. Eine Stellungnahme der Deutschen Forschungsgemeinschaft, Stand: April 2017, URL: \url{https://www.dfg.de/download/pdf/dfg_im_profil/geschaeftsstelle/publikationen/stellungnahmen_papiere/2017/170425_stellungnahme_replizierbarkeit_forschungsergebnisse_de.pdf} (letzter Zugriff am 21.04.2022)} Kurzum ging es um die existenzielle Frage, wie Wissenschaft praktiziert werden muss, damit wissenschaftliche Forschung, insbesondere die statistisch empirische, reliabel ist. Als Antwort auf diese Krise hat sich in den vergangenen Jahren die internationale Open Science Bewegung formiert, die in den Anfangsjahren stark auf die Frage nach Replizierbarkeit von Forschungsstudien fokussiert war.\footnote{}

\paragraph{Open-Bewegung} Die Open Science Bewegung ist Teil der breiten sozialen Open-Bewegung, welche unter den Begriffen ,,Open'', ,,Openness'' beziehungsweise ,,Free'' subsumiert, ,,Daten, Entwürfe, Fotos, Musikstücke oder sonstige Inhalte und Wissen'' \footnote{Wikimedia Deutschland e. V., Open Knowledge Foundation Deutschland e. V. (Hrsg.): ABC der Offenheit, Berlin 2019, S. 4f., URI: \url{https://commons.wikimedia.org/wiki/File:ABC_der_Offenheit_-_Brosch\%C3\%BCre_(2019).pdf}} aus allen gesellschaftlichen Bereichen wie Forschung, Politik oder Wirtschaft für die Wiederverwendbarkeit digital zur Verfügung stellen und dadurch Teilhabe als demokratisches Prinzip in einer freiheitlichen Gesellschaft stärken will.\footnote{Ebd.} Neben Open Science umfasst die Bewegung unter anderem Open Knowledge, Open GLAM, Open Government, Open Design, Open Innovation, wobei eine trennscharfe Abgrenzung nicht immer möglich und sinnvoll ist. So lässt sich Open Data auch als Querschnittsbereich auffassen, der in andere Bereiche wie Open Science hineinreicht.\footnote{Vgl. Taxonomie von Open Science, .Eine konzise Übersicht aller Bereiche siehe auch WMK, OKF (2019), ABC der Offenheit, S. 14-54}. Eine Vertreterin der ersten Stunde der Open-Bewegung und die wohl populärste ist die gemeinnützige Wikimedia Foundation, Inc. (WMF)\footnote{URL: \url{https://wikimediafoundation.org/de/} (letzter Zugriff am 22.04.2022)} mit Sitz in den USA.\footnote{Vgl. den Wikipedia-Eintrag zur Wikimedia Foundation, Seite ,,Wikimedia Foundation''. In: Wikipedia – Die freie Enzyklopädie. Bearbeitungsstand: 31. März 2022, 20:07 UTC. URL: \url{https://de.wikipedia.org/w/index.php?title=Wikimedia_Foundation\&oldid=221669459} (letzter Zugriff am 22.04.2022) In Deutschland vertreten durch den Verein Wikimedia Deutschland e. V., vgl. ebd.} Bereits seit 2001 entwickelt sie digitale Infrastrukturen, mit denen Wissen offen ausgetauscht und geteilt werden kann. Ihr bekanntestes und ältestes Projekt ist die freie Enzyklopädie Wikipedia\footnote{URL: \url{https://de.wikipedia.org/wiki/Wikipedia:Hauptseite} (letzter Zugriff am 22.04.2022)}. Die WMF engagiert sich aber nicht ausschließlich mit der Wikipedia in der Open-Bewegung, sondern hat inzwischen eine Vielzahl an digitalen ,,Schwesternprojekten'' wie das Wörterbuch Wictionary (2002)\footnote{URL: \url{}}, die Text- und Quellensammlung Wikisource (2003)\footnote{URL: \url{https://de.wikisource.org/wiki/Hauptseite} (letzter Zugriff am 22.04.2022)}, die Mediensammlung Wikimedia Commons (2004)\footnote{URL: \url{https://commons.wikimedia.org/wiki/Hauptseite} (letzter Zugriff am 22.04.2022)}, die Wissensdatenbank Wikidata (2012)\footnote{URL: \url{https://www.wikidata.org/wiki/Wikidata:Main_Page} (letzter Zugriff am 22.04.2022)} und weitere.\footnote{Eine Auflistung aller Wikimedia-Projekte ist auf der Homepage zu finden unter \url{https://www.wikimedia.de/projekte/} (letzter Zugriff am 22.04.2022)} Daneben stellt sie eine Reihe ihrer MediaWiki Software-Komponenten in Open Source zur Verfügung.\footnote{Eine Übersicht ist auf der Website zu finden unter URL: \url{https://doc.wikimedia.org/} (letzter Zugriff am 22.04.2022)} Eine weitere und mit der WMF koopierende Organisation der Open-Bewegung ist die Open Knowledge Foundation (OKF), die 2005 in London gegründete wurde\footnote{URL: \url{https://okfn.org/} (letzter Zugriff am 22.04.2022)} und von der es seit 2011 auch einen deutschen Ableger in Berlin gibt.\footnote{URL: \url{https://okfn.de/} (letzter Zugriff am 22.04.2022)} Auch diese hat ihren Schwerpunkt auf offenes Wissen gelegt.\footnote{}

\subsection{Open Science Initiativen und Aktivitäten} 

Auch wenn oben genannte WMF und OKF nicht ausschließlich auf Open Science spezialisiert sind, so kommen inzwischen auch aus diesen beiden Organisation Impulse zur Bewegung. Am OKF ist zum Beispiel das deutschsprachige OKF-Arbeitsgruppe ,,Open Science''. 




Innerhalb der Open-Science-Bewegung ist vor allem das interdisziplinäre \textit{Center for Open Science} (COS) zu nennen, welches in direkter Reaktion auf die Replikationskrise 2013 in den USA gegründet wurde\footnote{URL: \url{https://www.cos.io/?hsLang=en} (letzter Zugriff am 21.04.2022)}. Eine der ersten Aktivitäten des COS war das mit der University of Viginia gemeinsam großangelegte \textit{Reproducibility Project}, in dem sich eine Autorengruppe, welche sich ,,Open Science Collaboration' nannte, systematisch mit der Reproduzierbarkeit von 100 Forschungsstudien in der Psychologie auseinandersetzte.\footnote{Brian A. Nosek, Johanna Cohoon, Mallory C. Kidwell, Jeffrey R. Spies: Estimating the reproducibility of psychological science, in: American Association for the Advancement of Science (Hrsg.), Science, Band 349, Ausgabe 6251, Washington 2015, doi:10.1126/science.aac4716}.




Während von wissenschaftlicher Seite insbesondere Transparenz, Reproduzierbarkeit, Wiederverwendbarkeit und offene Kommunikation während des Forschungsprozesses betont wird, ist es von der Open-Bewegung her vor allem Partizipation über die Wissenschaft hinaus, die zentral ist. All diese Punkte werden unter den ,,Open Science Principles'' zusammengefasst.\footnote{} Um diese letztlich umsetzen und praktizieren zu können, wird die digitale Transformation genutzt und digitale webbasierte Technologien angeeignet und für eigenen Zwecke adapiert.
 Nach der Bestandsaufnahme, bei der die Rate nichtreplizierbarer Forschungsstudien wie bei vorausgegangenen Replikationsstudien signifikant hoch war, widmete sich das COS verstärkt den Strategien zur Überwindung der Replikationskrise, die im Kern als eine methodische Krise identifiziert wurde, aber auch zweifelhafte Forschungspraktiken aufdeckte.\footnote{Vgl. } Seit diesem Befund ist das Ziel daher, Strategien und Verfahren zu entwickeln, welche eine Qualitätssicherung wissenschaftlicher Forschung garantieren. Nicht ausschließlich veröffentlichte (ausgewählte) Studienergebnisse sollen der Fachwelt zugänglich sein, sondern der gesamte Forschungszyklus, das heißt von der Themenfindung, Fragestellung beziehungsweise Hypothesenbildung, über das Forschungsdesigns, die Datenerhebung, -analyse sowie -auswertung bis zur Veröffentlichung der Research Outputs sollen alle Phasen im Forschungsprozess zu jeder Zeit nachvollziehbar sein. Das möchte die Bewegung durch eine radikale Stärkung von Transparenz, Reproduzierbarkeit, Wiederverwendbarkeit und offener Kommunikation erreichen.\footnote{} Sie werden unter dem Begriff ,,Open Sciene Principles'' zusammengefasst.\footnote{} Ziel ist es mit diesen, die Anwendung fragwürdiger Methoden oder Forschungspraktiken durch Präregistrierung, Versionierung, offene Datenrepositorien oder Peer Review zu reduzieren.\footnote{Ebd.} 
 
 Hier noch die Grundsätze
 
 
 Damit läutete die Open-Science-Bewegung nicht weniger als einen Kulturwandel im Wissenschaftssystem ein, der bis heute andauert.\footnote{Einen Aufschwung erlebte Open Science im Zusammenhang mit der COVID-19-Pandemie, wo deren Grundsätze essentiel insbesondere bei der Impfstoffentwicklung war. Siehe dazu Lonni Besançon, Nathan Peiffer-Smadja, Corentin Segalas, Haiting Jiang, Paola Masuzzo, Cooper Smout, Eric Billy, Maxime Deforet, Clémence Leyrat: Open science saves lives: lessons from the COVID-19 pandemic, in: BMC Medical Research Methodology, Band 21, Artikelnr. 117, 2021, doi:10.1186/s12874-021-01304-y} Im Kern geht es auch darum, die Integrität von wissenschaftlicher Forschung zu wahren, sie gerade im sogenannten postfaktischen Zeitalter zu stärken, das heißt sie weniger anfällig für Betrug und Fälschung in einer digitalen Welt zu machen. Dabei nimmt sie ganz im Sinne der Open-Bewegung die digitale Transformation auch als Chance wahr, indem sie zur Realisierung ihrer Grundsätze digitale Technologien adaptiert und in das Wissenschaftssystem integriert.\footnote{Vgl. } Beispielhaft ist hier das \textit{Open Science Framework} (OSF)\footnote{\url{https://osf.io/}}, mit dem Forschenden kostenfrei die Möglichkeit gegeben werden soll, auf einer einzigen Plattform Open Science während der gesamten Forschungsarbeit zu praktizieren.\footnote{Es ist aus dem bereits erwähnten Reproducibility Project entstanden, siehe \url{https://osf.io/ezum7/} Auch für diese Masterarbeit wurde das OSF verwendet, welche dort als Projekt öffentlich zugänglich ist, siehe Kapitel 1.4.}. Es wird von zahlreichen wissenschaftlichen überwiegend US-amerikanischen Institutionen verwendet.\footnote{Zum Beispiel Princeton University, New York University, George Washington University, u.a. Siehe \url{https://www.cos.io/products/osf-institutions} (letzter Zugriff am 21.04.2022)} Inzwischen gibt es Open Science Inititativen und Organisationen auch an deutschen wissenschaftlichen Einrichtungen.\footnote{Erst kürzlich hat sich das sogenannte German Reproducibility Network (GRN) gegründet, das fachübergreifend gezielt Replikationsstudien und Open Science Praktiken unterstützt. Zu dessen Hauptakteuren gehören u.a. Berlin University Alliance, das Helmholtz Center (Open Science), das LMU Open Science Center (OSC), das Netzwerk der Open Science Initiativen (NOSI), die Deutsche Gesellschaft für Psychologie (DGPs), u.a. Siehe Ankündigung der Berlin University Alliance: German Reproducibility Network gestartet, News vom 01.02.2021, URL: \url{https://www.berlin-university-alliance.de/news/items/2021/210201-grn.html} (letzter Zugriff am 21.04.2022). Homepage des GRN unter URL: \url{https://reproducibilitynetwork.de/}}.


Auch bei wissenschafts- wie gesellschaftspolitischen Entscheidungen gewinnt Open Science auf Bundes- sowie auf EU-Ebene an Relevanz, wobei zu konstatieren ist, dass der Schwerpunkt zumindest in Deutschland 


Wenn auch noch nicht die volle Bandbreite von Open Science, so unterstützt die Deutsche Forschungsgemeinschaft (DFG) immerhin offiziell gezielt Open Access-Publikationen finanziell. 

Die Europäische Union hat Open Science zu einem von insgesamt drei Grundsatzzielen für die Forschungsarbeit in Europa erklärt  und die  Deutsche UNESCO-Kommission betont in ihrer Empfehlung für Open Science:

,,Darüber hinaus besteht mit Open Science eine Chance auf die praktische Umsetzung von seit Langem bestehenden politischen Forderungen: Mit Open Science kann Teilhabe an und Zugang zu wissenschaftlichen Erkenntnissen als Gemeingut und Menschenrecht praktisch umgesetzt werden, wie es bereits seit Ende des Zweiten Weltkriegs in der Allgemeinen Erklärung der Menschenrechte gefordert war.''
Und auch auf der EU-Ebene 

Ebenso die oben erwähnt WMF und OKF widmen sich enfalls der Adaptierung digitaler Technologien für die Anwendung in der Wissenschaft


Felloship Programm, Wikiversity Open Science


Im Kern geht darum

Und auch die Bewegung selbst hat sich inzwischen in diverse Gruppen untergliedert, die sich auf unterschiedliche Phases im Forschungsprozess konzentrieren. Die am weit verbreitesten ist wahrscheinlich Open Access

bekanntesten sind Open Access, Open Data, Ope Zusammengefasst eint alle Gruppen, dass es darum geht, den gesamten Forschungsprozess - und nicht nur die Forschungsergebnisse - zu teilen. 

 



Jüngst hat die DFG ein Projekt bewilligt, in der 
https://nachrichten.idw-online.de/2020/05/12/wie-steht-es-um-die-glaubwuerdigkeit-von-wissenschaft/

  


    





\section{Forschungsdatenmanagement}

\subsection{Digitale Forschungsdaten}

Digitale Forschungsdaten, die im Zuge des Forschungsprozesses erzeugt werden, sind Bestandteil auch in der Forschungsarbeit von Historiker*innen geworden. Mit ihnen rücken in den Geschichtswissenschaften computergestützte qualitative wie quantitative Analyse- und Auswertungsverfahren in den Fokus. Lehrstühle wie der für Digital History an der Humboldt-Universität zu Berlin haben sich darauf eingestellt und , digitale Konzepte, Methoden und Verfahren für die geschichtswissenschaftliche Forschung zu reflektieren und zu anzupassen.  Unstrittig ist, dass digitale Forschungsdaten wichtige Ressource bei der Erkenntnisgenerierung sind, Es können unterschiedlich vorliegen (Def. Forschungsdaten)

Wenn aber Forschungsdaten epistemologisch an Bedeutung für die Geschichtswissenschaften gewinnen, dann stellen sich unweigerlich Fragen nach dem wissenschaftlichen Umgang mit ihnen. Daraus wurde bereits die Notwendigkeit eines Forschungsdatenmanagements abgeleitet.  Das Ziel ist, Methoden, Verfahren und Maßnahmen zur Handhabung von Forschungsdaten zu entwickeln. Es bezieht sich zum einen auf die Forschungs- und Arbeitsphasen innerhalb des Forschungsprozesses. Zum anderen geht es darüber hinaus, das heißt Forschungsdaten sollen in die Forschung zurückgespielt werden können und langzeitig zur Verfügung stehen. Problematisch ist hierbei, dass aufgrund des großen Anteils projektbezogener Einzelförderung - bei der DFG immerhin mehr als ein Drittel im Jahr 2020  – nicht allen Forschungsvorhaben ein nachhaltiges Forschungsdatenmanagement inhärent ist. Da es entsprechende Forschungsgebiete in der Vergangenheit schlichtweg noch nicht gab, war der Umgang mit Forschungsdaten mehr von individuellen digitalen Kenntnissen und Kompetenzen des oder der Wissenschaftler*in abhängig als von allgemeingültigen wissenschaftlichen Kriterien sowie technischen Standards. Zeitökonomisch betrachtet bedeutet der wissenschaftliche Umgang mit digitalen Forschungsdaten zudem Arbeitsaufwand, der zu den routinierten Abläufen hinzukommt. Erst recht, wenn sich ganz neu mit dieser Thematik auseinandergesetzt werden muss. Das wirft die nachvollziehbare Frage nach dem Kosten-Nutzen-Verhältnis für die eigene Forschungsarbeit auf. Diese wird sich

Klar ist, dass diese Aufgabe allein auf individueller Ebene nicht bewältigt werden kann, sondern dafür entsprechende digitale Infrastrukturen, Dienste und Tools unterstützend bereitgestellt werden müssen. Aktuell gibt es nationale Anstrengungen wie die Initiative Nationale Forschungsdateninfrastruktur , die in dieser offenen Situation Positionen und Lösungsstrategien entwickeln. Die Geschichtswissenschaften werden voraussichtlich ab Januar 2023 offiziell mit dem Konsortium „nfdi4memory“ vertreten sein und damit verbunden eine zehnjährige Förderung erhalten (Stand 10/2021).

\subsection{FAIR, CARE and Open Data}

drei Modelle besprechen

FAIR-Modell mit FAIR Principles
CARE-Modell
5-Sterne-Modell von Open Data (Tim Berners Lee)

Auf diese Prinzipien beruft sich auch die Wikimedia Foundation, für ihre Produkte und macht diese auch für die Wissenschaft interessant. Mittlerweile gibt es diverse Kooperationen zwischen wissenschaftlichen Einrichtungen und der Wikimedia. So hat die Deutsche Nationalbibliothek ein Projekt gestartet, in dem sie die GND zugänglicher und nachnutzbarer für gestalten will und damit ihre strenge GND-Policy

FAIR and CARE Principles

\section{Wirtschaftliche Existenzvernichtung der Juden im Nationalsozialismus}

Die ersten grundlegenden, wissenschaftlichen Auseinandersetzungen mit der wirtschaftlichen Verfolgung, Verdrängung und Vernichtung der Juden im Nationalsozialismus erfolgten zwar schon früh in der BRD im Nachkriegsdeutschland.\footnote{Im Jahr 1966 erschien die Pionierstudie von Helmut Genschel. Erst 20 Jahre später folgte die nächste grundlegende Studie des israelischen Historikers Avraham Barkai, der an Gentschels Ergebnisse anknüpfte. Vgl. Benno Nietzel: Die Vernichtung der wirtschaftlichen Existenz der deutschen Juden 1933-1945. Ein Literatur und Forschungsbericht, in: Friedrich-Ebert-Stiftung (Hg.), Archiv für Sozialgeschichte, Band 49, Bonn 2009, S. 561-613} Allerdings blieben diese vereinzelt und ohne größere Resonanz. 

Erst Ende der 1990er Jahren trat in Deutschland eine längere Forschungswelle zum Thema auf, die eine Bandbreite an Studien hervorgebracht hat und in deren Folge sich ein eigenes Forschungsfeld zur wirtschaftlichen Existenzvernichtung der Juden im Nationalsozialismus etablierte.\footnote{Als wegweisend wird regelmäßig die Lokalstudie zu Arisierung in Hamburg des Historikers Frank Bajohr aus dem Jahr 1997/98 gewertet. Siehe zum Beispiel Nietzel 2009, S. 561 oder Christiane Fritsche: Ausgeplündert, zurückerstattet und entschädigt. Arisierung und Wiedergutmachung in Mannheim, 2. Aufl., Ubstadt-Weiher, Heidelberg, Neustadt a. d. W., Basel 2013, S. 21. Frank Bajohr: ,,Arisierung'' in Hamburg. Die Verdrängung der jüdischen Unternehmer 1933-1945, 2. Aufl., Hamburg 1998 (zuerst 1997). Auf Ursachen des Forschungsbooms kann im Rahmen dieser Arbeit nicht eingegangen werden. Siehe dazu auch Christoph Kreutzmüller, Vernichtung der jüdischen Gewerbetätigkeit im Nationalsozialismus. Abläufe, Blickwinkel und Begrifflichkeiten, Version: 2.0, in: Docupedia-Zeitgeschichte, 12.3.2020, URL: \url{http://docupedia.de/zg/Kreutzmueller_vernichtung_der_juedischen_Gewerbetaetigkeit_v2_de_2020}} Es lieferte innerhalb der NS-Forschung weitere Erklärungsansätze zur antisemitischen Verfolgungs- und Vernichtungspolitik, deren Antriebskräfte in der Vergangenheit unterschiedlich interpretiert wurden.\footnote{Siehe zu den unterschiedlichen Deutungen und Perspektiven (insbesondere Intentionalismus vs. Strukturalismus) Bajohr 1998, S. 10-14} Hierbei waren lange nationalsozialistische Akteure, kommunale Verwaltungsinstanzen und nicht-jüdische Nutznießer sowie deren Strategien, Verhalten und Handlungsoptionen Schwerpunkt der Forschung. Diese Fokussierung wurde in zunehmendem Maß als zu einseitig kritisiert, da insbesondere die jüdischen Betroffenen ganz ausgeblendet oder sie ausschließlich als passive Opfer gezeigt worden seien. Zudem entwickelte sich langsam ein wissenschaftlicher Diskurs über die Anwendung historischer Begrifflichkeiten in der Forschung.\footnote{Vgl. Ludolf Herbst, Christoph Kreutzmüller, Ingo Loose u.a., Einleitung, in: Ludolf Herbst, Christoph Kreutzmüller, Thomas Weihe (Hg.): Die Commerzbank und die Juden 1933-1945, München 2004, S. 10-13. Diese Selbstkritik war ohne Zweifel richtig und auch notwendig, da sie grundlegende konzeptionelle Probleme im Forschungsfeld aufdeckte. Dennoch ist die einseitige Perspektive auf Täter, Mittäter und Mitwisser vor dem Hintergrund des jahrzehntelangen Verdrängens in der deutschen Nachkriegs- und Tätergesellschaft bis hin zu Geschichtsrevisionismus und Opfer-Umkehrung ein verständliches Anliegen gewesen. Letztlich leistete die Geschichtswissenschaft damit zwar einen späten aber nicht weniger wichtigen Beitrag zur historischen Aufarbeitung der NS-Verbrechen.} Im Zentrum stand hierbei die Kritik, dass die meisten Studien die Bandbreite und Komplexität des Forschungsthemas unter dem diffusen Begriff ,,Arisierung'' untersuchten und diesen dabei unterschiedlich ausdehnten.\footnote{Vgl. Nietzel 2009, S. 562-565. Mitunter wird der Begriff bis in die Zwangsarbeit hinein ausgeweitet. Siehe Britta Bopf: ,,Arisierung'' in Köln. Die wirtschaftliche Existenzvernichtung der Juden 1933-1945, Köln 2004, S. 11.} Häufig lag der Schwerpunkt der Untersuchung jedoch auf jüdischen Unternehmern und der Übernahme deren Eigentums\footnote{Siehe zum Beispiel Barbara Händler-Lachmann/Thomas Werther: Vergessene Geschäfte, verlorene Geschichte. Jüdisches Wirtschaftsleben in Marburg und seine Vernichtung im Nationalsozialismus, Marburg 1992; Alex Bruns-Wüstefeld: Lohnende Geschäfte. Die ,,Entjudung'' der Wirtschaft am Beispiel Göttingens, Hannover 1997; Bajohr 1997/98, Einleitung, S. 9f.; Marian Rappl: ,,Arisierung'' in München. Die Verdrängung der jüdischen Gewerbetreibenden aus dem Wirtschaftsleben der Stadt 1933-1939, in: Kommission für bayerische Landesgeschichte bei der Bayerischen Akademie der Wissenschaften in Verbindung mit der Gesellschaft für fränkische Geschichte und der Schwäbischen Forschungsgemeinschaft (Hrsg.), Zeitschrift für bayerische Landesgeschichte, Bd. 63, Heft 1, München 2000, S. 82-123, hier S. 125; Heinz-Jürgen Priamus (Hrsg.): Was die Nationalsozialisten ,,Arisierung'' nannten. Wirtschaftsverbrechen in Gelsenkirchen während des ,,Dritten Reiches'', Essen 2007, S. 11ff.}, wodurch die historische Forschung zuweilen Schlagseite erlitt, da andere Aspekte der wirtschaftlichen Existenzvernichtung wie zum Beispiel die Verdrängung von Juden aus ihren Berufen unterbelichtet blieben.\footnote{Vgl. Nietzel 2009, S. 565.} Zusammengefasst war der Einwand, dass die bisher verwendeten Untersuchungsbegriffe ,,engführend''\footnote{Kreutzmüller 2016/2020,  URL: \url{http://docupedia.de/zg/Kreutzmueller_vernichtung_der_juedischen_Gewerbetaetigkeit_v2_de_2020}} dahingehend seien, das Geschehene nur einseitig zu rekonstruieren, zu dessen gesamtheitlicher Analyse folglich nicht taugen.\footnote{Vgl. Nietzel 2009, S. 564 und Herbst/Weihe, Commerzbank, 2004, S. 10ff.}

Ab Mitte der 2000er Jahre lässt sich daraufhin eine Weiterentwicklung beobachten, die vor allem von größeren universitären Forschungsprojekten vorangetrieben wurde und die mit der Verschiebung in der Forschungsperspektive sowie der begrifflichen Ausdifferenzierung einher ging.\footnote{Pionierarbeit leistet hier u.a. das Forschungsprojekt ,,Geschichte der Commerzbank von 1870 bis 1958'' am Lehrstuhl für Zeitgeschichte an der Humboldt-Universität zu Berlin unter Leitung von Prof. Dr. Ludolf Herbst sowie das Forschungsprojekt zur Vernichtung der jüdischen Gewerbetätigkeit im Nationalsozialismus in den drei Großstädten Berlin, Breslau, Frankfurt am Main, ebendort. Siehe Ludolf Herbst/Thomas Weihe (Hg.), Die Commerzbank und die Juden 1933-1945, München 2004; Christoph Kreutzmüller, Ausverkauf. Die Vernichtung der jüdischen Gewerbetätigkeit in Berlin 1930-45, Berlin 2012; Benno Nietzel, Handeln und Überleben: jüdische Unternehmer aus Frankfurt am Main 1924-1964, Göttingen 2012} Die neueren Studien unterschieden sich im Wesentlichen dadurch, dass sie die jüdischen Betroffenen als handelnde Akteure begriffen und deren \textit{agency} in den Blick nahmen. Außerdem versuchten sie erstmals mit den Begriffen ,,Arisierung'' oder ,,Entjudung'' zu brechen\footnote{Unwissenschaftlich insofern, als dass es sich um rassistisch konnotierte Begriffe handelt, die selbst eigentlich zu historisieren wären, anstatt diese in die Wissenschaftssprache aufzunehmen. Vgl. Nietzel 2009, S. 563.} und Phänomene des Forschungsthema durch eine wissenschaftliche Terminologie zu benennen. Dabei wurde ein prozessorientierter Zugang gewählt, der an die Holocaust-Forschung des US-amerikanischen Historikers Raul Hilberg anknüpfte. Hilberg analysierte den Massenmord an den Juden wegweisend als einen Prozess, der über Definition, Kennzeichnung, Enteignung, Konzentration und Mord mehrstufig verlief.\footnote{Raul Hilberg: Die Vernichtung der europäischen Juden, Band 1, Frankfurt am Main 1990 (zuerst englisch 1961), S. 85-163. Eine wichtige Ergänzung zu Hilbergs Thesen war, dass die wirtschaftliche Existenzvernichtung der Juden der Teilprozess, war, der ,,am längsten – nämlich über den Tod der Opfer hinaus – dauerte und demzufolge in alle anderen Prozesse hineinreichte''. Kreutzmüller 2012, S. 378} Als integraler Bestandteil dieses Prozesses wurde die Vernichtung der wirtschaftlichen Existenz der Juden im Nationalsozialismus als ein mehrschichtiger Gesamtprozess analysiert, der sich aus den abgrenzbaren, aber überlagernden und in Wechselbeziehung stehenden Teilprozessen Verdrängung, Besitztransfer, Liquidation und Vermögensentzug zusammensetzte. Diese schlossen folglich die Verdrängung der Juden aus dem Berufsleben, die Vernichtung der jüdischen Gewerbetätigkeit durch Besitzübernahme oder Liquidation sowie die Entziehung des Vermögens der Juden ein.\footnote{Exemplarisch wurden erstmals alle Teilprozesse systematisch im Rahmen der Erforschung der Geschichte der Commerzbank betrachtet. Siehe Herbst/Weihe, Commerzbank, 2004.}

Mit diesem Forschungsansatz konnte zum einen anhand der drei deutschen Großstädte Berlin, Frankfurt am Main und Breslau empirisch  gezeigt werden, dass die als jüdisch verfolgten Unternehmen nicht - wie bisher durch die Schwerpunktsetzung der historischen Forschung suggeriert - größtenteils in den Besitz nichtjüdischer Erwerber*innen übergingen, sondern schlichtweg liquidiert wurden.\footnote{Vgl. Kreutzmüller 2016/2020} Diesbezüglich lag der Erkenntnisfortschritt in der Freilegung des Teilprozess der Vernichtung der jüdischen Gewerbetätigkeit als ein ,,großangelegtes Liquidationsprogramm'', das bisher kaum als solches von der historischen Forschung reflektiert worden war.\footnote{Vgl. Nietzel 2012, S. 164 und Kreutzmüller 2012, S. 250.} Des Weiteren wurde durch den Wechsel der Forschungsperspektive systematisch herausgearbeitet, dass sich die jüdischen Betroffenen gegen ihre Entrechtung wehrten und dazu verschiedenen institutionelle wie individuelle Strategien nutzten.\footnote{Systematisch untersucht von Kreutzmüller, Ausverkauf, 2012, Kapitel IV. Abwehrstrategien jüdischer Gewerbetreibender, S. 257-357; Nietzel, Handeln und Überleben, 2012, Kapitel II.2 Erwartungen, Anpassung und Selbstbehauptung, S. 99-150.}

An diesen Forschungsstand anknüpfend, unternahm zuletzt der Historiker Benno Nietzel im Jahr 2009 den Versuch, die zahlreichen Forschungsstudien zur Vernichtung der wirtschaftlichen Existenz der Juden im Nationalsozialismus zu ordnen, indem er die bisherigen Forschungsfragen, Untersuchungsgegenstände sowie Forschungsergebnisse zusammenfasste und strukturierte.\footnote{Auch Nietzel sprach von "analaytischer Hilflosigkeit angesichts der Vielschichtigkeit und Komplexität des Prozesses [der wirtschaftlichen Existenzvernichtung der Juden, Anm. S.E.]", ebd. S. 564.}. Sein Ziel war es, die wirtschaftliche Existenzvernichtung der Juden als ein abgrenzbares Forschungsfeld abzustecken, einheitlich zu definieren und damit einer einer systematischeren Bearbeitung zuzuführen. Dafür definierte er fünf Teilbereiche des Forschungsfelds:
\begin{itemize}
\item Verdrängung der Juden aus dem Berufsleben (Angestellte, Beamte, Selbstständige wie Rechtsanwälte, Ärzte oder Wissenschaftler)
\item Vernichtung der jüdischen Gewerbetätigkeit (Besitztransfer und Liquidation)
\item staatliche Enteignung des jüdischen Vermögens (Privatbesitz, Firmenvermögen, Immobilienvermögen aus Grundbesitz) 
\item Entgrenzung (transnationale Perspektiven)
\item Wiedergutmachung nach 1945 in der BRD
\end{itemize}

Zwar betonte er deren überschneidende Beziehungen und Verhältnisse zueinander, nahm aber in erster Linie eine separierte Betrachtung zum Zwecke der inhaltlichen Erschließung und zur Herausarbeitung von Spezifika des Forschungsthemas vor.\footnote{Nietzel 2009, S. 562. Nietzel greift außerdem die Beteiligung von nichtjüdischen Unternehmen mit auf aber explizit nicht als eine eigene Kategorie sondern als Querschnittaspekt, weshalb dieser hier nicht berücksichtigt wird, da er strenggenommen zum Forschungsfeld der Unternehmensgeschichte gehört. Siehe zu Unternehmensgeschichte Ralf Ahrens, Unternehmensgeschichte, Version: 1.0, in: Docupedia-Zeitgeschichte, 1.11.2010, URL: \url{http://docupedia.de/zg/Ahrens_unternehmensgeschichte_v1_de_2010}} 

Neben den bereits erläuterten Teilprozessen ordnete Nietzel dem Forschungsfeld außerdem die historisch untrennbare materielle Wiedergutmachung nach 1945 in der BRD zu, welche zum einen die Restitution/ Rückerstattung und zum anderen die Entschädigung meint. Hiervon ausgenommen ist die Entziehung und die Restitution von Kulturgütern, die Nietzel dem eigenen Forschungsfeld der Provenienzforschung zuordnete.\footnote{Vgl. ebd. S. 273} Im Falle der Entgrenzung vor allem nach Kriegsbeginn geht um die europaweite Perspektive der wirtschaftlichen Existenzvernichtung. Im Sinne des transnationalen Forschungsansatzes stehen dabei der Transfer von Erfahrungswissen und der Export von Verfolgungspraktiken sowie deren Weiterentwicklung in den besetzten Gebieten im Fokus. Auch Kollaboration und die Rolle von deutschen Unternehmen bei der Ausplünderung der europäischen Juden werden in den Blick genommen.\footnote{Vgl. ebd. S. 602-608}

Nietzels Systematisierungsversuch wurde bisher auffallend wenig von der historischen Forschung rezipiert.\footnote{Aus Literaturrecherche und Interviews ging nicht hervor, dass Nietzels Systematik nachträglich kontrovers diskutiert oder weiterentwickelt wurde.} Lediglich der Historiker Christoph Kreutzmüller nahm 2016 darauf Bezug und ergänzte den neuesten Forschungsstand zur Vernichtung der jüdischen Gewerbetätigkeit.\footnote{Siehe Kreutzmüller 2016/2020,  URL: \url{http://docupedia.de/zg/Kreutzmueller_vernichtung_der_juedischen_Gewerbetaetigkeit_v2_de_2020}} Auch wenn dieser eine deutliche Professionalisierung darstellt, weil erstmals unter Einbeziehung aller relevanten Forschungsstudien konzeptionell mit dem komplexen Forschungsthema auseinandergesetzt wurde, so bleibt festzuhalten, dass der Begriff ,,Arisierung'' als Untersuchungsbegriff in der historischen Forschung nach wie vor zur Anwendung kommt.\footnote{Siehe Maren Janetzko: Die ,,Arisierung'' mittelständischer jüdischer Unternehmen in Bayern 1933-1939. Ein interregionaler Vergleich, Ansbach 2012, S. 17f; Claudia Flümann: ,,... doch nicht bei uns in Krefeld!". Arisierung, Enteignung, Wiedergutmachung in der Samt- und Seidenstadt 1933-1963, Krefeld 2015, S. 13 oder jüngst bei Monika Juliane Gibas: ,,Arisierung'' der Wirtschaft in Thüringen: Das Beispiel Arnstadt, in: Schlossmuseum Arnstadt (Hrsg.): Jüdische Familien aus Arnstadt und Plaue. Katalog zur Sonderausstellung im Schlossmuseum Arnstadt, Arnstadt 2021, S. 108-148.}  

Charakteristisch für das Forschungsfeld ist zudem, dass lokal- bzw. regionalgeschichtliche Studien dominieren. Zwar wurde das Thema auch in Form von Überblicks- oder Gesamtdarstellungen zum Deutschen Reich (in den Grenzen von 1937) abgehandelt, dies jedoch nur vereinzelt und vor allem in den Anfangsjahren der wissenschaftlichen Auseinandersetzung mit dem Thema.\footnote{Siehe zum Beispiel die bereits erwähnten grundlegenden Studien von Genschel 1966 und Barkai 1987. Danach erschienen sind noch: Günter
Plum, Wirtschaft und Erwerbsleben, in: Wolfgang Benz (Hrsg.), Die Juden in Deutschland 1933–
1945. Leben unter nationalsozialistischer Herrschaft, München 1988, S. 268–313. Dieter Ziegler, Die wirtschaftliche
Verfolgung der Juden im »Dritten Reich«, in: Heinz-Jürgen Priamus (Hrsg.), Was die
Nationalsozialisten ,,Arisierung'' nannten. Wirtschaftsverbrechen in Gelsenkirchen während des
»Dritten Reiches«, Essen 2007, S. 17–40. Für die Literaturanalyse wurden vier Überblicks- bzw. Gesamtdarstellungen und fünfzehn Lokalstudien erfasst. Es ist natürlich nicht auszuschließen, dass es mehr Darstellungen zum Deutschen Reich oder zu Europa gibt, aber eine Tendenz im Forschungsfeld hin zu lokalhistorischen Studien ist nichtsdestotrotz deutlich erkennbar.} In den letzten fünfzehn Jahren sind überwiegend Untersuchungen zu Klein- und Großstädten erschienen, deren Ergebnisse ebenfalls vereinzelt in Form von Sammelbänden zusammengefasst wurden.\footnote{Siehe zum Beispiel Christiane Fritsche u.a (Hrsg.), ,,Arisierung'' und ,,Wiedergutmachung'' in deutschen Städten, Köln 2014. Allerdings handelt es sich dabei um einen ,,partikularistischen Zugriff'' auf das Thema, dessen Stärken vor allem in der zusammenfassenden Darstellung der aktuellen Forschungsergebnisse liegt als im Generieren neuer Erkenntnisse. Siehe Rezension dazu: Jan Schleusener: Rezension zu: Fritsche, Christiane; Paulmann, Johannes (Hrsg.), ,,Arisierung'' und ,,Wiedergutmachung'' in deutschen Städten, Köln  2014. ISBN 978-3-412-22160-7, In: H-Soz-Kult, 10.12.2014, \url{www.hsozkult.de/publicationreview/id/reb-21747}.} Diese Entwicklung hat zwei Gründe:

Da sich die historische Forschung zum Thema, wie oben erläutert, früh auf die Vernichtung der jüdischen Gewerbetätigkeit in Deutschland konzentriert hat, ist sie wissenschaftlich begründet. Denn jene erfolgte erst ab 1938 mit der Einführung reichsweiter Gesetze und Regelungen.\footnote{Darunter fiel auch die antisemitische Definition, was unter einem "jüdischen Gewerbebetrieb" verstanden werden sollte.} Das heißt, dass die jüdische Gewerbetätigkeit für die nationalsozialistische Wirtschaftspolitik erst spät auf dem Plan stand.\footnote{Vgl. Nietzel 2009, S. 562, 565 und 576.} Anders sah es hingegen in der politischen Peripherie aus, wo bereits ab 1933 mit den Aprilboykotten jüdische Gewerbebetriebe gezielt verfolgt wurden und in deren Folge jüdische Gewerbebetriebe verschwanden. Es waren insbesondere also lokale Akteure gewesen, die den Vernichtungsprozess vorangetrieben hatten. Auch nach 1938 waren sie es, die die reichsweiten Gesetze und Bestimmungen umsetzten. Es ist daher wenig überraschend, dass die Wissenschaft überwiegend den lokalhistorischen Zugang gewählt hat, da in einer Überblicksdarstellung für Deutschland die Vernichtung der jüdischen Gewerbetätigkeit unmöglich in der notwendigen Dichte beschrieben und rekonstruiert werden kann.\footnote{Programmatisch war hier wieder die Lokalstudie zu Hamburg von Frank Bajohr Ende der neunziger Jahre. Siehe Bajohr 1997/98.} 

Neben der wissenschaftlichen Begründung, die von fast allen Studien vorgetragen wird\footnote{\textbf{hier Studien}}, wird in diesen seltener reflektiert, dass viele Forschungsprojekte dem Bereich der lokalen, insbesondere der städtischen Erinnerungskultur entsprungen sind, was zur lokalgeschichtlichen Dominanz sicherlich mit beigetragen hat.\footnote{\textbf{hier Projekte aufzählen}}. Als Erklärungsansatz für diese besondere Entwicklung scheinen die gesellschaftlichen Auf- und Umbruchszeiten der 1980er Jahre plausibel. In der Tradition der basisdemokratischen und dezentralen Graswurzelbegewegung (,,Grabe, wo du stehst'')\footnote{\textbf{Programmatisch war hier ???}} mit der Etablierung zahlreicher lokaler Geschichtswerkstätten ab Anfang der 1980er Jahre in der BRD war die Motivation verbunden, die nationalsozialistische Geschichte des eigenen Ortes kritisch aufzuarbeiten.\footnote{Siehe zur Geschichte und zum Einfluss der Bewegung: Jenny Wüstenberg, Zivilgesellschaft und Erinnerungspolitik in Deutschland seit 1945, Berlin Münster 2020, Kapitel 4 Grabe, wo stehst: Die Geschichtsbewegung und die Graswurzel-Erinnerungskultur S. 147-200 und Kapitel 5 Memorialästhetik und die Erinnerungsbewegungen der 1980er, S. 201-230.} Ab Mitte der 80er Jahre rückten zunehmend die jüdischen Opfer ins Bewusstsein und es stand ein angemessenes, innovatives Gedenken sowie die Schaffung von Gedenkorten im Fokus.\footnote{Das bekannteste Projekt ist wahrscheinlich das Stolperstein-Projekt des Künstlers Gunther Demnig. Vgl. Wüstenberg 2020, S. 209. Die erste Verlegung in Berlin-Kreuzberg im Jahr 1996 war von den Behörden noch nicht genehmigt worden und wurde erst später legalisiert. Siehe Projektwebsite, URl: \url{http://www.stolpersteine.eu/start/} (Letzter Zugriff am 26.01.2022)} Alles in allem waren die Akteure dieser Bewegung von einem emanzipatorischen (,,Geschichte von unten''), einem aufklärerischem (Lernen aus der Geschichte) sowie einem moralischen (Vergangenheit nicht vergessen) Antrieb geleitet. Sie wollten die etablierte Geschichtsforschung und Erinnerungspolitik durch Demokratisierung von unten und Partizipation von Grund auf verändern.\footnote{Das diese Ideale in der Praxis nicht vollkommen widerspruchs- und konfliktfrei blieben, zeigt sehr anschaulich der historische Abriss von Jenny Wüstenberg. Vgl. Wüstenberg 2020, S. 166f. und 182ff.} Diese Entwicklung hatte Rückkopplungseffekte auf die akademische Geschichtswissenschaft, die sich von einer sozialhistorischen Ausrichtung hin zu einer \textit{Alltagsgeschichte} als neuer Forschungsansatz weiterentwickelte. Kennzeichnend für 

Bevor abschließend 

Abschließend deutlich geworden ist, dass die Forschungsdaten historiographisch im  Kontext des Forschungsfelds zur Vernichtung der wirtschaftlichen Existenz der Juden im Nationalsozialismus entstanden sind, welches Teil der umfassenden NS-Forschung ist und insbesondere an die Holocaust-Forschung angeknüpft. Das Forschungsfeld wird seit circa 20 Jahren Jahren systematisch bearbeitet. Hierbei dominieren lokalgeschichtliche Zugänge. Sofern es also Forschungsdaten gibt, dann wurden diese in der Vergangenheit vorwiegend im Rahmen von Studien generiert, die sich bei ihrer wissenschaftlichen Analyse geografisch begrenzt haben. Dementsprechend sind die zugehörigen Forschungsdaten räumlich von begrenzter Aussage, da sie jeweils lediglich einen Ort oder eine Region abbilden. Damit handelt es sich bei diesen Lokalstudien gleichzeitig um Fallstudien, die genau genommen erst in ihrer Summe eine Gesamtdarstellung für das Deutsche Reich in den Grenzen von 1937 ergeben. Eine Synthese dieser bisher nebeneinander existierenden Forschungsergebnisse gibt es noch nicht.\footnote{Vgl. Nietzel S.} Die Herausforderung besteht darin, die in einem Zeitraum von über zwanzig Jahren publizierten, verschiedenen Lokalstudien in Bezug auf ihre Forschungsdaten erstmals zusammenzuführen und in ein (projekt)übergreifendes FDM zu überführen. Zu beachten ist hierbei, dass das Forschungsfeld nicht ausschließlich im akademischen Umfeld bearbeitet wurde und wird, sondern unterschiedlichste zivilgesellschaftliche Initiativen oder Einzelpersonen ebenfalls ein wesentlicher Treiber der Forschung waren und sind. Eine strikte Trennung in akademisch einerseits und nichtakademisch andererseits erscheint nicht sinnvoll, da sich beide Bereiche in der Vergangenheit gegenseitig bedingten und befruchteten.\footnote{Zum Verhältnis von akademischer und nichtakademischer historischer Forschung vgl. Wüstenberg 2020, S. 163ff. Überschneidungen gab es vor allem bei beim Organisieren auf personeller Ebene.} Das bedeutet, dass potentielle Anwender*innen von offenem FDM im Forschungsfeld sowie deren Nutzungsmotive und Nutzungserwartungen äußerst heterogen sind. Die sich daraus ableitenden Zielgruppen und Stakeholder von offenem FDM werden in Kapitel 3.1. separat definiert und beschrieben.

Auffällig ist, dass das Forschungsfeld inhaltlich in den letzten 20 Jahren enorm voranschritt, aber im Vergleich auf konzeptueller Ebene die Weiterentwicklung stagnierte. Wenn in ausnahmslos jeder Studie der Begriff ,,Arisierung'' (oder ,,Entjudung'') kritisch und problemorientiert hinterfragt wird, in der Konsequenz aber nicht aus der wissenschaftlichen Arbeit verbannt, sondern entgegen der eigenen Argumentation als Untersuchungsbegriff beibehalten wird, dann herrscht ein offensichtlicher Mangel an einer breiteren konzeptionellen und methodischen Auseinandersetzung im Forschungsfeld. Dafür spricht auch, dass es bis heute keine einheitliche Definition des Begriffs gibt.\footnote{Und die es auch in der Geschichte des Begriffs nie gegeben hat.\textbf{Vgl. Nietzel und Kreutzmüller}} Einerseits wird darunter speziell der Transfer von jüdischem Eigentum, insbesondere Firmeneigentum, in nicht-jüdischen Besitz und andererseits generisch der gesamte Prozess der wirtschaftlichen Existenzvernichtung der Juden gefasst, wobei dieser unterschiedlich ausgedehnt wurde\footnote{•} Einen allgemeingültigen wissenschaftlichen Konsens scheint es auf der methodischen Ebene im Forschungsfeld nicht zu geben. Unklar ist, warum nach den eindeutig nachvollziehbaren Gegeneinwänden und alternativen Vorschlägen aus dem Forschungsfeld selbst sich diese methodische Schwäche bis heute hartnäckig hält. 
Im Umkehrschluss stellt sich damit die Ausgangslage für das offene Forschungsdatenmanagement als nicht absolut eindeutig dar, was insofern problematisch ist, als dass das offenen FDM, sofern es digital laufen soll, aus entwicklungstechnischer Sicht widerspruchsfrei, in der Regel in Form eines Datenmodells, beschrieben werden muss. Die Weiterverwendung unwissenschaftlicher Begrifflichkeiten scheint an dieser Stelle erst recht nicht geeignet, da sie in keiner Weise zur Präzision beitragen. Als derzeit einzige Möglichkeit, sich im Forschungsfeld zwischen den unterschiedlichen Studien zu orientieren, bietet sich der in Kapitel 2.1. erläuterte Systematisierungsversuch des Historikers Nietzel an. Er wird in dieser Arbeit methodisch als Taxonomie aufgegriffen, die es ermöglicht, erstens die Berliner Forschungsdaten sowie die in Kapitel 2.2. weiter betrachteten Forschungsdaten zur Vernichtung der jüdischen Gewerbetätigkeit zu klassifizieren. Von hier aus wird deutlich sichtbar, dass diese inhaltlich lediglich einen kleinen Ausschnitt aus dem Gesamtkomplex der wirtschaftlichen Existenzvernichtung der Juden im NS abbilden, diesen also nur teilweise repräsentieren und darüber hinaus inhaltlich in den größeren Prozess der Verfolgung und Vernichtung der Juden in Deutschland eingebettet sind. Auch wenn im Rahmen dieser Arbeit der Schwerpunkt auf der Vernichtung der jüdischen Gewerbetätigkeit liegt, wird das FDM offen konzipiert, das bedeutet, dass es inhaltlich anschlussfähig erstens an die weiteren Unterkategorien des Forschungsfelds ist und zweitens in der Entwicklungsperspektive auch an benachbarte Forschungsfelder der Verfolgung und Vernichtung im Nationalsozialismus andocken kann. Damit läuft die Konzeption auf eine prototypische Lösung von offenem FDM hinaus, die übertragbar auch auf andere zeitgeschichtliche Forschungsfelder ist.