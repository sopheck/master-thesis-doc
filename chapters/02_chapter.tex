\onehalfspacing
Open Science zwar ein positiv besetzter und vielfach benutzter Begriff ist, aber bei genauerem Hinsehen pauschal kein Selbstläufer ist
Forschungsstand zu Open Science und Forschungsdatenmanagement
Open Science in Zusammenhang mit Forschungsdaten und Forschungsdatenmanagement praktiziert werden kann. Nach einem allgemeinen Überblick über Ursachen, Ziele und Konzepte von Open Science, wird anschließend auf Open Data fokussiert und 
Hier wird im Weiteren untersucht in welchem Verhältnis Open Data zum noch relativ jungen Forschungsfeld des Forschungsdatenmanagements (FDM) steht, das bezüglich digitaler Forschungsdaten in den letzten Jahren an Bedeutung in der Wissenschaft gewonnen hat. Dazu wird das Forschungsfeld selbst kurz überblickt. Zum Schluss bleibt das dritte Foschungsfeld der Vernichtung der wirtschaftlichen Existenz der Juden im Nationalsozialismus, an dem exemplarisch anhand von Foschungsdaten zur Vernichtung der jüdischen Gewerbetätigkeit der Bedarf sowie Anforderungen an offenes Forschungsdatenmanagement, welches Open Science-Ansätze integriert, analysiert werden.


Auch wenn von der Replikationskrise nicht direkt betroffen, so stellt sich die Frage nach Qualitätssicherung von wissenschaftlicher Forschung und Arbeit ebenso für die Geschichtswissenschaften - gerade in der digitalen Transformationphase, die tiefgreifende Veränderungen in der Forschungspraxis mit sich bringt. Die selbstkritische Haltung der Open Science-Bewegung gegenüber der eigenen Disziplin wird in dieser Arbeit aufgegriffen und deren abgeleitete Lösungsstrategien aufgenommen.



über den historiographischen Kontext Klarheit verschafft zur Einordnung der hier betrachteten historischen Forschungsdaten

Hinsichtlich der Verfahren und Strategien nicht alle der gesamten Open Science Taxonomie erläutern, sondern in Bezug auf Thema der Arbeit auf Forschungsdaten beschränken. Open Data und FAIR-Data, wo letztere seinen Ursprung aus dem Feld des FDM kommt  

Um ein solches offenes FDM in dieser Arbeit im Sinne der Praktikabilität und Überschaubarkeit im Griff zu behalten, muss es allerdings auch klar abgegrenzt sein. Hier bietet sich die Fokussierung auf einen handhabbaren thematischen Rahmen an, dessen Grenzen also zwischen wissenschaftlicher Schlüssigkeit/ Nachvollziehbarkeit und pragmatischer Machbarkeit ausloten. In einem ersten Schritt wird daher zur thematischen Einordnung der größere  historiographische Kontext dargestellt, in dem sich das offene FDM bewegt. Darauf folgt die thematische Eingrenzung auf Forschungsfeld zur Vernichtung der jüdischen Gewerbetätigkeit im Nationalsozialismus. Dieses bildet den Kontextrahmen, in dem das offene FDM zukünftig arbeiten wird und dessen Merkmale in einem Überblick vorgestellt werden.

\section{Open Science}

Open Science gewinnt auf wissenschaftlicher, wissenschaftspolitischer und gesellschaftlicher Ebene an Bedeutung. Doch was genau das Schlüsselwort ,,Open'' im Kontext von Wissenschaft aussagt, erschließt sich nicht sofort. Um Open Science zu verstehen, was also Open Science ist und warum diese als notwendig eingestuft wird, wird die gleichnamige Bewegung in den Blick genommen und deren Ursprünge überblickt.\footnote{Genau genommen ist das Konzept von Open Science, also im Kern eigene Forschungsmethoden,  -praktiken und -ergebnisse transparent für andere zu machen, schon älter und findet Anwendung bereits in der Renaissance. Für das Thema dieser Arbeit ist eine longue durée letztlich wissenschaftlicher Praxis jedoch nicht relevant. Daher wird sich auf die aktuelle Bewegung und deren direkte Ursprünge begrenzt. Siehe auch Paul A. David: Common Agency Contracting and the Emergence of ,,Open Science'' Institutions, in: The American Economic Review (Hrsg.), 2. Ausgabe, 1998, S. 15–21, URL (stable): \url{http://www.jstor.org/stable/116885.}.} Zudem wird der Versuch unternommen, den Begriff Open Science für eine Anwendung in dieser Arbeit zu definieren. Anhand von existierenden Technologien und Infrastruktuen wird abschließend herausgearbeitet, wo Open Science gegenwärtig steht, woraus sich wiederum Konsequenzen für die Implementierung eines offenen Forschungsdatenmanagements ergeben. 

\subsection{Ursprünge der Open Science-Bewegung}

Hinsichtlich der Entstehung der Open Science-Bewegung können zwei Entwicklungsstränge verfolgt werden. Zum einen lässt sie sich auf ein konkretes Ereignis innerhalb der Wissenschaft zurückverfolgen, nämlich auf die sogenannte Replikationskrise. Hier bezieht sich Open Science explizit auf die Transformation wissenschaftlicher Forschungsmethoden und -praktiken, um Forschung noch robuster zu machen. Zum anderen ist Open Science Teil der breiteren sozialen Open-Bewegung, welche von der Do-it-yourself-Bewegung, der Hacker-Bewegung der 1960/ 70er sowie der Freie-Software-Bewegung der 1980er Jahre
(Vorgänger der Open Source-Bewegung) stark beeinflusst ist.\footnote{Vgl. ayway media (Hrsg.): Das digitale Handbuch,
Kapitel C.15 Die ,,Open-Bewegung'', Vettelschloss 2016, S. 252}

\paragraph{Replikationskrise} Ab Mitte der 2010er Jahre erhielten in der Wissenschaft, vordergründig in der Psychologie sowie in den Lebens- und Naturwissenschaften, zunehmend Replikationsstudien Aufmerksamkeit. Diese konnten in sogenannten Replikationsversuchen eine statistisch signifikante Anzahl publizierter empirischer Forschungsergebnisse entweder falsifizieren oder nicht replizieren, weil die Daten nicht zur Verfügung standen.\footnote{Als erste Replikationsstudie dieser Art wird jene des Medizinwissenschaftlers und Statistiskers John Ioannidis aus dem Jahr 2005 gezählt, mit der er erstmals systematisch versuchte, veröffentlichte Untersuchungsergebnisse nachträglich zu replizieren/ reproduzieren. Siehe John P.A. Ioannidis: Why Most Published Research Findings Are False, PLoS Med 2(8): e124, 2005, doi:10.1371/journal.pmed.0020124. Es folgten eine Reihe weiterer Replikationsstudien auch in anderen Fächern wie den Sozialwissenschaften. Siehe zum Beispiel Marjan Bakker, Annette van Dijk, Jelte M. Wicherts: The Rules of the Game Called Psychological Science, in: Perspectives on Psychological Science, 7(6), 2012, S. 543-554, doi:10.1177/1745691612459060; Thomas Herndon, Michael Ash, Robert Pollin: Does high public debt consistently stifle economic growth? A critique of Reinhart and Rogoff, in: Cambridge Political Economy Society (Hrsg.), Cambridge Journal of Economics, Band 38, 2. Ausgabe, Oxford 2014, S. 257-279, URL (stable): \url{https://www.jstor.org/stable/24694929}; Jeremy Freese, David Peterson: Replication in Social Science, in: Annual Reviews (Hrsg), Annual Review of Sociology, Band 43, San Mateo 2017, S. 147-165, doi:10.1146/annurev-soc-060116-053450} Das löste die vielfach diskutierte ,,Replikationskrise'' in den betroffenen Fächern aus. Zum einen ging es, hinsichtlich der Falsifizierungen, nachträglich um Ursachenforschung, die sich auf Defizite insbesondere bei den Forschungsmethoden und in der Publikationspraxis wissenschaftlicher Journals fokussierte.\footnote{Diskutiert wurden insbesondere, wie das Institut für Psychologie an der Humboldt-Universität zu Berlin konzis berichtete, "p-hacking, selektives Berichten von (abhängigen) Variablen, Hypothesizing After the Results are Known (HARKING), nur signifikante Ergebnisse berichten, mehr Daten sammeln nachdem die bestehenden Daten keine positiven Ergebnisse hervorgebracht haben, Publikations Bias". Methodengruppe Berlin (Autorengruppe): Die Replikationskrise und Open Science, Blog Post, Humboldt-Universität zu Berlin, Lebenswissenschaftliche Fakultät Institut für Psychologie, Lehrstuhl für Psychologische Methodenlehre (Hrsg), URL: \url{http://methods-berlin.com/de/replikationskrise_open_science/} (letzter Zugriff am 21.04.2022). Siehe auch Klaus Fiedler, Norbert Schwarz: Questionable Research Practices Revisited, in: SAGE Publishing (Hrsg.), Social Psychological and Personality Science, Band 7, 1. Ausgabe, 2016, S. 45-52, doi:10.1177/1948550615612150; Annie Franco, Neil Malhotra, Gabor Simonovits: Publication bias in the social sciences. Unlocking the file drawer, in: American Association for the Advancement of Science (Hrsg.), Science, Band 345, Ausgabe 6203, Washington 2014, S. 1502-1505, doi:10.1126/science.1255484.} Aber auch die Replikationsstudien selbst wurden kritisch betrachtet.\footnote{Vgl. Deutsche Forschungsgemeinschaft (Hrsg.): Replizierbarkeit von Forschungsergebnissen. Eine Stellungnahme der Deutschen Forschungsgemeinschaft, Stand: April 2017, URL: \url{https://www.dfg.de/download/pdf/dfg_im_profil/geschaeftsstelle/publikationen/stellungnahmen_papiere/2017/170425_stellungnahme_replizierbarkeit_forschungsergebnisse_de.pdf} (letzter Zugriff am 21.04.2022).} Zum anderen war, hinsichtlich der Nichverfügbarkeit von Daten, eine wesentliche Eigenschaft von robuster evidenzbasierter Forschung, nämlich die Nachvollziehbarkeit ihrer Ergebnisse durch Replikation (als Bestandteil von Qualitätssicherung), nicht mehr gegeben und damit in der Konsequenz auch ein gesellschaftlicher Bedeutungsverlust von Wissenschaft bei der Wissensproduktion zu befürchten. 

Kurzum ging es um die existenzielle Frage, wie Wissenschaft praktiziert werden muss, damit wissenschaftliche Forschung, insbesondere die statistisch empirische, reliabel ist. Als Antwort auf diese Krise hat sich in den vergangenen Jahren die internationale Open Science-Bewegung formiert\footnote{Entsprechend der Internationalität der Open Science-Bewegung, existieren weltweit Open Science Initiativen, von denen allein in Deutschland hier nur eine Auswahl wiedergegeben werden kann: Berlin School of Public Engagement and Open Science als Kollaborationsprojekts des Museums für Naturkunde Berlin, der Humboldt-Universität zu Berlin und der Robert-Bosch-Stiftung, URL: \url{https://www.museumfuernaturkunde.berlin/de/future/wissenschaftscampus/berlin-school-public-engagement-and-open-science}; Open Science Working Group an der FU Berlin, URL: \url{https://www.fu-berlin.de/sites/open-science}; Open Science Center an der LMU München;  Initiative für Offene Wissenschaft und Innovation des Stifterverbands, URL: \url{https://www.stifterverband.org/open-science-innovation-netzwerke}.}, die in den Anfangsjahren stark auf die Frage nach Replizierbarkeit von Forschungsstudien fokussiert war. 

In Deutschland hat sich zuletzt das \textit{German Reproducibility Network} (GRN) gegründet, das fachübergreifend gezielt Replikationsstudien und Open Science Praktiken unterstützen möchte.\footnote{Zu dessen Hauptakteuren gehören u.a. Berlin University Alliance, das Helmholtz Center (Open Science), das LMU Open Science Center (OSC), das Netzwerk der Open Science Initiativen (NOSI), die Deutsche Gesellschaft für Psychologie (DGPs), u.a. Siehe Ankündigung der Berlin University Alliance: German Reproducibility Network gestartet, News vom 01.02.2021, URL: \url{https://www.berlin-university-alliance.de/news/items/2021/210201-grn.html}. Homepage des GRN unter URL: \url{https://reproducibilitynetwork.de/} (alle letzter Zugriff am 27.04.2022).} Auf internationaler Ebene ist vor allem das interdisziplinäre \textit{Center for Open Science} (COS) zu nennen, welches in direkter Reaktion auf die Replikationskrise 2013 in den USA gegründet wurde\footnote{URL: \url{https://www.cos.io/?hsLang=en} (letzter Zugriff am 21.04.2022).}. Eine der ersten Aktivitäten des COS war das mit der University of Viginia gemeinsam großangelegte \textit{Reproducibility Project}, in dem sich eine Autorengruppe, welche sich ,,Open Science Collaboration'' nannte, systematisch mit der Reproduzierbarkeit von 100 Forschungsstudien in der Psychologie auseinandersetzte.\footnote{Brian A. Nosek, Johanna Cohoon, Mallory C. Kidwell, Jeffrey R. Spies: Estimating the reproducibility of psychological science, in: American Association for the Advancement of Science (Hrsg.), Science, Band 349, Ausgabe 6251, Washington 2015, doi:10.1126/science.aac4716.}. Nach der Bestandsaufnahme, bei der die Rate nichtreplizierbarer Forschungsstudien wie bei vorausgegangenen Replikationsstudien signifikant hoch war, widmete sich das COS verstärkt den Strategien zur Überwindung der Replikationskrise, die im Kern ebenfalls als eine methodische Krise identifiziert wurde sowie zweifelhafte Forschungspraktiken aufdeckte.

\paragraph{Open-Bewegung} Die Open Science-Bewegung ist Teil der breiten sozialen Open-Bewegung, welche unter den Begriffen ,,Open'', ,,Openness'' beziehungsweise ,,Free'' subsumiert, ,,Daten, Entwürfe, Fotos, Musikstücke oder sonstige Inhalte und Wissen'' \footnote{Wikimedia Deutschland e. V., Open Knowledge Foundation Deutschland e. V. (Hrsg.): ABC der Offenheit, Berlin 2019, S. 4f., URL: \url{https://commons.wikimedia.org/wiki/File:ABC_der_Offenheit_-_Broschüre_(2019).pdf} (letzter Zugriff am 26.04.2022).} aus allen gesellschaftlichen Bereichen zur Weiterverbreitung sowie Wiederverwendbarkeit schrankenlos zur Verfügung stellen und dadurch Teilhabe als demokratisches Prinzip in einer freiheitlichen Gesellschaft stärken will. Außerdem sieht sie in dieser Kultur der Offenheit Potenzial für neue Innovationen\footnote{Ebd. sowie siehe auch Open Knowledge Foundation (Hrsg.): Why open data? URl: \url{https://okfn.org/opendata/why-open-data/} (letzter Zugriff am 26.04.2022).} Diese Forderungen sind zwar nicht grundsätzlich neu, bekamen aber mit der Verbreitung des World Wide Web (WWW) ab Mitte der 1990er Jahre\footnote{Veröffentlichung des ersten Webbrowsers Netscape in offener Lizenz, die Personen auf der ganzen Welt mit PC und Internetverbindung ermöglichte, frei im Web ,,zu surfen''} einen neuen Schub. Dies ist in der Natur des WWW selbst begründet. Denn dessen Schlüsseleigenschaft ist es - seit seiner Entstehung 1989 - Informationen system- und plattformunabhängig in einer gemeinsamen Netzwerkinfrastruktur zu übertragen und auszutauschen.\footnote{Erfunden wurde das WWW vom Physiker und Informatiker Tim Berners-Lee, der 1989 am CERN in Genf arbeitete und technischen Lösungen suchte, wie unter Forschern schnell und einfach kommuniziert werden kann. Die grundlegenden Technologien des WWW waren und sind es bis heute: HTML zur Darstellung und Verlinkung von Informationen (Hyper Text Markup Language), URI/ URL (Unified Ressource Identifier bzw. Locator) zur Lokalisierung einer Ressource z.B. eines HTML-Dokuments im Rechnernetz, HTTP (Hyper Text Transfer Protocol) zur Übertragung dieser Ressource im Rechnernetz. Zur detaillierten Historie, Funktionsweise und weiteren Entwicklung des WWW siehe zum Beispiel Tim Berners-Lee, Mark Fischetti: Weaving the web. The original design and ultimative destiny of the World Wide Web by its inventor, New York 2011. Niels Brügger: Web history, New York, Bern 2010. James Gilles, Robert Cailliau: How the Web was born. The story of the World Wide Web, Oxford University Press, 2000.} Damit eignete es sich auch, die Forderungen der Open-Bewegung technisch zu implementieren. Folglich werden überwiegend webbasierte Technologien in der Open-Bewegung eingesetzt, insbesondere die des Web 2.0, welche die Interaktionsmöglichkeiten im digitalen Raum erheblich erweiterten.\footnote{Vgl. Benedikt Fecher, Sönke Friesike: Open Science. One Term, Five Schools of Thought, Springer, 2014, S.11, doi:10.1007/978-3-319-00026-8\_2.} Eine wichtige Voraussetzung für viele heutige Open (Science) Projekte war zudem, dass die Technologien hinter dem WWW selbst von Anfang an offen waren, diese also (kosten)frei für jeden zur Verfügung standen und von jedem genutzt werden konnten.\footnote{Der Begründer Tim Berners-Lee hat sich von Anfang dafür eingesetzt das WWW offen zu halten. Er gründete 2012 in London das gemeinnützige Open Data Institute (ODI) mit, wodurch er selbst ein (einflussreicher) Vertreter der Open-Bewegung ist. URL: \url{https://theodi.org/} (letzter Zugriff am 27.04.2022).} 

Die Open Science-Bewegung kann in diesem Kontext als Weiterentwicklung der vor 20 Jahren gegründeten Open Access-Bewegung gesehen werden, in der sich Wissenschaftler*innen 2002/2003 zusammengeschlossen haben, um offenen Zugang zu wissenschaftlichen Forschungsergebnissen zu fördern.\footnote{Siehe Erklärung der ,,Budapest Open Access Initiative'' vom 14.02.2002, URL: \url{https://www.budapestopenaccessinitiative.org/read/} sowie ,,Berliner Erklärung über den offenen Zugang zu wissenschaftlichem Wissen'' vom 22. Oktober 2003, abgerufen auf der Website der Max Planck Gesellschaft, URL: \url{https://openaccess.mpg.de/Berliner-Erklaerung} (alle letzter Zugriff am 02.05.2022)} Daneben umfasst die Open-Bewegung unter anderem Open Knowledge, Open GLAM, Open Government, Open Design, Open Innovation, wobei es eine trennscharfe Abgrenzung nicht gibt. So lässt sich Open Data auch als Querschnittsbereich auffassen, der in andere Bereiche wie Open Science hineinreicht.\footnote{Vgl. Birgit Schmidt, Astrid Orth, Gwen Franck, Iryna Kuchma, Petr Knoth, José Carvalho: Stepping up Open Science Training for European Research, in: Publications (Hrsg), 2 Ausgabe, 2016, S. 3, doi:10.3390/publications4020016. Eine konzise Übersicht aller Bereiche siehe auch WMK, OKF (2019), ABC der Offenheit, S. 14-54}. Eine Vertreterin der ersten Stunde der Open-Bewegung und die wohl populärste ist die gemeinnützige Wikimedia Foundation, Inc. (WMF)\footnote{URL: \url{https://wikimediafoundation.org/de/} (letzter Zugriff am 22.04.2022)} mit Sitz in den USA.\footnote{Vgl. den Wikipedia-Eintrag zur Wikimedia Foundation, Seite ,,Wikimedia Foundation''. In: Wikipedia – Die freie Enzyklopädie. Bearbeitungsstand: 31. März 2022, 20:07 UTC. URL: \url{https://de.wikipedia.org/w/index.php?title=Wikimedia_Foundation\&oldid=221669459.} (letzter Zugriff am 22.04.2022) In Deutschland vertreten durch den Verein Wikimedia Deutschland e. V., vgl. ebd.} Bereits seit 2001 stellt sie digitale Dienste kostenfrei zur Verfügung, mit denen Wissen offen ausgetauscht und geteilt werden kann. Ihr bekanntestes und ältestes Projekt ist die freie Enzyklopädie \textit{Wikipedia}\footnote{URL: \url{https://de.wikipedia.org/wiki/Wikipedia:Hauptseite} (letzter Zugriff am 22.04.2022)}. Die WMF engagiert sich aber nicht ausschließlich mit der Wikipedia in der Open-Bewegung, sondern hat inzwischen eine Vielzahl an digitalen ,,Schwesternprojekten''\footnote{Zum Beispiel das Wörterbuch Wictionary (2002), URL: \url{https://de.wiktionary.org/}; die Text- und Quellensammlung Wikisource (2003), URL: \url{https://de.wikisource.org/wiki/Hauptseite}; die Mediensammlung Wikimedia Commons (2004), URL: \url{https://commons.wikimedia.org/wiki/Hauptseite}; die Wissensdatenbank Wikidata (2012), URL: \url{https://www.wikidata.org/wiki/Wikidata:Main_Page} (alle letzter Zugriff am 22.04.2022). Eine Auflistung aller Wikimedia-Projekte ist auf der Homepage zu finden unter \url{https://www.wikimedia.de/projekte/} (letzter Zugriff am 22.04.2022)} Daneben stellt sie eine Reihe ihrer MediaWiki Software-Komponenten in Open Source zur Verfügung.\footnote{Eine Übersicht ist auf der Website zu finden unter URL: \url{https://doc.wikimedia.org/} (letzter Zugriff am 22.04.2022)} Eine weitere und mit der WMF koopierende Organisation in der Open-Bewegung ist die Open Knowledge Foundation (OKF), die 2005 in London gegründete wurde\footnote{URL: \url{https://okfn.org/} (letzter Zugriff am 22.04.2022).} und von der es seit 2011 auch einen deutschen Ableger in Berlin gibt.\footnote{URL: \url{https://okfn.de/} (letzter Zugriff am 22.04.2022).}. Anders als die WMF hat die OKF kein zentrales Projekt mit einer homogenen Softwarelandschaft, sondern wirkt unterstütztend und begleitetend an kleineren Projekten.\footnote{Siehe Open Knowledge Foundation (Hrsg.): What we do? URL: \url{https://okfn.org/what-we-do/} (letzter Zugriff am 26.04.2022).}

Beide hier vorgestellten Initiativen engagieren sich ebenfalls in der Open Science. An der deutschsprachige OKF hat sich die Arbeitsgruppe Open Science gegründet, die wiederum von der Wikimedia Deutschland unterstützt wird.\footnote{Siehe Website der AG Open Science, URL: \url{https://ag-openscience.de/netzwerk/} (letzter Zugriff am 03.05.2022).} In der offenen AG kommen unterschiedliche Akteure aus der Wissenschaft zusammen, die gemeinsam Open Science-Ziele für die Wissenschaft formulieren.\footnote{Vgl. Open Science AG (Hrsg.): Mission Statement. Science - Open by default, Verison 1.0, Oktober 2014, URL: \url{https://ag-openscience.de/mission-statement/} (letzter Zugriff am 03.05.2022).} Die Wikimedia Deutschland gibt die Blogreihe „Freies Wissen und Wissenschaft“ heraus, in der bisher Stärken und Vorteile von Open Science für die traditionelle Wissenschaft herausgearbeitet wurden.\footnote{Wikimedia Deutschland (Hrsg.): Freies Wissen und Wissenschaft, Blogreihe, Teil 01-07, URL: \url{https://blog.wikimedia.de/2015/04/20/freies-wissen-und-wissenschaft-teil-01-science-2-0-die-digitalisierung-des-forschungsalltags/} (letzter Zugriff am 03.05.2022).} Außerdem hat sie zwischen 2016 und 2021 das interdisziplinäre Fellow-Programm \textit{Freies Wissen} durchgeführt, mit dem Nachwuchswissenschaftler*innen bei der Integration von Open Science in das eigene Forschungsprojekt gefördert wurden.\footnote{Sarah Behrens, Christopher Schwarzkopf, Anna-Katharina Gödeke, Dr. Dominik Scholl, Nico Schneider (2022): Fellow-Programm Freies Wissen 2016 - 2021, Zenodo, doi:10.5281/zenodo.5788379. Siehe auch Informations- und Kommunikationskanäle des Fellow Programms auf de.wikimedia.org, URL's: \url{https://www.wikimedia.de/projects/fellow-programm-freies-wissen/}, \url{https://de.wikiversity.org/wiki/Wikiversity:Fellow-Programm_Freies_Wissen}, \url{https://blog.wikimedia.de/c/fellow-programm-freies-wissen-de/} (alle letzter Zugriff am 03.05.2022)} Mit diesem Zugriff auf die Wissenschaft war der Effekt des Programms auch, dass Open Science-Multiplikatoren ausgebildet wurden, die die Idee und Praxis von Open Science in wissenschaftlichen Einrichtungen und Communities verbreiten und festigen.\footnote{Vgl. Moritz Schubotz, Isabella Peters, Benedikt Fecher, Dominik Scholl (2020): Lessons Learned aus dem Fellow-Programm Freies Wissen. Open-Access-Tage 2020 (OAT2020), Bielefeld, Germany, Zenodo, doi:10.5281/zenodo.4009144}

\subsection{Definition} 

Eine allgemeingültige Definition von Open Science, die hier eins zu eins übernommen werden kann, existiert nicht.\footnote{Bestätigt wird diese Aussage von dem öffentlichen Wiki ,,forschungsdaten.org'' der Universität Koblenz, welches seit 2019 von der Universität betrieben wird (vorher vom Helmholtz-Zentrum Potsdam und Deutschem GeoForschungsZentrum GFZ), in dem allein 11 Definitionen vorgstellt werden, vgl. URL: \url{https://www.forschungsdaten.org/index.php/Open_Science} (letzter Zugriff am 30.04.2022).} Erschwerend kommt hinzu, dass ebenfalls die Open Research oder Open Scholarship oft, aber nicht immer synonym verwendet werden.\footnote{Siehe zum Beispiel Freie Universität Berlin (Hrsg.): FDM Glossar. Open Science\/ Open Research\/ Open Scholarship, URL: \url{https://www.fu-berlin.de/sites/forschungsdatenmanagement/glossar/open-science-open-research-open-scholarship.html}, Ben Kaden: Drei Gründe für Forschungsdatenpublikationen, Blogartikel auf eDissPlus, DFG-Projekt: Elektronische Dissertationen Plus, 29.09.2016, URL: \url{https://www2.hu-berlin.de/edissplus/2016/09/29/gruende-fuer-forschungsdatenpublikationen/} (alle letzter Zugriff am 30.04.2022).} Hieraus ergibt sich ein Definitionsproblem für diese Arbeit, das sich aus dem IST-Stand von Open Science ergibt. Denn entsprechende Verfahren und Strukturen sowohl auf der technischen als auch auf der organisatorischen Ebene haben sich schlichtweg noch nicht etabliert. Zwar gibt es - wie der vorherige Abschnitt gezeigt hat - ein großes Bekenntnis zu Open Science, doch die feste Verankerung in das bestehende Wissenschaftssystem ist noch nicht erfolgt. Erst aber in diesem Prozess wird sich Open Science abschließend konsolidieren. 

Es können aber die sogenannten Open Science-Grundsätze als ,,weiche'' Definition und als Handlungsrahmen für diese Arbeit herangezogen werden. Sie werden von allen recherchierten Initiativen vorgetragen und können wie folgt zusammengefasst werden: Während von wissenschaftlicher Seite insbesondere Transparenz, offene Kommunikation, Kollaboration sowie Reproduzierbarkeit und Wiederverwendbarkeit der Forschungsergebnisse betont wird, ist es von der Open-Bewegung her vor allem öffentliche Partizipation und damit die Demokratisierung der Wissenschaft, die zentral ist. Open Science wird als moderne Wissenschaftspraxis gesehen, die traditionelle Wissenschaft dort transformiert, wo es - wie die Replikationskrise gezeigt hat - notwendig ist. Das primäre Ziel ist es, durch Open Science Integrität von Wissenschaft zu stärken sowie Qualität von Forschung im digitalen Zeitalter zu steigern.\footnote{Vgl. Ina Friebe: Forschungsqualität durch Open Science verbessern, veröffentlicht auf der Website der Berlin University Alliance (Hrsg.) am 12.05.2021, URL: \url{https://www.berlin-university-alliance.de/impressions/210512-lecture-series-o3/index.html} (letzer Zugriff am 27.04.2022).} Eine wichtige Eigenschaft dieser Grundsätze ist zudem, dass sie generisch, das heißt über alle wissenschaftlichen Domänen hinweg gültig sind.\footnote{Vgl. CODATA Coordinated Expert Group, Paul Arthur Berkman, Jan Brase, Richard Hartshorn, Simon Hodson, Wim Hugo, Sabina Leonelli, Barend Mons, Hana Pergl, Hans Pfeiffenberger: Open Science for a Global Transformation: CODATA coordinated submission to the UNESCO Open Science Consultation. Zenodo 2020, Version 1, S. 13 doi:10.5281/zenodo.3935461.} Von daher spricht Open Science nicht allein die lebens- und naturwissenschaftlichen Bereiche, sondern gleichermaßen auch die geisteswissenschaftlichen an und deren Grundsätze sind folglich auch auf die Forschungsdaten zur Vernichtung der jüdischen Gewerbetätigkeit im NS anwendbar.

Während diese Open Science-Grundsätze (manchmal auch Open Science-Principles) als gesetzt gelten können, bleibt die Antwort auf die Frage nach dem Open Science-Grad, also wie weit Offene Wissenschaft auf den Forschungsprozess ausgedehnt ist, abschließend uneindeutig. Es lassen sich zum jetztigen Zeitpunkt jedoch zwei Gruppen identifizieren und abstufen:
\begin{enumerate}
\item Grad: Auf der einen Seite können die Akteure zusammengefasst werden, die unter Open Science die Veröffentlichung aller Forschungs\textit{ergebnisse} verstehen.\footnote{Siehe zum Beispiel die Selbstverständnis-Erklärung des Arbeitskreises Open Science der Helmholtz-Gemeinschaft, URL: \url{https://os.helmholtz.de/open-science-in-der-helmholtz-gemeinschaft/stakeholder-und-ihre-rollen/arbeitskreis-open-science/selbstverstaendnis-des-arbeitskreises-open-science-der-helmholtz-gemeinschaft/} (letzter Zugriff am 01.05.2022). Auch das öffentliche Zenodo-Repositorium wird vielfach so verwendet, vgl. Kapitel 2.1.3 Infrastrukuren.} Sie sehen den Fortschritt in Open Science darin, dass nicht mehr textbasierte Publikationen wie wissenschaftliche Artikel, Monografien, Editionen, etc. zugänglich sind, sondern ebenfalls alle digitalen Ressourcen, wie Daten oder Software, die epistemologischen Wert besitzen, also zu den gewonnen Erkenntnissen beigetragen haben. Diese digitalen Ressourcen werden als Teil der Forschungsergebnisse interpretiert und diese müssen in der Konsequenz veröffentlicht werden. Der traditionelle Forschungsprozess an sich bleibt größtenteils unberührt. Lediglich dessen abschließende Phase, wenn es darum geht Ergebnisse zu kommunizieren, soll erweitert werden und hier Zugänglichkeit und Wiederverwendbarkeit von Forschungsergebnissen gefördert werden.
\item Grad: Auf der anderen Seite stehen die Akteure, vor allem aus dem Dunstkreis der Replikationskrise, die hier noch sehr viel weiter als oben genannte Akteure gehen. Denn sie wollen die Open Science-Grundsätze auf alle Phase des Forschungszyklus angewandt sehen und damit den gesamten Forschungsprozess transparent machen. Aus der Erfahrung der Replikationskrise heraus ist ihr Hauptargument, dass es, um Reliabilität von Wissenschaft zu gewährleisten, nicht ausreicht, nur publizierte Forschungsergebnisse zur Verfügung zu haben.\footnote{Vgl. Benedikt Fecher, Mathis Fräßdorf, Marcel Hebing, Gert G. Wagner: Replikationen, Reputation und gute wissenschaftliche Praxis, in: Information - Wissenschaft \& Praxis (Hrsg.), Bd. 68, Ausgabe 2-3, 2017, S. 154-158, doi:10.1515\/iwp-2017-0025.} Dabei stimmt sie den Forderungen der ersten Gruppen grundsätzlich zu, erweitert diese aber, indem sie die Praxis des Veröffentlichens ausschließlich \textit{publizierbarer} Forschungsergebnisse aufbrechen will. Genau hierin liegt der entscheidende Unterschied zu den Akteuren der ersten Gruppe. Denn bei dieser konsequenten Umsetzung der Open Science-Grundsätze, würden auch alle Rohdaten und Working Papers - also die Zwischenergebnisse -, vor ihrer Bereinigung bzw. vor dem Peer-Review, sowie dokumentierte Workflows der Forschungsarbeit mit Methodenentwicklung und Forschungsdesign zugänglich sein. Erst auf diese Weise - so die Argumentation - lasse sich der gesamte Erhebungs-, Verarbeitungs- sowie Analysesprozess von Forschungsdaten und damit der Erkenntnisprozess selbst in größtmöglicher Transparenz nachvollziehen und befähigt im Sinne einer Datenkritik, sowohl die Daten als auch die Ergebnisse nachträglich zu beurteilen und abschließend zu bewerten, was insbesondere für deren Nachnutzung von epistemologischer Bedeutung ist.\footnote{Und auch von lebensrettender Bedeutung, wie im Zusammenhang mit der COVID-19-Pandemie seit 2020 vielfach diskutiert wird. Den Open Science-Kerneigenschaften wie der globale ungehinderte Austausch von Daten, Papers und Zwischenergebnissen werden eine entscheidende Rolle bei der raschen Impfstoffentwicklung zugewiesen. Siehe Lonni Besançon, Nathan Peiffer-Smadja, Corentin Segalas, Haiting Jiang, Paola Masuzzo, Cooper Smout, Eric Billy, Maxime Deforet, Clémence Leyrat: Open science saves lives: lessons from the COVID-19 pandemic, in: BMC Medical Research Methodology, Band 21, Artikelnr. 117, 2021, doi:10.1186/s12874-021-01304-y und CODATA Coordinated Expert Group (2020): Open Science for a Global Transformation. CODATA coordinated submission to the UNESCO Open Science Consultation, Zenodo, doi:10.5281/zenodo.3935461.} 
\end{enumerate}  

Die vorgestellten Diffenzierungen von Open Science machen deutlich, dass es \textit{die} Open Science nicht gibt und bis zu welchem Grad sich Open Science am Ende durchsetzen wird, muss in dieser Arbeit offen bleiben. Letztendlich hängt diese Entwicklung stark vom Selbstverständnis der jeweiligen Initiatve, Einrichtung oder des jeweiligen Wissenschaftsbereichs sowie von anderen Variablen wie rechtliche oder forschungsethische Rahmenbedingungen ab. Es ist daher wahrscheinlich, dass sich Open Science unter der gemeinsamen Klammer der Open Science-Grundsätze zukünftig weiter ausdifferenzieren wird und unterschiedliche Grade nebeneinander existieren werden.

\subsection{Konzepte und Infrastrukturen} 

\paragraph{Konzepte}

In Bezug auf Konzepte wird häufig der \textit{Umbrella Term} herangezogen, um die verschiedenenen Open Science-Handlungsfelder in der Wissenschaft zu veranschaulichen and damit die Dimensionen von Open Science zu verdeutlichen (Abb. 2.1.). 

\begin{figure}[h]
    \centering
    \frame{\includegraphics[scale=0.7]{open_science_eosc-hub}}
    \caption{Definition der Open Science-Handlungsfelder nach der Europäischen Kommission.\protect\footnotemark}
    \label{fig:x cubed graph}
\end{figure} \footnotetext{URL: \url{https://www.eosc-hub.eu/open-science-info} (letzter Zugriff am 03.05.2022).}

Die Europäische Kommission zum Beispiel definiert für das große EU-Infrastrukturprojekt ,,European Open Science Cloud'' (EOSC)\footnote{Siehe Abschnitt ,,Infrastrukturen''.}, welche im Rahmen des Langzeitprogramms \textit{Horizon Europe} aufgebaut wird\footnote{Horizon Europe startete 2020 und läuft noch bis 2027 mit einem Förderungsumfang von insgesamt 95,5 Milliarden Euro (Phase 2021-27), URL: \url{https://ec.europa.eu/info/research-and-innovation/funding/funding-opportunities/funding-programmes-and-open-calls/horizon-europe_en} (letzter Zugriff am 03.05.2022)}, sechs Handlungsfelder - wie aus der Abbildung 2.1. hervorgeht. Dabei kombinieren die Handlungsfelder Praktiken aus der traditionellen Wissenschaft mit den Open Science-Grundsätzen und entwickeln daraus Lösungskonzepte für die wissenschaftliche Forschung nach Schwerpunkten. Open Data-Konzepte unter dem Dach der Open Science zum Beispiel konzentrieren sich auf den wissenschaftlichen Umgang mit den im Forschungsprozess anfallenden digitalen Forschungsdaten, während sich Open Access-Konzepte mit Fragen des freien Zugangs zu diesen und sonstigen wissenschaftlichen Materialen beschäftigen. Citizen Science-Konzepte entwickeln Lösungen, wie unter Beibehaltung wissenschaftlicher Integrität Partizipation an Wissenschaft gestärkt werden kann.\footnote{Siehe zum Beispiel die Citizen Science-Plattform ,,Bürger schaffen Wissen'', URL: \url{https://www.buergerschaffenwissen.de/} (letzter Zugriff am 03.05.2022).}

\begin{figure}[h]
    \centering
    \frame{\includegraphics[scale=0.2]{open_science_british-psy-soc}}
    \caption{Definition der Open Science-Handlungsfelder (hier Open Scholarship) nach der Britischen Gesellschaft für Psychologie (The British Psychological Society).\protect\footnotemark}
    \label{fig:x cubed graph}
\end{figure} \footnotetext{URL: \url{https://thepsychologist.bps.org.uk/volume-33/november-2020/bropenscience-broken-science} (letzter Zugriff am 03.05.2022). Zu sehen ist hier auch die unterschiedliche Begriffsverwendungen ,,Open Science'' und ,,Open Scholarship''}

Die Handlungsfelder können voneinander abweichen, wie ein Blick auf die Abbildung 2.2 zeigt. Die Abweichungen zwischen beiden Abbildungen lassen den Schluss zu, dass es ganz ähnlich zum Open Science-Grad letztlich vom konkreten (wissenschaftlichen) Kontext abhängt, welche Handlungsfelder unter Open Science definiert werden und es hier folglich eine strenge Vorgabe nicht gibt. Schließlich hängt diese Definition auch davon ab, wo und ob überhaupt Handlungsbedarf für Open Science gesehen wird. Dass die Replikationskrise dringenden Handlungsbedarf vorwiegend in den Lebens- und Naturwissenschaften offenbart hat, heißt nicht, dass dieser gleichermaßen auch in geisteswissenschaftlichen Fächern gesehen wird, wo vorwiegend heuristische Forschungsverfahren zur Anwendung kommen, die sich fundamental von den statistisch empirschen der Naturwissenschaften unterscheiden. Das bedeutet im Umkehrschluss, dass Handlungsbedarf gegebenfalls erst noch geschaffen werden muss oder aber - und die Frage muss erlaubt sein - überhaupt nicht notwendig ist. 

\paragraph{Infrastrukturen} Anhand der gegenwärtigen Anwendungsmöglichkeiten von Open Science in der eigenen Forschung können grob\footnote{Technische Überschneidungen sind nicht ausgeschlossen. Die Einteilung orientiert sich an mögliche Nutzungsszenarien.} drei Gruppen von Infrastrukuren unterschieden werden: 1. zentrale, 2. dezentrale und 3. nachgenutzte Infrastrukturen:

\begin{enumerate}
\item Begleitend zur Reproduzierbarkeitsstudie des COS wurde das \textit{Open Science Framework} (OSF)\footnote{URL: \url{https://osf.io/} (letzter Zugriff am 28.04.2022).} entwickelt, das im Hintergrund eine zentrale IT- Infrastruktur über eine Plattform bereitstellt, die bekannte Open Science Verfahren wie Präregistrierung, Preprints und Generierung von Permalinks ermöglicht. Zum Funktionsumfang gehören außerdem Projektversionierung sowie ein generisches Repositorium zum Speichern und Aggregieren multipler Inhalte unterschiedlicher Formaten. Im veröffentlichten, diese Arbeit von Beginn an begleitenden, OSF-Projekt ,,Master thesis: Open Science in History?''\footnote{URL: \url{https://osf.io/sc9yf/?view_only=aa5eb53a48ba4eaab512d049712d704a}, hier nur mit lesendem Zugriff auf das Projekt.} wurde zum Beispiel die LaTex-Version der schriftlichen Arbeit, welche mit Git versioniert und auf GitHub zugänglich ist, und die Zotero-Library mit der verwendeten Literatur über die Add-ons-Funktionalität sowie die prototypische Wikidata-Lösung für offenes Forschungsdatenmanagement als Komponente dem Projekt hinzugefügt. Lokal gespeicherte Materialien wie die Interviewtranskripte (.pdf), der Fragebogen (.pdf) und die Literaturauswertung (.csv) wurden manuell hochgeladen. Dafür stehen verschiedenen Server zur Verfügung, darunter auch in Deutschland (Frankfurt am Main). Heterogene Dienste und verteilte Ressourcen können also im OSF individuell zusammengeführt und dort synchron gehalten werden. Damit ist das OSF im Kern ein Projektmanagement-Tool, das den gesamten Forschungsprozess begleitet. Durch eine homogen gestaltete kollaborative Arbeitsumgebung werden Wissenschaftler*innen dabei unterstützt, automatisierte Open Science Worklows in den Forschungsalltag zu integrieren.\footnote{Vertrauensvorschuss erhält das COS vor allem durch eine konsequent transparente Politik wie zum Beispiel der Veröffentlichung aller Finanzberichte, URL: \url{https://www.cos.io/about/finances} (letzter Zugriff am 28.04.2022).} Dass das OSF steigende Anwenderzahlen insbesondere durch akademische Einrichtungen in den USA verzeichent,\footnote{Zum Beispiel Princeton University, New York University, George Washington University, u.a. Siehe \url{https://osf.io/institutions} (letzter Zugriff am 21.04.2022).}, weist darauf hin, dass es das Potential hat, sich zu einem Standard in diesem Bereich zu entwickeln. Eine mögliche negative Nebenfolge dieser Entwicklung ist die Entstehung einer Plattformabhängikeit, die zum Beispiel im Zusammenhang mit den sozialen Medien inzwischen kritisiert wird und gegen die sich Widerstand regt.\footnote{Gemeint sind hier Plattformen wie Facebook, Twitter, Google, Amazon, etc., wo die momentane Plattformökonomie Monopolstellung und Machtzentrierung fördert. Siehe zu dieser Problematik Justus Haucap: Plattformökonomie. Neue Wettbewerbsregeln –
Renaissance der Missbrauchsaufsicht, in: Wirtschaftsdienst 100 (Hrsg.), 2020, S. 20-29, doi:10.1007/s10273-020-2611-9. Siehe auch das jüngste Urteil des Europäischen Gerichtshofs (EuGH) zu Verbandsklagen gegen Facebook und dessen Datenschutzpraktiken, vgl. Alexander Fanta: EU-Gericht erlaubt Verbandsklagen gegen Facebook, netzpolotik.org, 28.04.2022, URL: \url{https://netzpolitik.org/2022/dsgvo-eu-gericht-erlaubt-verbandsklagen-gegen-facebook/} (letzter Zugriff am 30.04.2022). Zur Zeit in den Schlagzeilen und kontrovers diskutiert ist der Kauf von Twitter durch den Tech-Milliardär Elon Musk, vgl. Alexander Fanta: Der EU droht die Kraftprobe mit Elon Musks Twitter, netzpolitik.org, 26.04.2022, URL: \url{https://netzpolitik.org/2022/digitale-dienste-gesetz-der-eu-droht-die-kraftprobe-mit-elon-musks-twitter/} (letzter Zugriff am 30.04.2022)} Freilich steht hinter der Plattformökonomie selbst kein Automatismus und es nicht gesagt, dass das OSF irgendwann in einer Reihe mit den großen US-amerikanischen Digitalkonzernen\footnote{Gemeint sind hier Google, Facebook, Twitter, Amazon, etc.} stehen wird. Dennoch bleibt festzuhalten, dass das COS, als Akteur hinter dem OSF, mit seiner Plattform Gestaltungsmacht in der Frage hat, was Offenheit in der Wissenschaft bedeutet. Diese Macht wird mit steigenden Nutzerzahlen wachsen. Das COS sowie sonstige Anbieter von ähnlichen Open Science Produkten stehen hier in der besonderen Verantwortung und vor der herausfordernden Aufgabe, mögliche Machtgefälle und Abhängikeiten stetig zu reflektieren und zu kommunizieren, das heißt sich die Frage nach Vertrauenswürdigkeit und Legitimation immer wieder neu zu stellen. Denn was Open Science am Ende ist, ist nach der Open-Bewegung keine Frage von Einzelakteuren, sondern ein andauernder demokratischer Aushandlungsprozess.\footnote{Positiv hervorzuheben ist, dass das COS alle seine Softwareprodukte auf GitHUb in Open Source veröffentlicht. Siehe URL: \url{https://github.com/CenterForOpenScience} (letzter Zugriff am 30.04.2022).} 

\item Eine etwas andere Entwicklung ist derzeit in Europa zu beobachten, wo es ein zentrales, allumfassendes Infrastrukturangebot, wie das OSF, nicht gibt. Zwar existieren einzelne Projekte wie zum Beispiel das Repositorium \textit{Zenodo} (seit 2016)\footnote{URL: \url{https://zenodo.org/} (letzter Zugriff am 28.04.2022)}, doch ist dieses Infrastrukturangebot funktional auf die Archivierung, Verfügbarkeit und Zugänglichkeit einzelner digitaler Ressourcen zugeschnitten\footnote{Siehe Upload-Seite in Zenodo, URL: \url{https://zenodo.org/deposit/new} (letzter Zugriff am 30.04.2022)}, die wiederum von ,,Communities'' kuratiert werden können\footnote{Zum Beispiel die Community ,,Deutsch-jüdische Geschichte'', URL: \url{https://zenodo.org/communities/djg} (letzter Zugriff am 28.04.2022)}. Auf die Masterarbeit angewandt, konnte das GitHub-Repositorium mit der Versionierung hier nicht - analog zum OSF - synchronisiert werden. Zenodo bietet aber die Möglichkeit, automatisiert den jeweils aktuellen Repo-Release von GitHub als verpackte .zip-Archivdatei hochzuladen und zu veröffentlichen.\footnote{Siehe URL: \url{https://zenodo.org/account/settings/github/} (letzter Zugriff am 28.04.2022)} Der erste Release dieser Arbeit erfolgte aber üblicherweise erst mit deren Abgabe und damit in der finalen Phase des Enstehungsprozesses.\footnote{Zum Vergleich: Im OSF konnte die Arbeit während des gesamten Entstehungsprozesses eingesehen werden. Es kann freilich in Zenodo jederzeit manuell eine .zip-Archivdatei hochgeladen werden, was aber aufwändig insofern ist, dass es in die tägliche Forschungsarbeit als Workflow manuell integriert werden muss.} Das ist kein Beleg, aber ein Indiz dafür, dass der Schwerpunkt in Zenodo auf \textit{publizierbaren} Ressourcen liegt. Diese Vermutung wird auch von einer Stichprobenauswertung zur Nutzung von Zenodo in dessen globaler Suche nach ,,Datasets'' und ,,Publications | Articles'' gestützt.\footnote{Dies kann über die Versionsnummer der Ressource identifiziert werden. URL der Suchanfrage am 29.04.2022: \url{https://zenodo.org/search?page=1&size=20&type=dataset&type=publication&subtype=article&sort=mostrecent} Viele Artikel und Datensätze existieren häufig nur in einer Version (v1), was dafür spricht, dass insbesondere die finalen Ergebnisse auf Zenodo veröffentlicht werden. Es wäre an dieser Stelle interessant gewesen, einmal systematisch und mit computationalen Methoden zu evaluieren, wie Zenodo von Wissenschaftler*innen verwendet wird und empirisch gesicherte Aussagen zu treffen, bis zu welchem Grad Open Science tatsächlich praktiziert wird. Dies könnte zum Beispiel mit der von Zenodo bereitgestellten öffentlichen REST-API oder dem OAI-PMH Protokoll realisiert werden, URL: \url{https://developers.zenodo.org/} (letzter Zugriff am 29.04.20222). Diese Auswertung konnte im Rahmen der Arbeit nicht mehr geleistet werden.} Der Hauptunterschied zum OSF besteht darin, dass Zenodo bis auf GitHub-Releases keine Services zur Integration automatisierter Workflows in den Forschungsalltag im Portfolio hat. Wer mit Zenodo konsequent Open Science über den gesamten Forschungsprozess praktizieren will, muss dies über manuell iteratives Hochladen von Ressourcen machen. Mit der \textit{European Open Science Cloud} (EOSC, seit 2018)\footnote{URL: \url{https://eosc-portal.eu/} (letzter Zugriff am 27.04.2022)} gibt es aktuell außerdem ein großes europäisches Infrastrukturprojekt, das zum Ziel hat, Dienste, Daten und andere Ressourcen ,,from a wide range of national, regional and institutional public research infrastructures across Europe''\footnote{Europäische Kommission (Hrsg.): European Open Science Cloud, URL: \url{https://digital-strategy.ec.europa.eu/en/policies/open-science-cloud} (letzter Zugriff am 28.04.2022).} über das \textit{EOSC Portal}\footnote{URL: \url{https://eosc-portal.eu/} (letzter Zugriff am 28.04.2022).} zentral zu verzeichnen, die wiederum von EOSC-Nutzer*innen in eigenen Projekten verwaltet werden können. Der Unterschied zum OSF besteht hauptsächlich darin, dass die EOSC kein Infrastrukturangebot ist, auf der individuell Open Research praktiziert werden kann. Die EOSC ist selbst nur Aggregator bereits existierender Angebote, registriert und vernetzt diese miteinander. Sie ist mehr Verzeichnes als Plattform, das Sichtbarkeit und Recherchierbarkeit dezentraler Infrastrukturen ermöglicht. Die Möglichkeiten der Interkation sind daher auf diese Zwecke beschränkt.\footnote{Auch hier wurde testweise ein Projekt für die Masterarbeit angelegt. Eigene Ressourcen konnten nicht hochgeladen/ eingebunden, sondern nur in der Cloud registrierte Open Science Angebote in einer privaten Liste gespeichert werden..}

\item Neben dem Aufbau neuer Infrastrukturen für die Wissenschaft gibt es außerdem den Ansatz, bestehende und etablierte Infrastrukturen aus der weiter gefassten Open-Bewegung nutzbar zu machen. Hervorzuheben sind die Angebote der Wikimedia Foundation, die sich, wie in Kapitel 2.1.1 beschrieben, mit dem ,,Fellow-Programm Freies Wissen'' bereits aktiv in die Open Science-Bewegung eingebracht hat. Aktuell laufen unterschiedliche Projekte, die das sogenannte Wiki*versum in der wissenschaftlichen Forschungsarbeit nutzen. Aus dem Fellow Programm stammt das Wiki*versum-Projekt \textit{Die Datenlaube}, wo das Massenblatt ,,Die Gartenlaube – Illustrirtes Familienblatt'' aus dem 19. Jahrhundert mittels Commons, Wikisource und Wikidata kollaborativ erschlossen und analysiert wurde.\footnote{In Commons digitalisiert (\url{https://commons.wikimedia.org/w/index.php?title=Category:Gartenlaube_(Magazine)&oldid=334192328&uselang=de}), mit Wikisource transkribiert (\url{https://de.wikisource.org/w/index.php?title=Die_Gartenlaube&oldid=4048963}) und in Wikidata strukturiert erfasst und ausgewertet. Siehe zum Projekt auch das öffentliche Repositorium auf GitHub, URL: \url{https://github.com/DieDatenlaube} sowie das Blog, URL: \url{http://diedatenlaube.github.io}. Ein Überblick über das Projekt ist auf das Wikimedia-Blog veröffentlicht, siehe Christopher Schwarzkopf: Hilfe für die Datenlaube: mit [[Wikisource+Wikidata]] die freie Quellensammlung verbessern, Wikimedia Deutschland, 16. Oktober 2019, URL: \url{https://blog.wikimedia.de/2019/10/16/hilfe-fuer-die-datenlaube-mit-wikisourcewikidata-die-freie-quellensammlung-verbessern/} (letzter Zugriff am 01.05.2022).} Ein weiteres, nicht aus dem Fellow Programm stammendes Projekt ist die \textit{Bamberger Islam-Enzyklopädie}. Bei diesem wurde wissenschaftlich betreut in der deutschsprachigen Wikipedia eine Enzyklopädie zum Themenbereich Islam aufgebaut und wird in der Fortsetzung kollaborativ ergänzt.\footnote{Siehe Vorstellung des Projekts auf der Website der Universität Bamberg, URL: \url{https://www.uni-bamberg.de/islamwissenschaft/bie/} (letzter Zugriff am 01.05.2022). Beispielartikel in der Wikipedia \textit{Fādilīya}, URL: \url{https://de.wikipedia.org/w/index.php?title=Fādilīya&oldid=202323908.}} Vorteilhaft bei den Wiki*versum-Lösungen ist die Ausnutzung von Synergieeffekten. Die Wissenschaft kann die langjährigen Erfahrungen der Wikimedia bei der Implementierung von Offenheitskriterien für sich nutzen und deren Tools frei verwenden. Umgekehrt können dadurch gleichzeitig fundierte Erkenntnisse aus der wissenschaftlichen Forschung effizient in die Öffentlichkeit transferiert und das Wissen im Wiki*versum dadurch für alle verbessert werden. Die Projekte zeigen schließlich auch, dass vorhandene offene Infrastrukturen für die wissenschaftliche Forschung adaptiert und damit nutzbar gemacht werden können. Mit dem großen Angebotsspektrum bietet sich zudem für viele Open Sciene-Handlungsfelder eine Nutzungsoption. Auch wenn sich die WMF im Open Science Bereich engagiert, bleibt anzumerken, dass deren Angebote nicht auf die Bedürfnisse der Wissenschaft zugeschnitten sind, sondern in erster Linie dem Grundsatz des freien Wissens für alle folgen. Daher muss für jedes Projekt individuell evaluiert werden, inwiefern hier ein oder mehrere Wikimedia-Angebote für die eigene Forschungsarbeit in Frage kommen.\footnote{Dies wird auch in den beiden vorgestellten wissenschaftlichen Wiki*versum-Projekten so reflektiert..}    
\end{enumerate}

Der Blick auf die Infrastrukturebene zeigt, dass die Möglichkeiten von Open Science stark von den Infrastrukturen im Hintergrund abhängen und sie folglich die Basis bilden, auf die Open Science-Konzepte wie Technologien aus der Open Access, Open Data, Open Source etc. aufsetzen.

\section{Forschungsdatenmanagement}

Digitale Forschungsdaten, die im Zuge des Forschungsprozesses erzeugt werden, sind Bestandteil auch in der Forschungsarbeit von Historiker*innen geworden. Mit ihnen rücken in den Geschichtswissenschaften computergestützte qualitative wie quantitative Analyse- und Auswertungsverfahren in den Fokus. Lehrstühle wie der für Digital History an der Humboldt-Universität zu Berlin haben sich darauf eingestellt und , digitale Konzepte, Methoden und Verfahren für die geschichtswissenschaftliche Forschung zu reflektieren und zu anzupassen.  Unstrittig ist, dass digitale Forschungsdaten wichtige Ressource bei der Erkenntnisgenerierung sind, Es können unterschiedlich vorliegen (Def. Forschungsdaten)

Wenn aber Forschungsdaten epistemologisch an Bedeutung für die Geschichtswissenschaften gewinnen, dann stellen sich unweigerlich Fragen nach dem wissenschaftlichen Umgang mit ihnen. Daraus wurde bereits die Notwendigkeit eines Forschungsdatenmanagements abgeleitet. Das Ziel ist, Methoden, Verfahren und Maßnahmen zur Handhabung von Forschungsdaten zu entwickeln. Es bezieht sich zum einen auf die Forschungs- und Arbeitsphasen innerhalb des Forschungsprozesses. Zum anderen geht es darüber hinaus, das heißt Forschungsdaten sollen in die Forschung zurückgespielt werden können und langzeitig zur Verfügung stehen. Problematisch ist hierbei, dass aufgrund des großen Anteils projektbezogener Einzelförderung - bei der DFG immerhin mehr als ein Drittel im Jahr 2020  – nicht allen Forschungsvorhaben ein nachhaltiges Forschungsdatenmanagement inhärent ist. Da es entsprechende Forschungsgebiete in der Vergangenheit schlichtweg noch nicht gab, war der Umgang mit Forschungsdaten mehr von individuellen digitalen Kenntnissen und Kompetenzen des oder der Wissenschaftler*in abhängig als von allgemeingültigen wissenschaftlichen Kriterien sowie technischen Standards. Zeitökonomisch betrachtet bedeutet der wissenschaftliche Umgang mit digitalen Forschungsdaten zudem Arbeitsaufwand, der zu den routinierten Abläufen hinzukommt. Erst recht, wenn sich ganz neu mit dieser Thematik auseinandergesetzt werden muss. Das wirft die berechtigte Frage nach dem Kosten-Nutzen-Verhältnis für die eigene Forschungsarbeit auf.

Klar ist, dass diese Aufgabe allein auf individueller Ebene nicht bewältigt werden kann, sondern dafür entsprechende digitale Infrastrukturen, Dienste und Tools unterstützend bereitgestellt werden müssen. Aktuell gibt es nationale Anstrengungen wie die Initiative Nationale Forschungsdateninfrastruktur , die in dieser offenen Situation Positionen und Lösungsstrategien entwickeln. Die Geschichtswissenschaften werden voraussichtlich ab Januar 2023 offiziell mit dem Konsortium „nfdi4memory“ vertreten sein und damit verbunden eine zehnjährige Förderung erhalten (Stand 10/2021).

\subsection{Forschungsdaten und Forschungsdatenlebenszyklus}



\subsection{FAIR und Open Data}

drei Modelle besprechen

hier auch Vergleich zu 5-Sterne-Modell von Open Data (Tim Berners Lee)

Auf diese Prinzipien beruft sich auch die Wikimedia Foundation, für ihre Produkte und macht diese auch für die Wissenschaft interessant. Mittlerweile gibt es diverse Kooperationen zwischen wissenschaftlichen Einrichtungen und der Wikimedia. So hat die Deutsche Nationalbibliothek ein Projekt gestartet, in dem sie die GND zugänglicher und nachnutzbarer für gestalten will und damit ihre strenge GND-Policy

wie zum Beispiel dauerhafte digitale Identifikatoren\footnote{Dazu gehören das Digital Object Identifier System (DOI), URL: \url{https://www.doi.org/}; Uniform Resource Identifier (URI), URL: \url{https://www.w3.org/TR/uri-clarification/\#uri-schemes} oder Permalink (alle letzter Zugriff am 01.05.2022).}.