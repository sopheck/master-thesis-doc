\section{Transkripte}
\section{Codepoints}
\section{Anforderungskatalog}
Beschreibung der Forschungsdaten zur Vernichtung der jüdischen Gewerbetätigkeit mit den Anforderungen an offenes FDM
\begin{itemize}
    \item im Kontext des Forschungsfeld zur Vernichtung der wirtschaftlichen Existenz der Juden im Nationalsozialismus
    \item Bilden inhaltlich nur einen Ausschnitt aus dem Gesamtkomplex 
    \item Sie sind räumlich begrenzt.
    \item können in strukturierter als statistische Daten oder als unstrukturierte textuelle Daten 
    \item wurden sowohl im akademischen als auch im öffentlichen Umfeld generiert
\end{itemize}
\begin{table}
    \caption{Anforderungen an offenes Forschungsdatenmanagement}
    \label{tab:forschungsdatentabelle}
    \begin{tabular} { L{5cm}|L{5cm} }
    Forschungsdaten zur Vernichtung der jüdischen Gewerbetaetigkeit & Anforderung \\
    \hline 
    1 & Arisierung in Hamburg \\
    \hline 
    \end{tabular}
\end{table}
\section{Datenmodell mit Beispielerfassung}
\section{SPARQL-Beispielabfragen}